%!TEX root = std.tex
\rSec0[utilities]{General utilities library}

\rSec1[utilities.general]{General}

\pnum
This Clause describes utilities that are generally useful in \Cpp programs; some
of these utilities are used by other elements of the \Cpp standard library.
These utilities are summarized in Table~\ref{tab:util.lib.summary}.

\begin{libsumtab}{General utilities library summary}{tab:util.lib.summary}
\ref{utility}               & Utility components                & \tcode{<utility>}     \\ \rowsep
\ref{intseq}                & Compile-time integer sequences    & \tcode{<utility>}     \\ \rowsep
\ref{pairs}                 & Pairs                             & \tcode{<utility>}     \\ \rowsep
\ref{tuple}                 & Tuples                            & \tcode{<tuple>}       \\ \rowsep
\ref{optional}              & Optional objects                  & \tcode{<optional>}    \\ \rowsep
\ref{variant}               & Variants                          & \tcode{<variant>}     \\ \rowsep
\ref{any}                   & Storage for any type              & \tcode{<any>}         \\ \rowsep
\ref{template.bitset}       & Fixed-size sequences of bits      & \tcode{<bitset>}      \\ \rowsep
\ref{memory}                & Memory                            & \tcode{<memory>}      \\
                            &                                   & \tcode{<cstdlib>}     \\
\ref{smartptr}              & Smart pointers                    & \tcode{<memory>}      \\ \rowsep
\ref{mem.res}               & Memory resources                  & \tcode{<memory_resource>} \\ \rowsep
\ref{allocator.adaptor}     & Scoped allocators                 & \tcode{<scoped_allocator>} \\ \rowsep
\ref{function.objects}      & Function objects                  & \tcode{<functional>}  \\ \rowsep
\ref{meta}                  & Type traits                       & \tcode{<type_traits>} \\ \rowsep
\ref{ratio}                 & Compile-time rational arithmetic  & \tcode{<ratio>}       \\ \rowsep
\ref{time}                  & Time utilities                    & \tcode{<chrono>}      \\
                            &                                   & \tcode{<ctime>}       \\ \rowsep
\ref{type.index}            & Type indexes                      & \tcode{<typeindex>}   \\ \rowsep
\ref{execpol}               & Execution policies                & \tcode{<execution>}   \\
\end{libsumtab}

\rSec1[utility]{Utility components}

\pnum
This subclause contains some basic function and class templates that are used
throughout the rest of the library.

\indexlibrary{\idxhdr{utility}}%
\indexlibrary{\idxcode{rel_ops}}%
\synopsis{Header \tcode{<utility>} synopsis}

\pnum
The header \tcode{<utility>} defines several types and function templates
that are described in this Clause. It also defines the template \tcode{pair}
and various function templates that operate on \tcode{pair} objects.

\begin{codeblock}
#include <initializer_list>

namespace std {
  // \ref{operators}, operators:
  namespace rel_ops {
    template<class T> bool operator!=(const T&, const T&);
    template<class T> bool operator> (const T&, const T&);
    template<class T> bool operator<=(const T&, const T&);
    template<class T> bool operator>=(const T&, const T&);
  }

  // \ref{utility.swap}, swap:
  template <class T> void swap(T& a, T& b) noexcept(@\seebelow@);
  template <class T, size_t N> void swap(T (&a)[N], T (&b)[N]) noexcept(is_nothrow_swappable_v<T>);

  // \ref{utility.exchange}, exchange:
  template <class T, class U=T> T exchange(T& obj, U&& new_val);

  // \ref{forward}, forward/move:
  template <class T> 
    constexpr T&& forward(remove_reference_t<T>& t) noexcept;
  template <class T>
    constexpr T&& forward(remove_reference_t<T>&& t) noexcept;
  template <class T>
    constexpr remove_reference_t<T>&& move(T&&) noexcept;
  template <class T>
    constexpr conditional_t<
    !is_nothrow_move_constructible_v<T> && is_copy_constructible_v<T>,
    const T&, T&&> move_if_noexcept(T& x) noexcept;

  // \ref{utility.as_const}, as_const:
  template <class T> constexpr add_const_t<T>& as_const(T& t) noexcept;
  template <class T> void as_const(const T&&) = delete;

  // \ref{declval}, declval:
  template <class T>
    add_rvalue_reference_t<T> declval() noexcept;  // as unevaluated operand

@
\indexlibrary{\idxcode{index_sequence}}%
\indexlibrary{\idxcode{make_index_sequence}}%
\indexlibrary{\idxcode{index_sequence_for}}%
@
  // \ref{intseq}, Compile-time integer sequences
  template<class T, T...> struct integer_sequence;
  template<size_t... I>
    using index_sequence = integer_sequence<size_t, I...>;

  template<class T, T N>
    using make_integer_sequence = integer_sequence<T, @\seebelow{}@>;
  template<size_t N>
    using make_index_sequence = make_integer_sequence<size_t, N>;

  template<class... T>
    using index_sequence_for = make_index_sequence<sizeof...(T)>;

  // \ref{pairs}, pairs:
  template <class T1, class T2> struct pair;

  // \ref{pairs.spec}, pair specialized algorithms:
  template <class T1, class T2>
    constexpr bool operator==(const pair<T1, T2>&, const pair<T1, T2>&);
  template <class T1, class T2>
    constexpr bool operator< (const pair<T1, T2>&, const pair<T1, T2>&);
  template <class T1, class T2>
    constexpr bool operator!=(const pair<T1, T2>&, const pair<T1, T2>&);
  template <class T1, class T2>
    constexpr bool operator> (const pair<T1, T2>&, const pair<T1, T2>&);
  template <class T1, class T2>
    constexpr bool operator>=(const pair<T1, T2>&, const pair<T1, T2>&);
  template <class T1, class T2>
    constexpr bool operator<=(const pair<T1, T2>&, const pair<T1, T2>&);
  template <class T1, class T2>
    void swap(pair<T1, T2>& x, pair<T1, T2>& y) noexcept(noexcept(x.swap(y)));
  template <class T1, class T2>
    constexpr @\seebelow@ make_pair(T1&&, T2&&);

  // \ref{pair.astuple}, tuple-like access to pair:
  template <class T> class tuple_size;
  template <size_t I, class T> class tuple_element;

  template <class T1, class T2> struct tuple_size<pair<T1, T2>>;
  template <class T1, class T2> struct tuple_element<0, pair<T1, T2>>;
  template <class T1, class T2> struct tuple_element<1, pair<T1, T2>>;

  template<size_t I, class T1, class T2>
    constexpr tuple_element_t<I, pair<T1, T2>>&
      get(pair<T1, T2>&) noexcept;
  template<size_t I, class T1, class T2>
    constexpr tuple_element_t<I, pair<T1, T2>>&&
      get(pair<T1, T2>&&) noexcept;
  template<size_t I, class T1, class T2>
    constexpr const tuple_element_t<I, pair<T1, T2>>&
      get(const pair<T1, T2>&) noexcept;
  template<size_t I, class T1, class T2>
    constexpr const tuple_element_t<I, pair<T1, T2>>&&
      get(const pair<T1, T2>&&) noexcept;
  template <class T, class U>
    constexpr T& get(pair<T, U>& p) noexcept;
  template <class T, class U>
    constexpr const T& get(const pair<T, U>& p) noexcept;
  template <class T, class U>
    constexpr T&& get(pair<T, U>&& p) noexcept;
  template <class T, class U>
    constexpr const T&& get(const pair<T, U>&& p) noexcept;
  template <class T, class U>
    constexpr T& get(pair<U, T>& p) noexcept;
  template <class T, class U>
    constexpr const T& get(const pair<U, T>& p) noexcept;
  template <class T, class U>
    constexpr T&& get(pair<U, T>&& p) noexcept;
  template <class T, class U>
    constexpr const T&& get(const pair<U, T>&& p) noexcept;

  // \ref{pair.piecewise}, pair piecewise construction
  struct piecewise_construct_t { };
  constexpr piecewise_construct_t piecewise_construct{};
  template <class... Types> class tuple;  // defined in \tcode{<tuple>}

  // \ref{utility.inplace}, in-place construction
  struct in_place_tag {
    in_place_tag() = delete;
  };
  using in_place_t = in_place_tag(&)(@\unspec@);
  template <class T>
    using in_place_type_t = in_place_tag(&)(@\unspec@<T>);
  template <size_t I>
    using in_place_index_t = in_place_tag(&)(@\unspec@<I>);
  in_place_tag in_place(@\unspec@);
  template <class T>
    in_place_tag in_place(@\unspec@<T>);
  template <size_t I>
    in_place_tag in_place(@\unspec@<I>);
}
\end{codeblock}

\rSec2[operators]{Operators}

\pnum
To avoid redundant definitions of \tcode{operator!=} out of \tcode{operator==}
and operators \tcode{>}, \tcode{<=}, and \tcode{>=} out of \tcode{operator<},
the library provides the following:

\indexlibrary{\idxcode{operator"!=}}%
\begin{itemdecl}
template <class T> bool operator!=(const T& x, const T& y);
\end{itemdecl}

\begin{itemdescr}
\pnum
\requires
Type \tcode{T} is \tcode{EqualityComparable} (Table~\ref{tab:equalitycomparable}).

\pnum
\returns
\tcode{!(x == y)}.
\end{itemdescr}

\indexlibrary{\idxcode{operator>}}%
\begin{itemdecl}
template <class T> bool operator>(const T& x, const T& y);
\end{itemdecl}

\begin{itemdescr}
\pnum
\requires
Type \tcode{T} is \tcode{LessThanComparable} (Table~\ref{tab:lessthancomparable}).

\pnum
\returns
\tcode{y < x}.
\end{itemdescr}

\indexlibrary{\idxcode{operator<=}}%
\begin{itemdecl}
template <class T> bool operator<=(const T& x, const T& y);
\end{itemdecl}

\begin{itemdescr}
\pnum
\requires
Type \tcode{T} is \tcode{LessThanComparable} (Table~\ref{tab:lessthancomparable}).

\pnum
\returns
\tcode{!(y < x)}.
\end{itemdescr}

\indexlibrary{\idxcode{operator>=}}%
\begin{itemdecl}
template <class T> bool operator>=(const T& x, const T& y);
\end{itemdecl}

\begin{itemdescr}
\pnum
\requires
Type \tcode{T} is \tcode{LessThanComparable} (Table~\ref{tab:lessthancomparable}).

\pnum
\returns
\tcode{!(x < y)}.
\end{itemdescr}

\pnum
In this library, whenever a declaration is provided for an \tcode{operator!=},
\tcode{operator>}, \tcode{operator>=}, or \tcode{operator<=},
and requirements and semantics are not explicitly provided,
the requirements and semantics are as specified in this Clause.

\rSec2[utility.swap]{swap}

\indexlibrary{\idxcode{swap}}%
\begin{itemdecl}
template <class T> void swap(T& a, T& b) noexcept(@\seebelow@);
\end{itemdecl}

\begin{itemdescr}
\pnum
\remarks This function shall not participate in overload resolution
unless \tcode{is_move_constructible_v<T>} is \tcode{true} and
\tcode{is_move_assignable_v<T>} is \tcode{true}.
The expression inside \tcode{noexcept} is equivalent to:

\begin{codeblock}
is_nothrow_move_constructible_v<T> && is_nothrow_move_assignable_v<T>
\end{codeblock}

\pnum
\requires
Type
\tcode{T}
shall be
\tcode{MoveConstructible} (Table~\ref{tab:moveconstructible})
and
\tcode{MoveAssignable} (Table~\ref{tab:moveassignable}).

\pnum
\effects
Exchanges values stored in two locations.
\end{itemdescr}

\indexlibrary{\idxcode{swap}}%
\begin{itemdecl}
template <class T, size_t N>
  void swap(T (&a)[N], T (&b)[N]) noexcept(is_nothrow_swappable_v<T>);
\end{itemdecl}

\begin{itemdescr}
\pnum
\remarks
This function shall not participate in overload resolution
unless \tcode{is_swappable_v<T>} is \tcode{true}.

\pnum
\requires
\tcode{a[i]} shall be swappable with~(\ref{swappable.requirements}) \tcode{b[i]}
for all \tcode{i} in the range \range{0}{N}.

\pnum
\effects As if by \tcode{swap_ranges(a, a + N, b)}.
\end{itemdescr}

\rSec2[utility.exchange]{exchange}

\indexlibrary{\idxcode{exchange}}%
\begin{itemdecl}
template <class T, class U=T> T exchange(T& obj, U&& new_val);
\end{itemdecl}

\begin{itemdescr}
\pnum
\effects
Equivalent to:
\begin{codeblock}
T old_val = std::move(obj);
obj = std::forward<U>(new_val);
return old_val;
\end{codeblock}
\end{itemdescr}


\rSec2[forward]{forward/move helpers}

\pnum
The library provides templated helper functions to simplify
applying move semantics to an lvalue and to simplify the implementation
of forwarding functions.

\indexlibrary{\idxcode{forward}}%
\begin{itemdecl}
template <class T> constexpr T&& forward(remove_reference_t<T>& t) noexcept;
template <class T> constexpr T&& forward(remove_reference_t<T>&& t) noexcept;
\end{itemdecl}

\begin{itemdescr}
\pnum
\returns \tcode{static_cast<T\&\&>(t)}.

\pnum
\remarks If the second form is instantiated with an lvalue reference type, the program is ill-formed.

\pnum
\begin{example}
\begin{codeblock}
template <class T, class A1, class A2>
shared_ptr<T> factory(A1&& a1, A2&& a2) {
  return shared_ptr<T>(new T(std::forward<A1>(a1), std::forward<A2>(a2)));
}

struct A {
  A(int&, const double&);
};

void g() {
  shared_ptr<A> sp1 = factory<A>(2, 1.414); // error: 2 will not bind to \tcode{int\&}
  int i = 2;
  shared_ptr<A> sp2 = factory<A>(i, 1.414); // OK
}
\end{codeblock}

\pnum
In the first call to \tcode{factory},
\tcode{A1} is deduced as \tcode{int}, so 2 is forwarded
to \tcode{A}'s constructor as an rvalue.
In the second call to \tcode{factory},
\tcode{A1} is deduced as \tcode{int\&}, so \tcode{i} is forwarded
to \tcode{A}'s constructor as an lvalue. In
both cases, \tcode{A2} is deduced as \tcode{double}, so
1.414 is forwarded to \tcode{A}'s constructor as an rvalue.

\end{example}
\end{itemdescr}

\indexlibrary{\idxcode{move}!function}%
\begin{itemdecl}
template <class T> constexpr remove_reference_t<T>&& move(T&& t) noexcept;
\end{itemdecl}

\begin{itemdescr}
\pnum
\returns
\tcode{static_cast<remove_reference_t<T>\&\&>(t)}.

\pnum
\begin{example}
\begin{codeblock}
template <class T, class A1>
shared_ptr<T> factory(A1&& a1) {
  return shared_ptr<T>(new T(std::forward<A1>(a1)));
}

struct A {
  A();
  A(const A&);  // copies from lvalues
  A(A&&);       // moves from rvalues
};

void g() {
  A a;
  shared_ptr<A> sp1 = factory<A>(a);              // ``\tcode{a}'' binds to \tcode{A(const A\&)}
  shared_ptr<A> sp1 = factory<A>(std::move(a));   // ``\tcode{a}'' binds to \tcode{A(A\&\&)}
}
\end{codeblock}

\pnum
In the first call to \tcode{factory},
\tcode{A1} is deduced as \tcode{A\&}, so \tcode{a} is forwarded
as a non-const lvalue. This binds to the constructor \tcode{A(const A\&)},
which copies the value from \tcode{a}.
In the second call to \tcode{factory}, because of the call
\tcode{std::move(a)},
\tcode{A1} is deduced as \tcode{A}, so \tcode{a} is forwarded
as an rvalue. This binds to the constructor \tcode{A(A\&\&)},
which moves the value from \tcode{a}.

\end{example}
\end{itemdescr}

\indexlibrary{\idxcode{move_if_noexcept}}%
\begin{itemdecl}
template <class T> constexpr conditional_t<
  !is_nothrow_move_constructible_v<T> && is_copy_constructible_v<T>,
  const T&, T&&> move_if_noexcept(T& x) noexcept;
\end{itemdecl}

\begin{itemdescr}
\pnum
\returns \tcode{std::move(x)}.
\end{itemdescr}

\rSec2[utility.as_const]{Function template \tcode{as_const}}

\indexlibrary{\idxcode{as_const}}%
\begin{itemdecl}
template <class T> constexpr add_const_t<T>& as_const(T& t) noexcept;
\end{itemdecl}

\begin{itemdescr}
\pnum
\returns \tcode{t}.
\end{itemdescr}

\rSec2[declval]{Function template \tcode{declval}}

\pnum
The library provides the function template \tcode{declval} to simplify the definition of
expressions which occur as unevaluated operands (Clause~\ref{expr}).

\indexlibrary{\idxcode{declval}}%
\begin{itemdecl}
template <class T>
  add_rvalue_reference_t<T> declval() noexcept;  // as unevaluated operand
\end{itemdecl}

\begin{itemdescr}
\pnum
\remarks If this function is odr-used~(\ref{basic.def.odr}), the program is ill-formed.

\pnum
\remarks The template parameter \tcode{T} of \tcode{declval} may be an incomplete type.
\end{itemdescr}

\pnum
\begin{example}
\begin{codeblock}
template <class To, class From>
  decltype(static_cast<To>(declval<From>())) convert(From&&);
\end{codeblock}
declares a function template \tcode{convert} which only participates in overloading if the
type \tcode{From} can be explicitly converted to type \tcode{To}. For another example see class
template \tcode{common_type}~(\ref{meta.trans.other}).
\end{example}

\rSec2[utility.inplace]{In-place construction}

\pnum
The \tcode{in_place_t}, \tcode{in_place_type_t}, and \tcode{in_place_index_t}
function types are used as unique types to disambiguate constructor and function
overloading. Specifically, \tcode{optional} has a constructor with \tcode{in_place_t}
as the first parameter followed by a parameter pack; this indicates that \tcode{T}
should be constructed in-place (as if by a call to a placement \grammarterm{new-expression})
with the forwarded pack expansion as arguments for the initialization of \tcode{T}.

\pnum
\remarks
Calling \tcode{in_place} functions results in undefined behavior.
\begin{note} These functions might be instantiated into every object file.  \end{note}

\rSec1[intseq]{Compile-time integer sequences}

\rSec2[intseq.general]{In general}

\pnum
The library provides a class template that can represent an integer sequence.
When used as an argument to a function template the parameter pack defining the
sequence can be deduced and used in a pack expansion.

\pnum
\begin{example}

\begin{codeblock}
template<class F, class Tuple, std::size_t... I>
  decltype(auto) apply_impl(F&& f, Tuple&& t, index_sequence<I...>) {
    return std::forward<F>(f)(std::get<I>(std::forward<Tuple>(t))...);
  }

template<class F, class Tuple>
  decltype(auto) apply(F&& f, Tuple&& t) {
    using Indices = make_index_sequence<std::tuple_size_v<std::decay_t<Tuple>>>;
    return apply_impl(std::forward<F>(f), std::forward<Tuple>(t), Indices());
  }
\end{codeblock}

\end{example}
\begin{note}
The \tcode{index_sequence} alias template is provided for the common case of
an integer sequence of type \tcode{size_t}.
\end{note}

\rSec2[intseq.intseq]{Class template \tcode{integer_sequence}}

\indexlibrary{\idxcode{integer_sequence}}%
\begin{codeblock}
namespace std {
  template<class T, T... I>
  struct integer_sequence {
    using value_type = T;
    static constexpr size_t size() noexcept { return sizeof...(I); }
  };
}
\end{codeblock}

\pnum
\tcode{T} shall be an integer type.

\rSec2[intseq.make]{Alias template \tcode{make_integer_sequence}}

\indexlibrary{\idxcode{make_integer_sequence}}%
\begin{itemdecl}
template<class T, T N>
  using make_integer_sequence = integer_sequence<T, @\seebelow{}@>;
\end{itemdecl}

\begin{itemdescr}
\pnum
If \tcode{N} is negative the program is ill-formed. The alias template
\tcode{make_integer_sequence} denotes a specialization of
\tcode{integer_sequence} with \tcode{N} template non-type arguments.
The type \tcode{make_integer_sequence<T, N>} denotes the type
\tcode{integer_sequence<T, 0, 1, ..., N-1>}.
\begin{note} \tcode{make_integer_sequence<int, 0>} denotes the type
\tcode{integer_sequence<int>} \end{note}
\end{itemdescr}

\rSec1[pairs]{Pairs}

\rSec2[pairs.general]{In general}

\pnum
The library provides a template for heterogeneous pairs of values.
The library also provides a matching function template to simplify
their construction and several templates that provide access to \tcode{pair}
objects as if they were \tcode{tuple} objects (see~\ref{tuple.helper}
and~\ref{tuple.elem}).%
\indexlibrary{\idxcode{pair}}%
\indextext{\idxcode{pair}!tuple interface to}%
\indextext{\idxcode{tuple}!and pair@and \tcode{pair}}%

\rSec2[pairs.pair]{Class template \tcode{pair}}

\indexlibrary{\idxcode{pair}}%
\begin{codeblock}
// defined in header \tcode{<utility>}

namespace std {
  template <class T1, class T2>
  struct pair {
    using first_type  = T1;
    using second_type = T2;

    T1 first;
    T2 second;
    pair(const pair&) = default;
    pair(pair&&) = default;
    constexpr pair();
    @\EXPLICIT@ constexpr pair(const T1& x, const T2& y);
    template<class U, class V> @\EXPLICIT@ constexpr pair(U&& x, V&& y);
    template<class U, class V> @\EXPLICIT@ constexpr pair(const pair<U, V>& p);
    template<class U, class V> @\EXPLICIT@ constexpr pair(pair<U, V>&& p);
    template <class... Args1, class... Args2>
      pair(piecewise_construct_t,
           tuple<Args1...> first_args, tuple<Args2...> second_args);

    pair& operator=(const pair& p);
    template<class U, class V> pair& operator=(const pair<U, V>& p);
    pair& operator=(pair&& p) noexcept(@\seebelow@);
    template<class U, class V> pair& operator=(pair<U, V>&& p);

    void swap(pair& p) noexcept(@\seebelow@);
  };
}
\end{codeblock}

\pnum
Constructors and member functions of \tcode{pair} shall not throw exceptions unless one of
the element-wise operations specified to be called for that operation
throws an exception.

\pnum
The defaulted move and copy constructor, respectively, of pair shall
be a \tcode{constexpr} function if and only if all required element-wise
initializations for copy and move, respectively, would satisfy the
requirements for a \tcode{constexpr} function.

\indexlibrary{\idxcode{pair}!constructor}%
\begin{itemdecl}
constexpr pair();
\end{itemdecl}

\begin{itemdescr}
\pnum
\effects
Value-initializes \tcode{first} and \tcode{second}.

\pnum
\remarks
This constructor shall not participate in overload resolution unless
\tcode{is_default_construct\-ible_v<first_type>} is \tcode{true} and
\tcode{is_default_constructible_v<second_type>} is \tcode{true}.
\begin{note} This behaviour can be implemented by a constructor template
with default template arguments. \end{note}
\end{itemdescr}

\indexlibrary{\idxcode{pair}!constructor}%
\begin{itemdecl}
@\EXPLICIT@ constexpr pair(const T1& x, const T2& y);
\end{itemdecl}

\begin{itemdescr}
\pnum
\effects
The constructor initializes \tcode{first} with \tcode{x} and \tcode{second}
with \tcode{y}.

\pnum
\remarks This constructor shall not participate in overload resolution
unless \tcode{is_copy_construct\-ible_v<first_type>} is \tcode{true} and
\tcode{is_copy_constructible_v<second_type>} is \tcode{true}.
The constructor is explicit if and only if
\tcode{is_convertible_v<const first_type\&, first_type>} is \tcode{false} or
\tcode{is_convertible_v<const second_type\&, second_type>} is \tcode{false}.
\end{itemdescr}

\indexlibrary{\idxcode{pair}!constructor}%
\begin{itemdecl}
template<class U, class V> @\EXPLICIT@ constexpr pair(U&& x, V&& y);
\end{itemdecl}

\begin{itemdescr}
\pnum
\effects
The constructor initializes \tcode{first} with
\tcode{std::forward<U>(x)} and \tcode{second}
with \tcode{std::forward<\brk{}V>(y)}.

\pnum
\remarks
This constructor shall not participate in overload resolution unless
\tcode{is_constructible_v<first_type, U\&\&>} is \tcode{true} and
\tcode{is_constructible_v<second_type, V\&\&>} is \tcode{true}.
The constructor is explicit if and only if
\tcode{is_convertible_v<U\&\&, first_type>} is \tcode{false} or
\tcode{is_convertible_v<V\&\&, second_type>} is \tcode{false}.
\end{itemdescr}

\indexlibrary{\idxcode{pair}!constructor}%
\begin{itemdecl}
template<class U, class V> @\EXPLICIT@ constexpr pair(const pair<U, V>& p);
\end{itemdecl}

\begin{itemdescr}
\pnum
\effects
The constructor initializes members from the corresponding members of the argument.

\pnum
\remarks This constructor shall not participate in overload resolution unless
\tcode{is_constructible_v<first_type, const U\&>} is \tcode{true} and
\tcode{is_constructible_v<second_type, const V\&>} is \tcode{true}.
The constructor is explicit if and only if
\tcode{is_convertible_v<const U\&, first_type>} is \tcode{false} or
\tcode{is_convertible_v<const V\&, second_type>} is \tcode{false}.
\end{itemdescr}

\indexlibrary{\idxcode{pair}!constructor}%
\begin{itemdecl}
template<class U, class V> @\EXPLICIT@ constexpr pair(pair<U, V>&& p);
\end{itemdecl}

\begin{itemdescr}
\pnum
\effects
The constructor initializes \tcode{first} with
\tcode{std::forward<U>(p.first)}
and \tcode{second} with
\tcode{std::\brk{}forward<V>(p.second)}.

\pnum
\remarks This constructor shall not participate in overload resolution unless
\tcode{is_constructible_v<first_type, U\&\&>} is \tcode{true} and
\tcode{is_constructible_v<second_type, V\&\&>} is \tcode{true}.
The constructor is explicit if and only if
\tcode{is_convertible_v<U\&\&, first_type>} is \tcode{false} or
\tcode{is_convertible_v<V\&\&, second_type>} is \tcode{false}.
\end{itemdescr}

\indexlibrary{\idxcode{pair}!constructor}%
\begin{itemdecl}
template<class... Args1, class... Args2>
  pair(piecewise_construct_t,
       tuple<Args1...> first_args, tuple<Args2...> second_args);
\end{itemdecl}

\begin{itemdescr}
\pnum
\requires \tcode{is_constructible_v<first_type, Args1\&\&...>} is \tcode{true}
and \tcode{is_constructible_v<sec\-ond_type, Args2\&\&...>} is \tcode{true}.

\pnum
\effects The constructor initializes \tcode{first} with arguments of types
\tcode{Args1...} obtained by forwarding the elements of \tcode{first_args}
and initializes \tcode{second} with arguments of types \tcode{Args2...}
obtained by forwarding the elements of \tcode{second_args}. (Here, forwarding
an element \tcode{x} of type \tcode{U} within a \tcode{tuple} object means calling
\tcode{std::forward<U>(x)}.) This form of construction, whereby constructor
arguments for \tcode{first} and \tcode{second} are each provided in a separate
\tcode{tuple} object, is called \defn{piecewise construction}.
\end{itemdescr}

\indexlibrarymember{operator=}{pair}%
\begin{itemdecl}
pair& operator=(const pair& p);
\end{itemdecl}

\begin{itemdescr}
\pnum
\requires \tcode{is_copy_assignable_v<first_type>} is \tcode{true}
and \tcode{is_copy_assignable_v<second_type>} is \tcode{true}.

\pnum
\effects Assigns \tcode{p.first} to \tcode{first} and \tcode{p.second} to \tcode{second}.

\pnum
\returns \tcode{*this}.
\end{itemdescr}

\indexlibrarymember{operator=}{pair}%
\begin{itemdecl}
template<class U, class V> pair& operator=(const pair<U, V>& p);
\end{itemdecl}

\begin{itemdescr}
\pnum
\requires \tcode{is_assignable_v<first_type\&, const U\&>} is \tcode{true}
and \tcode{is_assignable_v<second_type\&, const V\&>} is \tcode{true}.

\pnum
\effects Assigns \tcode{p.first} to \tcode{first} and \tcode{p.second} to \tcode{second}.

\pnum
\returns \tcode{*this}.
\end{itemdescr}

\indexlibrarymember{operator=}{pair}%
\begin{itemdecl}
pair& operator=(pair&& p) noexcept(@\seebelow@);
\end{itemdecl}

\begin{itemdescr}
\pnum
\remarks The expression inside \tcode{noexcept} is equivalent to:

\begin{codeblock}
is_nothrow_move_assignable_v<T1> && is_nothrow_move_assignable_v<T2>
\end{codeblock}

\pnum
\requires \tcode{is_move_assignable_v<first_type>} is \tcode{true}
and \tcode{is_move_assignable_v<second_type>} is \tcode{true}.

\pnum
\effects
Assigns to \tcode{first} with \tcode{std::forward<first_type>(p.first)}
and to \tcode{second} with\\ \tcode{std::forward<second_type>(p.second)}.

\pnum
\returns \tcode{*this}.
\end{itemdescr}

\indexlibrarymember{operator=}{pair}%
\begin{itemdecl}
template<class U, class V> pair& operator=(pair<U, V>&& p);
\end{itemdecl}

\begin{itemdescr}
\pnum
\requires \tcode{is_assignable_v<first_type\&, U\&\&>} is \tcode{true}
and\\ \tcode{is_assignable_v<second_type\&, V\&\&>} is \tcode{true}.

\pnum
\effects
Assigns to \tcode{first} with \tcode{std::forward<U>(p.first)}
and to \tcode{second} with\\ \tcode{std::forward<V>(p.second)}.

\pnum
\returns \tcode{*this}.
\end{itemdescr}

\indexlibrarymember{swap}{pair}%
\begin{itemdecl}
void swap(pair& p) noexcept(@\seebelow@);
\end{itemdecl}

\begin{itemdescr}
\pnum
\remarks The expression inside \tcode{noexcept} is equivalent to:

\begin{codeblock}
is_nothrow_swappable_v<first_type> &&
is_nothrow_swappable_v<second_type>
\end{codeblock}

\pnum
\requires
\tcode{first} shall be swappable with~(\ref{swappable.requirements})
\tcode{p.first} and \tcode{second} shall be swappable with \tcode{p.second}.

\pnum
\effects Swaps
\tcode{first} with \tcode{p.first} and
\tcode{second} with \tcode{p.second}.
\end{itemdescr}

\rSec2[pairs.spec]{Specialized algorithms}

\indexlibrarymember{operator==}{pair}%
\begin{itemdecl}
template <class T1, class T2>
  constexpr bool operator==(const pair<T1, T2>& x, const pair<T1, T2>& y);
\end{itemdecl}

\begin{itemdescr}
\pnum
\returns
\tcode{x.first == y.first \&\& x.second == y.second}.
\end{itemdescr}

\indexlibrarymember{operator<}{pair}%
\begin{itemdecl}
template <class T1, class T2>
  constexpr bool operator<(const pair<T1, T2>& x, const pair<T1, T2>& y);
\end{itemdecl}

\begin{itemdescr}
\pnum
\returns
\tcode{x.first < y.first || (!(y.first < x.first) \&\& x.second < y.second)}.
\end{itemdescr}

\indexlibrarymember{operator"!=}{pair}%
\begin{itemdecl}
template <class T1, class T2>
  constexpr bool operator!=(const pair<T1, T2>& x, const pair<T1, T2>& y);
\end{itemdecl}

\begin{itemdescr}
\pnum
\returns \tcode{!(x == y)}.
\end{itemdescr}

\indexlibrarymember{operator>}{pair}%
\begin{itemdecl}
template <class T1, class T2>
  constexpr bool operator>(const pair<T1, T2>& x, const pair<T1, T2>& y);
\end{itemdecl}

\begin{itemdescr}
\pnum
\returns \tcode{y < x}.
\end{itemdescr}

\indexlibrarymember{operator>=}{pair}%
\begin{itemdecl}
template <class T1, class T2>
  constexpr bool operator>=(const pair<T1, T2>& x, const pair<T1, T2>& y);
\end{itemdecl}

\begin{itemdescr}
\pnum
\returns \tcode{!(x < y)}.
\end{itemdescr}

\indexlibrarymember{operator<=}{pair}%
\begin{itemdecl}
template <class T1, class T2>
  constexpr bool operator<=(const pair<T1, T2>& x, const pair<T1, T2>& y);
\end{itemdecl}

\begin{itemdescr}
\pnum
\returns \tcode{!(y < x)}.
\end{itemdescr}


\indexlibrary{\idxcode{swap}!\idxcode{pair}}%
\begin{itemdecl}
template<class T1, class T2> void swap(pair<T1, T2>& x, pair<T1, T2>& y)
  noexcept(noexcept(x.swap(y)));
\end{itemdecl}

\begin{itemdescr}
\pnum
\effects As if by \tcode{x.swap(y)}.

\pnum
\remarks
This function shall not participate in overload resolution unless
\tcode{is_swappable_v<T1>} is \tcode{true} and
\tcode{is_swappable_v<T2>} is \tcode{true}.
\end{itemdescr}

\indexlibrary{\idxcode{make_pair}}%
\begin{itemdecl}
template <class T1, class T2>
  constexpr pair<V1, V2> make_pair(T1&& x, T2&& y);
\end{itemdecl}

\begin{itemdescr}
\pnum
\returns \tcode{pair<V1, V2>(std::forward<T1>(x), std::forward<T2>(y))},
where \tcode{V1} and \tcode{V2} are determined as follows: Let \tcode{Ui} be
\tcode{decay_t<Ti>} for each \tcode{Ti}. Then each \tcode{Vi} is \tcode{X\&}
if \tcode{Ui} equals \tcode{reference_wrapper<X>}, otherwise \tcode{Vi} is
\tcode{Ui}.
\end{itemdescr}

\pnum
\begin{example}
In place of:

\begin{codeblock}
  return pair<int, double>(5, 3.1415926);   // explicit types
\end{codeblock}

a \Cpp program may contain:

\begin{codeblock}
  return make_pair(5, 3.1415926);           // types are deduced
\end{codeblock}
\end{example}

\rSec2[pair.astuple]{Tuple-like access to pair}

\indexlibrary{\idxcode{tuple_size}}%
\begin{itemdecl}
template <class T1, class T2>
struct tuple_size<pair<T1, T2>>
  : integral_constant<size_t, 2> { };
\end{itemdecl}

\indexlibrary{\idxcode{tuple_element}}%
\begin{itemdecl}
tuple_element<0, pair<T1, T2>>::type
\end{itemdecl}
\begin{itemdescr}
\pnum\textit{Value:} the type \tcode{T1}.
\end{itemdescr}

\indexlibrary{\idxcode{tuple_element}}%
\begin{itemdecl}
tuple_element<1, pair<T1, T2>>::type
\end{itemdecl}
\begin{itemdescr}
\pnum\textit{Value:} the type T2.
\end{itemdescr}

\indexlibrarymember{get}{pair}%
\begin{itemdecl}
template<size_t I, class T1, class T2>
  constexpr tuple_element_t<I, pair<T1, T2>>&
    get(pair<T1, T2>& p) noexcept;
template<size_t I, class T1, class T2>
  constexpr const tuple_element_t<I, pair<T1, T2>>&
    get(const pair<T1, T2>& p) noexcept;
template<size_t I, class T1, class T2>
  constexpr tuple_element_t<I, pair<T1, T2>>&&
    get(pair<T1, T2>&& p) noexcept;
template<size_t I, class T1, class T2>
  constexpr const tuple_element_t<I, pair<T1, T2>>&&
    get(const pair<T1, T2>&& p) noexcept;
\end{itemdecl}
\begin{itemdescr}

\pnum
\returns If \tcode{I == 0} returns a reference to \tcode{p.first};
if \tcode{I == 1} returns a reference to \tcode{p.second};
otherwise the program is ill-formed.
\end{itemdescr}

\indexlibrarymember{get}{pair}%
\begin{itemdecl}
template <class T, class U>
  constexpr T& get(pair<T, U>& p) noexcept;
template <class T, class U>
  constexpr const T& get(const pair<T, U>& p) noexcept;
template <class T, class U>
  constexpr T&& get(pair<T, U>&& p) noexcept;
template <class T, class U>
  constexpr const T&& get(const pair<T, U>&& p) noexcept;
\end{itemdecl}
\begin{itemdescr}
\pnum
\requires \tcode{T} and \tcode{U} are distinct types. Otherwise, the program is ill-formed.

\pnum
\returns A reference to \tcode{p.first}.
\end{itemdescr}

\indexlibrarymember{get}{pair}%
\begin{itemdecl}
template <class T, class U>
  constexpr T& get(pair<U, T>& p) noexcept;
template <class T, class U>
  constexpr const T& get(const pair<U, T>& p) noexcept;
template <class T, class U>
  constexpr T&& get(pair<U, T>&& p) noexcept;
template <class T, class U>
  constexpr const T&& get(const pair<U, T>&& p) noexcept;
\end{itemdecl}
\begin{itemdescr}

\pnum
\requires \tcode{T} and \tcode{U} are distinct types. Otherwise, the program is ill-formed.

\pnum
\returns A reference to \tcode{p.second}.
\end{itemdescr}

\rSec2[pair.piecewise]{Piecewise construction}

\indexlibrary{\idxcode{piecewise_construct_t}}%
\indexlibrary{\idxcode{piecewise_construct}}%
\begin{itemdecl}
struct piecewise_construct_t { };
constexpr piecewise_construct_t piecewise_construct{};
\end{itemdecl}

\pnum
The \tcode{struct} \tcode{piecewise_construct_t} is an empty structure type
used as a unique type to disambiguate constructor and function overloading. Specifically,
\tcode{pair} has a constructor with \tcode{piecewise_construct_t} as the
first argument, immediately followed by two \tcode{tuple}~(\ref{tuple}) arguments used
for piecewise construction of the elements of the \tcode{pair} object.

\rSec1[tuple]{Tuples}

\rSec2[tuple.general]{In general}

\pnum
\indexlibrary{\idxcode{tuple}}%
This subclause describes the tuple library that provides a tuple type as
the class template \tcode{tuple} that can be instantiated with any number
of arguments. Each template argument specifies
the type of an element in the \tcode{tuple}.  Consequently, tuples are
heterogeneous, fixed-size collections of values. An instantiation of \tcode{tuple} with
two arguments is similar to an instantiation of \tcode{pair} with the same two arguments.
See~\ref{pairs}.

\pnum
\synopsis{Header \tcode{<tuple>} synopsis}

\indexlibrary{\idxhdr{tuple}}%
\begin{codeblock}
namespace std {
  // \ref{tuple.tuple}, class template \tcode{tuple}:
  template <class... Types> class tuple;

  // \ref{tuple.creation}, tuple creation functions:
  const @\unspec@ ignore;

  template <class... Types>
    constexpr tuple<@\placeholder{VTypes}@...> make_tuple(Types&&...);
  template <class... Types>
    constexpr tuple<Types&&...> forward_as_tuple(Types&&...) noexcept;

  template<class... Types>
    constexpr tuple<Types&...> tie(Types&...) noexcept;

  template <class... Tuples>
    constexpr tuple<@\placeholder{Ctypes}@...> tuple_cat(Tuples&&...);

  // \ref{tuple.apply}, calling a function with a tuple of arguments:
  template <class F, class Tuple>
    constexpr decltype(auto) apply(F&& f, Tuple&& t);

  template <class T, class Tuple>
    constexpr T make_from_tuple(Tuple&& t);

  // \ref{tuple.helper}, tuple helper classes:
  template <class T> class tuple_size;  // not defined
  template <class T> class tuple_size<const T>;
  template <class T> class tuple_size<volatile T>;
  template <class T> class tuple_size<const volatile T>;

  template <class... Types> class tuple_size<tuple<Types...>>;

  template <size_t I, class T> class tuple_element;    // not defined
  template <size_t I, class T> class tuple_element<I, const T>;
  template <size_t I, class T> class tuple_element<I, volatile T>;
  template <size_t I, class T> class tuple_element<I, const volatile T>;

  template <size_t I, class... Types> class tuple_element<I, tuple<Types...>>;

  template <size_t I, class T>
    using tuple_element_t = typename tuple_element<I, T>::type;

  // \ref{tuple.elem}, element access:
  template <size_t I, class... Types>
    constexpr tuple_element_t<I, tuple<Types...>>&
      get(tuple<Types...>&) noexcept;
  template <size_t I, class... Types>
    constexpr tuple_element_t<I, tuple<Types...>>&&
      get(tuple<Types...>&&) noexcept;
  template <size_t I, class... Types>
    constexpr const tuple_element_t<I, tuple<Types...>>&
      get(const tuple<Types...>&) noexcept;
  template <size_t I, class... Types>
    constexpr const tuple_element_t<I, tuple<Types...>>&&
      get(const tuple<Types...>&&) noexcept;
  template <class T, class... Types>
    constexpr T& get(tuple<Types...>& t) noexcept;
  template <class T, class... Types>
    constexpr T&& get(tuple<Types...>&& t) noexcept;
  template <class T, class... Types>
    constexpr const T& get(const tuple<Types...>& t) noexcept;
  template <class T, class... Types>
    constexpr const T&& get(const tuple<Types...>&& t) noexcept;

  // \ref{tuple.rel}, relational operators:
  template<class... TTypes, class... UTypes>
    constexpr bool operator==(const tuple<TTypes...>&, const tuple<UTypes...>&);
  template<class... TTypes, class... UTypes>
    constexpr bool operator<(const tuple<TTypes...>&, const tuple<UTypes...>&);
  template<class... TTypes, class... UTypes>
    constexpr bool operator!=(const tuple<TTypes...>&, const tuple<UTypes...>&);
  template<class... TTypes, class... UTypes>
    constexpr bool operator>(const tuple<TTypes...>&, const tuple<UTypes...>&);
  template<class... TTypes, class... UTypes>
    constexpr bool operator<=(const tuple<TTypes...>&, const tuple<UTypes...>&);
  template<class... TTypes, class... UTypes>
    constexpr bool operator>=(const tuple<TTypes...>&, const tuple<UTypes...>&);

  // \ref{tuple.traits}, allocator-related traits
  template <class... Types, class Alloc>
    struct uses_allocator<tuple<Types...>, Alloc>;

  // \ref{tuple.special}, specialized algorithms:
  template <class... Types>
    void swap(tuple<Types...>& x, tuple<Types...>& y) noexcept(@\seebelow@);

  // \ref{tuple.helper}, tuple helper classes
  template <class T> constexpr size_t tuple_size_v
    = tuple_size<T>::value;
}
\end{codeblock}

\rSec2[tuple.tuple]{Class template \tcode{tuple}}

\indexlibrary{\idxcode{tuple}}%
\begin{codeblock}
namespace std {
  template <class... Types>
  class tuple  {
  public:

    // \ref{tuple.cnstr}, \tcode{tuple} construction
    constexpr tuple();
    @\EXPLICIT@ constexpr tuple(const Types&...); // only if \tcode{sizeof...(Types) >= 1}
    template <class... UTypes>
      @\EXPLICIT@ constexpr tuple(UTypes&&...); // only if \tcode{sizeof...(Types) >= 1}

    tuple(const tuple&) = default;
    tuple(tuple&&) = default;

    template <class... UTypes>
      @\EXPLICIT@ constexpr tuple(const tuple<UTypes...>&);
    template <class... UTypes>
      @\EXPLICIT@ constexpr tuple(tuple<UTypes...>&&);

    template <class U1, class U2>
      @\EXPLICIT@ constexpr tuple(const pair<U1, U2>&);       // only if \tcode{sizeof...(Types) == 2}
    template <class U1, class U2>
      @\EXPLICIT@ constexpr tuple(pair<U1, U2>&&);            // only if \tcode{sizeof...(Types) == 2}

    // allocator-extended constructors
    template <class Alloc>
      tuple(allocator_arg_t, const Alloc& a);
    template <class Alloc>
      @\EXPLICIT@ tuple(allocator_arg_t, const Alloc& a, const Types&...);
    template <class Alloc, class... UTypes>
      @\EXPLICIT@ tuple(allocator_arg_t, const Alloc& a, UTypes&&...);
    template <class Alloc>
      tuple(allocator_arg_t, const Alloc& a, const tuple&);
    template <class Alloc>
      tuple(allocator_arg_t, const Alloc& a, tuple&&);
    template <class Alloc, class... UTypes>
      @\EXPLICIT@ tuple(allocator_arg_t, const Alloc& a, const tuple<UTypes...>&);
    template <class Alloc, class... UTypes>
      @\EXPLICIT@ tuple(allocator_arg_t, const Alloc& a, tuple<UTypes...>&&);
    template <class Alloc, class U1, class U2>
      @\EXPLICIT@ tuple(allocator_arg_t, const Alloc& a, const pair<U1, U2>&);
    template <class Alloc, class U1, class U2>
      @\EXPLICIT@ tuple(allocator_arg_t, const Alloc& a, pair<U1, U2>&&);

    // \ref{tuple.assign}, \tcode{tuple} assignment
    tuple& operator=(const tuple&);
    tuple& operator=(tuple&&) noexcept(@\seebelow@);

    template <class... UTypes>
      tuple& operator=(const tuple<UTypes...>&);
    template <class... UTypes>
      tuple& operator=(tuple<UTypes...>&&);

    template <class U1, class U2>
      tuple& operator=(const pair<U1, U2>&);    // only if \tcode{sizeof...(Types) == 2}
    template <class U1, class U2>
      tuple& operator=(pair<U1, U2>&&);         // only if \tcode{sizeof...(Types) == 2}

    // \ref{tuple.swap}, \tcode{tuple} swap
    void swap(tuple&) noexcept(@\seebelow@);
  };
}
\end{codeblock}

\rSec3[tuple.cnstr]{Construction}

\pnum
For each \tcode{tuple} constructor, an exception is thrown only if the construction of
one of the types in \tcode{Types} throws an exception.

\pnum
The defaulted move and copy constructor, respectively, of
\tcode{tuple} shall be a \tcode{constexpr} function if and only if all
required element-wise initializations for copy and move, respectively,
would satisfy the requirements for a \tcode{constexpr} function. The
defaulted move and copy constructor of \tcode{tuple<>} shall be
\tcode{constexpr} functions.

\pnum
In the constructor descriptions that follow, let $i$ be in the range
\range{0}{sizeof...(Types)} in order, $T_i$ be the $i^{th}$ type in \tcode{Types}, and
$U_i$ be the $i^{th}$ type in a template parameter pack named \tcode{UTypes}, where indexing
is zero-based.

\indexlibrary{\idxcode{tuple}!constructor}%
\begin{itemdecl}
constexpr tuple();
\end{itemdecl}

\begin{itemdescr}
\pnum
\effects Value-initializes each element.

\pnum
\remarks
This constructor shall not participate in overload resolution unless
\tcode{is_default_construct\-ible_v<$T_i$>} is \tcode{true} for all $i$.
\begin{note} This behaviour can be implemented by a constructor template
with default template arguments. \end{note}
\end{itemdescr}

\indexlibrary{\idxcode{tuple}!constructor}%
\begin{itemdecl}
@\EXPLICIT@ constexpr tuple(const Types&...);
\end{itemdecl}

\begin{itemdescr}
\pnum
\effects The constructor initializes each element with the value of the
corresponding parameter.

\pnum
\remarks This constructor shall not participate in overload resolution unless
\tcode{sizeof...(Types) >= 1} and \tcode{is_copy_constructible_v<$T_i$>}
is \tcode{true} for all $i$. The constructor is explicit if and only if
\tcode{is_convertible_v<const $T_i$\&, $T_i$>} is \tcode{false}
for at least one $i$.
\end{itemdescr}

\indexlibrary{\idxcode{tuple}!constructor}%
\begin{itemdecl}
template <class... UTypes>
  @\EXPLICIT@ constexpr tuple(UTypes&&... u);
\end{itemdecl}

\begin{itemdescr}
\pnum
\effects The constructor initializes the elements in the tuple with the
corresponding value in \tcode{std::for\-ward<UTypes>(u)}.

\pnum
\remarks This constructor shall not participate in overload resolution unless
\tcode{sizeof...(Types)} \tcode{==} \tcode{sizeof...(UTypes)} and
\tcode{sizeof...(Types) >= 1} and \tcode{is_constructible_v<$T_i$, $U_i$\&\&>}
is \tcode{true} for all $i$. The constructor is explicit if and only if
\tcode{is_convertible_v<$U_i$\&\&, $T_i$>} is \tcode{false}
for at least one $i$.
\end{itemdescr}

\indexlibrary{\idxcode{tuple}!constructor}%
\begin{itemdecl}
tuple(const tuple& u) = default;
\end{itemdecl}

\begin{itemdescr}
\pnum
\requires \tcode{is_copy_constructible_v<$T_i$>} is \tcode{true} for all $i$.

\pnum
\effects Initializes each element of \tcode{*this} with the
corresponding element of \tcode{u}.
\end{itemdescr}

\indexlibrary{\idxcode{tuple}!constructor}%
\begin{itemdecl}
tuple(tuple&& u) = default;
\end{itemdecl}

\begin{itemdescr}
\pnum
\requires \tcode{is_move_constructible_v<$T_i$>} is \tcode{true} for all $i$.

\pnum
\effects For all $i$, initializes the $i^{th}$ element of \tcode{*this} with
\tcode{std::forward<$T_i$>(get<$i$>(u))}.
\end{itemdescr}

\indexlibrary{\idxcode{tuple}!constructor}%
\begin{itemdecl}
template <class... UTypes> @\EXPLICIT@ constexpr tuple(const tuple<UTypes...>& u);
\end{itemdecl}

\begin{itemdescr}
\pnum
\effects The constructor initializes each element of \tcode{*this}
with the corresponding element of \tcode{u}.

\pnum
\remarks This constructor shall not participate in overload resolution unless
\begin{itemize}
\item
\tcode{sizeof...(Types)} \tcode{==} \tcode{sizeof...(UTypes)} and
\item
\tcode{is_constructible_v<$T_i$, const $U_i$\&>} is \tcode{true} for all $i$, and
\item
\tcode{sizeof...(Types) != 1}, or
(when \tcode{Types...} expands to \tcode{T} and \tcode{UTypes...} expands to \tcode{U})\linebreak
\tcode{!is_convertible_v<const tuple<U>\&, T> \&\& !is_constructible_v<T, const tuple<U>\&>\linebreak{}\&\& !is_same_v<T, U>}
is \tcode{true}.
\end{itemize}
The constructor is explicit if and only if
\tcode{is_convertible_v<const $U_i$\&, $T_i$>} is \tcode{false}
for at least one $i$.
\end{itemdescr}

\indexlibrary{\idxcode{tuple}!constructor}%
\begin{itemdecl}
template <class... UTypes> @\EXPLICIT@ constexpr tuple(tuple<UTypes...>&& u);
\end{itemdecl}

\begin{itemdescr}
\pnum
\effects For all $i$, the constructor
initializes the $i^{th}$ element of \tcode{*this} with
\tcode{std::forward<$U_i$>(get<$i$>(u))}.

\pnum
\remarks This constructor shall not participate in overload resolution unless

\begin{itemize}
\item
\tcode{sizeof...(Types)} \tcode{==} \tcode{sizeof...(UTypes)}, and
\item
\tcode{is_constructible_v<$T_i$, $U_i$\&\&>} is \tcode{true} for all $i$, and
\item
\tcode{sizeof...(Types) != 1}, or
(when \tcode{Types...} expands to \tcode{T} and \tcode{UTypes...} expands to \tcode{U})\linebreak
\tcode{!is_convertible_v<tuple<U>, T> \&\& !is_constructible_v<T, tuple<U>> \&\&\linebreak{}!is_same_v<T, U>}
is \tcode{true}.
\end{itemize}
The constructor is explicit if and only if
\tcode{is_convertible_v<$U_i$\&\&, $T_i$>} is \tcode{false}
for at least one $i$.
\end{itemdescr}

\indexlibrary{\idxcode{tuple}!constructor}%
\indexlibrary{\idxcode{pair}}%
\begin{itemdecl}
template <class U1, class U2> @\EXPLICIT@ constexpr tuple(const pair<U1, U2>& u);
\end{itemdecl}

\begin{itemdescr}
\pnum
\effects The constructor initializes the first element with \tcode{u.first} and the
second element with \tcode{u.second}.

\pnum
\remarks This constructor shall not participate in overload resolution unless
\tcode{sizeof...(Types) == 2},
\tcode{is_constructible_v<$T_0$, const U1\&>} is \tcode{true} and
\tcode{is_constructible_v<$T_1$, const U2\&>} is \tcode{true}.

\pnum
The constructor is explicit if and only if
\tcode{is_convertible_v<const U1\&, $T_0$>} is \tcode{false} or
\tcode{is_convertible_v<const U2\&, $T_1$>} is \tcode{false}.
\end{itemdescr}

\indexlibrary{\idxcode{tuple}!constructor}%
\indexlibrary{\idxcode{pair}}%
\begin{itemdecl}
template <class U1, class U2> @\EXPLICIT@ constexpr tuple(pair<U1, U2>&& u);
\end{itemdecl}

\begin{itemdescr}
\pnum
\effects The constructor initializes the first element with
\tcode{std::forward<U1>(u.first)} and the
second element with \tcode{std::forward<U2>(u.second)}.

\pnum
\remarks This constructor shall not participate in overload resolution unless
\tcode{sizeof...(Types) == 2},
\tcode{is_constructible_v<$T_0$, U1\&\&>} is \tcode{true} and
\tcode{is_constructible_v<$T_1$, U2\&\&>} is \tcode{true}.

\pnum
The constructor is explicit if and only if
\tcode{is_convertible_v<U1\&\&, $T_0$>} is \tcode{false} or
\tcode{is_convertible_v<U2\&\&, $T_1$>} is \tcode{false}.
\end{itemdescr}

\indexlibrary{\idxcode{tuple}!constructor}%
\begin{itemdecl}
template <class Alloc>
  tuple(allocator_arg_t, const Alloc& a);
template <class Alloc>
  @\EXPLICIT@ tuple(allocator_arg_t, const Alloc& a, const Types&...);
template <class Alloc, class... UTypes>
  @\EXPLICIT@ tuple(allocator_arg_t, const Alloc& a, UTypes&&...);
template <class Alloc>
  tuple(allocator_arg_t, const Alloc& a, const tuple&);
template <class Alloc>
  tuple(allocator_arg_t, const Alloc& a, tuple&&);
template <class Alloc, class... UTypes>
  @\EXPLICIT@ tuple(allocator_arg_t, const Alloc& a, const tuple<UTypes...>&);
template <class Alloc, class... UTypes>
  @\EXPLICIT@ tuple(allocator_arg_t, const Alloc& a, tuple<UTypes...>&&);
template <class Alloc, class U1, class U2>
  @\EXPLICIT@ tuple(allocator_arg_t, const Alloc& a, const pair<U1, U2>&);
template <class Alloc, class U1, class U2>
  @\EXPLICIT@ tuple(allocator_arg_t, const Alloc& a, pair<U1, U2>&&);
\end{itemdecl}

\begin{itemdescr}
\pnum
\requires \tcode{Alloc} shall meet the requirements for an
\tcode{Allocator}~(\ref{allocator.requirements}).

\pnum
\effects Equivalent to the preceding constructors except that each element is constructed with
uses-allocator construction~(\ref{allocator.uses.construction}).
\end{itemdescr}

\rSec3[tuple.assign]{Assignment}

\pnum
For each \tcode{tuple} assignment operator, an exception is thrown only if the
assignment of one of the types in \tcode{Types} throws an exception.
In the function descriptions that follow, let $i$ be in the range \range{0}{sizeof...\brk{}(Types)}
in order, $T_i$ be the $i^{th}$ type in \tcode{Types}, and $U_i$ be the $i^{th}$ type in a
template parameter pack named \tcode{UTypes}, where indexing is zero-based.

\indexlibrarymember{operator=}{tuple}%
\begin{itemdecl}
tuple& operator=(const tuple& u);
\end{itemdecl}

\begin{itemdescr}
\pnum
\requires \tcode{is_copy_assignable_v<$T_i$>} is \tcode{true} for all $i$.

\pnum
\effects Assigns each element of \tcode{u} to the corresponding
element of \tcode{*this}.

\pnum
\returns \tcode{*this}.
\end{itemdescr}

\indexlibrarymember{operator=}{tuple}%
\begin{itemdecl}
tuple& operator=(tuple&& u) noexcept(@\seebelow@);
\end{itemdecl}

\begin{itemdescr}
\pnum
\remarks The expression inside \tcode{noexcept} is equivalent to the logical \textsc{and} of the
following expressions:

\begin{codeblock}
is_nothrow_move_assignable_v<@$T_i$@>
\end{codeblock}

where $T_i$ is the $i^{th}$ type in \tcode{Types}.

\pnum
\requires \tcode{is_move_assignable_v<$T_i$>} is \tcode{true} for all $i$.

\pnum
\effects For all $i$, assigns \tcode{std::forward<$T_i$>(get<$i$>(u))} to
\tcode{get<$i$>(*this)}.

\pnum
\returns \tcode{*this}.
\end{itemdescr}

\indexlibrarymember{operator=}{tuple}%
\begin{itemdecl}
template <class... UTypes>
  tuple& operator=(const tuple<UTypes...>& u);
\end{itemdecl}

\begin{itemdescr}
\pnum
\requires
\tcode{sizeof...(Types) == sizeof...(UTypes)} and
\tcode{is_assignable_v<$T_i$\&, const $U_i$\&>} is \tcode{true} for all $i$.

\pnum
\effects  Assigns each element of \tcode{u} to the corresponding element
of \tcode{*this}.

\pnum
\returns \tcode{*this}.
\end{itemdescr}

\indexlibrarymember{operator=}{tuple}%
\begin{itemdecl}
template <class... UTypes>
  tuple& operator=(tuple<UTypes...>&& u);
\end{itemdecl}

\begin{itemdescr}
\pnum
\requires
\tcode{is_assignable_v<Ti\&, Ui\&\&> == true} for all \tcode{i}.
\tcode{sizeof...(Types)} \tcode{==}\\\tcode{sizeof...(UTypes)}.

\pnum
\effects For all $i$, assigns \tcode{std::forward<$U_i$>(get<$i$>(u))} to
\tcode{get<$i$>(*this)}.

\pnum
\returns \tcode{*this}.
\end{itemdescr}

\indexlibrarymember{operator=}{tuple}%
\indexlibrary{\idxcode{pair}}%
\begin{itemdecl}
template <class U1, class U2> tuple& operator=(const pair<U1, U2>& u);
\end{itemdecl}

\begin{itemdescr}
\pnum
\requires \tcode{sizeof...(Types) == 2}.
\tcode{is_assignable_v<$T_0$\&, const U1\&>} is \tcode{true} for the first type $T_0$ in
\tcode{Types} and \tcode{is_assignable_v<$T_1$\&, const U2\&>} is \tcode{true} for the
second type $T_1$ in \tcode{Types}.

\pnum
\effects  Assigns \tcode{u.first} to the first element of \tcode{*this}
and \tcode{u.second} to the second element of \tcode{*this}.

\pnum
\returns \tcode{*this}.
\end{itemdescr}

\indexlibrarymember{operator=}{tuple}%
\indexlibrary{\idxcode{pair}}%
\begin{itemdecl}
template <class U1, class U2> tuple& operator=(pair<U1, U2>&& u);
\end{itemdecl}

\begin{itemdescr}
\pnum
\requires \tcode{sizeof...(Types) == 2}.
\tcode{is_assignable_v<$T_0$\&, U1\&\&>} is \tcode{true} for the first type $T_0$ in
\tcode{Types} and \tcode{is_assignable_v<$T_1$\&, U2\&\&>} is \tcode{true} for the second
type $T_1$ in \tcode{Types}.

\pnum
\effects Assigns \tcode{std::forward<U1>(u.first)} to the first
element of \tcode{*this} and\\ \tcode{std::forward<U2>(u.second)} to the
second element of \tcode{*this}.

\pnum
\returns \tcode{*this}.
\end{itemdescr}

\rSec3[tuple.swap]{\tcode{swap}}

\indexlibrarymember{swap}{tuple}%
\begin{itemdecl}
void swap(tuple& rhs) noexcept(@\seebelow@);
\end{itemdecl}

\begin{itemdescr}
\pnum
\remarks The expression inside \tcode{noexcept} is equivalent to the logical
\textsc{and} of the following expressions:

\begin{codeblock}
is_nothrow_swappable_v<@$T_i$@>
\end{codeblock}

where $T_i$ is the $i^{th}$ type in \tcode{Types}.

\pnum
\requires
Each element in \tcode{*this} shall be swappable with~(\ref{swappable.requirements})
the corresponding element in \tcode{rhs}.

\pnum
\effects Calls \tcode{swap} for each element in \tcode{*this} and its
corresponding element in \tcode{rhs}.

\pnum
\throws Nothing unless one of the element-wise \tcode{swap} calls throws an exception.
\end{itemdescr}

\rSec3[tuple.creation]{Tuple creation functions}

\pnum
In the function descriptions that follow, let $i$ be in the range \range{0}{sizeof...(TTypes)}
in order and let $T_i$ be the $i^{th}$ type in a template parameter pack named \tcode{TTypes};
let $j$ be in the range \range{0}{sizeof...(UTypes)} in order and $U_j$ be the $j^{th}$ type
in a template parameter pack named \tcode{UTypes}, where indexing is zero-based.

\indexlibrary{\idxcode{make_tuple}}%
\indexlibrary{\idxcode{tuple}!\idxcode{make_tuple}}%
\begin{itemdecl}
template<class... Types>
  constexpr tuple<@\placeholder{VTypes}@...> make_tuple(Types&&... t);
\end{itemdecl}

\begin{itemdescr} \pnum Let \tcode{$U_i$} be \tcode{decay_t<$T_i$>} for each
$T_i$ in \tcode{Types}. Then each $V_i$ in \tcode{VTypes} is
\tcode{X\&} if $U_i$ equals \tcode{reference_wrapper<X>}, otherwise
$V_i$ is $U_i$.

\pnum
\returns \tcode{tuple<VTypes...>(std::forward<Types>(t)...)}.

\pnum
\begin{example}

\begin{codeblock}
int i; float j;
make_tuple(1, ref(i), cref(j))
\end{codeblock}

creates a tuple of type

\begin{codeblock}
tuple<int, int&, const float&>
\end{codeblock}

\end{example}

\end{itemdescr}

\indexlibrary{\idxcode{forward_as_tuple}}%
\indexlibrary{\idxcode{tuple}!\idxcode{forward_as_tuple}}%
\begin{itemdecl}
template<class... Types>
  constexpr tuple<Types&&...> forward_as_tuple(Types&&... t) noexcept;
\end{itemdecl}

\begin{itemdescr}
\pnum
\effects Constructs a tuple of references to the arguments in \tcode{t} suitable
for forwarding as arguments to a function. Because the result may contain references
to temporary variables, a program shall ensure that the return value of this
function does not outlive any of its arguments. (e.g., the program should typically
not store the result in a named variable).

\pnum
\returns \tcode{tuple<Types\&\&...>(std::forward<Types>(t)...)}.
\end{itemdescr}

\indexlibrary{\idxcode{tie}}%
\indexlibrary{\idxcode{ignore}}%
\indexlibrary{\idxcode{tuple}!\idxcode{tie}}%
\begin{itemdecl}
template<class... Types>
  constexpr tuple<Types&...> tie(Types&... t) noexcept;
\end{itemdecl}

\begin{itemdescr}
\pnum
\returns  \tcode{tuple<Types\&...>(t...)}.  When an
argument in \tcode{t} is \tcode{ignore}, assigning
any value to the corresponding tuple element has no effect.

\pnum
\begin{example}
\tcode{tie} functions allow one to create tuples that unpack
tuples into variables. \tcode{ignore} can be used for elements that
are not needed:
\begin{codeblock}
int i; std::string s;
tie(i, ignore, s) = make_tuple(42, 3.14, "C++");
// \tcode{i == 42}, \tcode{s == "C++"}
\end{codeblock}
\end{example}
\end{itemdescr}

\indexlibrary{\idxcode{tuple_cat}}
\begin{itemdecl}
template <class... Tuples>
  constexpr tuple<@\placeholder{CTypes}@...> tuple_cat(Tuples&&... tpls);
\end{itemdecl}

\begin{itemdescr}
\pnum
In the following paragraphs, let $T_i$ be the $i^{th}$ type in \tcode{Tuples},
$U_i$ be \tcode{remove_reference_t<Ti>}, and $tp_i$ be the $i^{th}$
parameter in the function parameter pack \tcode{tpls}, where all indexing is
zero-based.

\pnum
\requires For all $i$, $U_i$ shall be the type
$\cv_i$ \tcode{tuple<$Args_i...$>}, where $\cv_i$ is the (possibly empty) $i^{th}$
cv-qualifier-seq and $Args_i$ is the parameter pack representing the element
types in $U_i$. Let ${A_{ik}}$ be the ${k_i}^{th}$ type in $Args_i$. For all
$A_{ik}$ the following requirements shall be satisfied: If $T_i$ is
deduced as an lvalue reference type, then
\tcode{is_constructible_v<$A_{ik}$, $cv_i$ $A_{ik}$\&> == true}, otherwise
\tcode{is_constructible_v<$A_{ik}$, $cv_i A_{ik}$\&\&> == true}.

\pnum
\remarks The types in \tcode{\placeholder{Ctypes}} shall be equal to the ordered
sequence of the extended types
\tcode{$Args_0$..., $Args_1$...,} ... \tcode{$Args_{n-1}$...}, where $n$ is
equal to \tcode{sizeof...(Tuples)}. Let \tcode{$e_i$...} be the $i^{th}$
ordered sequence of tuple elements of the resulting \tcode{tuple} object
corresponding to the type sequence $Args_i$.

\pnum
\returns A \tcode{tuple} object constructed by initializing the ${k_i}^{th}$
type element $e_{ik}$ in \tcode{$e_i$...} with\\
\tcode{get<$k_i$>(std::forward<$T_i$>($tp_i$))} for each valid $k_i$ and
each group $e_i$ in order.

\pnum
\realnote An implementation may support additional types in the parameter
pack \tcode{Tuples} that support the \tcode{tuple}-like protocol, such as
\tcode{pair} and \tcode{array}.
\end{itemdescr}

\rSec3[tuple.apply]{Calling a function with a \tcode{tuple} of arguments}

\indexlibrary{\idxcode{apply}}%
\begin{itemdecl}
template <class F, class Tuple>
  constexpr decltype(auto) apply(F&& f, Tuple&& t);
\end{itemdecl}

\begin{itemdescr}
\pnum
\effects
Given the exposition only function:
\begin{codeblock}
template <class F, class Tuple, size_t... I>
constexpr decltype(auto) apply_impl(F&& f, Tuple&& t, index_sequence<I...>) { // exposition only
  return @\textit{INVOKE}@(std::forward<F>(f), std::get<I>(std::forward<Tuple>(t))...);
}
\end{codeblock}
Equivalent to:
\begin{codeblock}
return apply_impl(std::forward<F>(f), std::forward<Tuple>(t),
                  make_index_sequence<tuple_size_v<decay_t<Tuple>>>{});
\end{codeblock}
\end{itemdescr}

\indexlibrary{\idxcode{make_from_tuple}}%
\begin{itemdecl}
template <class T, class Tuple>
  constexpr T make_from_tuple(Tuple&& t);
\end{itemdecl}

\begin{itemdescr}
\pnum
\effects
Given the exposition-only function:
\begin{codeblock}
template <class T, class Tuple, size_t... I>
constexpr T make_from_tuple_impl(Tuple&& t, index_sequence<I...>) { // exposition only
  return T(get<I>(std::forward<Tuple>(t))...);
}
\end{codeblock}
Equivalent to:
\begin{codeblock}
return make_from_tuple_impl<T>(forward<Tuple>(t),
                               make_index_sequence<tuple_size_v<decay_t<Tuple>>>{});
\end{codeblock}
\begin{note} The type of \tcode{T} must be supplied
as an explicit template parameter,
as it cannot be deduced from the argument list. \end{note}
\end{itemdescr}

\rSec3[tuple.helper]{Tuple helper classes}

\indexlibrary{\idxcode{tuple_size}!in~general}%
\begin{itemdecl}
template <class T> struct tuple_size;
\end{itemdecl}

\begin{itemdescr}
\pnum
\remarks All specializations of \tcode{tuple_size<T>} shall meet the
\tcode{UnaryTypeTrait} requirements~(\ref{meta.rqmts}) with a
\tcode{BaseCharacteristic} of \tcode{integral_constant<size_t, N>}
for some \tcode{N}.
\end{itemdescr}

\indexlibrary{\idxcode{tuple_size}}%
\begin{itemdecl}
template <class... Types>
class tuple_size<tuple<Types...>>
  : public integral_constant<size_t, sizeof...(Types)> { };
\end{itemdecl}

\indexlibrary{\idxcode{tuple_element}}%
\begin{itemdecl}
template <size_t I, class... Types>
class tuple_element<I, tuple<Types...>> {
public:
  using type = TI;
};
\end{itemdecl}

\begin{itemdescr}
\pnum
\requires \tcode{I < sizeof...(Types)}.
The program is ill-formed if \tcode{I} is out of bounds.

\pnum
\ctype \tcode{TI} is the
type of the \tcode{I}th element of \tcode{Types},
where indexing is zero-based.
\end{itemdescr}

\indexlibrary{\idxcode{tuple_size}}%
\begin{itemdecl}
template <class T> class tuple_size<const T>;
template <class T> class tuple_size<volatile T>;
template <class T> class tuple_size<const volatile T>;
\end{itemdecl}

\begin{itemdescr}
\pnum
Let \tcode{\placeholder{TS}} denote \tcode{tuple_size<T>} of the \cv-unqualified type \tcode{T}. Then each
of the three templates shall meet the \tcode{UnaryTypeTrait} requirements~(\ref{meta.rqmts})
with a \tcode{BaseCharacteristic} of
\begin{codeblock}
integral_constant<size_t, @\placeholder{TS}@::value>
\end{codeblock}

\pnum
In addition to being available via inclusion of the \tcode{<tuple>} header,
the three templates are available when either of the headers \tcode{<array>} or
\tcode{<utility>} are included.
\end{itemdescr}

\indexlibrary{\idxcode{tuple_element}}%
\begin{itemdecl}
template <size_t I, class T> class tuple_element<I, const T>;
template <size_t I, class T> class tuple_element<I, volatile T>;
template <size_t I, class T> class tuple_element<I, const volatile T>;
\end{itemdecl}

\begin{itemdescr}
\pnum
Let \tcode{\placeholder{TE}} denote \tcode{tuple_element_t<I, T>} of the \cv-unqualified type \tcode{T}. Then
each of the three templates shall meet the \tcode{TransformationTrait}
requirements~(\ref{meta.rqmts}) with a member typedef \tcode{type} that names the following
type:

\begin{itemize}
\item
for the first specialization, \tcode{add_const_t<\placeholder{TE}>},
\item
for the second specialization, \tcode{add_volatile_t<\placeholder{TE}>}, and
\item
for the third specialization, \tcode{add_cv_t<\placeholder{TE}>}.
\end{itemize}

\pnum
In addition to being available via inclusion of the \tcode{<tuple>} header,
the three templates are available when either of the headers \tcode{<array>} or
\tcode{<utility>} are included.
\end{itemdescr}

\rSec3[tuple.elem]{Element access}

\indexlibrarymember{get}{tuple}%
\begin{itemdecl}
template <size_t I, class... Types>
  constexpr tuple_element_t<I, tuple<Types...>>&
    get(tuple<Types...>& t) noexcept;
template <size_t I, class... Types>
  constexpr tuple_element_t<I, tuple<Types...>>&&
    get(tuple<Types...>&& t) noexcept;        // Note A
template <size_t I, class... Types>
  constexpr tuple_element_t<I, tuple<Types...>> const&
    get(const tuple<Types...>& t) noexcept;   // Note B
template <size_t I, class... Types>
  constexpr const tuple_element_t<I, tuple<Types...>>&& get(const tuple<Types...>&& t) noexcept; 
\end{itemdecl}

\begin{itemdescr}
\pnum
\requires \tcode{I < sizeof...(Types)}.
The program is ill-formed if \tcode{I} is out of bounds.

\pnum
\returns  A reference to the \tcode{I}th element of \tcode{t}, where
indexing is zero-based.

\pnum
\begin{note}[Note A]
If a \tcode{T} in \tcode{Types} is some reference type \tcode{X\&},
the return type is \tcode{X\&}, not \tcode{X\&\&}.
However, if the element type is a non-reference type \tcode{T},
the return type is \tcode{T\&\&}.
\end{note}

\pnum
\begin{note}[Note B]
Constness is shallow. If a \tcode{T}
in \tcode{Types} is some
reference type \tcode{X\&}, the return type is \tcode{X\&}, not \tcode{const X\&}.
However, if the element type is a non-reference type \tcode{T},
the return type is \tcode{const T\&}.
This is consistent with how constness is defined to work
for member variables of reference type.
\end{note}
\end{itemdescr}

\indexlibrarymember{get}{tuple}%
\begin{itemdecl}
template <class T, class... Types>
  constexpr T& get(tuple<Types...>& t) noexcept;
template <class T, class... Types>
  constexpr T&& get(tuple<Types...>&& t) noexcept;
template <class T, class... Types>
  constexpr const T& get(const tuple<Types...>& t) noexcept;
template <class T, class... Types>
  constexpr const T&& get(const tuple<Types...>&& t) noexcept;
\end{itemdecl}

\begin{itemdescr}
\pnum
\requires The type \tcode{T} occurs exactly once in \tcode{Types...}.
Otherwise, the program is ill-formed.

\pnum
\returns A reference to the element of \tcode{t} corresponding to the type
\tcode{T} in \tcode{Types...}.

\pnum
\begin{example}
\begin{codeblock}
  const tuple<int, const int, double, double> t(1, 2, 3.4, 5.6);
  const int& i1 = get<int>(t);        // OK. Not ambiguous. i1 == 1
  const int& i2 = get<const int>(t);  // OK. Not ambiguous. i2 == 2
  const double& d = get<double>(t);   // ERROR. ill-formed
\end{codeblock}
\end{example}
\end{itemdescr}

\pnum
\begin{note} The reason \tcode{get} is a
nonmember function is that if this functionality had been
provided as a member function, code where the type
depended on a template parameter would have required using
the \tcode{template} keyword. \end{note}

\rSec3[tuple.rel]{Relational operators}

\indexlibrarymember{operator==}{tuple}%
\begin{itemdecl}
template<class... TTypes, class... UTypes>
  constexpr bool operator==(const tuple<TTypes...>& t, const tuple<UTypes...>& u);
\end{itemdecl}

\begin{itemdescr}
\pnum
\requires  For all \tcode{i},
where \tcode{0 <= i} and
\tcode{i < sizeof...(TTypes)}, \tcode{get<i>(t) == get<i>(u)} is a valid expression
returning a type that is convertible to \tcode{bool}.
\tcode{sizeof...(TTypes)} \tcode{==}
\tcode{sizeof...(UTypes)}.

\pnum
\returns  \tcode{true} if \tcode{get<i>(t) == get<i>(u)} for all
\tcode{i}, otherwise \tcode{false}.
For any two zero-length tuples \tcode{e} and \tcode{f}, \tcode{e == f} returns \tcode{true}.

\pnum
\effects  The elementary comparisons are performed in order from the
zeroth index upwards.  No comparisons or element accesses are
performed after the first equality comparison that evaluates to
\tcode{false}.
\end{itemdescr}

\indexlibrarymember{operator<}{tuple}%
\begin{itemdecl}
template<class... TTypes, class... UTypes>
  constexpr bool operator<(const tuple<TTypes...>& t, const tuple<UTypes...>& u);
\end{itemdecl}

\begin{itemdescr}
\pnum
\requires  For all \tcode{i},
where \tcode{0 <= i} and
\tcode{i < sizeof...(TTypes)}, both \tcode{get<i>(t) < get<i>(u)}
and \tcode{get<i>(u) < get<i>(t)}
are valid expressions returning types that are
convertible to \tcode{bool}.
\tcode{sizeof...(TTypes)} \tcode{==}
\tcode{sizeof...(UTypes)}.

\pnum\returns  The result of a lexicographical comparison
between \tcode{t} and \tcode{u}. The result is defined
as: \tcode{(bool)(get<0>(t) < get<0>(u)) ||
(!(bool)(get<0>(u) < get<0>(t)) \&\& t$_{\mathrm{tail}}$ <
u$_{\mathrm{tail}}$)}, where \tcode{r$_{\mathrm{tail}}$} for some
tuple \tcode{r} is a tuple containing all but the first element
of \tcode{r}.  For any two zero-length tuples \tcode{e}
and \tcode{f}, \tcode{e < f} returns \tcode{false}.
\end{itemdescr}

\indexlibrarymember{operator"!=}{tuple}%
\begin{itemdecl}
template<class... TTypes, class... UTypes>
  constexpr bool operator!=(const tuple<TTypes...>& t, const tuple<UTypes...>& u);
\end{itemdecl}
\begin{itemdescr}
\pnum\returns \tcode{!(t == u)}.
\end{itemdescr}

\indexlibrarymember{operator>}{tuple}%
\begin{itemdecl}
template<class... TTypes, class... UTypes>
  constexpr bool operator>(const tuple<TTypes...>& t, const tuple<UTypes...>& u);
\end{itemdecl}
\begin{itemdescr}
\pnum\returns \tcode{u < t}.
\end{itemdescr}

\indexlibrarymember{operator<=}{tuple}%
\begin{itemdecl}
template<class... TTypes, class... UTypes>
  constexpr bool operator<=(const tuple<TTypes...>& t, const tuple<UTypes...>& u);
\end{itemdecl}
\begin{itemdescr}
\pnum\returns \tcode{!(u < t)}.
\end{itemdescr}

\indexlibrarymember{operator>=}{tuple}%
\begin{itemdecl}
template<class... TTypes, class... UTypes>
  constexpr bool operator>=(const tuple<TTypes...>& t, const tuple<UTypes...>& u);
\end{itemdecl}

\begin{itemdescr}
\pnum\returns \tcode{!(t < u)}.
\end{itemdescr}

\pnum \begin{note} The above definitions for comparison operators
do not require \tcode{t$_{\mathrm{tail}}$}
(or \tcode{u$_{\mathrm{tail}}$}) to be constructed. It may not
even be possible, as \tcode{t} and \tcode{u} are not required to be copy
constructible. Also, all comparison operators are short circuited;
they do not perform element accesses beyond what is required to determine the
result of the comparison. \end{note}

\rSec3[tuple.traits]{Tuple traits}

\indexlibrary{\idxcode{uses_allocator<tuple>}}%
\begin{itemdecl}
template <class... Types, class Alloc>
  struct uses_allocator<tuple<Types...>, Alloc> : true_type { };
\end{itemdecl}

\begin{itemdescr}
\pnum
\requires \tcode{Alloc} shall be an \tcode{Allocator}~(\ref{allocator.requirements}).

\pnum
\begin{note} Specialization of this trait informs other library components that
\tcode{tuple} can be constructed with an allocator, even though it does not have
a nested \tcode{allocator_type}. \end{note}
\end{itemdescr}

\rSec3[tuple.special]{Tuple specialized algorithms}

\indexlibrary{\idxcode{swap}}%
\begin{itemdecl}
template <class... Types>
  void swap(tuple<Types...>& x, tuple<Types...>& y) noexcept(@\seebelow@);
\end{itemdecl}

\begin{itemdescr}
\pnum
\remarks This function shall not participate in overload resolution
unless \tcode{is_swappable_v<$T_i$>} is \tcode{true}
for all $i$, where $0 <= i$ and $i <$ \tcode{sizeof...(Types)}.
The expression inside \tcode{noexcept} is equivalent to:

\begin{codeblock}
noexcept(x.swap(y))
\end{codeblock}

\pnum
\effects As if by \tcode{x.swap(y)}.
\end{itemdescr}

\rSec1[optional]{Optional objects}

\rSec2[optional.general]{In general}

\pnum
This subclause describes class template \tcode{optional} that represents
optional objects.
An \defn{optional object} is an
object that contains the storage for another object and manages the lifetime of
this contained object, if any. The contained object may be initialized after
the optional object has been initialized, and may be destroyed before the
optional object has been destroyed. The initialization state of the contained
object is tracked by the optional object.

\rSec2[optional.syn]{Header \tcode{<optional>} synopsis}

\indexlibrary{\idxhdr{optional}}%
\begin{codeblock}
namespace std {
  // \ref{optional.optional}, optional
  template <class T> class optional;

  // \ref{optional.nullopt}, no-value state indicator
  struct nullopt_t{@\seebelow@};
  constexpr nullopt_t nullopt(@\unspec@);

  // \ref{optional.bad.access}, class \tcode{bad_optional_access}
  class bad_optional_access;

  // \ref{optional.relops}, relational operators
  template <class T>
  constexpr bool operator==(const optional<T>&, const optional<T>&);
  template <class T>
  constexpr bool operator!=(const optional<T>&, const optional<T>&);
  template <class T>
  constexpr bool operator<(const optional<T>&, const optional<T>&);
  template <class T>
  constexpr bool operator>(const optional<T>&, const optional<T>&);
  template <class T>
  constexpr bool operator<=(const optional<T>&, const optional<T>&);
  template <class T>
  constexpr bool operator>=(const optional<T>&, const optional<T>&);

  // \ref{optional.nullops}, comparison with \tcode{nullopt}
  template <class T> constexpr bool operator==(const optional<T>&, nullopt_t) noexcept;
  template <class T> constexpr bool operator==(nullopt_t, const optional<T>&) noexcept;
  template <class T> constexpr bool operator!=(const optional<T>&, nullopt_t) noexcept;
  template <class T> constexpr bool operator!=(nullopt_t, const optional<T>&) noexcept;
  template <class T> constexpr bool operator<(const optional<T>&, nullopt_t) noexcept;
  template <class T> constexpr bool operator<(nullopt_t, const optional<T>&) noexcept;
  template <class T> constexpr bool operator<=(const optional<T>&, nullopt_t) noexcept;
  template <class T> constexpr bool operator<=(nullopt_t, const optional<T>&) noexcept;
  template <class T> constexpr bool operator>(const optional<T>&, nullopt_t) noexcept;
  template <class T> constexpr bool operator>(nullopt_t, const optional<T>&) noexcept;
  template <class T> constexpr bool operator>=(const optional<T>&, nullopt_t) noexcept;
  template <class T> constexpr bool operator>=(nullopt_t, const optional<T>&) noexcept;

  // \ref{optional.comp_with_t}, comparison with \tcode{T}
  template <class T> constexpr bool operator==(const optional<T>&, const T&);
  template <class T> constexpr bool operator==(const T&, const optional<T>&);
  template <class T> constexpr bool operator!=(const optional<T>&, const T&);
  template <class T> constexpr bool operator!=(const T&, const optional<T>&);
  template <class T> constexpr bool operator<(const optional<T>&, const T&);
  template <class T> constexpr bool operator<(const T&, const optional<T>&);
  template <class T> constexpr bool operator<=(const optional<T>&, const T&);
  template <class T> constexpr bool operator<=(const T&, const optional<T>&);
  template <class T> constexpr bool operator>(const optional<T>&, const T&);
  template <class T> constexpr bool operator>(const T&, const optional<T>&);
  template <class T> constexpr bool operator>=(const optional<T>&, const T&);
  template <class T> constexpr bool operator>=(const T&, const optional<T>&);

  // \ref{optional.specalg}, specialized algorithms
  template <class T> void swap(optional<T>&, optional<T>&) noexcept(@\seebelow@);
  template <class T> constexpr optional<@\seebelow@> make_optional(T&&);
  template <class T, class... Args>
    constexpr optional<T> make_optional(Args&&... args);
  template <class T, class U, class... Args>
    constexpr optional<T> make_optional(initializer_list<U> il, Args&&... args);

  // \ref{optional.hash}, hash support
  template <class T> struct hash;
  template <class T> struct hash<optional<T>>;
}
\end{codeblock}

\pnum
A program that necessitates the instantiation of template \tcode{optional} for
a reference type, or for possibly cv-qualified types \tcode{in_place_t} or
\tcode{nullopt_t} is ill-formed.

\rSec2[optional.optional]{Class template \tcode{optional}}

\indexlibrary{\idxcode{optional}}%
\begin{codeblock}
template <class T> class optional {
public:
  using value_type = T;

  // \ref{optional.ctor}, constructors
  constexpr optional() noexcept;
  constexpr optional(nullopt_t) noexcept;
  optional(const optional &);
  optional(optional &&) noexcept(@\seebelow@);
  constexpr optional(const T &);
  constexpr optional(T &&);
  template <class... Args> constexpr explicit optional(in_place_t, Args &&...);
  template <class U, class... Args>
    constexpr explicit optional(in_place_t, initializer_list<U>, Args &&...);

  // \ref{optional.dtor}, destructor
  ~optional();

  // \ref{optional.assign}, assignment
  optional &operator=(nullopt_t) noexcept;
  optional &operator=(const optional &);
  optional &operator=(optional &&) noexcept(@\seebelow@);
  template <class U> optional &operator=(U &&);
  template <class... Args> void emplace(Args &&...);
  template <class U, class... Args>
    void emplace(initializer_list<U>, Args &&...);

  // \ref{optional.swap}, swap
  void swap(optional &) noexcept(@\seebelow@);

  // \ref{optional.observe}, observers
  constexpr T const *operator->() const;
  constexpr T *operator->();
  constexpr T const &operator*() const &;
  constexpr T &operator*() &;
  constexpr T &&operator*() &&;
  constexpr const T &&operator*() const &&;
  constexpr explicit operator bool() const noexcept;
  constexpr bool has_value() const noexcept;
  constexpr T const &value() const &;
  constexpr T &value() &;
  constexpr T &&value() &&;
  constexpr const T &&value() const &&;
  template <class U> constexpr T value_or(U &&) const &;
  template <class U> constexpr T value_or(U &&) &&;

  // \ref{optional.mod}, modifiers
  void reset() noexcept;

private:
  T *val; // \expos
};
\end{codeblock}

\pnum
Any instance of \tcode{optional<T>} at any given time either contains a value or does not contain a value.
When an instance of \tcode{optional<T>} \defnx{contains a value}{contains a value!\idxcode{optional}},
it means that an object of type \tcode{T}, referred to as the optional object's \defnx{contained value}{contained value!\idxcode{optional}},
is allocated within the storage of the optional object.
Implementations are not permitted to use additional storage, such as dynamic memory, to allocate its contained value.
The contained value shall be allocated in a region of the \tcode{optional<T>} storage suitably aligned for the type \tcode{T}.
When an object of type \tcode{optional<T>} is contextually converted to \tcode{bool},
the conversion returns \tcode{true} if the object contains a value;
otherwise the conversion returns \tcode{false}.

\pnum
Member \tcode{val} is provided for exposition only. When an \tcode{optional<T>} object contains a value, \tcode{val} points to the contained value.

\pnum
\tcode{T} shall be an object type and shall satisfy the requirements of \tcode{Destructible} (Table~\ref{tab:destructible}).

\rSec3[optional.ctor]{Constructors}

\indexlibrary{\idxcode{optional}!constructor}%
\begin{itemdecl}
constexpr optional() noexcept;
constexpr optional(nullopt_t) noexcept;
\end{itemdecl}

\begin{itemdescr}
\pnum
\postconditions
\tcode{*this} does not contain a value.

\pnum
\remarks
No contained value is initialized.
For every object type \tcode{T} these constructors shall be \tcode{constexpr} constructors (\ref{dcl.constexpr}).
\end{itemdescr}

\indexlibrary{\idxcode{optional}!constructor}%
\begin{itemdecl}
optional(const optional& rhs);
\end{itemdecl}

\begin{itemdescr}
\pnum
\requires
\tcode{is_copy_constructible_v<T>} is \tcode{true}.

\pnum
\effects
If \tcode{rhs} contains a value, initializes the contained value as if
direct-non-list-initializing an object of type \tcode{T} with the expression \tcode{*rhs}.

\pnum
\postconditions
\tcode{bool(rhs) == bool(*this)}.

\pnum
\throws
Any exception thrown by the selected constructor of \tcode{T}.
\end{itemdescr}

\indexlibrary{\idxcode{optional}!constructor}%
\begin{itemdecl}
optional(optional&& rhs) noexcept(@\seebelow@);
\end{itemdecl}

\begin{itemdescr}
\pnum
\requires
\tcode{is_move_constructible_v<T>} is \tcode{true}.

\pnum
\effects
If \tcode{rhs} contains a value, initializes the contained value as if
direct-non-list-initializing an object of type \tcode{T} with the expression \tcode{std::move(*rhs)}.
\tcode{bool(rhs)} is unchanged.

\pnum
\postconditions
\tcode{bool(rhs) == bool(*this)}.

\pnum
\throws
Any exception thrown by the selected constructor of \tcode{T}.

\pnum
\remarks
The expression inside \tcode{noexcept} is equivalent to
\tcode{is_nothrow_move_constructible_v<T>}.
\end{itemdescr}

\indexlibrary{\idxcode{optional}!constructor}%
\begin{itemdecl}
constexpr optional(const T& v);
\end{itemdecl}

\begin{itemdescr}
\pnum
\requires
\tcode{is_copy_constructible_v<T>} is \tcode{true}.

\pnum
\effects
Initializes the contained value as if direct-non-list-initializing an object of type \tcode{T} with the expression \tcode{v}.

\pnum
\postconditions
\tcode{*this} contains a value.

\pnum
\throws
Any exception thrown by the selected constructor of \tcode{T}.

\pnum
\remarks
If \tcode{T}'s selected constructor is a \tcode{constexpr} constructor, this constructor shall be a \tcode{constexpr} constructor.
\end{itemdescr}

\indexlibrary{\idxcode{optional}!constructor}%
\begin{itemdecl}
constexpr optional(T&& v);
\end{itemdecl}

\begin{itemdescr}
\pnum
\requires
\tcode{is_move_constructible_v<T>} is \tcode{true}.

\pnum
\effects
Initializes the contained value as if direct-non-list-initializing an object of type \tcode{T} with the expression \tcode{std::move(v)}.

\pnum
\postconditions
\tcode{*this} contains a value.

\pnum
\throws
Any exception thrown by the selected constructor of \tcode{T}.

\pnum
\remarks
If \tcode{T}'s selected constructor is a \tcode{constexpr} constructor, this constructor shall be a \tcode{constexpr} constructor.
\end{itemdescr}

\indexlibrary{\idxcode{optional}!constructor}%
\begin{itemdecl}
template <class... Args> constexpr explicit optional(in_place_t, Args&&... args);
\end{itemdecl}

\begin{itemdescr}
\pnum
\requires
\tcode{is_constructible_v<T, Args\&\&...>} is \tcode{true}.

\pnum
\effects
Initializes the contained value as if direct-non-list-initializing an object of type \tcode{T} with the arguments \tcode{std::forward<Args>(args)...}.

\pnum
\postconditions
\tcode{*this} contains a value.

\pnum
\throws
Any exception thrown by the selected constructor of \tcode{T}.

\pnum
\remarks
If \tcode{T}'s constructor selected for the initialization is a \tcode{constexpr} constructor, this constructor shall be a \tcode{constexpr} constructor.
\end{itemdescr}

\indexlibrary{\idxcode{optional}!constructor}%
\begin{itemdecl}
template <class U, class... Args>
  constexpr explicit optional(in_place_t, initializer_list<U> il, Args&&... args);
\end{itemdecl}

\begin{itemdescr}

\pnum
\requires
\tcode{is_constructible_v<T, initializer_list<U>\&, Args\&\&...>} is \tcode{true}.

\pnum
\effects
Initializes the contained value as if direct-non-list-initializing an object of type \tcode{T} with the arguments \tcode{il, std::forward<Args>(args)...}.

\pnum
\postconditions
\tcode{*this} contains a value.

\pnum
\throws
Any exception thrown by the selected constructor of \tcode{T}.

\pnum
\remarks
The function shall not participate in overload resolution unless \tcode{is_constructible_v<T, initializer_list<U>\&, Args\&\&...>} is \tcode{true}.
If \tcode{T}'s constructor selected for the initialization is a \tcode{constexpr} constructor, this constructor shall be a \tcode{constexpr} constructor.
\end{itemdescr}

\rSec3[optional.dtor]{Destructor}

\indexlibrary{\idxcode{optional}!destructor}%
\begin{itemdecl}
~optional();
\end{itemdecl}

\begin{itemdescr}
\pnum
\effects
If \tcode{is_trivially_destructible_v<T> != true} and \tcode{*this} contains a value, calls \tcode{val->T::\~T()}.

\pnum
\remarks
If \tcode{is_trivially_destructible_v<T> == true} then this destructor shall be a trivial destructor.
\end{itemdescr}

\rSec3[optional.assign]{Assignment}

\indexlibrarymember{operator=}{optional}%
\begin{itemdecl}
optional<T>& operator=(nullopt_t) noexcept;
\end{itemdecl}

\begin{itemdescr}
\pnum
\effects
If \tcode{*this} contains a value, calls \tcode{val->T::\~T()} to destroy the contained value; otherwise no effect.

\pnum
\returns
\tcode{*this}.

\pnum
\postconditions
\tcode{*this} does not contain a value.
\end{itemdescr}

\indexlibrarymember{operator=}{optional}%
\begin{itemdecl}
optional<T>& operator=(const optional& rhs);
\end{itemdecl}

\begin{itemdescr}
\pnum
\requires
\tcode{is_copy_constructible_v<T>} is \tcode{true} and \tcode{is_copy_assignable_v<T>} is \tcode{true}.

\pnum
\effects
See Table~\ref{tab:optional.assign.copy}.
\begin{lib2dtab2}{\tcode{optional::operator=(const optional\&)} effects}{tab:optional.assign.copy}
{\tcode{*this} contains a value}
{\tcode{*this} does not contain a value}

\rowhdr{\tcode{rhs} contains a value} &
assigns \tcode{*rhs} to the contained value &
initializes the contained value as if direct-non-list-initializing an object of type \tcode{T} with \tcode{*rhs} \\
\rowsep

\rowhdr{\tcode{rhs} does not contain a value} &
destroys the contained value by calling \tcode{val->T::\~T()} &
no effect \\
\end{lib2dtab2}

\pnum
\returns
\tcode{*this}.

\pnum
\postconditions
\tcode{bool(rhs) == bool(*this)}.

\pnum
\remarks
If any exception is thrown, the result of the expression \tcode{bool(*this)} remains unchanged.
If an exception is thrown during the call to \tcode{T}'s copy constructor, no effect.
If an exception is thrown during the call to \tcode{T}'s copy assignment,
the state of its contained value is as defined by the exception safety guarantee of \tcode{T}'s copy assignment.
\end{itemdescr}

\indexlibrarymember{operator=}{optional}%
\begin{itemdecl}
optional<T>& operator=(optional&& rhs) noexcept(@\seebelow@);
\end{itemdecl}

\begin{itemdescr}
\pnum
\requires
\tcode{is_move_constructible_v<T>} is \tcode{true} and \tcode{is_move_assignable_v<T>} is \tcode{true}.

\pnum
\effects
See Table~\ref{tab:optional.assign.move}.
The result of the expression \tcode{bool(rhs)} remains unchanged.
\begin{lib2dtab2}{\tcode{optional::operator=(optional\&\&)} effects}{tab:optional.assign.move}
{\tcode{*this} contains a value}
{\tcode{*this} does not contain a value}

\rowhdr{\tcode{rhs} contains a value} &
assigns \tcode{std::move(*rhs)} to the contained value &
initializes the contained value as if direct-non-list-initializing an object of type \tcode{T} with \tcode{std::move(*rhs)} \\
\rowsep

\rowhdr{\tcode{rhs} does not contain a value} &
destroys the contained value by calling \tcode{val->T::\~T()} &
no effect \\
\end{lib2dtab2}

\pnum
\returns
\tcode{*this}.

\pnum
\postconditions
\tcode{bool(rhs) == bool(*this)}.

\pnum
\remarks
The expression inside \tcode{noexcept} is equivalent to:
\begin{codeblock}
is_nothrow_move_assignable_v<T> && is_nothrow_move_constructible_v<T>
\end{codeblock}

\pnum
If any exception is thrown, the result of the expression \tcode{bool(*this)} remains unchanged.
If an exception is thrown during the call to \tcode{T}'s move constructor,
the state of \tcode{*rhs.val} is determined by the exception safety guarantee of \tcode{T}'s move constructor.
If an exception is thrown during the call to \tcode{T}'s move assignment,
the state of \tcode{*val} and \tcode{*rhs.val} is determined by the exception safety guarantee of \tcode{T}'s move assignment.
\end{itemdescr}

\indexlibrarymember{operator=}{optional}%
\begin{itemdecl}
template <class U> optional<T>& operator=(U&& v);
\end{itemdecl}

\begin{itemdescr}
\pnum
\requires
\tcode{is_constructible_v<T, U>} is \tcode{true} and \tcode{is_assignable_v<T\&, U>} is \tcode{true}.

\pnum
\effects
If \tcode{*this} contains a value, assigns \tcode{std::forward<U>(v)} to the contained value; otherwise initializes the contained value as if direct-non-list-initializing object of type \tcode{T} with \tcode{std::forward<U>(v)}.

\pnum
\returns
\tcode{*this}.

\pnum
\postconditions
\tcode{*this} contains a value.

\pnum
\remarks
If any exception is thrown, the result of the expression \tcode{bool(*this)} remains unchanged. If an exception is thrown during the call to \tcode{T}'s constructor, the state of \tcode{v} is determined by the exception safety guarantee of \tcode{T}'s constructor. If an exception is thrown during the call to \tcode{T}'s assignment, the state of \tcode{*val} and \tcode{v} is determined by the exception safety guarantee of \tcode{T}'s assignment.
The function shall not participate in overload resolution unless
\tcode{is_same_v<decay_t<U>, T>} is \tcode{true}.

\pnum
\realnotes
The reason for providing such generic assignment and then constraining it so that effectively \tcode{T} == \tcode{U} is to guarantee that assignment of the form \tcode{o = \{\}} is unambiguous.
\end{itemdescr}

\indexlibrarymember{emplace}{optional}%
\begin{itemdecl}
template <class... Args> void emplace(Args&&... args);
\end{itemdecl}

\begin{itemdescr}
\pnum
\requires
\tcode{is_constructible_v<T, Args\&\&...>} is \tcode{true}.

\pnum
\effects
Calls \tcode{*this = nullopt}. Then initializes the contained value as if direct-non-list-initializing an object of type \tcode{T} with the arguments \tcode{std::forward<Args>(args)...}.

\pnum
\postconditions
\tcode{*this} contains a value.

\pnum
\throws
Any exception thrown by the selected constructor of \tcode{T}.

\pnum
\remarks
If an exception is thrown during the call to \tcode{T}'s constructor, \tcode{*this} does not contain a value, and the previous \tcode{*val} (if any) has been destroyed.
\end{itemdescr}

\indexlibrarymember{emplace}{optional}%
\begin{itemdecl}
template <class U, class... Args> void emplace(initializer_list<U> il, Args&&... args);
\end{itemdecl}

\begin{itemdescr}
\pnum
\effects
Calls \tcode{*this = nullopt}. Then initializes the contained value as if direct-non-list-initializing an object of type \tcode{T} with the arguments \tcode{il, std::forward<Args>(args)...}.

\pnum
\postconditions
\tcode{*this} contains a value.

\pnum
\throws
Any exception thrown by the selected constructor of \tcode{T}.

\pnum
\remarks
If an exception is thrown during the call to \tcode{T}'s constructor, \tcode{*this} does not contain a value, and the previous \tcode{*val} (if any) has been destroyed.
This function shall not participate in overload resolution unless \tcode{is_constructible_v<T, initializer_list<U>\&, Args\&\&...>} is \tcode{true}.
\end{itemdescr}

\rSec3[optional.swap]{Swap}

\indexlibrarymember{swap}{optional}%
\begin{itemdecl}
void swap(optional& rhs) noexcept(@\seebelow@);
\end{itemdecl}

\begin{itemdescr}
\pnum
\requires
Lvalues of type \tcode{T} shall be swappable and \tcode{is_move_constructible_v<T>} is \tcode{true}.

\pnum
\effects
See Table~\ref{tab:optional.swap}.
\begin{lib2dtab2}{\tcode{optional::swap(optional\&)} effects}{tab:optional.swap}
{\tcode{*this} contains a value}
{\tcode{*this} does not contain a value}

\rowhdr{\tcode{rhs} contains a value} &
calls \tcode{swap(*(*this), *rhs)} &
initializes the contained value of \tcode{*this} as if
direct-non-list-initializing an object of type \tcode{T} with the expression \tcode{std::move(*rhs)},
followed by \tcode{rhs.val->T::\~T()};
postcondition is that \tcode{*this} contains a value and \tcode{rhs} does not contain a value \\
\rowsep

\rowhdr{\tcode{rhs} does not contain a value} &
initializes the contained value of \tcode{rhs} as if
direct-non-list-initializing an object of type \tcode{T} with the expression \tcode{std::move(*(*this))},
followed by \tcode{val->T::\~T()};
postcondition is that \tcode{*this} does not contain a value and \tcode{rhs} contains a value &
no effect \\
\end{lib2dtab2}

\pnum
\throws
Any exceptions thrown by the operations in the relevant part of Table~\ref{tab:optional.swap}.

\pnum
\remarks
The expression inside \tcode{noexcept} is equivalent to:
\begin{codeblock}
is_nothrow_move_constructible_v<T> && noexcept(swap(declval<T&>(), declval<T&>()))
\end{codeblock}
If any exception is thrown, the results of the expressions \tcode{bool(*this)} and \tcode{bool(rhs)} remain unchanged.
If an exception is thrown during the call to function \tcode{swap},
the state of \tcode{*val} and \tcode{*rhs.val} is determined by the exception safety guarantee of \tcode{swap} for lvalues of \tcode{T}.
If an exception is thrown during the call to \tcode{T}'s move constructor,
the state of \tcode{*val} and \tcode{*rhs.val} is determined by the exception safety guarantee of \tcode{T}'s move constructor.
\end{itemdescr}

\rSec3[optional.observe]{Observers}

\indexlibrarymember{operator->}{optional}%
\begin{itemdecl}
constexpr T const* operator->() const;
constexpr T* operator->();
\end{itemdecl}

\begin{itemdescr}
\pnum
\requires
\tcode{*this} contains a value.

\pnum
\returns
\tcode{val}.

\pnum
\throws
Nothing.

\pnum
\remarks
Unless \tcode{T} is a user-defined type with overloaded unary \tcode{operator\&}, these functions shall be \tcode{constexpr} functions.
\end{itemdescr}

\indexlibrarymember{operator*}{optional}%
\begin{itemdecl}
constexpr T const& operator*() const &;
constexpr T& operator*() &;
\end{itemdecl}

\begin{itemdescr}
\pnum
\requires
\tcode{*this} contains a value.

\pnum
\returns
\tcode{*val}.

\pnum
\throws
Nothing.

\pnum
\remarks
These functions shall be \tcode{constexpr} functions.
\end{itemdescr}

\indexlibrarymember{operator*}{optional}%
\begin{itemdecl}
constexpr T&& operator*() &&;
constexpr const T&& operator*() const &&;
\end{itemdecl}

\begin{itemdescr}
\pnum
\requires
\tcode{*this} contains a value.

\pnum
\effects
Equivalent to: \tcode{return std::move(*val);}
\end{itemdescr}

\indexlibrarymember{operator bool}{optional}%
\begin{itemdecl}
constexpr explicit operator bool() const noexcept;
\end{itemdecl}

\begin{itemdescr}
\pnum
\returns
\tcode{true} if and only if \tcode{*this} contains a value.

\pnum
\remarks
This function shall be a \tcode{constexpr} function.
\end{itemdescr}

\indexlibrarymember{has_value}{optional}%
\begin{itemdecl}
constexpr bool has_value() const noexcept;
\end{itemdecl}

\begin{itemdescr}
\pnum
\returns \tcode{true} if and only if \tcode{*this} contains a value.

\pnum
\remarks This function shall be a \tcode{constexpr} function.
\end{itemdescr}

\indexlibrarymember{value}{optional}%
\begin{itemdecl}
constexpr T const& value() const &;
constexpr T& value() &;
\end{itemdecl}

\begin{itemdescr}
\pnum
\effects
Equivalent to:
\begin{codeblock}
return bool(*this) ? *val : throw bad_optional_access();
\end{codeblock}
\end{itemdescr}

\indexlibrarymember{value}{optional}%
\begin{itemdecl}
constexpr T&& value() &&;
constexpr const T&& value() const &&;
\end{itemdecl}

\begin{itemdescr}

\pnum
\effects
Equivalent to:
\begin{codeblock}
return bool(*this) ? std::move(*val) : throw bad_optional_access();
\end{codeblock}
\end{itemdescr}

\indexlibrarymember{value_or}{optional}%
\begin{itemdecl}
template <class U> constexpr T value_or(U&& v) const &;
\end{itemdecl}

\begin{itemdescr}
\pnum
\effects
Equivalent to:
\begin{codeblock}
return bool(*this) ? **this : static_cast<T>(std::forward<U>(v));
\end{codeblock}

\pnum
\remarks
If \tcode{is_copy_constructible_v<T> \&\& is_convertible_v<U\&\&, T>} is \tcode{false},
the program is ill-formed.
\end{itemdescr}

\indexlibrarymember{value_or}{optional}%
\begin{itemdecl}
template <class U> constexpr T value_or(U&& v) &&;
\end{itemdecl}

\begin{itemdescr}
\pnum
\effects
Equivalent to:
\begin{codeblock}
return bool(*this) ? std::move(**this) : static_cast<T>(std::forward<U>(v));
\end{codeblock}

\pnum
\remarks
If \tcode{is_move_constructible_v<T> \&\& is_convertible_v<U\&\&, T>} is \tcode{false},
the program is ill-formed.
\end{itemdescr}

\rSec3[optional.mod]{Modifiers}

\indexlibrarymember{reset}{optional}%
\begin{itemdecl}
void reset() noexcept;
\end{itemdecl}

\begin{itemdescr}
\pnum
\effects
If \tcode{*this} contains a value, calls \tcode{val->T::\~T()} to destroy the contained value;
otherwise no effect.

\pnum
\postconditions
\tcode{*this} does not contain a value.
\end{itemdescr}

\rSec2[optional.nullopt]{No-value state indicator}

\indexlibrary{\idxcode{nullopt_t}}%
\indexlibrary{\idxcode{nullopt}}%
\begin{itemdecl}
struct nullopt_t{@\seebelow@};
constexpr nullopt_t nullopt(@\unspec@);
\end{itemdecl}

\pnum
The struct \tcode{nullopt_t} is an empty structure type used as a unique type to indicate the state of not containing a value for \tcode{optional} objects.
In particular, \tcode{optional<T>} has a constructor with \tcode{nullopt_t} as a single argument;
this indicates that an optional object not containing a value shall be constructed.

\pnum
Type \tcode{nullopt_t} shall not have a default constructor. It shall be a literal type. Constant \tcode{nullopt} shall be initialized with an argument of literal type.

\rSec2[optional.bad.access]{Class \tcode{bad_optional_access}}

\begin{codeblock}
class bad_optional_access : public logic_error {
public:
  bad_optional_access();
};
\end{codeblock}

\pnum
The class \tcode{bad_optional_access} defines the type of objects thrown as exceptions to report the situation where an attempt is made to access the value of an optional object that does not contain a value.

\indexlibrary{\idxcode{bad_optional_access}!constructor}%
\indexlibrarymember{what}{bad_optional_access}%
\begin{itemdecl}
bad_optional_access();
\end{itemdecl}

\begin{itemdescr}
\pnum
\effects
Constructs an object of class \tcode{bad_optional_access}.

\pnum
\postconditions
\tcode{what()} returns an
\impldef{return value of \tcode{bad_optional_access::what}}
\ntbs.
\end{itemdescr}

\rSec2[optional.relops]{Relational operators}

\indexlibrarymember{operator==}{optional}%
\begin{itemdecl}
template <class T> constexpr bool operator==(const optional<T>& x, const optional<T>& y);
\end{itemdecl}

\begin{itemdescr}
\pnum
\requires
The expression \tcode{*x == *y} shall be well-formed and
its result shall be convertible to \tcode{bool}.
\begin{note} \tcode{T} need not be \tcode{EqualityComparable}. \end{note}

\pnum
\returns
If \tcode{bool(x) != bool(y)}, \tcode{false}; otherwise if \tcode{bool(x) == false}, \tcode{true}; otherwise \tcode{*x == *y}.

\pnum
\remarks
Specializations of this function template
for which \tcode{*x == *y} is a core constant expression
shall be \tcode{constexpr} functions.
\end{itemdescr}

\indexlibrarymember{operator"!=}{optional}%
\begin{itemdecl}
template <class T> constexpr bool operator!=(const optional<T>& x, const optional<T>& y);
\end{itemdecl}

\begin{itemdescr}
\pnum
\requires
The expression \tcode{*x != *y} shall be well-formed and
its result shall be convertible to \tcode{bool}.

\pnum
\returns
If \tcode{bool(x) != bool(y)}, \tcode{true};
otherwise, if \tcode{bool(x) == false}, \tcode{false};
otherwise \tcode{*x != *y}.

\pnum
\remarks
Specializations of this function template
for which \tcode{*x != *y} is a core constant expression
shall be constexpr functions.
\end{itemdescr}

\indexlibrarymember{operator<}{optional}%
\begin{itemdecl}
template <class T> constexpr bool operator<(const optional<T>& x, const optional<T>& y);
\end{itemdecl}

\begin{itemdescr}
\pnum
\requires
\tcode{*x < *y} shall be well-formed
and its result shall be convertible to \tcode{bool}.

\pnum
\returns
If \tcode{!y}, \tcode{false};
otherwise, if \tcode{!x}, \tcode{true};
otherwise \tcode{*x < *y}.

\pnum
\remarks
Specializations of this function template
for which \tcode{*x < *y} is a core constant expression
shall be \tcode{constexpr} functions.
\end{itemdescr}

\indexlibrarymember{operator>}{optional}%
\begin{itemdecl}
template <class T> constexpr bool operator>(const optional<T>& x, const optional<T>& y);
\end{itemdecl}

\begin{itemdescr}
\pnum
\requires
The expression \tcode{*x > *y} shall be well-formed and
its result shall be convertible to \tcode{bool}.

\pnum
\returns
If \tcode{!x}, \tcode{false};
otherwise, if \tcode{!y}, \tcode{true};
otherwise \tcode{*x > *y}.

\pnum
\remarks
Specializations of this function template
for which \tcode{*x > *y} is a core constant expression
shall be constexpr functions.
\end{itemdescr}

\indexlibrarymember{operator<=}{optional}%
\begin{itemdecl}
template <class T> constexpr bool operator<=(const optional<T>& x, const optional<T>& y);
\end{itemdecl}

\begin{itemdescr}
\pnum
\requires
The expression \tcode{*x <= *y} shall be well-formed and
its result shall be convertible to \tcode{bool}.

\pnum
\returns
If \tcode{!x}, \tcode{true};
otherwise, if \tcode{!y}, \tcode{false};
otherwise \tcode{*x <= *y}.

\pnum
\remarks
Specializations of this function template
for which \tcode{*x <= *y} is a core constant expression
shall be constexpr functions.
\end{itemdescr}

\indexlibrarymember{operator>=}{optional}%
\begin{itemdecl}
template <class T> constexpr bool operator>=(const optional<T>& x, const optional<T>& y);
\end{itemdecl}

\begin{itemdescr}
\pnum
\requires
The expression \tcode{*x >= *y} shall be well-formed and
its result shall be convertible to \tcode{bool}.

\pnum
\returns
If \tcode{!y}, \tcode{true};
otherwise, if \tcode{!x}, \tcode{false};
otherwise \tcode{*x >= *y}.

\pnum
\remarks
Specializations of this function template
for which \tcode{*x >= *y} is a core constant expression
shall be constexpr functions.
\end{itemdescr}

\rSec2[optional.nullops]{Comparison with \tcode{nullopt}}

\indexlibrarymember{operator==}{optional}%
\begin{itemdecl}
template <class T> constexpr bool operator==(const optional<T>& x, nullopt_t) noexcept;
template <class T> constexpr bool operator==(nullopt_t, const optional<T>& x) noexcept;
\end{itemdecl}

\begin{itemdescr}
\pnum
\returns
\tcode{!x}.
\end{itemdescr}

\indexlibrarymember{operator"!=}{optional}%
\begin{itemdecl}
template <class T> constexpr bool operator!=(const optional<T>& x, nullopt_t) noexcept;
template <class T> constexpr bool operator!=(nullopt_t, const optional<T>& x) noexcept;
\end{itemdecl}

\begin{itemdescr}
\pnum
\returns
\tcode{bool(x)}.
\end{itemdescr}

\indexlibrarymember{operator<}{optional}%
\begin{itemdecl}
template <class T> constexpr bool operator<(const optional<T>& x, nullopt_t) noexcept;
\end{itemdecl}

\begin{itemdescr}
\pnum
\returns
\tcode{false}.
\end{itemdescr}

\indexlibrarymember{operator<}{optional}%
\begin{itemdecl}
template <class T> constexpr bool operator<(nullopt_t, const optional<T>& x) noexcept;
\end{itemdecl}

\begin{itemdescr}
\pnum
\returns
\tcode{bool(x)}.
\end{itemdescr}

\indexlibrarymember{operator<=}{optional}%
\begin{itemdecl}
template <class T> constexpr bool operator<=(const optional<T>& x, nullopt_t) noexcept;
\end{itemdecl}

\begin{itemdescr}
\pnum
\returns
\tcode{!x}.
\end{itemdescr}

\indexlibrarymember{operator<=}{optional}%
\begin{itemdecl}
template <class T> constexpr bool operator<=(nullopt_t, const optional<T>& x) noexcept;
\end{itemdecl}

\begin{itemdescr}
\pnum
\returns
\tcode{true}.
\end{itemdescr}

\indexlibrarymember{operator>}{optional}%
\begin{itemdecl}
template <class T> constexpr bool operator>(const optional<T>& x, nullopt_t) noexcept;
\end{itemdecl}

\begin{itemdescr}
\pnum
\returns
\tcode{bool(x)}.
\end{itemdescr}

\indexlibrarymember{operator>}{optional}%
\begin{itemdecl}
template <class T> constexpr bool operator>(nullopt_t, const optional<T>& x) noexcept;
\end{itemdecl}

\begin{itemdescr}
\pnum
\returns
\tcode{false}.
\end{itemdescr}

\indexlibrarymember{operator>=}{optional}%
\begin{itemdecl}
template <class T> constexpr bool operator>=(const optional<T>& x, nullopt_t) noexcept;
\end{itemdecl}

\begin{itemdescr}
\pnum
\returns
\tcode{true}.
\end{itemdescr}

\indexlibrarymember{operator>=}{optional}%
\begin{itemdecl}
template <class T> constexpr bool operator>=(nullopt_t, const optional<T>& x) noexcept;
\end{itemdecl}

\begin{itemdescr}
\pnum
\returns
\tcode{!x}.
\end{itemdescr}

\rSec2[optional.comp_with_t]{Comparison with \tcode{T}}

\indexlibrarymember{operator==}{optional}%
\begin{itemdecl}
template <class T> constexpr bool operator==(const optional<T>& x, const T& v);
\end{itemdecl}

\begin{itemdescr}
\pnum
\effects
Equivalent to: \tcode{return bool(x) ? *x == v : false;}
\end{itemdescr}

\indexlibrarymember{operator==}{optional}%
\begin{itemdecl}
template <class T> constexpr bool operator==(const T& v, const optional<T>& x);
\end{itemdecl}

\begin{itemdescr}
\pnum
\effects
Equivalent to: \tcode{return bool(x) ? v == *x : false;}
\end{itemdescr}

\indexlibrarymember{operator"!=}{optional}%
\begin{itemdecl}
template <class T> constexpr bool operator!=(const optional<T>& x, const T& v);
\end{itemdecl}

\begin{itemdescr}
\pnum
\effects
Equivalent to: \tcode{return bool(x) ? *x != v : true;}
\end{itemdescr}

\indexlibrarymember{operator"!=}{optional}%
\begin{itemdecl}
template <class T> constexpr bool operator!=(const T& v, const optional<T>& x);
\end{itemdecl}

\begin{itemdescr}
\pnum
\effects
Equivalent to: \tcode{return bool(x) ? v != *x : true;}
\end{itemdescr}

\indexlibrarymember{operator<}{optional}%
\begin{itemdecl}
template <class T> constexpr bool operator<(const optional<T>& x, const T& v);
\end{itemdecl}

\begin{itemdescr}
\pnum
\effects
Equivalent to: \tcode{return bool(x) ? *x < v : true;}
\end{itemdescr}

\indexlibrarymember{operator<}{optional}%
\begin{itemdecl}
template <class T> constexpr bool operator<(const T& v, const optional<T>& x);
\end{itemdecl}

\begin{itemdescr}
\pnum
\effects
Equivalent to: \tcode{return bool(x) ? v < *x : false;}
\end{itemdescr}

\indexlibrarymember{operator<=}{optional}%
\begin{itemdecl}
template <class T> constexpr bool operator<=(const optional<T>& x, const T& v);
\end{itemdecl}

\begin{itemdescr}
\pnum
\effects
Equivalent to: \tcode{return bool(x) ? *x <= v : true;}
\end{itemdescr}

\indexlibrarymember{operator<=}{optional}%
\begin{itemdecl}
template <class T> constexpr bool operator<=(const T& v, const optional<T>& x);
\end{itemdecl}

\begin{itemdescr}
\pnum
\effects
Equivalent to: \tcode{return bool(x) ? v <= *x : false;}
\end{itemdescr}

\indexlibrarymember{operator>}{optional}%
\begin{itemdecl}
template <class T> constexpr bool operator>(const optional<T>& x, const T& v);
\end{itemdecl}

\begin{itemdescr}
\pnum
\effects
Equivalent to: \tcode{return bool(x) ? *x > v : false;}
\end{itemdescr}

\indexlibrarymember{operator>}{optional}%
\begin{itemdecl}
template <class T> constexpr bool operator>(const T& v, const optional<T>& x);
\end{itemdecl}

\begin{itemdescr}
\pnum
\effects
Equivalent to: \tcode{return bool(x) ? v > *x : true;}
\end{itemdescr}

\indexlibrarymember{operator>=}{optional}%
\begin{itemdecl}
template <class T> constexpr bool operator>=(const optional<T>& x, const T& v);
\end{itemdecl}

\begin{itemdescr}
\pnum
\effects
Equivalent to: \tcode{return bool(x) ? *x >= v : false;}
\end{itemdescr}

\indexlibrarymember{operator>=}{optional}%
\begin{itemdecl}
template <class T> constexpr bool operator>=(const T& v, const optional<T>& x);
\end{itemdecl}

\begin{itemdescr}
\pnum
\effects
Equivalent to: \tcode{return bool(x) ? v >= *x : true;}
\end{itemdescr}


\rSec2[optional.specalg]{Specialized algorithms}

\indexlibrary{\idxcode{swap}!\idxcode{optional}}%
\begin{itemdecl}
template <class T> void swap(optional<T>& x, optional<T>& y) noexcept(noexcept(x.swap(y)));
\end{itemdecl}

\begin{itemdescr}
\pnum
\effects
Calls \tcode{x.swap(y)}.
\end{itemdescr}

\indexlibrary{\idxcode{make_optional}}%
\begin{itemdecl}
template <class T> constexpr optional<decay_t<T>> make_optional(T&& v);
\end{itemdecl}

\begin{itemdescr}
\pnum
\returns
\tcode{optional<decay_t<T>>(std::forward<T>(v))}.
\end{itemdescr}

\indexlibrary{\idxcode{make_optional}}%
\begin{itemdecl}
template <class T, class...Args>
  constexpr optional<T> make_optional(Args&&... args);
\end{itemdecl}

\begin{itemdescr}
\pnum
\effects Equivalent to: \tcode{return optional<T>(in_place, std::forward<Args>(args)...);}
\end{itemdescr}

\indexlibrary{\idxcode{make_optional}}%
\begin{itemdecl}
template <class T, class U, class... Args>
  constexpr optional<T> make_optional(initializer_list<U> il, Args&&... args);
\end{itemdecl}

\begin{itemdescr}
\pnum
\effects Equivalent to: \tcode{return optional<T>(in_place, il, std::forward<Args>(args)...);}
\end{itemdescr}

\rSec2[optional.hash]{Hash support}

\indexlibrary{\idxcode{hash}!\idxcode{optional}}%
\begin{itemdecl}
template <class T> struct hash<optional<T>>;
\end{itemdecl}

\begin{itemdescr}
\pnum
\requires
The template specialization \tcode{hash<T>} shall meet the requirements of class template \tcode{hash} (\ref{unord.hash}).
The template specialization \tcode{hash<optional<T>>} shall meet the requirements of class template \tcode{hash}.
For an object \tcode{o} of type \tcode{optional<T>}, if \tcode{bool(o) == true},
\tcode{hash<optional<T>>()(o)} shall evaluate to the same value as \tcode{hash<T>()(*o)};
otherwise it evaluates to an unspecified value.
\end{itemdescr}


\rSec1[variant]{Variants}

\rSec2[variant.general]{In general}

\pnum
A variant object holds and manages the lifetime of a value.
If the \tcode{variant} holds a value, that value's type has to be one
of the template argument types given to variant.
These template arguments are called alternatives.

\indexlibrary{\idxhdr{variant}}%
\synopsis{Header \tcode{<variant>} synopsis}

\begin{codeblock}
namespace std {
  // \ref{variant.variant}, variant
  template <class... Types> class variant;

  // \ref{variant.helper}, variant helper classes
  template <class T> struct variant_size; // not defined
  template <class T> struct variant_size<const T>;
  template <class T> struct variant_size<volatile T>;
  template <class T> struct variant_size<const volatile T>;
  template <class T> constexpr size_t variant_size_v
    = variant_size<T>::value;

  template <class... Types>
    struct variant_size<variant<Types...>>;

  template <size_t I, class T> struct variant_alternative; // not defined
  template <size_t I, class T> struct variant_alternative<I, const T>;
  template <size_t I, class T> struct variant_alternative<I, volatile T>;
  template <size_t I, class T> struct variant_alternative<I, const volatile T>;
  template <size_t I, class T>
    using variant_alternative_t = typename variant_alternative<I, T>::type;

  template <size_t I, class... Types>
    struct variant_alternative<I, variant<Types...>>;

  constexpr size_t variant_npos = -1;

  // \ref{variant.get}, value access
  template <class T, class... Types>
    constexpr bool holds_alternative(const variant<Types...>&) noexcept;

  template <size_t I, class... Types>
    constexpr variant_alternative_t<I, variant<Types...>>&
    get(variant<Types...>&);
  template <size_t I, class... Types>
    constexpr variant_alternative_t<I, variant<Types...>>&&
    get(variant<Types...>&&);
  template <size_t I, class... Types>
    constexpr variant_alternative_t<I, variant<Types...>> const&
    get(const variant<Types...>&);
  template <size_t I, class... Types>
    constexpr variant_alternative_t<I, variant<Types...>> const&&
    get(const variant<Types...>&&);

  template <class T, class... Types>
    constexpr T& get(variant<Types...>&);
  template <class T, class... Types>
    constexpr T&& get(variant<Types...>&&);
  template <class T, class... Types>
    constexpr const T& get(const variant<Types...>&);
  template <class T, class... Types>
    constexpr const T&& get(const variant<Types...>&&);

  template <size_t I, class... Types>
    constexpr add_pointer_t<variant_alternative_t<I, variant<Types...>>>
    get_if(variant<Types...>*) noexcept;
  template <size_t I, class... Types>
    constexpr add_pointer_t<const variant_alternative_t<I, variant<Types...>>>
    get_if(const variant<Types...>*) noexcept;

  template <class T, class... Types>
    constexpr add_pointer_t<T> get_if(variant<Types...>*) noexcept;
  template <class T, class... Types>
    constexpr add_pointer_t<const T> get_if(const variant<Types...>*) noexcept;

  // \ref{variant.relops}, relational operators
  template <class... Types>
    constexpr bool operator==(const variant<Types...>&,
                              const variant<Types...>&);
  template <class... Types>
    constexpr bool operator!=(const variant<Types...>&,
                              const variant<Types...>&);
  template <class... Types>
    constexpr bool operator<(const variant<Types...>&,
                             const variant<Types...>&);
  template <class... Types>
    constexpr bool operator>(const variant<Types...>&,
                             const variant<Types...>&);
  template <class... Types>
    constexpr bool operator<=(const variant<Types...>&,
                              const variant<Types...>&);
  template <class... Types>
    constexpr bool operator>=(const variant<Types...>&,
                              const variant<Types...>&);

  // \ref{variant.visit}, visitation
  template <class Visitor, class... Variants>
  constexpr @\seebelow@ visit(Visitor&&, Variants&&...);

  // \ref{variant.monostate}, class \tcode{monostate}
  struct monostate;

  // \ref{variant.monostate.relops}, \tcode{monostate} relational operators
  constexpr bool operator<(monostate, monostate) noexcept;
  constexpr bool operator>(monostate, monostate) noexcept;
  constexpr bool operator<=(monostate, monostate) noexcept;
  constexpr bool operator>=(monostate, monostate) noexcept;
  constexpr bool operator==(monostate, monostate) noexcept;
  constexpr bool operator!=(monostate, monostate) noexcept;

  // \ref{variant.specalg}, specialized algorithms
  template <class... Types>
  void swap(variant<Types...>&, variant<Types...>&) noexcept(@\seebelow@);

  // \ref{variant.bad.access}, class \tcode{bad_variant_access}
  class bad_variant_access;

  // \ref{variant.hash}, hash support
  template <class T> struct hash;
  template <class... Types> struct hash<variant<Types...>>;
  template <> struct hash<monostate>;

  // \ref{variant.traits}, allocator-related traits
  template <class T, class Alloc> struct uses_allocator;
  template <class... Types, class Alloc>
  struct uses_allocator<variant<Types...>, Alloc>;
}
\end{codeblock}

\indexlibrary{\idxcode{variant}}%
\rSec2[variant.variant]{Class template \tcode{variant}}

\begin{codeblock}
namespace std {
  template <class... Types>
  class variant {
  public:
    // \ref{variant.ctor}, constructors
    constexpr variant() noexcept(@\seebelow@);
    variant(const variant&);
    variant(variant&&) noexcept(@\seebelow@);

    template <class T> constexpr variant(T&&) noexcept(@\seebelow@);

    template <class T, class... Args>
      constexpr explicit variant(in_place_type_t<T>, Args&&...);
    template <class T, class U, class... Args>
      constexpr explicit variant(in_place_type_t<T>, initializer_list<U>, Args&&...);

    template <size_t I, class... Args>
      constexpr explicit variant(in_place_index_t<I>, Args&&...);
    template <size_t I, class U, class... Args>
      constexpr explicit variant(in_place_index_t<I>, initializer_list<U>, Args&&...);

    // allocator-extended constructors
    template <class Alloc>
      variant(allocator_arg_t, const Alloc&);
    template <class Alloc>
      variant(allocator_arg_t, const Alloc&, const variant&);
    template <class Alloc>
      variant(allocator_arg_t, const Alloc&, variant&&);
    template <class Alloc, class T>
      variant(allocator_arg_t, const Alloc&, T&&);
    template <class Alloc, class T, class... Args>
      variant(allocator_arg_t, const Alloc&, in_place_type_t<T>, Args&&...);
    template <class Alloc, class T, class U, class... Args>
      variant(allocator_arg_t, const Alloc&, in_place_type_t<T>, initializer_list<U>, Args&&...);
    template <class Alloc, size_t I, class... Args>
      variant(allocator_arg_t, const Alloc&, in_place_index_t<I>, Args&&...);
    template <class Alloc, size_t I, class U, class... Args>
      variant(allocator_arg_t, const Alloc&, in_place_index_t<I>, initializer_list<U>, Args&&...);

    // \ref{variant.dtor}, destructor
    ~variant();

    // \ref{variant.assign}, assignment
    variant& operator=(const variant&);
    variant& operator=(variant&&) noexcept(@\seebelow@);

    template <class T> variant& operator=(T&&) noexcept(@\seebelow@);

    // \ref{variant.mod}, modifiers
    template <class T, class... Args> void emplace(Args&&...);
    template <class T, class U, class... Args>
      void emplace(initializer_list<U>, Args&&...);
    template <size_t I, class... Args> void emplace(Args&&...);
    template <size_t I, class U, class... Args>
      void emplace(initializer_list<U>, Args&&...);

    // \ref{variant.status}, value status
    constexpr bool valueless_by_exception() const noexcept;
    constexpr size_t index() const noexcept;

    // \ref{variant.swap}, swap
    void swap(variant&) noexcept(@\seebelow@);
  };
}
\end{codeblock}

\pnum
Any instance of \tcode{variant} at any given time either holds a value
of one of its alternative types, or it holds no value.
When an instance of \tcode{variant} holds a value of alternative type \tcode{T},
it means that a value of type \tcode{T}, referred to as the \tcode{variant}
object's contained value, is allocated within the storage of the
\tcode{variant} object.
Implementations are not permitted to use additional storage, such as dynamic
memory, to allocate the contained value.
The contained value shall be allocated in a region of the \tcode{variant}
storage suitably aligned for all types in \tcode{Types...}.
It is \impldef{whether \tcode{variant} supports over-aligned types}
whether over-aligned types are supported.

\pnum
All types in \tcode{Types...} shall be (possibly cv-qualified)
object types, (possibly cv-qualified) \tcode{void}, or references.
\begin{note}
Implementations could decide to store references in a wrapper.
\end{note}

\rSec3[variant.ctor]{Constructors}

\pnum
In the descriptions that follow, let $i$ be in the range \range{0}{sizeof...(Types)},
and $T_i$ be the $i^{th}$ type in \tcode{Types...}.

\indexlibrary{\idxcode{variant}!constructor}%
\begin{itemdecl}
constexpr variant() noexcept(@\seebelow@);
\end{itemdecl}

\begin{itemdescr}
\pnum
\effects
Constructs a \tcode{variant} holding a value-initialized value of type $T_0$.

\pnum
\postconditions
\tcode{valueless_by_exception()} is \tcode{false} and \tcode{index()} is \tcode{0}.

\pnum
\throws
Any exception thrown by the value-initialization of $T_0$.

\pnum
\remarks
This function shall be \tcode{constexpr} if and only if the
value-initialization of the alternative type $T_0$ would satisfy the
requirements for a \tcode{constexpr} function.
The expression inside \tcode{noexcept} is equivalent to
\tcode{is_nothrow_default_constructible_v<$T_0$>}.
This function shall not participate in overload resolution unless
\tcode{is_default_constructible_v<$T_0$>} is \tcode{true}.
\begin{note} See also class \tcode{monostate}. \end{note}
\end{itemdescr}

\indexlibrary{\idxcode{variant}!constructor}%
\begin{itemdecl}
variant(const variant& w);
\end{itemdecl}

\begin{itemdescr}
\pnum
\effects
If \tcode{w} holds a value, initializes the \tcode{variant} to hold the same
alternative as \tcode{w} and direct-initializes the contained value
with \tcode{get<j>(w)}, where \tcode{j} is \tcode{w.index()}.
Otherwise, initializes the \tcode{variant} to not hold a value.

\pnum
\throws
Any exception thrown by direct-initializing any $T_i$ for all $i$.

\pnum
\remarks
This function shall not participate in overload resolution unless
\tcode{is_copy_constructible_v<$T_i$>} is \tcode{true} for all $i$.
\end{itemdescr}

\indexlibrary{\idxcode{variant}!constructor}%
\begin{itemdecl}
variant(variant&& w) noexcept(@\seebelow@);
\end{itemdecl}

\begin{itemdescr}
\pnum
\effects
If \tcode{w} holds a value, initializes the \tcode{variant} to hold the same
alternative as \tcode{w} and direct-initializes the contained value with
\tcode{get<j>(std::move(w))}, where \tcode{j} is \tcode{w.index()}.
Otherwise, initializes the \tcode{variant} to not hold a value.

\pnum
\throws
Any exception thrown by move-constructing any $T_i$ for all $i$.

\pnum
\remarks
The expression inside \tcode{noexcept} is equivalent to the logical AND of
\tcode{is_nothrow_move_constructible_v<$T_i$>} for all $i$.
This function shall not participate in overload resolution unless
\tcode{is_move_constructible_v<$T_i$>} is \tcode{true} for all $i$.
\end{itemdescr}

\indexlibrary{\idxcode{variant}!constructor}%
\begin{itemdecl}
template <class T> constexpr variant(T&& t) noexcept(@\seebelow@);
\end{itemdecl}

\begin{itemdescr}
\pnum
Let $T_j$ be a type that is determined as follows:
build an imaginary function \tcode{\textit{FUN}($T_i$)} for each alternative type $T_i$. The overload \tcode{\textit{FUN}($T_j$)} selected by overload
resolution for the expression \tcode{\textit{FUN}(std::forward<T>(\brk{}t))} defines
the alternative $T_j$ which is the type of the contained value after
construction.

\pnum
\effects
Initializes \tcode{*this} to hold the alternative type $T_j$ and
direct-initializes the contained value as if direct-non-list-initializing it
with \tcode{std::forward<T>(t)}.

\pnum
\postconditions
\tcode{holds_alternative<$T_j$>(*this)} is \tcode{true}.

\pnum
\throws
Any exception thrown by the initialization of the selected alternative $T_j$.

\pnum
\remarks
This function shall not participate in overload resolution unless
\tcode{is_same_v<decay_t<T>, variant>} is \tcode{false}, unless \tcode{is_constructible_v<$T_j$, T>} is \tcode{true}, and unless the expression
\tcode{\textit{FUN}(}\brk\tcode{std::forward<T>(t))} (with \tcode{\textit{FUN}}
being the above-mentioned set of imaginary functions) is well formed.

\pnum
\begin{note}
\begin{codeblock}
variant<string, string> v("abc");
\end{codeblock}
is ill-formed, as both alternative types have an equally viable constructor
for the argument. \end{note}

\pnum
The expression inside \tcode{noexcept} is equivalent to
\tcode{is_nothrow_constructible_v<$T_j$, T>}.
If $T_j$'s selected constructor is a constexpr constructor,
this constructor shall be a constexpr constructor.
\end{itemdescr}

\indexlibrary{\idxcode{variant}!constructor}%
\begin{itemdecl}
template <class T, class... Args> constexpr explicit variant(in_place_type_t<T>, Args&&... args);
\end{itemdecl}

\begin{itemdescr}
\pnum
\effects
Initializes the contained value of type \tcode{T} with the arguments \tcode{std::forward<Args>(args)...}.

\pnum
\postconditions
\tcode{holds_alternative<T>(*this)} is \tcode{true}.

\pnum
\throws
Any exception thrown by calling the selected constructor of \tcode{T}.

\pnum
\remarks
This function shall not participate in overload resolution unless there is
exactly one occurrence of \tcode{T} in \tcode{Types...} and
\tcode{is_constructible_v<T, Args...>} is \tcode{true}.
If \tcode{T}'s selected constructor is a constexpr constructor, this
constructor shall be a constexpr constructor.
\end{itemdescr}

\indexlibrary{\idxcode{variant}!constructor}%
\begin{itemdecl}
template <class T, class U, class... Args>
  constexpr explicit variant(in_place_type_t<T>, initializer_list<U> il, Args&&... args);
\end{itemdecl}

\begin{itemdescr}
\pnum
\effects
Initializes the contained value as if constructing an object of type \tcode{T}
with the arguments \tcode{il, std::forward<Args>(args)...}.

\pnum
\postconditions
\tcode{holds_alternative<T>(*this)} is \tcode{true}.

\pnum
\throws
Any exception thrown by calling the selected constructor of \tcode{T}.

\pnum
\remarks
This function shall not participate in overload resolution unless there is
exactly one occurrence of \tcode{T} in \tcode{Types...} and
\tcode{is_constructible_v<T, initializer_list<U>\&, Args...>} is \tcode{true}.
If \tcode{T}'s selected constructor is a constexpr constructor, this
constructor shall be a constexpr constructor.
\end{itemdescr}

\indexlibrary{\idxcode{variant}!constructor}%
\begin{itemdecl}
template <size_t I, class... Args> constexpr explicit variant(in_place_index_t<I>, Args&&... args);
\end{itemdecl}

\begin{itemdescr}
\pnum
\effects
Initializes the contained value as if constructing an object of type $T_I$
with the arguments \tcode{std::forward<Args>(args)...}.

\pnum
\postconditions
\tcode{index()} is \tcode{I}.

\pnum
\throws
Any exception thrown by calling the selected constructor of $T_I$.

\pnum
\remarks
This function shall not participate in overload resolution unless \tcode{I} is
less than \tcode{sizeof...(Types)} and \tcode{is_constructible_v<$T_I$, Args...>} is \tcode{true}.
If $T_I$'s selected constructor is a \tcode{constexpr} constructor, this
constructor shall be a constexpr constructor.
\end{itemdescr}

\indexlibrary{\idxcode{variant}!constructor}%
\begin{itemdecl}
template <size_t I, class U, class... Args>
  constexpr explicit variant(in_place_index_t<I>, initializer_list<U> il, Args&&... args);
\end{itemdecl}

\begin{itemdescr}
\pnum
\effects
Initializes the contained value as if constructing an object of type
$T_I$ with the arguments \tcode{il, std::forward<Args>(args)...}.

\pnum
\postconditions
\tcode{index()} is \tcode{I}.

\pnum
\remarks
This function shall not participate in overload resolution unless \tcode{I} is
less than \tcode{sizeof...(Types)} and
\tcode{is_constructible_v<$T_I$, initializer_list<U>\&, Args...>} is \tcode{true}.
If $T_I$'s selected constructor is a \tcode{constexpr} constructor, this
constructor shall be a constexpr constructor.
\end{itemdescr}

\indexlibrary{\idxcode{variant}!constructor}%
\begin{itemdecl}
// allocator-extended constructors
template <class Alloc>
  variant(allocator_arg_t, const Alloc& a);
template <class Alloc>
  variant(allocator_arg_t, const Alloc& a, const variant& v);
template <class Alloc>
  variant(allocator_arg_t, const Alloc& a, variant&& v);
template <class Alloc, class T>
  variant(allocator_arg_t, const Alloc& a, T&& t);
template <class Alloc, class T, class... Args>
  variant(allocator_arg_t, const Alloc& a, in_place_type_t<T>, Args&&... args);
template <class Alloc, class T, class U, class... Args>
  variant(allocator_arg_t, const Alloc& a, in_place_type_t<T>,
          initializer_list<U> il, Args&&... args);
template <class Alloc, size_t I, class... Args>
  variant(allocator_arg_t, const Alloc& a, in_place_index_t<I>, Args&&... args);
template <class Alloc, size_t I, class U, class... Args>
  variant(allocator_arg_t, const Alloc& a, in_place_index_t<I>,
          initializer_list<U> il, Args&&... args);
\end{itemdecl}

\begin{itemdescr}
\pnum
\requires
\tcode{Alloc} shall meet the requirements for an Allocator~(\ref{allocator.requirements}).

\pnum
\effects
Equivalent to the preceding constructors except that the contained value is
constructed with uses-allocator construction~(\ref{allocator.uses.construction}).
\end{itemdescr}

\rSec3[variant.dtor]{Destructor}

\indexlibrary{\idxcode{variant}!destructor}%
\begin{itemdecl}
~variant();
\end{itemdecl}

\begin{itemdescr}
\pnum
\effects
If \tcode{valueless_by_exception()} is \tcode{false},
destroys the currently contained value.

\pnum
\remarks
If \tcode{is_trivially_destructible_v<$T_i$> == true} for all $T_i$
then this destructor shall be a trivial destructor.
\end{itemdescr}

\rSec3[variant.assign]{Assignment}

\indexlibrarymember{operator=}{variant}%
\begin{itemdecl}
variant& operator=(const variant& rhs);
\end{itemdecl}

\begin{itemdescr}
\pnum
\effects
\begin{itemize}
\item
If neither \tcode{*this} nor \tcode{rhs} holds a value, there is no effect.  Otherwise,
\item
if \tcode{*this} holds a value but \tcode{rhs} does not, destroys the value
contained in \tcode{*this} and sets \tcode{*this} to not hold a value.  Otherwise,
\item
if \tcode{index() == rhs.index()}, assigns the value contained in \tcode{rhs}
to the value contained in \tcode{*this}. Otherwise,
\item
copies the value contained in \tcode{rhs} to a temporary, then destroys any
value contained in \tcode{*this}. Sets \tcode{*this} to hold the same
alternative index as \tcode{rhs} and initializes the value contained in
\tcode{*this} as if direct-non-list-initializing an object of type $T_j$
with \tcode{std::forward<$T_j$>(TMP),} with \tcode{TMP} being the temporary and
$j$ being \tcode{rhs.index()}.
\end{itemize}

\pnum
\returns \tcode{*this}.

\pnum
\postconditions \tcode{index() == rhs.index()}.

\pnum
\remarks
This function shall not participate in overload resolution unless
\tcode{is_copy_constructible_v<$T_i$> \&\& is_move_constructible_v<$T_i$> \&\& is_copy_assignable_v<$T_i$>}
is \tcode{true} for all $i$.
\begin{itemize}
\item If an exception is thrown during the call to $T_j$'s copy assignment,
the state of the contained value is as defined by the exception safety
guarantee of $T_j$'s copy assignment; \tcode{index()} will be $j$.
\item If an exception is thrown during the call to $T_j$'s copy construction
(with $j$ being \tcode{rhs.index()}), \tcode{*this} will remain unchanged.
\item If an exception is thrown during the call to $T_j$'s move construction,
the \tcode{variant} will hold no value.
\end{itemize}
\end{itemdescr}

\indexlibrarymember{operator=}{variant}%
\begin{itemdecl}
variant& operator=(variant&& rhs) noexcept(@\seebelow@);
\end{itemdecl}

\begin{itemdescr}
\pnum
\effects
\begin{itemize}
\item
If neither \tcode{*this} nor \tcode{rhs} holds a value, there is no effect. Otherwise,
\item
if \tcode{*this} holds a value but \tcode{rhs} does not, destroys the value
contained in \tcode{*this} and sets \tcode{*this} to not hold a value. Otherwise,
\item
if \tcode{index() == rhs.index()}, assigns \tcode{get<$j$>(std::move(rhs))} to
the value contained in \tcode{*this}, with $j$ being \tcode{index()}. Otherwise,
\item
destroys any value contained in \tcode{*this}. Sets \tcode{*this} to hold the
same alternative index as \tcode{rhs} and initializes the value contained in
\tcode{*this} as if direct-non-list-initializing an object of type $T_j$
with \tcode{get<$j$>(std::move(rhs))} with $j$ being \tcode{rhs.index()}.
\end{itemize}

\pnum
\returns \tcode{*this}.

\pnum
\remarks
This function shall not participate in overload resolution unless
\tcode{is_move_constructible_v<$T_i$> \&\& is_move_assignable_v<$T_i$>} is
\tcode{true} for all $i$.
The expression inside \tcode{noexcept} is equivalent to:
\tcode{is_nothrow_move_constructible_v<$T_i$> \&\& is_nothrow_move_assignable_v<$T_i$>} for all $i$.
\begin{itemize}
\item If an exception is thrown during the call to $T_j$'s move construction
(with $j$ being \tcode{rhs.index())}, the \tcode{variant} will hold no value.
\item If an exception is thrown during the call to $T_j$'s move assignment,
the state of the contained value is as defined by the exception safety
guarantee of $T_j$'s move assignment; \tcode{index()} will be $j$.
\end{itemize}
\end{itemdescr}

\indexlibrarymember{operator=}{variant}%
\begin{itemdecl}
template <class T> variant& operator=(T&& t) noexcept(@\seebelow@);
\end{itemdecl}

\begin{itemdescr}
\pnum
Let $T_j$ be a type that is determined as follows:
build an imaginary function \tcode{\textit{FUN}($T_i$)} for each alternative type
$T_i$. The overload \tcode{\textit{FUN}($T_j$)} selected by overload
resolution for the expression \tcode{\textit{FUN}(std::forward<T>(\brk{}t))} defines
the alternative $T_j$ which is the type of the contained value after
assignment.

\pnum
\effects
If \tcode{*this} holds a $T_j$, assigns \tcode{std::forward<T>(t)} to
the value contained in \tcode{*this}. Otherwise, destroys any value contained
in \tcode{*this}, sets \tcode{*this} to hold the alternative type $T_j$
as selected by the imaginary function overload resolution described above,
and direct-initializes the contained value as if direct-non-list-initializing
it with \tcode{std::forward<T>(t)}.

\pnum
\postconditions
\tcode{holds_alternative<$T_j$>(*this)} is \tcode{true}, with $T_j$
selected by the imaginary function overload resolution described above.

\pnum
\returns \tcode{*this}.

\pnum
\remarks
This function shall not participate in overload resolution unless
\tcode{is_same_v<decay_t<T>, variant>} is \tcode{false}, unless
\tcode{is_assignable_v<$T_j$\&, T> \&\& is_constructible_v<$T_j$, T>} is \tcode{true},
and unless the expression \tcode{\textit{FUN}(std::forward<T>(t))} (with
\tcode{\textit{FUN}} being the above-mentioned set of imaginary functions)
is well formed.

\pnum
\begin{note}
\begin{codeblock}
variant<string, string> v;
v = "abc";
\end{codeblock}
is ill-formed, as both alternative types have an equally viable constructor
for the argument. \end{note}

\pnum
The expression inside \tcode{noexcept} is equivalent to:
\tcode{is_nothrow_assignable_v<$T_j$\&, T> \&\& is_nothrow_constructible_v<$T_j$, T>}.
\begin{itemize}
\item If an exception is thrown during the assignment of \tcode{std::forward<T>(t)}
to the value contained in \tcode{*this}, the state of the contained value and
\tcode{t} are as defined by the exception safety guarantee of the assignment
expression; \tcode{valueless_by_exception()} will be \tcode{false}.
\item If an exception is thrown during the initialization of the contained value,
the \tcode{variant} object might not hold a value.
\end{itemize}
\end{itemdescr}

\rSec3[variant.mod]{Modifiers}

\indexlibrarymember{emplace}{variant}%
\begin{itemdecl}
template <class T, class... Args> void emplace(Args&&... args);
\end{itemdecl}

\begin{itemdescr}
\pnum
\effects
Equivalent to \tcode{emplace<$I$>(std::forward<Args>(args)...)} where $I$
is the zero-based index of \tcode{T} in \tcode{Types...}.

\pnum
\remarks
This function shall not participate in overload resolution unless
\tcode{is_constructible_v<T, Args...>} is \tcode{true}, and \tcode{T} occurs
exactly once in \tcode{Types...}.
\end{itemdescr}

\indexlibrarymember{emplace}{variant}%
\begin{itemdecl}
template <class T, class U, class... Args> void emplace(initializer_list<U> il, Args&&... args);
\end{itemdecl}

\begin{itemdescr}
\pnum
\effects
Equivalent to \tcode{emplace<$I$>(il, std::forward<Args>(args)...)} where
$I$ is the zero-based index of \tcode{T} in \tcode{Types...}.

\pnum
\remarks
This function shall not participate in overload resolution unless
\tcode{is_constructible_v<T, initializer_list<U>\&, Args...>} is \tcode{true},
and \tcode{T} occurs exactly once in \tcode{Types...}.
\end{itemdescr}

\indexlibrarymember{emplace}{variant}%
\begin{itemdecl}
template <size_t I, class... Args> void emplace(Args&&... args);
\end{itemdecl}

\begin{itemdescr}
\pnum
\requires
\tcode{I < sizeof...(Types)}.

\pnum
\effects
Destroys the currently contained value if \tcode{valueless_by_exception()}
is \tcode{false}. Then direct-initializes the contained value as if
constructing a value of type $T_I$ with the arguments
\tcode{std::forward<Ar\-gs>(args)...}.

\pnum
\postconditions
\tcode{index()} is \tcode{I}.

\pnum
\throws
Any exception thrown during the initialization of the contained value.

\pnum
\remarks
This function shall not participate in overload resolution unless
\tcode{is_constructible_v<$T_I$, Args...>} is \tcode{true}.
If an exception is thrown during the initialization of the contained value,
the \tcode{variant} might not hold a value.
\end{itemdescr}

\indexlibrarymember{emplace}{variant}%
\begin{itemdecl}
template <size_t I, class U, class... Args> void emplace(initializer_list<U> il, Args&&... args);
\end{itemdecl}

\begin{itemdescr}
\pnum
\requires
\tcode{I < sizeof...(Types)}.

\pnum
\effects
Destroys the currently contained value if \tcode{valueless_by_exception()}
is \tcode{false}. Then direct-initializes the contained value as if
constructing a value of type $T_I$ with the arguments
\tcode{il, std::forward<Args>(args)...}.

\pnum
\postconditions
\tcode{index()} is \tcode{I}.

\pnum
\throws
Any exception thrown during the initialization of the contained value.

\pnum
\remarks
This function shall not participate in overload resolution unless
\tcode{is_constructible_v<$T_I$, initializer_list<U>\&, Args...>} is \tcode{true}.
If an exception is thrown during the initialization of the contained value,
the \tcode{variant} might not hold a value.
\end{itemdescr}

\rSec3[variant.status]{Value status}

\indexlibrarymember{valueless_by_exception}{variant}%
\begin{itemdecl}
constexpr bool valueless_by_exception() const noexcept;
\end{itemdecl}

\begin{itemdescr}
\pnum
\effects
Returns \tcode{false} if and only if the \tcode{variant} holds a value.

\pnum
\begin{note}
A \tcode{variant} might not hold a value if an exception is thrown during a
type-changing assignment or emplacement. The latter means that even a
\tcode{variant<float, int>} can become \tcode{valueless_by_exception()}, for
instance by
\begin{codeblock}
struct S { operator int() { throw 42; }};
variant<float, int> v{12.f};
v.emplace<1>(S());
\end{codeblock}
\end{note}
\end{itemdescr}

\indexlibrarymember{index}{variant}%
\begin{itemdecl}
constexpr size_t index() const noexcept;
\end{itemdecl}

\begin{itemdescr}
\pnum
\effects
If \tcode{valueless_by_exception()} is \tcode{true}, returns \tcode{variant_npos}.
Otherwise, returns the zero-based index of the alternative of the contained value.
\end{itemdescr}

\rSec3[variant.swap]{Swap}

\indexlibrarymember{swap}{variant}%
\begin{itemdecl}
void swap(variant& rhs) noexcept(@\seebelow@);
\end{itemdecl}

\begin{itemdescr}
\pnum
\effects
\begin{itemize}
\item
if \tcode{valueless_by_exception() \&\& rhs.valueless_by_exception()} no effect. Otherwise,
\item
if \tcode{index() == rhs.index()}, calls \tcode{swap(get<$i$>(*this), get<$i$>(rhs))} where $i$ is \tcode{index()}. Otherwise,
\item
exchanges values of \tcode{rhs} and \tcode{*this}.
\end{itemize}

\pnum
\throws
Any exception thrown by \tcode{swap(get<$i$>(*this), get<$i$>(rhs))} with $i$
being \tcode{index()} and \tcode{variant}'s move constructor and move assignment operator.

\pnum
\remarks
This function shall not participate in overload resolution unless
\tcode{is_swappable_v<$T_i$>} is \tcode{true} for all $i$.
If an exception is thrown during the call to function \tcode{swap(get<$i$>(*this), get<$i$>(rhs))},
the states of the contained values of \tcode{*this} and of \tcode{rhs} are
determined by the exception safety guarantee of \tcode{swap} for lvalues of
$T_i$ with $i$ being \tcode{index()}.
If an exception is thrown during the exchange of the values of \tcode{*this}
and \tcode{rhs}, the states of the values of \tcode{*this} and of \tcode{rhs}
are determined by the exception safety guarantee of \tcode{variant}'s move
constructor and move assignment operator.
The expression inside \tcode{noexcept} is equivalent to the logical AND of
\tcode{is_nothrow_move_constructible_v<$T_i$> \&\& is_nothrow_swappable_v<$T_i$>} for all $i$.
\end{itemdescr}

\rSec2[variant.helper]{\tcode{variant} helper classes}

\indexlibrary{\idxcode{variant_size}}%
\begin{itemdecl}
template <class T> struct variant_size;
\end{itemdecl}

\begin{itemdescr}
\pnum
\remarks
All specializations of \tcode{variant_size<T>} shall meet the
\tcode{UnaryTypeTrait} requirements~(\ref{meta.rqmts}) with a \tcode{BaseCharacteristic} of \tcode{integral_constant<size_t, N>} for some \tcode{N}.
\end{itemdescr}

\indexlibrary{\idxcode{variant_size}}%
\begin{itemdecl}
template <class T> class variant_size<const T>;
template <class T> class variant_size<volatile T>;
template <class T> class variant_size<const volatile T>;
\end{itemdecl}

\begin{itemdescr}
\pnum
Let \tcode{VS} denote \tcode{variant_size<T>} of the cv-unqualified
type \tcode{T}. Then each of the three templates shall meet the
\tcode{UnaryTypeTrait} requirements~(\ref{meta.rqmts}) with a
\tcode{BaseCharacteristic} of \tcode{integral_constant<size_t, VS::value>}.
\end{itemdescr}

\indexlibrary{\idxcode{variant_size}}%
\begin{itemdecl}
template <class... Types>
struct variant_size<variant<Types...>>
  : integral_constant<size_t, sizeof...(Types)> { };
\end{itemdecl}
% No itemdescr needed for variant_size<variant<Types...>>

\indexlibrary{\idxcode{variant_alternative}}%
\begin{itemdecl}
template <size_t I, class T> class variant_alternative<I, const T>;
template <size_t I, class T> class variant_alternative<I, volatile T>;
template <size_t I, class T> class variant_alternative<I, const volatile T>;
\end{itemdecl}

\begin{itemdescr}
\pnum
Let \tcode{VA} denote \tcode{variant_alternative<I, T>} of the
cv-unqualified type \tcode{T}. Then each of the three templates shall
meet the \tcode{TransformationTrait} requirements~(\ref{meta.rqmts}) with a
member typedef \tcode{type} that names the following type:
\begin{itemize}
\item for the first specialization, \tcode{add_const_t<VA::type>},
\item for the second specialization, \tcode{add_volatile_t<VA::type>}, and
\item for the third specialization, \tcode{add_cv_t<VA::type>}.
\end{itemize}
\end{itemdescr}

\indexlibrary{\idxcode{variant_alternative}}%
\begin{itemdecl}
variant_alternative<I, variant<Types...>>::type
\end{itemdecl}

\begin{itemdescr}
\pnum
\requires \tcode{I < sizeof...(Types)}.

\pnum
\textit{Value:} The type $T_I$.
\end{itemdescr}

\rSec2[variant.get]{Value access}

\indexlibrary{\idxcode{holds_alternative}}
\indexlibrary{\idxcode{variant}!\idxcode{holds_alternative}}
\begin{itemdecl}
template <class T, class... Types>
  constexpr bool holds_alternative(const variant<Types...>& v) noexcept;
\end{itemdecl}

\begin{itemdescr}
\pnum
\requires
The type \tcode{T} occurs exactly once in \tcode{Types...}.
Otherwise, the program is ill-formed.

\pnum
\returns
\tcode{true} if \tcode{index()} is equal to the zero-based index of \tcode{T} in \tcode{Types...}.
\end{itemdescr}

\indexlibrarymember{get}{variant}%
\begin{itemdecl}
template <size_t I, class... Types>
  constexpr variant_alternative_t<I, variant<Types...>>& get(variant<Types...>& v);
template <size_t I, class... Types>
  constexpr variant_alternative_t<I, variant<Types...>>&& get(variant<Types...>&& v);
template <size_t I, class... Types>
  constexpr variant_alternative_t<I, variant<Types...>> const& get(const variant<Types...>& v);
template <size_t I, class... Types>
  constexpr variant_alternative_t<I, variant<Types...>> const&& get(const variant<Types...>&& v);
\end{itemdecl}

\begin{itemdescr}
\pnum
\requires
\tcode{I < sizeof...(Types)}.
Otherwise the program is ill-formed.

\pnum
\effects
If \tcode{v.index()} is \tcode{I}, returns a reference to the object stored in
the \tcode{variant}. Otherwise, throws an exception of type \tcode{bad_variant_access}.
\end{itemdescr}

\indexlibrarymember{get}{variant}%
\begin{itemdecl}
template <class T, class... Types> constexpr T& get(variant<Types...>& v);
template <class T, class... Types> constexpr T&& get(variant<Types...>&& v);
template <class T, class... Types> constexpr const T& get(const variant<Types...>& v);
template <class T, class... Types> constexpr const T&& get(const variant<Types...>&& v);
\end{itemdecl}

\begin{itemdescr}
\pnum
\requires
The type \tcode{T} occurs exactly once in \tcode{Types...}.
Otherwise, the program is ill-formed.

\pnum
\effects
If \tcode{v} holds a value of type \tcode{T}, returns a reference to that value.
Otherwise, throws an exception of type \tcode{bad_variant_access}.
\end{itemdescr}

\indexlibrary{\idxcode{get_if}}%
\indexlibrary{\idxcode{variant}!\idxcode{get_if}}%
\begin{itemdecl}
template <size_t I, class... Types>
  constexpr add_pointer_t<variant_alternative_t<I, variant<Types...>>>
  get_if(variant<Types...>* v) noexcept;
template <size_t I, class... Types>
  constexpr add_pointer_t<const variant_alternative_t<I, variant<Types...>>>
  get_if(const variant<Types...>* v) noexcept;
\end{itemdecl}

\begin{itemdescr}
\pnum
\requires
\tcode{I < sizeof...(Types)}, and $T_I$ is not (possibly cv-qualified) \tcode{void}.
Otherwise the program is ill-formed.

\pnum
\returns
A pointer to the value stored in the \tcode{variant}, if \tcode{v != nullptr}
and \tcode{v->index() == I}. Otherwise, returns \tcode{nullptr}.
\end{itemdescr}

\indexlibrary{\idxcode{get_if}}%
\indexlibrary{\idxcode{variant}!\idxcode{get_if}}%
\begin{itemdecl}
template <class T, class... Types>
  constexpr add_pointer_t<T> get_if(variant<Types...>* v) noexcept;
template <class T, class... Types>
  constexpr add_pointer_t<const T> get_if(const variant<Types...>* v) noexcept;
\end{itemdecl}

\begin{itemdescr}
\pnum
\requires
The type \tcode{T} occurs exactly once in \tcode{Types...}, and \tcode{T} is
not (possibly cv-qualified) \tcode{void}. Otherwise, the program is ill-formed.

\pnum
\effects
Equivalent to: \tcode{return get_if<$i$>(v);} with $i$ being the zero-based
index of \tcode{T} in \tcode{Types...}.
\end{itemdescr}

\rSec2[variant.relops]{Relational operators}

\indexlibrarymember{operator==}{variant}%
\begin{itemdecl}
template <class... Types>
  constexpr bool operator==(const variant<Types...>& v, const variant<Types...>& w);
\end{itemdecl}

\begin{itemdescr}
\pnum
\requires
\tcode{get<$i$>(v) == get<$i$>(w)} is a valid expression returning a type that is
convertible to \tcode{bool}, for all $i$.

\pnum
\returns
If \tcode{v.index() != w.index()}, \tcode{false};
otherwise if \tcode{v.valueless_by_exception()}, \tcode{true};
otherwise \tcode{get<$i$>(v) == get<$i$>(w)} with $i$ being \tcode{v.index()}.
\end{itemdescr}

\indexlibrarymember{operator"!=}{variant}%
\begin{itemdecl}
template <class... Types>
  constexpr bool operator!=(const variant<Types...>& v, const variant<Types...>& w);
\end{itemdecl}

\begin{itemdescr}
\pnum
\requires
\tcode{get<$i$>(v) != get<$i$>(w)} is a valid expression returning a type that is
convertible to \tcode{bool}, for all $i$.

\pnum
\returns
If \tcode{v.index() != w.index()}, \tcode{true};
otherwise if \tcode{v.valueless_by_exception()}, \tcode{false};
otherwise \tcode{get<$i$>(v) != get<$i$>(w)} with $i$ being \tcode{v.index()}.
\end{itemdescr}

\indexlibrarymember{operator<}{variant}%
\begin{itemdecl}
template <class... Types>
  constexpr bool operator<(const variant<Types...>& v, const variant<Types...>& w);
\end{itemdecl}

\begin{itemdescr}
\pnum
\requires
\tcode{get<$i$>(v) < get<$i$>(w)} is a valid expression returning a type that is
convertible to \tcode{bool}, for all $i$.

\pnum
\returns
If \tcode{w.valueless_by_exception()}, \tcode{false};
otherwise if \tcode{v.valueless_by_exception()}, \tcode{true};
otherwise, if \tcode{v.index() < w.index()}, \tcode{true};
otherwise if \tcode{v.index() > w.index()}, \tcode{false};
otherwise \tcode{get<$i$>(v) < get<$i$>(w)} with $i$ being \tcode{v.index()}.
\end{itemdescr}

\indexlibrarymember{operator>}{variant}%
\begin{itemdecl}
template <class... Types>
  constexpr bool operator>(const variant<Types...>& v, const variant<Types...>& w);
\end{itemdecl}

\begin{itemdescr}
\pnum
\requires
\tcode{get<$i$>(v) > get<$i$>(w)} is a valid expression returning a type that is
convertible to \tcode{bool}, for all $i$.

\pnum
\returns
If \tcode{v.valueless_by_exception()}, \tcode{false};
otherwise if \tcode{w.valueless_by_exception()}, \tcode{true};
otherwise, if \tcode{v.index() > w.index()}, \tcode{true};
otherwise if \tcode{v.index() < w.index()}, \tcode{false};
otherwise \tcode{get<$i$>(v) > get<$i$>(w)} with $i$ being \tcode{v.index()}.
\end{itemdescr}

\indexlibrarymember{operator<=}{variant}%
\begin{itemdecl}
template <class... Types>
  constexpr bool operator<=(const variant<Types...>& v, const variant<Types...>& w);
\end{itemdecl}

\begin{itemdescr}
\pnum
\requires
\tcode{get<$i$>(v) <= get<$i$>(w)} is a valid expression returning a type that is
convertible to \tcode{bool}, for all $i$.

\pnum
\returns
If \tcode{v.valueless_by_exception()}, \tcode{true};
otherwise if \tcode{w.valueless_by_exception()}, \tcode{false};
otherwise, if \tcode{v.index() < w.index()}, \tcode{true};
otherwise if \tcode{v.index() > w.index()}, \tcode{false};
otherwise \tcode{get<$i$>(v) <= get<$i$>(w)} with $i$ being \tcode{v.index()}.
\end{itemdescr}

\indexlibrarymember{operator>=}{variant}%
\begin{itemdecl}
template <class... Types>
  constexpr bool operator>=(const variant<Types...>& v, const variant<Types...>& w);
\end{itemdecl}

\begin{itemdescr}
\pnum
\requires
\tcode{get<$i$>(v) >= get<$i$>(w)} is a valid expression returning a type that is
convertible to \tcode{bool}, for all $i$.

\pnum
\returns
If \tcode{w.valueless_by_exception()}, \tcode{true};
otherwise if \tcode{v.valueless_by_exception()}, \tcode{false};
otherwise, if \tcode{v.index() > w.index()}, \tcode{true};
otherwise if \tcode{v.index() < w.index()}, \tcode{false};
otherwise \tcode{get<$i$>(v) >= get<$i$>(w)} with $i$ being \tcode{v.index()}.
\end{itemdescr}

\rSec2[variant.visit]{Visitation}

\indexlibrary{\idxcode{visit}}%
\indexlibrary{\idxcode{variant}!\idxcode{visit}}%
\begin{itemdecl}
template <class Visitor, class... Variants>
  constexpr @\seebelow@ visit(Visitor&& vis, Variants&&... vars);
\end{itemdecl}

\begin{itemdescr}
\pnum
\requires
The expression in the \effects element shall be a valid expression of the same
 type and value category, for all combinations of alternative types of all
 variants. Otherwise, the program is ill-formed.

\effects
Let \tcode{is...} be \tcode{vars.index()...}. Returns \tcode{\textit{INVOKE}(forward<Visitor>(vis), get<is>(}\brk{}
\tcode{forward<Variants>(vars))...);}.

\pnum
\remarks
The return type is the common type of all possible \tcode{\textit{INVOKE}}
expressions of the \effects element.

\pnum
\throws
\tcode{bad_variant_access} if any \tcode{variant} in \tcode{vars} is \tcode{valueless_by_exception()}.

\complexity
For \tcode{sizeof...(Variants) <= 1}, the invocation of the callable object is
implemented in constant time, i.e. it does not depend on \tcode{sizeof...(Types).}
For \tcode{sizeof...(Variants) > 1}, the invocation of the callable object has
no complexity requirements.
\end{itemdescr}

\indexlibrary{\idxcode{monostate}}%
\rSec2[variant.monostate]{Class \tcode{monostate}}

\begin{itemdecl}
struct monostate{};
\end{itemdecl}

\begin{itemdescr}
The class \tcode{monostate} can serve as a first alternative type for
a \tcode{variant} to make the \tcode{variant} type default constructible.
\end{itemdescr}


\rSec2[variant.monostate.relops]{\tcode{monostate} relational operators}

\indexlibrary{\idxcode{operator<}!\idxcode{monostate}}%
\indexlibrary{\idxcode{operator>}!\idxcode{monostate}}%
\indexlibrary{\idxcode{operator<=}!\idxcode{monostate}}%
\indexlibrary{\idxcode{operator>=}!\idxcode{monostate}}%
\indexlibrary{\idxcode{operator==}!\idxcode{monostate}}%
\indexlibrary{\idxcode{operator"!=}!\idxcode{monostate}}%
\begin{itemdecl}
constexpr bool operator<(monostate, monostate) noexcept { return false; }
constexpr bool operator>(monostate, monostate) noexcept { return false; }
constexpr bool operator<=(monostate, monostate) noexcept { return true; }
constexpr bool operator>=(monostate, monostate) noexcept { return true; }
constexpr bool operator==(monostate, monostate) noexcept { return true; }
constexpr bool operator!=(monostate, monostate) noexcept { return false; }
\end{itemdecl}

\begin{itemdescr}
\pnum
\begin{note} \tcode{monostate} objects have only a single state; they thus always compare equal.\end{note}
\end{itemdescr}


\rSec2[variant.specalg]{Specialized algorithms}

\indexlibrary{\idxcode{swap}!\idxcode{variant}}%
\begin{itemdecl}
template <class... Types> void swap(variant<Types...>& v, variant<Types...>& w) noexcept(@\seebelow@);
\end{itemdecl}

\begin{itemdescr}
\pnum
\effects Equivalent to \tcode{v.swap(w)}.

\pnum
\remarks The expression inside \tcode{noexcept} is equivalent to \tcode{noexcept(v.swap(w))}.
\end{itemdescr}

\indexlibrary{\idxcode{bad_variant_access}}%
\rSec2[variant.bad.access]{Class \tcode{bad_variant_access}}

\begin{codeblock}
class bad_variant_access : public exception {
public:
  bad_variant_access() noexcept;
  virtual const char* what() const noexcept;
};
\end{codeblock}

\pnum
Objects of type \tcode{bad_variant_access} are thrown to report invalid
accesses to the value of a \tcode{variant} object.

\indexlibrary{\idxcode{bad_variant_access}!constructor}%
\begin{itemdecl}
bad_variant_access() noexcept;
\end{itemdecl}

\begin{itemdescr}
\pnum
Constructs a \tcode{bad_variant_access} object.
\end{itemdescr}

\indexlibrarymember{what}{bad_variant_access}%
\begin{itemdecl}
const char* what() const noexcept override;
\end{itemdecl}

\begin{itemdescr}
\pnum
\returns An \impldef{return value of \tcode{bad_variant_access::what}} \ntbs.
\end{itemdescr}

\rSec2[variant.hash]{Hash support}

\indexlibrary{\idxcode{hash}!\idxcode{variant}}%
\begin{itemdecl}
template <class... Types> struct hash<variant<Types...>>;
\end{itemdecl}

\begin{itemdescr}
\pnum
The template specialization \tcode{hash<T>} shall meet the requirements
of class template \tcode{hash}~(\ref{unord.hash}) for all \tcode{T} in \tcode{Types...}.
The template specialization \tcode{hash<variant<Types...>>} shall meet
the requirements of class template \tcode{hash}.
\end{itemdescr}

\indexlibrary{\idxcode{hash}!\idxcode{monostate}}%
\begin{itemdecl}
template <> struct hash<monostate>;
\end{itemdecl}

\begin{itemdescr}
\pnum
The template specialization hash<monostate> shall meet the requirements of class template hash.
\end{itemdescr}


\rSec2[variant.traits]{Allocator-related traits}

\indexlibrary{\idxcode{uses_allocator}!\idxcode{variant}}%
\begin{itemdecl}
template <class... Types, class Alloc>
  struct uses_allocator<variant<Types...>, Alloc> : true_type { };
\end{itemdecl}

\begin{itemdescr}
\pnum
\requires \tcode{Alloc} shall be an Allocator~(\ref{allocator.requirements}).

\pnum
\begin{note}
Specialization of this trait informs other library components
that variant can be constructed with an allocator,
even though it does not have a nested \tcode{allocator_type}.
\end{note}
\end{itemdescr}


\rSec1[any]{Storage for any type}

\pnum
This section describes components that C++ programs may use to perform operations on objects of a discriminated type.

\pnum
\begin{note}
The discriminated type may contain values of different types but does not attempt conversion between them,
i.e. \tcode{5} is held strictly as an \tcode{int} and is not implicitly convertible either to \tcode{"5"} or to \tcode{5.0}.
This indifference to interpretation but awareness of type effectively allows safe, generic containers of single values, with no scope for surprises from ambiguous conversions.
\end{note}

\indexlibrary{\idxhdr{any}}%
\rSec2[any.synop]{Header \tcode{<any>} synopsis}

\begin{codeblock}
namespace std {
  // \ref{any.bad_any_cast}, class \tcode{bad_any_cast}
  class bad_any_cast;

  // \ref{any.class}, class \tcode{any}
  class any;

  // \ref{any.nonmembers}, non-member functions
  void swap(any& x, any& y) noexcept;

  template <class T, class... Args>
    any make_any(Args&& ...args);
  template <class T, class U, class... Args>
    any make_any(initializer_list<U> il, Args&& ...args);

  template<class ValueType>
    ValueType any_cast(const any& operand);
  template<class ValueType>
    ValueType any_cast(any& operand);
  template<class ValueType>
    ValueType any_cast(any&& operand);
  
  template<class ValueType>
    const ValueType* any_cast(const any* operand) noexcept;
  template<class ValueType>
    ValueType* any_cast(any* operand) noexcept;
}
\end{codeblock}

\rSec2[any.bad_any_cast]{Class \tcode{bad_any_cast}}

\indexlibrary{\idxcode{bad_any_cast}}%
\begin{codeblock}
class bad_any_cast : public bad_cast {
public:
  const char* what() const noexcept override;
};
\end{codeblock}

\pnum
Objects of type \tcode{bad_any_cast} are thrown by a failed \tcode{any_cast}~(\ref{any.nonmembers}).

\rSec2[any.class]{Class \tcode{any}}

\begin{codeblock}
class any {
public:
  // \ref{any.cons}, construction and destruction
  constexpr any() noexcept;

  any(const any& other);
  any(any&& other) noexcept;

  template <class ValueType> any(ValueType&& value);

  template <class T, class... Args>
    explicit any(in_place_type_t<T>, Args&&...);
  template <class T, class U, class... Args>
    explicit any(in_place_type_t<T>, initializer_list<U>, Args&&...);

  ~any();

  // \ref{any.assign}, assignments
  any &operator=(const any& rhs);
  any &operator=(any&& rhs) noexcept;

  template <class ValueType> any& operator=(ValueType&& rhs);

  // \ref{any.modifiers}, modifiers
  template <class T, class... Args>
    void emplace(Args&& ...);
  template <class T, class U, class... Args>
    void emplace(initializer_list<U>, Args&&...);
  void reset() noexcept;
  void swap(any& rhs) noexcept;

  // \ref{any.observers}, observers
  bool has_value() const noexcept;
  const type_info& type() const noexcept;
};
\end{codeblock}

\pnum
\begin{note} \tcode{any} is not a literal type. \end{note}

\pnum
An object of class \tcode{any} stores an instance of any type that satisfies the constructor requirements or it has no value,
and this is referred to as the \defn{state} of the class \tcode{any} object.
The stored instance is called the \defn{contained object}.
Two states are equivalent if either they both have no value, or both have a value and the contained objects are equivalent.

\pnum
The non-member \tcode{any_cast} functions provide type-safe access to the contained object.

\pnum
Implementations should avoid the use of dynamically allocated memory for a small contained object.
\begin{example}
where the object constructed is holding only an \tcode{int}.
\end{example}
Such small-object optimization shall only be applied to types \tcode{T} for which
\tcode{is_nothrow_move_constructible_v<T>} is \tcode{true}.

\rSec3[any.cons]{Construction and destruction}

\indexlibrary{\idxcode{any}!constructor}%
\begin{itemdecl}
constexpr any() noexcept;
\end{itemdecl}

\begin{itemdescr}
\pnum
\postconditions
\tcode{has_value()} is \tcode{false}.
\end{itemdescr}

\indexlibrary{\idxcode{any}!constructor}%
\begin{itemdecl}
any(const any& other);
\end{itemdecl}

\begin{itemdescr}
\pnum
\effects
Constructs an object of type \tcode{any} with an equivalent state as \tcode{other}.

\pnum
\throws
Any exceptions arising from calling the selected constructor of the contained object.
\end{itemdescr}

\indexlibrary{\idxcode{any}!constructor}%
\begin{itemdecl}
any(any&& other) noexcept;
\end{itemdecl}

\begin{itemdescr}
\pnum
\effects
Constructs an object of type \tcode{any} with a state equivalent to the original state of \tcode{other}.

\pnum
\postconditions
\tcode{other} is left in a valid but otherwise unspecified state.
\end{itemdescr}

\indexlibrary{\idxcode{any}!constructor}%
\begin{itemdecl}
template<class ValueType>
any(ValueType&& value);
\end{itemdecl}

\begin{itemdescr}
\pnum
Let \tcode{T} be equal to \tcode{decay_t<ValueType>}.

\pnum
\requires
\tcode{T} shall satisfy the \tcode{CopyConstructible} requirements.
If \tcode{is_copy_constructible_v<T>} is \tcode{false}, the program is ill-formed.

\pnum
\effects
Constructs an object of type \tcode{any} that contains an object of type \tcode{T} direct-initialized with \tcode{std::forward<ValueType>(value)}.

\pnum
\remarks
This constructor shall not participate in overload resolution if \tcode{decay_t<ValueType>} is the same type as \tcode{any}.

\pnum
\throws
Any exception thrown by the selected constructor of \tcode{T}.
\end{itemdescr}

\indexlibrary{\idxcode{any}!constructor}%
\begin{itemdecl}
template <class T, class... Args>
  explicit any(in_place_type_t<T>, Args&&... args);
\end{itemdecl}

\begin{itemdescr}
\pnum
\requires \tcode{is_constructible_v<T, Args...>} is \tcode{true}.

\pnum
\effects Initializes the contained value as if direct-non-list-initializing an object of
type \tcode{T} with the arguments \tcode{std::forward<Args>(args)...}.

\pnum
\postconditions \tcode{*this} contains a value of type \tcode{T}.

\pnum
\throws Any exception thrown by the selected constructor of \tcode{T}.
\end{itemdescr}

\indexlibrary{\idxcode{any}!constructor}%
\begin{itemdecl}
template <class T, class U, class... Args>
  explicit any(in_place_type_t<T>, initializer_list<U> il, Args&&... args);
\end{itemdecl}

\begin{itemdescr}
\pnum
\requires \tcode{is_constructible_v<T, initializer_list<U>\&, Args...>} is \tcode{true}.

\pnum
\effects Initializes the contained value as if direct-non-list-initializing an object of
type \tcode{T} with the arguments \tcode{il, std::forward<Args>(args)...}.

\pnum
\postconditions \tcode{*this} contains a value.

\pnum
\throws Any exception thrown by the selected constructor of \tcode{T}.

\pnum
\remarks The function shall not participate in overload resolution
unless \tcode{is_constructible_v<T, initializer_list<U>\&, Args...>} is \tcode{true}.
\end{itemdescr}

\indexlibrary{\idxcode{any}!destructor}
\begin{itemdecl}
~any();
\end{itemdecl}

\begin{itemdescr}
\pnum
\effects
As if by \tcode{reset()}.
\end{itemdescr}

\rSec3[any.assign]{Assignment}

\indexlibrarymember{operator=}{any}%
\begin{itemdecl}
any& operator=(const any& rhs);
\end{itemdecl}

\begin{itemdescr}
\pnum
\effects
As if by \tcode{any(rhs).swap(*this)}.
No effects if an exception is thrown.

\pnum
\returns
\tcode{*this}.

\pnum
\throws
Any exceptions arising from the copy constructor of the contained object.
\end{itemdescr}

\indexlibrarymember{operator=}{any}%
\begin{itemdecl}
any& operator=(any&& rhs) noexcept;
\end{itemdecl}

\begin{itemdescr}
\pnum
\effects
As if by \tcode{any(std::move(rhs)).swap(*this)}.

\pnum
\returns
\tcode{*this}.

\pnum
\postconditions
The state of \tcode{*this} is equivalent to the original state of \tcode{rhs}
and \tcode{rhs} is left in a valid but otherwise unspecified state.
\end{itemdescr}

\indexlibrarymember{operator=}{any}%
\begin{itemdecl}
template<class ValueType>
any& operator=(ValueType&& rhs);
\end{itemdecl}

\begin{itemdescr}
\pnum
Let \tcode{T} be equal to \tcode{decay_t<ValueType>}.

\pnum
\requires
\tcode{T} shall satisfy the \tcode{CopyConstructible} requirements.
If \tcode{is_copy_constructible_v<T>} is \tcode{false}, the program is ill-formed.

\pnum
\effects
Constructs an object \tcode{tmp} of type \tcode{any} that contains an object of type \tcode{T} direct-initialized with \tcode{std::forward<ValueType>(rhs)}, and \tcode{tmp.swap(*this)}.
No effects if an exception is thrown.

\pnum
\returns
\tcode{*this}.

\pnum
\remarks
This operator shall not participate in overload resolution if \tcode{decay_t<ValueType>} is the same type as \tcode{any}.

\pnum
\throws
Any exception thrown by the selected constructor of \tcode{T}.
\end{itemdescr}

\rSec3[any.modifiers]{Modifiers}

\indexlibrarymember{emplace}{any}%
\begin{itemdecl}
template <class T, class... Args>
  void emplace(Args&&... args);
\end{itemdecl}

\begin{itemdescr}
\pnum
\requires \tcode{is_constructible_v<T, Args...>} is \tcode{true}.

\pnum
\effects Calls \tcode{reset()}.
Then initializes the contained value as if direct-non-list-initializing
an object of type \tcode{T} with the arguments \tcode{std::forward<Args>(args)...}.

\pnum
\postconditions \tcode{*this} contains a value.

\pnum
\throws Any exception thrown by the selected constructor of \tcode{T}.

\pnum
\remarks If an exception is thrown during the call to \tcode{T}'s constructor,
\tcode{*this} does not contain a value, and any previously contained object
has been destroyed.
\end{itemdescr}

\indexlibrarymember{emplace}{any}%
\begin{itemdecl}
template <class T, class U, class... Args>
  void emplace(initializer_list<U> il, Args&&... args);
\end{itemdecl}

\begin{itemdescr}
\pnum
\effects Calls \tcode{reset()}. Then initializes the contained value
as if direct-non-list-initializing an object of type \tcode{T} with the arguments
\tcode{il, std::forward<Args> (args)...}.

\pnum
\postconditions \tcode{*this} contains a value.

\pnum
\throws Any exception thrown by the selected constructor of \tcode{T}.

\pnum
\remarks If an exception is thrown during the call to \tcode{T}'s constructor,
\tcode{*this} does not contain a value, and any previously contained object
has been destroyed.
The function shall not participate in overload resolution unless
\tcode{is_constructible_v<T, initializer_list<U>\&, Args...>} is \tcode{true}.
\end{itemdescr}

\indexlibrarymember{reset}{any}%
\begin{itemdecl}
void reset() noexcept;
\end{itemdecl}

\begin{itemdescr}
\pnum
\effects
If \tcode{has_value()} is \tcode{true}, destroys the contained object.

\pnum
\postconditions
\tcode{has_value()} is \tcode{false}.
\end{itemdescr}

\indexlibrarymember{swap}{any}%
\begin{itemdecl}
void swap(any& rhs) noexcept;
\end{itemdecl}

\begin{itemdescr}

\pnum
\effects
Exchanges the states of \tcode{*this} and \tcode{rhs}.
\end{itemdescr}

\rSec3[any.observers]{Observers}

\indexlibrarymember{has_value}{any}%
\begin{itemdecl}
bool has_value() const noexcept;
\end{itemdecl}

\begin{itemdescr}
\pnum
\returns
\tcode{true} if \tcode{*this} contains an object, otherwise \tcode{false}.
\end{itemdescr}

\indexlibrarymember{type}{any}%
\begin{itemdecl}
const type_info& type() const noexcept;
\end{itemdecl}

\begin{itemdescr}
\pnum
\returns
\tcode{typeid(T)} if \tcode{*this} has a contained object of type \tcode{T},
otherwise \tcode{typeid(void)}.

\pnum
\begin{note}
Useful for querying against types known either at compile time or only at runtime.
\end{note}
\end{itemdescr}

\rSec2[any.nonmembers]{Non-member functions}

\indexlibrary{\idxcode{swap}!\idxcode{any}}%
\begin{itemdecl}
void swap(any& x, any& y) noexcept;
\end{itemdecl}

\begin{itemdescr}
\pnum
\effects
As if by \tcode{x.swap(y)}.
\end{itemdescr}

\indexlibrary{\idxcode{make_any}}%
\begin{itemdecl}
template <class T, class... Args>
  any make_any(Args&& ...args);
\end{itemdecl}

\begin{itemdescr}
\pnum
\effects
Equivalent to: \tcode{return any(in_place<T>, std::forward<Args>(args)...);}
\end{itemdescr}

\indexlibrary{\idxcode{make_any}}%
\begin{itemdecl}
template <class T, class U, class... Args>
  any make_any(initializer_list<U> il, Args&& ...args);
\end{itemdecl}

\begin{itemdescr}
\pnum
\effects
Equivalent to: \tcode{return any(in_place<T>, il, std::forward<Args>(args)...);}
\end{itemdescr}

\indexlibrary{\idxcode{any_cast}}%
\begin{itemdecl}
template<class ValueType>
  ValueType any_cast(const any& operand);
template<class ValueType>
  ValueType any_cast(any& operand);
template<class ValueType>
  ValueType any_cast(any&& operand);
\end{itemdecl}

\begin{itemdescr}
\pnum
\requires
\tcode{is_reference_v<ValueType>} is \tcode{true} or \tcode{is_copy_constructible_v<ValueType>} is \tcode{true}.
Otherwise the program is ill-formed.

\pnum
\returns
For the first form, \tcode{*any_cast<add_const_t<remove_reference_t<ValueType>>>(\&operand)}.
For the second and third forms, \tcode{*any_cast<remove_reference_t<ValueType>>(\&operand)}.

\pnum
\throws
\tcode{bad_any_cast} if \tcode{operand.type() != typeid(remove_reference_t<ValueType>)}.

\pnum
\begin{example}
\begin{codeblock}
any x(5);                                   // \tcode{x} holds \tcode{int}
assert(any_cast<int>(x) == 5);              // cast to value
any_cast<int&>(x) = 10;                     // cast to reference
assert(any_cast<int>(x) == 10);

x = "Meow";                                 // \tcode{x} holds \tcode{const char*}
assert(strcmp(any_cast<const char*>(x), "Meow") == 0);
any_cast<const char*&>(x) = "Harry";
assert(strcmp(any_cast<const char*>(x), "Harry") == 0);

x = string("Meow");                         // \tcode{x} holds \tcode{string}
string s, s2("Jane");
s = move(any_cast<string&>(x));             // move from \tcode{any}
assert(s == "Meow");
any_cast<string&>(x) = move(s2);            // move to \tcode{any}
assert(any_cast<const string&>(x) == "Jane");

string cat("Meow");
const any y(cat);                           // \tcode{const y} holds \tcode{string}
assert(any_cast<const string&>(y) == cat);

any_cast<string&>(y);                       // error; cannot
                                            // \tcode{any_cast} away const
\end{codeblock}
\end{example}
\end{itemdescr}

\indexlibrary{\idxcode{any_cast}}%
\begin{itemdecl}
template<class ValueType>
  const ValueType* any_cast(const any* operand) noexcept;
template<class ValueType>
  ValueType* any_cast(any* operand) noexcept;
\end{itemdecl}

\begin{itemdescr}
\pnum
\returns
If \tcode{operand != nullptr \&\& operand->type() == typeid(ValueType)},
a pointer to the object contained by \tcode{operand};
otherwise, \tcode{nullptr}.

\pnum
\begin{example}
\begin{codeblock}
bool is_string(const any& operand) {
  return any_cast<string>(&operand) != nullptr;
}
\end{codeblock}
\end{example}
\end{itemdescr}

\rSec1[template.bitset]{Class template \tcode{bitset}}%
\indexlibrary{\idxcode{bitset}}%

\synopsis{Header \tcode{<bitset>} synopsis}%
\indexlibrary{\idxhdr{bitset}}%

\begin{codeblock}
#include <string>
#include <iosfwd>               // for \tcode{istream}, \tcode{ostream}
namespace std {
  template <size_t N> class bitset;

  // \ref{bitset.operators} bitset operators:
  template <size_t N>
    bitset<N> operator&(const bitset<N>&, const bitset<N>&) noexcept;
  template <size_t N>
    bitset<N> operator|(const bitset<N>&, const bitset<N>&) noexcept;
  template <size_t N>
    bitset<N> operator^(const bitset<N>&, const bitset<N>&) noexcept;
  template <class charT, class traits, size_t N>
    basic_istream<charT, traits>&
    operator>>(basic_istream<charT, traits>& is, bitset<N>& x);
  template <class charT, class traits, size_t N>
    basic_ostream<charT, traits>&
    operator<<(basic_ostream<charT, traits>& os, const bitset<N>& x);
}
\end{codeblock}

\pnum
The header
\tcode{<bitset>}
defines a
class template
and several related functions for representing
and manipulating fixed-size sequences of bits.

\indexlibrary{\idxcode{bitset}}%
\begin{codeblock}
namespace std {
  template<size_t N> class bitset {
  public:
    // bit reference:
    class reference {
      friend class bitset;
      reference() noexcept;
    public:
     ~reference() noexcept;
      reference& operator=(bool x) noexcept;             // for \tcode{b[i] = x;}
      reference& operator=(const reference&) noexcept;   // for \tcode{b[i] = b[j];}
      bool operator~() const noexcept;                   // flips the bit
      operator bool() const noexcept;                    // for \tcode{x = b[i];}
      reference& flip() noexcept;                        // for \tcode{b[i].flip();}
    };

    // \ref{bitset.cons} constructors:
    constexpr bitset() noexcept;
    constexpr bitset(unsigned long long val) noexcept;
    template<class charT, class traits, class Allocator>
      explicit bitset(
        const basic_string<charT, traits, Allocator>& str,
        typename basic_string<charT, traits, Allocator>::size_type pos = 0,
        typename basic_string<charT, traits, Allocator>::size_type n =
          basic_string<charT, traits, Allocator>::npos,
          charT zero = charT('0'), charT one = charT('1'));
    template <class charT>
      explicit bitset(
        const charT* str,
        typename basic_string<charT>::size_type n = basic_string<charT>::npos,
        charT zero = charT('0'), charT one = charT('1'));

    // \ref{bitset.members} bitset operations:
    bitset<N>& operator&=(const bitset<N>& rhs) noexcept;
    bitset<N>& operator|=(const bitset<N>& rhs) noexcept;
    bitset<N>& operator^=(const bitset<N>& rhs) noexcept;
    bitset<N>& operator<<=(size_t pos) noexcept;
    bitset<N>& operator>>=(size_t pos) noexcept;
    bitset<N>& set() noexcept;
    bitset<N>& set(size_t pos, bool val = true);
    bitset<N>& reset() noexcept;
    bitset<N>& reset(size_t pos);
    bitset<N>  operator~() const noexcept;
    bitset<N>& flip() noexcept;
    bitset<N>& flip(size_t pos);

    // element access:
    constexpr bool operator[](size_t pos) const;       // for \tcode{b[i];}
    reference operator[](size_t pos);                  // for \tcode{b[i];}

    unsigned long to_ulong() const;
    unsigned long long to_ullong() const;
    template <class charT = char,
        class traits = char_traits<charT>,
        class Allocator = allocator<charT>>
      basic_string<charT, traits, Allocator>
      to_string(charT zero = charT('0'), charT one = charT('1')) const;
    size_t count() const noexcept;
    constexpr size_t size() const noexcept;
    bool operator==(const bitset<N>& rhs) const noexcept;
    bool operator!=(const bitset<N>& rhs) const noexcept;
    bool test(size_t pos) const;
    bool all() const noexcept;
    bool any() const noexcept;
    bool none() const noexcept;
    bitset<N> operator<<(size_t pos) const noexcept;
    bitset<N> operator>>(size_t pos) const noexcept;
  };

  // \ref{bitset.hash} hash support
  template <class T> struct hash;
  template <size_t N> struct hash<bitset<N>>;
}
\end{codeblock}

\pnum
The class template
\tcode{bitset<N>}%
describes an object that can store a sequence consisting of a fixed number of
bits, \tcode{N}.

\pnum
Each bit represents either the value zero (reset) or one (set).
To
\term{toggle}
a bit is to change the value zero to one, or the value one to
zero.
Each bit has a non-negative position \tcode{pos}.
When converting
between an object of class
\tcode{bitset<N>}
and a value of some
integral type, bit position \tcode{pos} corresponds to the
\term{bit value}
\tcode{1 \shl pos}.
The integral value corresponding to two
or more bits is the sum of their bit values.

\pnum
The functions described in this subclause can report three kinds of
errors, each associated with a distinct exception:

\begin{itemize}
\item
an
\term{invalid-argument}
error is associated with exceptions of type
\tcode{invalid_argument}~(\ref{invalid.argument});
\indexlibrary{\idxcode{invalid_argument}}%
\item
an
\term{out-of-range}
error is associated with exceptions of type
\tcode{out_of_range}~(\ref{out.of.range});
\indexlibrary{\idxcode{out_of_range}}%
\item
an
\term{overflow}
error is associated with exceptions of type
\tcode{overflow_error}~(\ref{overflow.error}).
\indexlibrary{\idxcode{overflow_error}}%
\end{itemize}

\rSec2[bitset.cons]{\tcode{bitset} constructors}

\indexlibrary{\idxcode{bitset}!constructor}%
\begin{itemdecl}
constexpr bitset() noexcept;
\end{itemdecl}

\begin{itemdescr}
\pnum
\effects
Constructs an object of class
\tcode{bitset<N>},
initializing all bits to zero.
\end{itemdescr}

\indexlibrary{\idxcode{bitset}!constructor}%
\begin{itemdecl}
constexpr bitset(unsigned long long val) noexcept;
\end{itemdecl}

\begin{itemdescr}
\pnum
\effects
Constructs an object of class
\tcode{bitset<N>},
initializing the first \tcode{M} bit positions to the corresponding bit
values in \tcode{val}.
\tcode{M} is the smaller of \tcode{N} and the number of bits in the value
representation~(\ref{basic.types}) of \tcode{unsigned long long}.
If \tcode{M < N}, the remaining bit positions are initialized to zero.
\end{itemdescr}

\indexlibrary{\idxcode{bitset}!constructor}%
\begin{itemdecl}
template <class charT, class traits, class Allocator>
explicit
bitset(const basic_string<charT, traits, Allocator>& str,
       typename basic_string<charT, traits, Allocator>::size_type pos = 0,
       typename basic_string<charT, traits, Allocator>::size_type n =
         basic_string<charT, traits, Allocator>::npos,
         charT zero = charT('0'), charT one = charT('1'));
\end{itemdecl}

\begin{itemdescr}
\pnum
\throws
\tcode{out_of_range}
if
\tcode{pos > str.size()}
or \tcode{invalid_argument} if an invalid character is found (see below).%
\indexlibrary{\idxcode{out_of_range}}

\pnum
\effects
Determines the effective length
\tcode{rlen} of the initializing string as the smaller of
\tcode{n} and
\tcode{str.size() - pos}.

The function then throws%
\indexlibrary{\idxcode{invalid_argument}}
\tcode{invalid_argument}
if any of the \tcode{rlen}
characters in \tcode{str} beginning at position \tcode{pos} is
other than \tcode{zero} or \tcode{one}. The function uses \tcode{traits::eq()}
to compare the character values.

Otherwise, the function constructs an object of class
\tcode{bitset<N>},
initializing the first \tcode{M} bit
positions to values determined from the corresponding characters in the string
\tcode{str}.
\tcode{M} is the smaller of \tcode{N} and \tcode{rlen}.

\pnum
An element of the constructed object has value zero if the
corresponding character in \tcode{str}, beginning at position
\tcode{pos}, is
\tcode{zero}.
Otherwise, the element has the value one.
Character position \tcode{pos + M - 1} corresponds to bit position zero.
Subsequent decreasing character positions correspond to increasing bit positions.

\pnum
If \tcode{M < N}, remaining bit positions are initialized to zero.
\end{itemdescr}

\indexlibrary{\idxcode{bitset}!constructor}%
\begin{itemdecl}
template <class charT>
  explicit bitset(
    const charT* str,
    typename basic_string<charT>::size_type n = basic_string<charT>::npos,
    charT zero = charT('0'), charT one = charT('1'));
\end{itemdecl}

\begin{itemdescr}
\pnum
\effects Constructs an object of class \tcode{bitset<N>} as if by:
\begin{codeblock}
bitset(
  n == basic_string<charT>::npos
    ? basic_string<charT>(str)
    : basic_string<charT>(str, n),
  0, n, zero, one)
\end{codeblock}

\end{itemdescr}


\rSec2[bitset.members]{\tcode{bitset} members}

\indexlibrarymember{operator\&=}{bitset}%
\begin{itemdecl}
bitset<N>& operator&=(const bitset<N>& rhs) noexcept;
\end{itemdecl}

\begin{itemdescr}
\pnum
\effects
Clears each bit in
\tcode{*this}
for which the corresponding bit in \tcode{rhs} is clear, and leaves all other bits unchanged.

\pnum
\returns
\tcode{*this}.
\end{itemdescr}

\indexlibrarymember{operator"|=}{bitset}%
\begin{itemdecl}
bitset<N>& operator|=(const bitset<N>& rhs) noexcept;
\end{itemdecl}

\begin{itemdescr}
\pnum
\effects
Sets each bit in
\tcode{*this}
for which the corresponding bit in \tcode{rhs} is set, and leaves all other bits unchanged.

\pnum
\returns
\tcode{*this}.
\end{itemdescr}

\indexlibrarymember{operator\caret=}{bitset}%
\begin{itemdecl}
bitset<N>& operator^=(const bitset<N>& rhs) noexcept;
\end{itemdecl}

\begin{itemdescr}
\pnum
\effects
Toggles each bit in
\tcode{*this}
for which the corresponding bit in \tcode{rhs} is set, and leaves all other bits unchanged.

\pnum
\returns
\tcode{*this}.
\end{itemdescr}

\indexlibrarymember{operator\shl=}{bitset}%
\begin{itemdecl}
bitset<N>& operator<<=(size_t pos) noexcept;
\end{itemdecl}

\begin{itemdescr}
\pnum
\effects
Replaces each bit at position \tcode{I} in
\tcode{*this}
with a value determined as follows:

\begin{itemize}
\item
If \tcode{I < pos}, the new value is zero;
\item
If \tcode{I >= pos}, the new value is the previous
value of the bit at position \tcode{I - pos}.
\end{itemize}

\pnum
\returns
\tcode{*this}.
\end{itemdescr}

\indexlibrarymember{operator\shr=}{bitset}%
\begin{itemdecl}
bitset<N>& operator>>=(size_t pos) noexcept;
\end{itemdecl}

\begin{itemdescr}
\pnum
\effects
Replaces each bit at position \tcode{I} in
\tcode{*this}
with a value determined as follows:

\begin{itemize}
\item
If \tcode{pos >= N - I}, the new value is zero;
\item
If \tcode{pos < N - I}, the new value is the previous value of the bit at position \tcode{I + pos}.
\end{itemize}

\pnum
\returns
\tcode{*this}.
\end{itemdescr}

\indexlibrarymember{set}{bitset}%
\begin{itemdecl}
bitset<N>& set() noexcept;
\end{itemdecl}

\begin{itemdescr}
\pnum
\effects
Sets all bits in
\tcode{*this}.

\pnum
\returns
\tcode{*this}.
\end{itemdescr}

\indexlibrarymember{set}{bitset}%
\begin{itemdecl}
bitset<N>& set(size_t pos, bool val = true);
\end{itemdecl}

\begin{itemdescr}
\pnum
\throws
\tcode{out_of_range}
if \tcode{pos} does not correspond to a valid bit position.%
\indexlibrary{\idxcode{out_of_range}}

\pnum
\effects
Stores a new value in the bit at position \tcode{pos} in
\tcode{*this}.
If \tcode{val} is nonzero, the stored value is one, otherwise it is zero.

\pnum
\returns
\tcode{*this}.
\end{itemdescr}

\indexlibrarymember{reset}{bitset}%
\begin{itemdecl}
bitset<N>& reset() noexcept;
\end{itemdecl}

\begin{itemdescr}
\pnum
\effects
Resets all bits in
\tcode{*this}.

\pnum
\returns
\tcode{*this}.
\end{itemdescr}

\indexlibrarymember{reset}{bitset}%
\begin{itemdecl}
bitset<N>& reset(size_t pos);
\end{itemdecl}

\begin{itemdescr}
\pnum
\throws
\tcode{out_of_range}
if \tcode{pos} does not correspond to a valid bit position.
\indexlibrary{\idxcode{out_of_range}}%

\pnum
\effects
Resets the bit at position \tcode{pos} in
\tcode{*this}.

\pnum
\returns
\tcode{*this}.
\end{itemdescr}

\indexlibrarymember{operator\~{}}{bitset}%
\begin{itemdecl}
bitset<N> operator~() const noexcept;
\end{itemdecl}

\begin{itemdescr}
\pnum
\effects
Constructs an object \tcode{x} of class
\tcode{bitset<N>}
and initializes it with
\tcode{*this}.

\pnum
\returns
\tcode{x.flip()}.
\end{itemdescr}

\indexlibrarymember{flip}{bitset}%
\begin{itemdecl}
bitset<N>& flip() noexcept;
\end{itemdecl}

\begin{itemdescr}
\pnum
\effects
Toggles all bits in
\tcode{*this}.

\pnum
\returns
\tcode{*this}.
\end{itemdescr}

\indexlibrarymember{flip}{bitset}%
\begin{itemdecl}
bitset<N>& flip(size_t pos);
\end{itemdecl}

\begin{itemdescr}
\pnum
\throws
\tcode{out_of_range}
if \tcode{pos} does not correspond to a valid bit position.%
\indexlibrary{\idxcode{out_of_range}}

\pnum
\effects
Toggles the bit at position \tcode{pos} in
\tcode{*this}.

\pnum
\returns
\tcode{*this}.
\end{itemdescr}

\indexlibrarymember{to_ulong}{bitset}%
\begin{itemdecl}
unsigned long to_ulong() const;
\end{itemdecl}

\begin{itemdescr}
\pnum
\throws
\tcode{overflow_error}%
\indexlibrary{\idxcode{overflow_error}}
if the integral value \tcode{x} corresponding to the bits in
\tcode{*this}
cannot be represented as type
\tcode{unsigned long}.

\pnum
\returns
\tcode{x}.
\end{itemdescr}

\indexlibrarymember{to_ullong}{bitset}%
\begin{itemdecl}
unsigned long long to_ullong() const;
\end{itemdecl}

\begin{itemdescr}
\pnum
\indexlibrary{\idxcode{overflow_error}}%
\throws
\tcode{overflow_error}
if the integral value \tcode{x} corresponding to the bits in
\tcode{*this}
cannot be represented as type
\tcode{unsigned long long}.

\pnum
\returns
\tcode{x}.
\end{itemdescr}

\indexlibrarymember{to_string}{bitset}%
\begin{itemdecl}
template <class charT = char,
    class traits = char_traits<charT>,
    class Allocator = allocator<charT>>
  basic_string<charT, traits, Allocator>
  to_string(charT zero = charT('0'), charT one = charT('1')) const;
\end{itemdecl}

\begin{itemdescr}
\pnum
\effects
Constructs a string object of the appropriate type
and initializes it to a string of length \tcode{N} characters.
Each character is determined by the value of its corresponding bit position in
\tcode{*this}.
Character position \tcode{N - 1} corresponds to bit position zero.
Subsequent decreasing character positions correspond to increasing bit
positions.
Bit value zero becomes the character \tcode{zero},
bit value one becomes the character
\tcode{one}.

\pnum
\returns
The created object.
\end{itemdescr}

\indexlibrarymember{count}{bitset}%
\begin{itemdecl}
size_t count() const noexcept;
\end{itemdecl}

\begin{itemdescr}
\pnum
\returns
A count of the number of bits set in
\tcode{*this}.
\end{itemdescr}

\indexlibrarymember{size}{bitset}%
\begin{itemdecl}
constexpr size_t size() const noexcept;
\end{itemdecl}

\begin{itemdescr}
\pnum
\returns
\tcode{N}.
\end{itemdescr}

\indexlibrarymember{operator==}{bitset}%
\begin{itemdecl}
bool operator==(const bitset<N>& rhs) const noexcept;
\end{itemdecl}

\begin{itemdescr}
\pnum
\returns
\tcode{true} if the value of each bit in
\tcode{*this}
equals the value of the corresponding bit in \tcode{rhs}.
\end{itemdescr}

\indexlibrarymember{operator"!=}{bitset}%
\begin{itemdecl}
bool operator!=(const bitset<N>& rhs) const noexcept;
\end{itemdecl}

\begin{itemdescr}
\pnum
\returns
\tcode{true} if
\tcode{!(*this == rhs)}.
\end{itemdescr}

\indexlibrarymember{test}{bitset}%
\begin{itemdecl}
bool test(size_t pos) const;
\end{itemdecl}

\begin{itemdescr}
\pnum
\throws
\tcode{out_of_range}
if \tcode{pos} does not correspond to a valid bit position.%
\indexlibrary{\idxcode{out_of_range}}

\pnum
\returns
\tcode{true}
if the bit at position \tcode{pos}
in
\tcode{*this}
has the value one.
\end{itemdescr}

\indexlibrarymember{all}{bitset}%
\begin{itemdecl}
bool all() const noexcept;
\end{itemdecl}

\begin{itemdescr}
\pnum
\returns \tcode{count() == size()}.
\end{itemdescr}

\indexlibrarymember{any}{bitset}%
\begin{itemdecl}
bool any() const noexcept;
\end{itemdecl}

\begin{itemdescr}
\pnum
\returns \tcode{count() != 0}.
\end{itemdescr}

\indexlibrarymember{none}{bitset}%
\begin{itemdecl}
bool none() const noexcept;
\end{itemdecl}

\begin{itemdescr}
\pnum
\returns \tcode{count() == 0}.
\end{itemdescr}

\indexlibrarymember{operator\shl}{bitset}%
\begin{itemdecl}
bitset<N> operator<<(size_t pos) const noexcept;
\end{itemdecl}

\begin{itemdescr}
\pnum
\returns
\tcode{bitset<N>(*this) \shl= pos}.
\end{itemdescr}

\indexlibrarymember{operator\shr}{bitset}%
\begin{itemdecl}
bitset<N> operator>>(size_t pos) const noexcept;
\end{itemdecl}

\begin{itemdescr}
\pnum
\returns
\tcode{bitset<N>(*this) \shr= pos}.
\end{itemdescr}

\indexlibrarymember{operator[]}{bitset}%
\begin{itemdecl}
constexpr bool operator[](size_t pos) const;
\end{itemdecl}

\begin{itemdescr}
\pnum
\requires
\tcode{pos} shall be valid.

\pnum
\returns
\tcode{true} if the bit at position \tcode{pos} in \tcode{*this} has the value
one, otherwise \tcode{false}.

\pnum
\throws Nothing.
\end{itemdescr}

\indexlibrarymember{operator[]}{bitset}%
\begin{itemdecl}
bitset<N>::reference operator[](size_t pos);
\end{itemdecl}

\begin{itemdescr}
\pnum
\requires
\tcode{pos} shall be valid.

\pnum
\returns
An object of type
\tcode{bitset<N>::reference}
such that
\tcode{(*this)[pos] == this->test(pos)},
and such that
\tcode{(*this)[pos] = val}
is equivalent to
\tcode{this->set(pos, val)}.

\pnum
\throws Nothing.

\pnum
\remarks For the purpose of determining the presence of a data
race~(\ref{intro.multithread}), any access or update through the resulting
reference potentially accesses or modifies, respectively, the entire
underlying bitset.
\end{itemdescr}

\rSec2[bitset.hash]{\tcode{bitset} hash support}

\indexlibrary{\idxcode{hash_code}}%
\begin{itemdecl}
template <size_t N> struct hash<bitset<N>>;
\end{itemdecl}

\begin{itemdescr}
\pnum The template specialization shall meet the requirements of class template
\tcode{hash}~(\ref{unord.hash}).
\end{itemdescr}


\rSec2[bitset.operators]{\tcode{bitset} operators}

\indexlibrarymember{operator\&}{bitset}%
\begin{itemdecl}
bitset<N> operator&(const bitset<N>& lhs, const bitset<N>& rhs) noexcept;
\end{itemdecl}

\begin{itemdescr}
\pnum
\returns
\tcode{bitset<N>(lhs) \&= rhs}.
\end{itemdescr}

\indexlibrarymember{operator"|}{bitset}%
\begin{itemdecl}
bitset<N> operator|(const bitset<N>& lhs, const bitset<N>& rhs) noexcept;
\end{itemdecl}

\begin{itemdescr}
\pnum
\returns
\tcode{bitset<N>(lhs) |= rhs}.
\end{itemdescr}

\indexlibrarymember{operator\caret}{bitset}%
\begin{itemdecl}
bitset<N> operator^(const bitset<N>& lhs, const bitset<N>& rhs) noexcept;
\end{itemdecl}

\begin{itemdescr}
\pnum
\returns
\tcode{bitset<N>(lhs) \caret= rhs}.
\end{itemdescr}

\indexlibrarymember{operator\shr}{bitset}%
\begin{itemdecl}
template <class charT, class traits, size_t N>
  basic_istream<charT, traits>&
  operator>>(basic_istream<charT, traits>& is, bitset<N>& x);
\end{itemdecl}

\begin{itemdescr}
\pnum
A formatted input function~(\ref{istream.formatted}).

\pnum
\effects
Extracts up to \tcode{N} characters from \tcode{is}.
Stores these characters in a temporary object \tcode{str} of type
\tcode{basic_string<charT, traits>},
then evaluates the expression
\tcode{x = bitset<N>(str)}.
Characters are extracted and stored until any of the following occurs:

\begin{itemize}
\item
\tcode{N} characters have been extracted and stored;
\item
end-of-file occurs on the input sequence;%
\indextext{end-of-file}
\item
the next input character is neither
\tcode{is.widen('0')}
nor
\tcode{is.widen('1')}
(in which case the input character is not extracted).
\end{itemize}

\pnum
If no characters are stored in \tcode{str}, calls
\tcode{is.setstate(ios_base::failbit)}
(which may throw
\tcode{ios_base::failure}~(\ref{iostate.flags})).

\pnum
\returns
\tcode{is}.
\end{itemdescr}

\indexlibrarymember{operator\shl}{bitset}%
\begin{itemdecl}
template <class charT, class traits, size_t N>
  basic_ostream<charT, traits>&
  operator<<(basic_ostream<charT, traits>& os, const bitset<N>& x);
\end{itemdecl}

\begin{itemdescr}
\pnum
\returns
\begin{codeblock}
os @\shl@ x.template to_string<charT, traits, allocator<charT>>(
  use_facet<ctype<charT>>(os.getloc()).widen('0'),
  use_facet<ctype<charT>>(os.getloc()).widen('1'))
\end{codeblock}
(see~\ref{ostream.formatted}).
\end{itemdescr}

\rSec1[memory]{Memory}

\rSec2[memory.general]{In general}

\pnum
This subclause describes the contents of the header
\tcode{<memory>}~(\ref{memory.syn}) and some
of the contents of the header \tcode{<cstdlib>}~(\ref{cstdlib.syn}).

\rSec2[memory.syn]{Header \tcode{<memory>} synopsis}

\pnum
The header \tcode{<memory>} defines several types and function templates that
describe properties of pointers and pointer-like types, manage memory
for containers and other template types, destroy objects, and
construct multiple objects in
uninitialized memory
buffers~(\ref{pointer.traits}--\ref{specialized.algorithms}).
The header also defines the templates
\tcode{unique_ptr}, \tcode{shared_ptr}, \tcode{weak_ptr}, and various function
templates that operate on objects of these types~(\ref{smartptr}).

\indexlibrary{\idxhdr{memory}}%
\begin{codeblock}
namespace std {
  // \ref{pointer.traits}, pointer traits
  template <class Ptr> struct pointer_traits;
  template <class T> struct pointer_traits<T*>;

  // \ref{util.dynamic.safety}, pointer safety
  enum class pointer_safety { relaxed, preferred, strict };
  void declare_reachable(void* p);
  template <class T> T* undeclare_reachable(T* p);
  void declare_no_pointers(char* p, size_t n);
  void undeclare_no_pointers(char* p, size_t n);
  pointer_safety get_pointer_safety() noexcept;

  // \ref{ptr.align}, pointer alignment function
  void* align(size_t alignment, size_t size, void*& ptr, size_t& space);

  // \ref{allocator.tag}, allocator argument tag
  struct allocator_arg_t { };
  constexpr allocator_arg_t allocator_arg{};

  // \ref{allocator.uses}, \tcode{uses_allocator}
  template <class T, class Alloc> struct uses_allocator;

  // \ref{allocator.traits}, allocator traits
  template <class Alloc> struct allocator_traits;

  // \ref{default.allocator}, the default allocator:
  template <class T> class allocator;
  template <class T, class U>
    bool operator==(const allocator<T>&, const allocator<U>&) noexcept;
  template <class T, class U>
    bool operator!=(const allocator<T>&, const allocator<U>&) noexcept;

  // \ref{specialized.algorithms}, specialized algorithms:
  template <class T> constexpr T* addressof(T& r) noexcept;
  template <class ForwardIterator>
    void uninitialized_default_construct(ForwardIterator first, ForwardIterator last);
  template <class ExecutionPolicy, class ForwardIterator>
    void uninitialized_default_construct(ExecutionPolicy&& exec, // see \ref{algorithms.parallel.overloads}
                                         ForwardIterator first, ForwardIterator last);
  template <class ForwardIterator, class Size>
    ForwardIterator uninitialized_default_construct_n(ForwardIterator first, Size n);
  template <class ExecutionPolicy, class ForwardIterator, class Size>
    ForwardIterator uninitialized_default_construct_n(ExecutionPolicy&& exec, // see \ref{algorithms.parallel.overloads}
                                                      ForwardIterator first, Size n);
  template <class ForwardIterator>
    void uninitialized_value_construct(ForwardIterator first, ForwardIterator last);
  template <class ExecutionPolicy, class ForwardIterator>
    void uninitialized_value_construct(ExecutionPolicy&& exec, // see \ref{algorithms.parallel.overloads}
                                       ForwardIterator first, ForwardIterator last);
  template <class ForwardIterator, class Size>
    ForwardIterator uninitialized_value_construct_n(ForwardIterator first, Size n);
  template <class ExecutionPolicy, class ForwardIterator, class Size>
    ForwardIterator uninitialized_value_construct_n(ExecutionPolicy&& exec, // see \ref{algorithms.parallel.overloads}
                                                    ForwardIterator first, Size n);
  template <class InputIterator, class ForwardIterator>
    ForwardIterator uninitialized_copy(InputIterator first, InputIterator last,
                                       ForwardIterator result);
  template <class ExecutionPolicy, class InputIterator, class ForwardIterator>
    ForwardIterator uninitialized_copy(ExecutionPolicy&& exec, // see \ref{algorithms.parallel.overloads}
                                       InputIterator first, InputIterator last,
                                       ForwardIterator result);
  template <class InputIterator, class Size, class ForwardIterator>
    ForwardIterator uninitialized_copy_n(InputIterator first, Size n,
                                         ForwardIterator result);
  template <class ExecutionPolicy, class InputIterator, class Size, class ForwardIterator>
    ForwardIterator uninitialized_copy_n(ExecutionPolicy&& exec, // see \ref{algorithms.parallel.overloads}
                                         InputIterator first, Size n,
                                         ForwardIterator result);
  template <class InputIterator, class ForwardIterator>
    ForwardIterator uninitialized_move(InputIterator first, InputIterator last,
                                       ForwardIterator result);
  template <class ExecutionPolicy, class InputIterator, class ForwardIterator>
    ForwardIterator uninitialized_move(ExecutionPolicy&& exec, // see \ref{algorithms.parallel.overloads}
                                       InputIterator first, InputIterator last,
                                       ForwardIterator result);
  template <class InputIterator, class Size, class ForwardIterator>
    pair<InputIterator, ForwardIterator>
      uninitialized_move_n(InputIterator first, Size n, ForwardIterator result);
  template <class ExecutionPolicy, class InputIterator, class Size, class ForwardIterator>
    pair<InputIterator, ForwardIterator>
      uninitialized_move_n(ExecutionPolicy&& exec, // see \ref{algorithms.parallel.overloads}
                           InputIterator first, Size n, ForwardIterator result);
  template <class ForwardIterator, class T>
    void uninitialized_fill(ForwardIterator first, ForwardIterator last,
                            const T& x);
  template <class ExecutionPolicy, class ForwardIterator, class T>
    void uninitialized_fill(ExecutionPolicy&& exec, // see \ref{algorithms.parallel.overloads}
                            ForwardIterator first, ForwardIterator last,
                            const T& x);
  template <class ForwardIterator, class Size, class T>
    ForwardIterator uninitialized_fill_n(ForwardIterator first, Size n, const T& x);
  template <class ExecutionPolicy, class ForwardIterator, class Size, class T>
    ForwardIterator uninitialized_fill_n(ExecutionPolicy&& exec, // see \ref{algorithms.parallel.overloads}
                                         ForwardIterator first, Size n, const T& x);
  template <class T>
    void destroy_at(T* location);
  template <class ForwardIterator>
    void destroy(ForwardIterator first, ForwardIterator last);
  template <class ExecutionPolicy, class ForwardIterator>
    void destroy(ExecutionPolicy&& exec, // see \ref{algorithms.parallel.overloads}
                 ForwardIterator first, ForwardIterator last);
  template <class ForwardIterator, class Size>
    ForwardIterator destroy_n(ForwardIterator first, Size n);
  template <class ExecutionPolicy, class ForwardIterator, class Size>
    ForwardIterator destroy_n(ExecutionPolicy&& exec, // see \ref{algorithms.parallel.overloads}
                              ForwardIterator first, Size n);

  // \ref{unique.ptr} class template unique_ptr:
  template <class T> struct default_delete;
  template <class T> struct default_delete<T[]>;
  template <class T, class D = default_delete<T>> class unique_ptr;
  template <class T, class D> class unique_ptr<T[], D>;

  template <class T, class... Args> unique_ptr<T> make_unique(Args&&... args);
  template <class T> unique_ptr<T> make_unique(size_t n);
  template <class T, class... Args> @\unspec@ make_unique(Args&&...) = delete;

  template <class T, class D> void swap(unique_ptr<T, D>& x, unique_ptr<T, D>& y) noexcept;

  template <class T1, class D1, class T2, class D2>
    bool operator==(const unique_ptr<T1, D1>& x, const unique_ptr<T2, D2>& y);
  template <class T1, class D1, class T2, class D2>
    bool operator!=(const unique_ptr<T1, D1>& x, const unique_ptr<T2, D2>& y);
  template <class T1, class D1, class T2, class D2>
    bool operator<(const unique_ptr<T1, D1>& x, const unique_ptr<T2, D2>& y);
  template <class T1, class D1, class T2, class D2>
    bool operator<=(const unique_ptr<T1, D1>& x, const unique_ptr<T2, D2>& y);
  template <class T1, class D1, class T2, class D2>
    bool operator>(const unique_ptr<T1, D1>& x, const unique_ptr<T2, D2>& y);
  template <class T1, class D1, class T2, class D2>
    bool operator>=(const unique_ptr<T1, D1>& x, const unique_ptr<T2, D2>& y);

  template <class T, class D>
    bool operator==(const unique_ptr<T, D>& x, nullptr_t) noexcept;
  template <class T, class D>
    bool operator==(nullptr_t, const unique_ptr<T, D>& y) noexcept;
  template <class T, class D>
    bool operator!=(const unique_ptr<T, D>& x, nullptr_t) noexcept;
  template <class T, class D>
    bool operator!=(nullptr_t, const unique_ptr<T, D>& y) noexcept;
  template <class T, class D>
    bool operator<(const unique_ptr<T, D>& x, nullptr_t);
  template <class T, class D>
    bool operator<(nullptr_t, const unique_ptr<T, D>& y);
  template <class T, class D>
    bool operator<=(const unique_ptr<T, D>& x, nullptr_t);
  template <class T, class D>
    bool operator<=(nullptr_t, const unique_ptr<T, D>& y);
  template <class T, class D>
    bool operator>(const unique_ptr<T, D>& x, nullptr_t);
  template <class T, class D>
    bool operator>(nullptr_t, const unique_ptr<T, D>& y);
  template <class T, class D>
    bool operator>=(const unique_ptr<T, D>& x, nullptr_t);
  template <class T, class D>
    bool operator>=(nullptr_t, const unique_ptr<T, D>& y);

  // \ref{util.smartptr.weak.bad}, class bad_weak_ptr:
  class bad_weak_ptr;

  // \ref{util.smartptr.shared}, class template shared_ptr:
  template<class T> class shared_ptr;

  // \ref{util.smartptr.shared.create}, shared_ptr creation
  template<class T, class... Args> shared_ptr<T> make_shared(Args&&... args);
  template<class T, class A, class... Args>
    shared_ptr<T> allocate_shared(const A& a, Args&&... args);

  // \ref{util.smartptr.shared.cmp}, shared_ptr comparisons:
  template<class T, class U>
    bool operator==(shared_ptr<T> const& a, shared_ptr<U> const& b) noexcept;
  template<class T, class U>
    bool operator!=(shared_ptr<T> const& a, shared_ptr<U> const& b) noexcept;
  template<class T, class U>
    bool operator<(shared_ptr<T> const& a, shared_ptr<U> const& b) noexcept;
  template<class T, class U>
    bool operator>(shared_ptr<T> const& a, shared_ptr<U> const& b) noexcept;
  template<class T, class U>
    bool operator<=(shared_ptr<T> const& a, shared_ptr<U> const& b) noexcept;
  template<class T, class U>
    bool operator>=(shared_ptr<T> const& a, shared_ptr<U> const& b) noexcept;

  template <class T>
    bool operator==(const shared_ptr<T>& x, nullptr_t) noexcept;
  template <class T>
    bool operator==(nullptr_t, const shared_ptr<T>& y) noexcept;
  template <class T>
    bool operator!=(const shared_ptr<T>& x, nullptr_t) noexcept;
  template <class T>
    bool operator!=(nullptr_t, const shared_ptr<T>& y) noexcept;
  template <class T>
    bool operator<(const shared_ptr<T>& x, nullptr_t) noexcept;
  template <class T>
    bool operator<(nullptr_t, const shared_ptr<T>& y) noexcept;
  template <class T>
    bool operator<=(const shared_ptr<T>& x, nullptr_t) noexcept;
  template <class T>
    bool operator<=(nullptr_t, const shared_ptr<T>& y) noexcept;
  template <class T>
    bool operator>(const shared_ptr<T>& x, nullptr_t) noexcept;
  template <class T>
    bool operator>(nullptr_t, const shared_ptr<T>& y) noexcept;
  template <class T>
    bool operator>=(const shared_ptr<T>& x, nullptr_t) noexcept;
  template <class T>
    bool operator>=(nullptr_t, const shared_ptr<T>& y) noexcept;

  // \ref{util.smartptr.shared.spec}, shared_ptr specialized algorithms:
  template<class T> void swap(shared_ptr<T>& a, shared_ptr<T>& b) noexcept;

  // \ref{util.smartptr.shared.cast}, shared_ptr casts:
  template<class T, class U>
    shared_ptr<T> static_pointer_cast(shared_ptr<U> const& r) noexcept;
  template<class T, class U>
    shared_ptr<T> dynamic_pointer_cast(shared_ptr<U> const& r) noexcept;
  template<class T, class U>
    shared_ptr<T> const_pointer_cast(shared_ptr<U> const& r) noexcept;

  // \ref{util.smartptr.getdeleter}, shared_ptr get_deleter:
  template<class D, class T> D* get_deleter(shared_ptr<T> const& p) noexcept;

  // \ref{util.smartptr.shared.io}, shared_ptr I/O:
  template<class E, class T, class Y>
    basic_ostream<E, T>& operator<< (basic_ostream<E, T>& os, shared_ptr<Y> const& p);

  // \ref{util.smartptr.weak}, class template weak_ptr:
  template<class T> class weak_ptr;

  // \ref{util.smartptr.weak.spec}, weak_ptr specialized algorithms:
  template<class T> void swap(weak_ptr<T>& a, weak_ptr<T>& b) noexcept;

  // \ref{util.smartptr.ownerless}, class template owner_less:
  template<class T = void> struct owner_less;

  // \ref{util.smartptr.enab}, class template enable_shared_from_this:
  template<class T> class enable_shared_from_this;

  // \ref{util.smartptr.shared.atomic}, shared_ptr atomic access:
  template<class T>
    bool atomic_is_lock_free(const shared_ptr<T>* p);

  template<class T>
    shared_ptr<T> atomic_load(const shared_ptr<T>* p);
  template<class T>
    shared_ptr<T> atomic_load_explicit(const shared_ptr<T>* p, memory_order mo);

  template<class T>
    void atomic_store(shared_ptr<T>* p, shared_ptr<T> r);
  template<class T>
    void atomic_store_explicit(shared_ptr<T>* p, shared_ptr<T> r, memory_order mo);

  template<class T>
    shared_ptr<T> atomic_exchange(shared_ptr<T>* p, shared_ptr<T> r);
  template<class T>
    shared_ptr<T> atomic_exchange_explicit(shared_ptr<T>* p, shared_ptr<T> r,
                                           memory_order mo);

  template<class T>
    bool atomic_compare_exchange_weak(
      shared_ptr<T>* p, shared_ptr<T>* v, shared_ptr<T> w);
  template<class T>
    bool atomic_compare_exchange_strong(
      shared_ptr<T>* p, shared_ptr<T>* v, shared_ptr<T> w);
  template<class T>
    bool atomic_compare_exchange_weak_explicit(
      shared_ptr<T>* p, shared_ptr<T>* v, shared_ptr<T> w,
      memory_order success, memory_order failure);
  template<class T>
    bool atomic_compare_exchange_strong_explicit(
      shared_ptr<T>* p, shared_ptr<T>* v, shared_ptr<T> w,
      memory_order success, memory_order failure);

  // \ref{util.smartptr.hash} hash support
  template <class T> struct hash;
  template <class T, class D> struct hash<unique_ptr<T, D>>;
  template <class T> struct hash<shared_ptr<T>>;

  // \ref{allocator.uses.trait} uses_allocator
  template <class T, class Alloc> constexpr bool uses_allocator_v
    = uses_allocator<T, Alloc>::value;
}
\end{codeblock}

\rSec2[pointer.traits]{Pointer traits}

\pnum
The class template \tcode{pointer_traits} supplies a uniform interface to certain
attributes of pointer-like types.

\indexlibrary{\idxcode{pointer_traits}}%
\begin{codeblock}
namespace std {
  template <class Ptr> struct pointer_traits {
    using pointer         = Ptr;
    using element_type    = @\seebelow@;
    using difference_type = @\seebelow@;

    template <class U> using rebind = @\seebelow@;

    static pointer pointer_to(@\seebelow@ r);
  };

  template <class T> struct pointer_traits<T*> {
    using pointer         = T*;
    using element_type    = T;
    using difference_type = ptrdiff_t;

    template <class U> using rebind = U*;

    static pointer pointer_to(@\seebelow@ r) noexcept;
  };
}
\end{codeblock}

\rSec3[pointer.traits.types]{Pointer traits member types}

\indexlibrarymember{element_type}{pointer_traits}%
\begin{itemdecl}
using element_type = @\seebelow@;
\end{itemdecl}

\begin{itemdescr}
\pnum
\ctype \tcode{Ptr::element_type} if
the \grammarterm{qualified-id} \tcode{Ptr::element_type} is valid and denotes a
type~(\ref{temp.deduct}); otherwise, \tcode{T} if
\tcode{Ptr} is a class template instantiation of the form \tcode{SomePointer<T, Args>},
where \tcode{Args} is zero or more type arguments; otherwise, the specialization is
ill-formed.
\end{itemdescr}

\indexlibrarymember{difference_type}{pointer_traits}%
\begin{itemdecl}
using difference_type = @\seebelow@;
\end{itemdecl}

\begin{itemdescr}
\pnum
\ctype \tcode{Ptr::difference_type} if
the \grammarterm{qualified-id} \tcode{Ptr::difference_type} is valid and denotes a
type~(\ref{temp.deduct}); otherwise,
\tcode{ptrdiff_t}.
\end{itemdescr}

\indexlibrarymember{rebind}{pointer_traits}%
\begin{itemdecl}
template <class U> using rebind = @\seebelow@;
\end{itemdecl}

\begin{itemdescr}
\pnum
\templalias \tcode{Ptr::rebind<U>} if
the \grammarterm{qualified-id} \tcode{Ptr::rebind<U>} is valid and denotes a
type~(\ref{temp.deduct}); otherwise,
\tcode{SomePointer<U, Args>} if
\tcode{Ptr} is a class template instantiation of the form \tcode{SomePointer<T, Args>},
where \tcode{Args} is zero or more type arguments; otherwise, the instantiation of
\tcode{rebind} is ill-formed.
\end{itemdescr}

\rSec3[pointer.traits.functions]{Pointer traits member functions}

\indexlibrarymember{pointer_to}{pointer_traits}%
\begin{itemdecl}
static pointer pointer_traits::pointer_to(@\seebelow@ r);
static pointer pointer_traits<T*>::pointer_to(@\seebelow@ r) noexcept;
\end{itemdecl}

\begin{itemdescr}
\pnum
\remarks If \tcode{element_type} is (possibly cv-qualified) \tcode{void}, the type of
\tcode{r} is unspecified; otherwise, it is \tcode{element_type\&}.

\pnum
\returns The first member function returns a pointer to \tcode{r}
obtained by calling \tcode{Ptr::pointer_to(r)} through which
indirection is valid; an instantiation of this function is
ill-formed if \tcode{Ptr} does not have a matching \tcode{pointer_to} static member
function. The second member function returns \tcode{addressof(r)}.
\end{itemdescr}

\rSec2[util.dynamic.safety]{Pointer safety}

\pnum
A complete object is \techterm{declared reachable} while the number of calls to
\tcode{declare_reachable} with an argument referencing the object exceeds the
number of calls to \tcode{undeclare_reachable} with an argument referencing the
object.

\indexlibrary{\idxcode{declare_reachable}}%
\begin{itemdecl}
void declare_reachable(void* p);
\end{itemdecl}

\begin{itemdescr}
\pnum
\requires \tcode{p} shall be a safely-derived
pointer~(\ref{basic.stc.dynamic.safety}) or a null pointer value.

\pnum
\effects If \tcode{p} is not null, the complete object referenced by \tcode{p}
is subsequently declared reachable~(\ref{basic.stc.dynamic.safety}).

\pnum
\throws May throw \tcode{bad_alloc} if the system cannot allocate
additional memory that may be required to track objects declared reachable.
\end{itemdescr}

\indexlibrary{\idxcode{undeclare_reachable}}%
\begin{itemdecl}
template <class T> T* undeclare_reachable(T* p);
\end{itemdecl}

\begin{itemdescr}
\pnum
\requires If \tcode{p} is not null, the complete object referenced by \tcode{p}
shall have been previously declared reachable, and shall be
live~(\ref{basic.life}) from the time of the call until the last
\tcode{undeclare_reachable(p)} call on the object.

\pnum
\returns A safely derived copy of \tcode{p} which shall compare equal to \tcode{p}.

\pnum
\throws Nothing.

\pnum \begin{note} It is expected that calls to \tcode{declare_reachable(p)} will consume
a small amount of memory in addition to that occupied by the referenced object until the
matching call to \tcode{undeclare_reachable(p)} is encountered. Long running programs
should arrange that calls are matched. \end{note} \end{itemdescr}

\indexlibrary{\idxcode{declare_no_pointers}}%
\begin{itemdecl}
void declare_no_pointers(char* p, size_t n);
\end{itemdecl}

\begin{itemdescr}
\pnum
\requires No bytes in the specified range
are currently registered with
\tcode{declare_no_pointers()}. If the specified range is in an allocated object,
then it must be entirely within a single allocated object. The object must be
live until the corresponding \tcode{undeclare_no_pointers()} call. \begin{note} In
a garbage-collecting implementation, the fact that a region in an object is
registered with \tcode{declare_no_pointers()} should not prevent the object from
being collected. \end{note}

\pnum
\effects The \tcode{n} bytes starting at \tcode{p} no longer contain
traceable pointer locations, independent of their type. Hence
indirection through a pointer located there is undefined if the object
it points to was created by global \tcode{operator new} and not
previously declared reachable. \begin{note} This may be used to inform a
garbage collector or leak detector that this region of memory need not
be traced. \end{note}

\pnum
\throws Nothing.

\pnum
\begin{note} Under some conditions implementations may need to allocate memory.
However, the request can be ignored if memory allocation fails. \end{note}
\end{itemdescr}

\indexlibrary{\idxcode{undeclare_no_pointers}}%
\begin{itemdecl}
void undeclare_no_pointers(char* p, size_t n);
\end{itemdecl}

\begin{itemdescr}
\pnum
\requires The same range must previously have been passed to
\tcode{declare_no_pointers()}.

\pnum
\effects Unregisters a range registered with \tcode{declare_no_pointers()} for
destruction. It must be called before the lifetime of the object ends.

\pnum
\throws Nothing.
\end{itemdescr}

\indexlibrary{\idxcode{get_pointer_safety}}%
\begin{itemdecl}
pointer_safety get_pointer_safety() noexcept;
\end{itemdecl}

\begin{itemdescr}
\pnum
\returns \tcode{pointer_safety::strict} if the implementation has strict pointer
safety~(\ref{basic.stc.dynamic.safety}). It is
\impldef{whether \tcode{get_pointer_safety} returns
\tcode{pointer_safety::relaxed} or
\tcode{pointer_safety::\brk{}preferred} if the implementation has relaxed pointer safety}
whether
\tcode{get_pointer_safety} returns \tcode{pointer_safety::relaxed} or
\tcode{pointer_safety::preferred} if the implementation has relaxed pointer
safety.\footnote{\tcode{pointer_safety::preferred} might be returned to indicate
that a leak detector is running so that the program can avoid spurious leak
reports.}
\end{itemdescr}


\rSec2[ptr.align]{Align}

\indexlibrary{\idxcode{align}}%
\begin{itemdecl}
void* align(size_t alignment, size_t size, void*& ptr, size_t& space);
\end{itemdecl}

\begin{itemdescr}
\pnum
\effects If it is possible to fit \tcode{size} bytes
of storage aligned by \tcode{alignment} into the buffer pointed to by
\tcode{ptr} with length \tcode{space}, the function updates
\tcode{ptr} to represent the first possible address of such storage
and decreases \tcode{space} by the number of bytes used for alignment.
Otherwise, the function does nothing.

\pnum
\requires

\begin{itemize}
\item \tcode{alignment} shall be a power of two

\item \tcode{ptr} shall represent the address of contiguous storage of at least
\tcode{space} bytes
\end{itemize}

\pnum
\returns A null pointer if the requested aligned buffer
would not fit into the available space, otherwise the adjusted value
of \tcode{ptr}.

\pnum
\begin{note} The function updates its \tcode{ptr}
and \tcode{space} arguments so that it can be called repeatedly
with possibly different \tcode{alignment} and \tcode{size}
arguments for the same buffer.  \end{note}
\end{itemdescr}

\rSec2[allocator.tag]{Allocator argument tag}

\indexlibrary{\idxcode{allocator_arg_t}}%
\indexlibrary{\idxcode{allocator_arg}}%
\begin{itemdecl}
namespace std {
  struct allocator_arg_t { };
  constexpr allocator_arg_t allocator_arg{};
}
\end{itemdecl}

\pnum
The \tcode{allocator_arg_t} struct is an empty structure type used as a unique type to
disambiguate constructor and function overloading. Specifically, several types (see
\tcode{tuple}~\ref{tuple}) have constructors with \tcode{allocator_arg_t} as the first
argument, immediately followed by an argument of a type that satisfies the
\tcode{Allocator} requirements~(\ref{allocator.requirements}).

\rSec2[allocator.uses]{\tcode{uses_allocator}}

\rSec3[allocator.uses.trait]{\tcode{uses_allocator} trait}

\indexlibrary{\idxcode{uses_allocator}}%
\begin{itemdecl}
template <class T, class Alloc> struct uses_allocator;
\end{itemdecl}

\begin{itemdescr}
\pnum
\remarks automatically detects whether \tcode{T} has a nested \tcode{allocator_type} that
is convertible from \tcode{Alloc}. Meets the BinaryTypeTrait
requirements~(\ref{meta.rqmts}). The implementation shall provide a definition that is
derived from \tcode{true_type} if the \grammarterm{qualified-id} \tcode{T::allocator_type}
is valid and denotes a type~(\ref{temp.deduct}) and
\tcode{is_convertible_v<Alloc, T::allocator_type> != false}, otherwise it shall be
derived from \tcode{false_type}. A program may specialize this template to derive from
\tcode{true_type} for a user-defined type \tcode{T} that does not have a nested
\tcode{allocator_type} but nonetheless can be constructed with an allocator where
either:

\begin{itemize}
\item the first argument of a constructor has type \tcode{allocator_arg_t} and the
second argument has type \tcode{Alloc} or

\item the last argument of a constructor has type \tcode{Alloc}.
\end{itemize}
\end{itemdescr}

\rSec3[allocator.uses.construction]{uses-allocator construction}

\pnum
\defn{Uses-allocator construction} with allocator \tcode{Alloc} refers to the
construction of an object \tcode{obj} of type \tcode{T}, using constructor arguments
\tcode{v1, v2, ..., vN} of types \tcode{V1, V2, ..., VN}, respectively, and an allocator
\tcode{alloc} of type \tcode{Alloc}, according to the following rules:

\begin{itemize}
\item if \tcode{uses_allocator_v<T, Alloc>} is \tcode{false} and
\tcode{is_constructible_v<T, V1, V2, ..., VN>} is \tcode{true}, then \tcode{obj} is
initialized as \tcode{obj(v1, v2, ..., vN)};

\item otherwise, if \tcode{uses_allocator_v<T, Alloc>} is \tcode{true} and
\tcode{is_constructible_v<T, allocator_arg_t, Alloc,} \tcode{V1, V2, ..., VN>} is
\tcode{true}, then \tcode{obj} is initialized as \tcode{obj(allocator_arg, alloc, v1,
v2, ..., vN)};

\item otherwise, if \tcode{uses_allocator_v<T, Alloc>} is \tcode{true} and
\tcode{is_constructible_v<T, V1, V2, ..., VN, Alloc>} is \tcode{true}, then
\tcode{obj} is initialized as \tcode{obj(v1, v2, ..., vN, alloc)};

\item otherwise, the request for uses-allocator construction is ill-formed. \begin{note}
An error will result if \tcode{uses_allocator_v<T, Alloc>} is \tcode{true} but the
specific constructor does not take an allocator. This definition prevents a silent
failure to pass the allocator to an element. \end{note}
\end{itemize}

\rSec2[allocator.traits]{Allocator traits}

\pnum
The class template \tcode{allocator_traits} supplies a uniform interface to all
allocator types.
An allocator cannot be a non-class type, however, even if \tcode{allocator_traits}
supplies the entire required interface. \begin{note} Thus, it is always possible to create
a derived class from an allocator. \end{note}

\indexlibrary{\idxcode{allocator_traits}}%
\begin{codeblock}
namespace std {
  template <class Alloc> struct allocator_traits {
    using allocator_type     = Alloc;

    using value_type         = typename Alloc::value_type;

    using pointer            = @\seebelow@;
    using const_pointer      = @\seebelow@;
    using void_pointer       = @\seebelow@;
    using const_void_pointer = @\seebelow@;

    using difference_type    = @\seebelow@;
    using size_type          = @\seebelow@;

    using propagate_on_container_copy_assignment = @\seebelow@;
    using propagate_on_container_move_assignment = @\seebelow@;
    using propagate_on_container_swap            = @\seebelow@;
    using is_always_equal                        = @\seebelow@;

    template <class T> using rebind_alloc = @\seebelow@;
    template <class T> using rebind_traits = allocator_traits<rebind_alloc<T>>;

    static pointer allocate(Alloc& a, size_type n);
    static pointer allocate(Alloc& a, size_type n, const_void_pointer hint);

    static void deallocate(Alloc& a, pointer p, size_type n);

    template <class T, class... Args>
      static void construct(Alloc& a, T* p, Args&&... args);

    template <class T>
      static void destroy(Alloc& a, T* p);

    static size_type max_size(const Alloc& a) noexcept;

    static Alloc select_on_container_copy_construction(const Alloc& rhs);
  };
}
\end{codeblock}

\rSec3[allocator.traits.types]{Allocator traits member types}

\indexlibrarymember{pointer}{allocator_traits}%
\begin{itemdecl}
using pointer = @\seebelow@;
\end{itemdecl}

\begin{itemdescr}
\pnum
\ctype \tcode{Alloc::pointer} if
the \grammarterm{qualified-id} \tcode{Alloc::pointer} is valid and denotes a
type~(\ref{temp.deduct}); otherwise, \tcode{value_type*}.
\end{itemdescr}

\indexlibrarymember{const_pointer}{allocator_traits}%
\begin{itemdecl}
using const_pointer = @\seebelow@;
\end{itemdecl}

\begin{itemdescr}
\pnum
\ctype \tcode{Alloc::const_pointer} if
the \grammarterm{qualified-id} \tcode{Alloc::const_pointer} is valid and denotes a
type~(\ref{temp.deduct}); otherwise,
\tcode{pointer_traits<pointer>::rebind<\brk{}const value_type>}.
\end{itemdescr}

\indexlibrarymember{void_pointer}{allocator_traits}%
\begin{itemdecl}
using void_pointer = @\seebelow@;
\end{itemdecl}

\begin{itemdescr}
\pnum
\ctype \tcode{Alloc::void_pointer} if
the \grammarterm{qualified-id} \tcode{Alloc::void_pointer} is valid and denotes a
type~(\ref{temp.deduct}); otherwise,
\tcode{pointer_traits<pointer>::rebind<\brk{}void>}.
\end{itemdescr}

\indexlibrarymember{const_void_pointer}{allocator_traits}%
\begin{itemdecl}
using const_void_pointer = @\seebelow@;
\end{itemdecl}

\begin{itemdescr}
\pnum
\ctype \tcode{Alloc::const_void_pointer} if
the \grammarterm{qualified-id} \tcode{Alloc::const_void_pointer} is valid and denotes a
type~(\ref{temp.deduct}); otherwise,
\tcode{pointer_traits<pointer>::\brk{}rebind<const void>}.
\end{itemdescr}

\indexlibrarymember{difference_type}{allocator_traits}%
\begin{itemdecl}
using difference_type = @\seebelow@;
\end{itemdecl}

\begin{itemdescr}
\pnum
\ctype \tcode{Alloc::difference_type} if
the \grammarterm{qualified-id} \tcode{Alloc::difference_type} is valid and denotes a
type~(\ref{temp.deduct}); otherwise,
\tcode{pointer_traits<pointer>::dif\-ference_type}.
\end{itemdescr}

\indexlibrarymember{size_type}{allocator_traits}%
\begin{itemdecl}
using size_type = @\seebelow@;
\end{itemdecl}

\begin{itemdescr}
\pnum
\ctype \tcode{Alloc::size_type} if
the \grammarterm{qualified-id} \tcode{Alloc::size_type} is valid and denotes a
type~(\ref{temp.deduct}); otherwise,
\tcode{make_unsigned_t<difference_type>}.
\end{itemdescr}

\indexlibrarymember{propagate_on_container_copy_assignment}{allocator_traits}%
\begin{itemdecl}
using propagate_on_container_copy_assignment = @\seebelow@;
\end{itemdecl}

\begin{itemdescr}
\pnum
\ctype \tcode{Alloc::propagate_on_container_copy_assignment} if
the \grammarterm{qualified-id} \tcode{Alloc::propagate_on_container_copy_assignment} is valid and denotes a
type~(\ref{temp.deduct}); otherwise
\tcode{false_type}.
\end{itemdescr}

\indexlibrarymember{propagate_on_container_move_assignment}{allocator_traits}%
\begin{itemdecl}
using propagate_on_container_move_assignment = @\seebelow@;
\end{itemdecl}

\begin{itemdescr}
\pnum
\ctype \tcode{Alloc::propagate_on_container_move_assignment} if
the \grammarterm{qualified-id} \tcode{Alloc::propagate_on_container_move_assignment} is valid and denotes a
type~(\ref{temp.deduct}); otherwise
\tcode{false_type}.
\end{itemdescr}

\indexlibrarymember{propagate_on_container_swap}{allocator_traits}%
\begin{itemdecl}
using propagate_on_container_swap = @\seebelow@;
\end{itemdecl}

\begin{itemdescr}
\pnum
\ctype \tcode{Alloc::propagate_on_container_swap} if
the \grammarterm{qualified-id} \tcode{Alloc::propagate_on_container_swap} is valid and denotes a
type~(\ref{temp.deduct}); otherwise
\tcode{false_type}.
\end{itemdescr}

\indexlibrarymember{is_always_equal}{allocator_traits}%
\begin{itemdecl}
using is_always_equal = @\seebelow@;
\end{itemdecl}

\begin{itemdescr}
\pnum
\ctype \tcode{Alloc::is_always_equal} if
the \grammarterm{qualified-id} \tcode{Alloc::is_always_equal}
is valid and denotes a type~(\ref{temp.deduct});
otherwise \tcode{is_empty<Alloc>::type}.
\end{itemdescr}

\indexlibrarymember{rebind_alloc}{allocator_traits}%
\begin{itemdecl}
template <class T> using rebind_alloc = @\seebelow@;
\end{itemdecl}

\begin{itemdescr}
\pnum
\templalias \tcode{Alloc::rebind<T>::other} if
the \grammarterm{qualified-id} \tcode{Alloc::rebind<T>::other} is valid and denotes a
type~(\ref{temp.deduct}); otherwise,
\tcode{Alloc<T, Args>} if \tcode{Alloc} is a class template instantiation
of the form \tcode{Alloc<U, Args>}, where \tcode{Args} is zero or more type arguments;
otherwise, the instantiation of \tcode{rebind_alloc} is ill-formed.
\end{itemdescr}

\rSec3[allocator.traits.members]{Allocator traits static member functions}

\indexlibrarymember{allocate}{allocator_traits}%
\begin{itemdecl}
static pointer allocate(Alloc& a, size_type n);
\end{itemdecl}

\begin{itemdescr}
\pnum
\returns \tcode{a.allocate(n)}.
\end{itemdescr}

\indexlibrarymember{allocate}{allocator_traits}%
\begin{itemdecl}
static pointer allocate(Alloc& a, size_type n, const_void_pointer hint);
\end{itemdecl}

\begin{itemdescr}
\pnum
\returns \tcode{a.allocate(n, hint)} if that expression is well-formed; otherwise, \tcode{a.allocate(n)}.
\end{itemdescr}

\indexlibrarymember{deallocate}{allocator_traits}%
\begin{itemdecl}
static void deallocate(Alloc& a, pointer p, size_type n);
\end{itemdecl}

\begin{itemdescr}
\pnum
\effects Calls \tcode{a.deallocate(p, n)}.

\pnum
\throws Nothing.
\end{itemdescr}

\indexlibrarymember{construct}{allocator_traits}%
\begin{itemdecl}
template <class T, class... Args>
  static void construct(Alloc& a, T* p, Args&&... args);
\end{itemdecl}

\begin{itemdescr}
\pnum
\effects Calls \tcode{a.construct(p, std::forward<Args>(args)...)}
if that call is well-formed;
otherwise, invokes \tcode{::new (static_cast<void*>(p)) T(std::forward<Args>(args)...)}.
\end{itemdescr}

\indexlibrarymember{destroy}{allocator_traits}%
\begin{itemdecl}
template <class T>
  static void destroy(Alloc& a, T* p);
\end{itemdecl}

\begin{itemdescr}
\pnum
\effects Calls \tcode{a.destroy(p)} if that call is well-formed; otherwise, invokes
\tcode{p->\~{}T()}.
\end{itemdescr}

\indexlibrarymember{max_size}{allocator_traits}%
\begin{itemdecl}
static size_type max_size(const Alloc& a) noexcept;
\end{itemdecl}

\begin{itemdescr}
\pnum
\returns \tcode{a.max_size()} if that expression is well-formed; otherwise,
\tcode{numeric_limits<size_type>::\brk{}max()/sizeof(value_type)}.
\end{itemdescr}

\indexlibrarymember{select_on_container_copy_construction}{allocator_traits}%
\begin{itemdecl}
static Alloc select_on_container_copy_construction(const Alloc& rhs);
\end{itemdecl}

\begin{itemdescr}
\pnum
\returns \tcode{rhs.select_on_container_copy_construction()} if that expression is
well-formed; otherwise, \tcode{rhs}.
\end{itemdescr}

\rSec2[default.allocator]{The default allocator}

\pnum
All specializations of the default allocator satisfy the
allocator completeness requirements~\ref{allocator.requirements.completeness}.

\indexlibrary{\idxcode{allocator}}%
\begin{codeblock}
namespace std {
  template <class T> class allocator {
   public:
    using value_type      = T;
    using propagate_on_container_move_assignment = true_type;
    using is_always_equal = true_type;

    allocator() noexcept;
    allocator(const allocator&) noexcept;
    template <class U> allocator(const allocator<U>&) noexcept;
   ~allocator();

    T* allocate(size_t n);
    void deallocate(T* p, size_t n);
  };
}
\end{codeblock}

\rSec3[allocator.members]{\tcode{allocator} members}

\pnum
Except for the destructor, member functions of the default allocator shall not introduce
data races~(\ref{intro.multithread}) as a result of concurrent calls to those member
functions from different threads. Calls to these functions that allocate or deallocate a
particular unit of storage shall occur in a single total order, and each such
deallocation call shall happen before the next allocation (if any) in this order.

\indexlibrarymember{allocate}{allocator}%
\begin{itemdecl}
T* allocate(size_t n);
\end{itemdecl}

\begin{itemdescr}
\pnum
\returns
A pointer to the initial element of an array of storage of size \tcode{n}
\tcode{* sizeof(T)}, aligned appropriately for objects of type \tcode{T}.

\pnum
\remarks
the storage is obtained by calling \tcode{::operator new}~(\ref{new.delete}),
but it is unspecified when or how often this
function is called.

\pnum
\throws
\tcode{bad_alloc} if the storage cannot be obtained.
\end{itemdescr}

\indexlibrarymember{deallocate}{allocator}%
\begin{itemdecl}
void deallocate(T* p, size_t n);
\end{itemdecl}

\begin{itemdescr}
\pnum
\requires
\tcode{p} shall be a pointer value obtained from \tcode{allocate()}.
\tcode{n} shall equal the value passed as the first argument
to the invocation of allocate which returned \tcode{p}.

\pnum
\effects
Deallocates the storage referenced by \tcode{p} .

\pnum
\remarks
Uses
\tcode{::operator delete}~(\ref{new.delete}),
but it is unspecified
when this function is called.
\end{itemdescr}

\rSec3[allocator.globals]{\tcode{allocator} globals}

\indexlibrarymember{operator==}{allocator}%
\begin{itemdecl}
template <class T, class U>
  bool operator==(const allocator<T>&, const allocator<U>&) noexcept;
\end{itemdecl}

\begin{itemdescr}
\pnum
\returns
\tcode{true}.
\end{itemdescr}

\indexlibrarymember{operator"!=}{allocator}%
\begin{itemdecl}
template <class T, class U>
  bool operator!=(const allocator<T>&, const allocator<U>&) noexcept;
\end{itemdecl}

\begin{itemdescr}
\pnum
\returns
\tcode{false}.
\end{itemdescr}

\rSec2[specialized.algorithms]{Specialized algorithms}

\pnum
Throughout this subclause,
the names of template parameters are used to express type requirements.
\begin{itemize}
\item
If an algorithm's template parameter is named \tcode{InputIterator},
the template argument shall satisfy the requirements
of an input iterator~(\ref{input.iterators}).
\item
If an algorithm's template parameter is named \tcode{ForwardIterator},
the template argument shall satisfy the requirements
of a forward iterator~(\ref{forward.iterators}), and
is required to have the property that no exceptions are thrown
from increment, assignment, comparison, or indirection through valid iterators.
\end{itemize}
Unless otherwise specified,
if an exception is thrown in the following algorithms there are no effects.

\rSec3[specialized.addressof]{\tcode{addressof}}

\indexlibrary{\idxcode{addressof}}%
\begin{itemdecl}
template <class T> constexpr T* addressof(T& r) noexcept;
\end{itemdecl}

\begin{itemdescr}
\pnum
\returns The actual address of the object or function referenced by \tcode{r}, even in the
presence of an overloaded \tcode{operator\&}.

\pnum
\remarks An expression \tcode{addressof(E)}
is a constant subexpression~(\ref{defns.const.subexpr})
if \tcode{E} is an lvalue constant subexpression.
\end{itemdescr}

\rSec3[uninitialized.construct.default]{\tcode{uninitialized_default_construct}}

\indexlibrary{\idxcode{uninitialized_default_construct}}%
\begin{itemdecl}
template <class ForwardIterator>
  void uninitialized_default_construct(ForwardIterator first, ForwardIterator last);
\end{itemdecl}

\begin{itemdescr}
\pnum
\effects
Equivalent to:
\begin{codeblock}
for (; first != last; ++first)
  ::new (static_cast<void*>(addressof(*first)))
    typename iterator_traits<ForwardIterator>::value_type;
\end{codeblock}
\end{itemdescr}

\indexlibrary{\idxcode{uninitialized_default_construct_n}}%
\begin{itemdecl}
template <class ForwardIterator, class Size>
  ForwardIterator uninitialized_default_construct_n(ForwardIterator first, Size n);
\end{itemdecl}

\begin{itemdescr}
\pnum
\effects
Equivalent to:
\begin{codeblock}
for (; n>0; (void)++first, --n)
  ::new (static_cast<void*>(addressof(*first)))
    typename iterator_traits<ForwardIterator>::value_type;
return first;
\end{codeblock}
\end{itemdescr}

\rSec3[uninitialized.construct.value]{\tcode{uninitialized_value_construct}}

\indexlibrary{\idxcode{uninitialized_value_construct}}%
\begin{itemdecl}
template <class ForwardIterator>
  void uninitialized_value_construct(ForwardIterator first, ForwardIterator last);
\end{itemdecl}

\begin{itemdescr}
\pnum
\effects
Equivalent to:
\begin{codeblock}
for (; first != last; ++first)
  ::new (static_cast<void*>(addressof(*first)))
    typename iterator_traits<ForwardIterator>::value_type();
\end{codeblock}
\end{itemdescr}

\indexlibrary{\idxcode{uninitialized_value_construct_n}}%
\begin{itemdecl}
template <class ForwardIterator, class Size>
  ForwardIterator uninitialized_value_construct_n(ForwardIterator first, Size n);
\end{itemdecl}

\begin{itemdescr}
\pnum
\effects
Equivalent to:
\begin{codeblock}
for (; n>0; (void)++first, --n)
  ::new (static_cast<void*>(addressof(*first)))
    typename iterator_traits<ForwardIterator>::value_type();
return first;
\end{codeblock}
\end{itemdescr}

\rSec3[uninitialized.copy]{\tcode{uninitialized_copy}}

\indexlibrary{\idxcode{uninitialized_copy}}%
\begin{itemdecl}
template <class InputIterator, class ForwardIterator>
  ForwardIterator uninitialized_copy(InputIterator first, InputIterator last,
                                     ForwardIterator result);
\end{itemdecl}

\begin{itemdescr}
\pnum
\effects
As if by:
\begin{codeblock}
for (; first != last; ++result, (void) ++first)
  ::new (static_cast<void*>(addressof(*result)))
    typename iterator_traits<ForwardIterator>::value_type(*first);
\end{codeblock}

\pnum
\returns
\tcode{result}.
\end{itemdescr}

\indexlibrary{\idxcode{uninitialized_copy_n}}%
\begin{itemdecl}
template <class InputIterator, class Size, class ForwardIterator>
  ForwardIterator uninitialized_copy_n(InputIterator first, Size n,
                                       ForwardIterator result);
\end{itemdecl}

\begin{itemdescr}
\pnum
\effects
As if by:
\begin{codeblock}
for ( ; n > 0; ++result, (void) ++first, --n) {
  ::new (static_cast<void*>(addressof(*result)))
    typename iterator_traits<ForwardIterator>::value_type(*first);
}
\end{codeblock}

\pnum
\returns \tcode{result}.
\end{itemdescr}

\rSec3[uninitialized.move]{\tcode{uninitialized_move}}

\indexlibrary{\idxcode{uninitialized_move}}%
\begin{itemdecl}
template <class InputIterator, class ForwardIterator>
  ForwardIterator uninitialized_move(InputIterator first, InputIterator last,
                                     ForwardIterator result);
\end{itemdecl}

\begin{itemdescr}
\pnum
\effects
Equivalent to:
\begin{codeblock}
for (; first != last; (void)++result, ++first)
  ::new (static_cast<void*>(addressof(*result)))
    typename iterator_traits<ForwardIterator>::value_type(std::move(*first));
return result;
\end{codeblock}

\pnum
\remarks
If an exception is thrown, some objects in the range \range{first}{last}
are left in a valid but unspecified state.
\end{itemdescr}

\indexlibrary{\idxcode{uninitialized_move_n}}%
\begin{itemdecl}
template <class InputIterator, class Size, class ForwardIterator>
  pair<InputIterator, ForwardIterator>
    uninitialized_move_n(InputIterator first, Size n, ForwardIterator result);
\end{itemdecl}

\begin{itemdescr}
\pnum
\effects
Equivalent to:
\begin{codeblock}
for (; n > 0; ++result, (void) ++first, --n)
  ::new (static_cast<void*>(addressof(*result)))
    typename iterator_traits<ForwardIterator>::value_type(std::move(*first));
return {first,result};
\end{codeblock}

\pnum
\remarks
If an exception is thrown, some objects in the range \range{first}{std::next(first,n)}
are left in a valid but unspecified state.
\end{itemdescr}

\rSec3[uninitialized.fill]{\tcode{uninitialized_fill}}

\indexlibrary{\idxcode{uninitialized_fill}}%
\begin{itemdecl}
template <class ForwardIterator, class T>
  void uninitialized_fill(ForwardIterator first, ForwardIterator last,
                          const T& x);
\end{itemdecl}

\begin{itemdescr}
\pnum
\effects
As if by:
\begin{codeblock}
for (; first != last; ++first)
  ::new (static_cast<void*>(addressof(*first)))
    typename iterator_traits<ForwardIterator>::value_type(x);
\end{codeblock}
\end{itemdescr}

\indexlibrary{\idxcode{uninitialized_fill_n}}%
\begin{itemdecl}
template <class ForwardIterator, class Size, class T>
  ForwardIterator uninitialized_fill_n(ForwardIterator first, Size n, const T& x);
\end{itemdecl}

\begin{itemdescr}
\pnum
\effects
As if by:
\begin{codeblock}
for (; n--; ++first)
  ::new (static_cast<void*>(addressof(*first)))
    typename iterator_traits<ForwardIterator>::value_type(x);
return first;
\end{codeblock}
\end{itemdescr}

\rSec3[specialized.destroy]{\tcode{destroy}}

\indexlibrary{\idxcode{destroy_at}}%
\begin{itemdecl}
template <class T>
  void destroy_at(T* location);
\end{itemdecl}

\begin{itemdescr}
\pnum
\effects
Equivalent to:
\begin{codeblock}
location->~T();
\end{codeblock}
\end{itemdescr}

\indexlibrary{\idxcode{destroy}}%
\begin{itemdecl}
template <class ForwardIterator>
  void destroy(ForwardIterator first, ForwardIterator last);
\end{itemdecl}

\begin{itemdescr}
\pnum
\effects
Equivalent to:
\begin{codeblock}
for (; first!=last; ++first)
  destroy_at(addressof(*first));
\end{codeblock}
\end{itemdescr}

\indexlibrary{\idxcode{destroy_n}}%
\begin{itemdecl}
template <class ForwardIterator, class Size>
  ForwardIterator destroy_n(ForwardIterator first, Size n);
\end{itemdecl}

\begin{itemdescr}
\pnum
\effects
Equivalent to:
\begin{codeblock}
for (; n > 0; (void)++first, --n)
  destroy_at(addressof(*first));
return first;
\end{codeblock}
\end{itemdescr}

\rSec2[c.malloc]{C library memory allocation}

\pnum
\indextext{\idxhdr{cstdlib}}%
\begin{note}
The header \tcode{<cstdlib>}~(\ref{cstdlib.syn})
declares the functions described in this subclause.
\end{note}

\indexlibrary{\idxcode{aligned_alloc}}%
\indexlibrary{\idxcode{calloc}}%
\indexlibrary{\idxcode{malloc}}%
\indexlibrary{\idxcode{realloc}}%
\begin{itemdecl}
void* aligned_alloc(size_t alignment, size_t size);
void* calloc(size_t nmemb, size_t size);
void* malloc(size_t size);
void* realloc(void* ptr, size_t size);
\end{itemdecl}

\begin{itemdescr}
\pnum
\effects
These functions have the semantics specified in the C standard library.

\pnum
\remarks
These functions do not attempt to allocate
storage by calling \tcode{::operator new()}~(\ref{support.dynamic}).
\indexlibrary{\idxcode{new}!\idxcode{operator}}%

\pnum
Storage allocated directly with these functions
is implicitly declared reachable
(see~\ref{basic.stc.dynamic.safety}) on allocation, ceases to be declared
reachable on deallocation, and need not cease to be declared reachable as the
result of an \tcode{undeclare_reachable()} call. \begin{note} This allows existing
C libraries to remain unaffected by restrictions on pointers that are not safely
derived, at the expense of providing far fewer garbage collection and leak
detection options for \tcode{malloc()}-allocated objects. It also allows
\tcode{malloc()} to be implemented with a separate allocation arena, bypassing
the normal \tcode{declare_reachable()} implementation. The above functions
should never intentionally be used as a replacement for
\tcode{declare_reachable()}, and newly written code is strongly encouraged to
treat memory allocated with these functions as though it were allocated with
\tcode{operator new}. \end{note}
\end{itemdescr}

\indexlibrary{\idxcode{free}}%
\begin{itemdecl}
void free(void* ptr);
\end{itemdecl}

\begin{itemdescr}
\pnum
\effects
This function has the semantics specified in the C standard library.

\pnum
\remarks
This function does not attempt to
deallocate storage by calling
\tcode{::operator delete()}\indexlibrary{\idxcode{delete}!\idxcode{operator}}.
\end{itemdescr}

\xref ISO C~7.22.3.

\rSec1[smartptr]{Smart pointers}

\rSec2[unique.ptr]{Class template \tcode{unique_ptr}}

\pnum
A \defn{unique pointer} is an object that owns another object and
manages that other object through a pointer. More precisely, a unique pointer
is an object \textit{u} that stores a pointer to a second object \textit{p} and
will dispose of \textit{p} when \textit{u} is itself destroyed (e.g., when
leaving block scope~(\ref{stmt.dcl})). In this context, \textit{u} is said
to \defn{own} \tcode{p}.

\pnum
The mechanism by which \textit{u} disposes of \textit{p} is known as
\textit{p}'s associated \defn{deleter}, a function object whose correct
invocation results in \textit{p}'s appropriate disposition (typically its deletion).

\pnum
Let the notation \textit{u.p} denote the pointer stored by \textit{u}, and
let \textit{u.d} denote the associated deleter. Upon request, \textit{u} can
\defn{reset} (replace) \textit{u.p} and \textit{u.d} with another pointer and
deleter, but must properly dispose of its owned object via the associated
deleter before such replacement is considered completed.

\pnum
Additionally, \textit{u} can, upon request, \defn{transfer ownership} to another
unique pointer \textit{u2}. Upon completion of such a transfer, the following
postconditions hold:

\begin{itemize}
\item \textit{u2.p} is equal to the pre-transfer \textit{u.p},
\item \textit{u.p} is equal to \tcode{nullptr}, and
\item if the pre-transfer \textit{u.d} maintained state, such state has been
transferred to \textit{u2.d}.
\end{itemize}

As in the case of a reset, \textit{u2} must properly dispose of its pre-transfer
owned object via the pre-transfer associated deleter before the ownership
transfer is considered complete. \begin{note} A deleter's state need never be
copied, only moved or swapped as ownership is transferred. \end{note}

\pnum
Each object of a type \tcode{U} instantiated from the \tcode{unique_ptr} template
specified in this subclause has the strict ownership semantics, specified above,
of a unique pointer. In partial satisfaction of these semantics, each such \tcode{U}
is \tcode{MoveConstructible} and \tcode{MoveAssignable}, but is not
\tcode{CopyConstructible} nor \tcode{CopyAssignable}.
The template parameter \tcode{T} of \tcode{unique_ptr} may be an incomplete type.

\pnum
\begin{note} The uses
of \tcode{unique_ptr} include providing exception safety for
dynamically allocated memory, passing ownership of dynamically allocated
memory to a function, and returning dynamically allocated memory from a
function. \end{note}

\begin{codeblock}
namespace std {
  template<class T> struct default_delete;
  template<class T> struct default_delete<T[]>;

  template<class T, class D = default_delete<T>> class unique_ptr;
  template<class T, class D> class unique_ptr<T[], D>;

  template<class T, class... Args> unique_ptr<T> make_unique(Args&&... args);
  template<class T> unique_ptr<T> make_unique(size_t n);
  template<class T, class... Args> @\unspec@ make_unique(Args&&...) = delete;

  template<class T, class D> void swap(unique_ptr<T, D>& x, unique_ptr<T, D>& y) noexcept;

  template<class T1, class D1, class T2, class D2>
    bool operator==(const unique_ptr<T1, D1>& x, const unique_ptr<T2, D2>& y);
  template<class T1, class D1, class T2, class D2>
    bool operator!=(const unique_ptr<T1, D1>& x, const unique_ptr<T2, D2>& y);
  template<class T1, class D1, class T2, class D2>
    bool operator<(const unique_ptr<T1, D1>& x, const unique_ptr<T2, D2>& y);
  template<class T1, class D1, class T2, class D2>
    bool operator<=(const unique_ptr<T1, D1>& x, const unique_ptr<T2, D2>& y);
  template<class T1, class D1, class T2, class D2>
    bool operator>(const unique_ptr<T1, D1>& x, const unique_ptr<T2, D2>& y);
  template<class T1, class D1, class T2, class D2>
    bool operator>=(const unique_ptr<T1, D1>& x, const unique_ptr<T2, D2>& y);

  template <class T, class D>
    bool operator==(const unique_ptr<T, D>& x, nullptr_t) noexcept;
  template <class T, class D>
    bool operator==(nullptr_t, const unique_ptr<T, D>& y) noexcept;
  template <class T, class D>
    bool operator!=(const unique_ptr<T, D>& x, nullptr_t) noexcept;
  template <class T, class D>
    bool operator!=(nullptr_t, const unique_ptr<T, D>& y) noexcept;
  template <class T, class D>
    bool operator<(const unique_ptr<T, D>& x, nullptr_t);
  template <class T, class D>
    bool operator<(nullptr_t, const unique_ptr<T, D>& y);
  template <class T, class D>
    bool operator<=(const unique_ptr<T, D>& x, nullptr_t);
  template <class T, class D>
    bool operator<=(nullptr_t, const unique_ptr<T, D>& y);
  template <class T, class D>
    bool operator>(const unique_ptr<T, D>& x, nullptr_t);
  template <class T, class D>
    bool operator>(nullptr_t, const unique_ptr<T, D>& y);
  template <class T, class D>
    bool operator>=(const unique_ptr<T, D>& x, nullptr_t);
  template <class T, class D>
    bool operator>=(nullptr_t, const unique_ptr<T, D>& y);

}
\end{codeblock}

\rSec3[unique.ptr.dltr]{Default deleters}

\rSec4[unique.ptr.dltr.general]{In general}

\pnum
The class template \tcode{default_delete} serves as the default deleter (destruction policy)
for the class template \tcode{unique_ptr}.

\pnum
The template parameter \tcode{T} of \tcode{default_delete} may be
an incomplete type.

\rSec4[unique.ptr.dltr.dflt]{\tcode{default_delete}}

\begin{codeblock}
namespace std {
  template <class T> struct default_delete {
    constexpr default_delete() noexcept = default;
    template <class U> default_delete(const default_delete<U>&) noexcept;
    void operator()(T*) const;
  };
}
\end{codeblock}

\indexlibrary{\idxcode{default_delete}!constructor}%
\begin{itemdecl}
template <class U> default_delete(const default_delete<U>& other) noexcept;
\end{itemdecl}

\begin{itemdescr}
\pnum
\effects Constructs a \tcode{default_delete} object
from another \tcode{default_delete<U>} object.

\pnum
\remarks This constructor shall not participate in overload resolution unless
\tcode{U*} is implicitly convertible to \tcode{T*}.
\end{itemdescr}

\indexlibrarymember{operator()}{default_delete}%
\begin{itemdecl}
void operator()(T* ptr) const;
\end{itemdecl}

\begin{itemdescr}
\pnum
\effects Calls \tcode{delete} on \tcode{ptr}.

\pnum
\remarks If \tcode{T} is an incomplete type, the program is ill-formed.
\end{itemdescr}

\rSec4[unique.ptr.dltr.dflt1]{\tcode{default_delete<T[]>}}

\begin{codeblock}
namespace std {
  template <class T> struct default_delete<T[]> {
    constexpr default_delete() noexcept = default;
    template <class U> default_delete(const default_delete<U[]>&) noexcept;
    template <class U> void operator()(U* ptr) const;
  };
}
\end{codeblock}

\indexlibrary{\idxcode{default_delete}!constructor}
\begin{itemdecl}
template <class U> default_delete(const default_delete<U[]>& other) noexcept;
\end{itemdecl}

\begin{itemdescr}
\pnum
\effects
constructs a \tcode{default_delete} object from another \tcode{default_delete<U[]>} object.

\pnum
\remarks
This constructor shall not participate in overload resolution unless \tcode{U(*)[]} is
convertible to \tcode{T(*)[]}.
\end{itemdescr}

\indexlibrarymember{operator()}{default_delete}%
\begin{itemdecl}
template <class U> void operator()(U* ptr) const;
\end{itemdecl}

\begin{itemdescr}
\pnum
\effects
Calls \tcode{delete[]} on \tcode{ptr}.

\pnum
\remarks If U is an incomplete type, the program is ill-formed.
This function shall not participate in overload resolution
unless \tcode{U(*)[]} is convertible to \tcode{T(*)[]}.
\end{itemdescr}

\rSec3[unique.ptr.single]{\tcode{unique_ptr} for single objects}

\indexlibrary{\idxcode{unique_ptr}}%
\begin{codeblock}
namespace std {
  template <class T, class D = default_delete<T>> class unique_ptr {
  public:
    using pointer      = @\seebelow@;
    using element_type = T;
    using deleter_type = D;

    // \ref{unique.ptr.single.ctor}, constructors
    constexpr unique_ptr() noexcept;
    explicit unique_ptr(pointer p) noexcept;
    unique_ptr(pointer p, @\seebelow@ d1) noexcept;
    unique_ptr(pointer p, @\seebelow@ d2) noexcept;
    unique_ptr(unique_ptr&& u) noexcept;
    constexpr unique_ptr(nullptr_t) noexcept
      : unique_ptr() { }
    template <class U, class E>
      unique_ptr(unique_ptr<U, E>&& u) noexcept;

    // \ref{unique.ptr.single.dtor}, destructor
    ~unique_ptr();

    // \ref{unique.ptr.single.asgn}, assignment
    unique_ptr& operator=(unique_ptr&& u) noexcept;
    template <class U, class E> unique_ptr& operator=(unique_ptr<U, E>&& u) noexcept;
    unique_ptr& operator=(nullptr_t) noexcept;

    // \ref{unique.ptr.single.observers}, observers
    add_lvalue_reference_t<T> operator*() const;
    pointer operator->() const noexcept;
    pointer get() const noexcept;
    deleter_type& get_deleter() noexcept;
    const deleter_type& get_deleter() const noexcept;
    explicit operator bool() const noexcept;

    // \ref{unique.ptr.single.modifiers} modifiers
    pointer release() noexcept;
    void reset(pointer p = pointer()) noexcept;
    void swap(unique_ptr& u) noexcept;

    // disable copy from lvalue
    unique_ptr(const unique_ptr&) = delete;
    unique_ptr& operator=(const unique_ptr&) = delete;
  };
}
\end{codeblock}

\pnum
The default type for the template parameter \tcode{D} is
\tcode{default_delete}. A client-supplied template argument
\tcode{D} shall be a function
object type~(\ref{function.objects}), lvalue reference to function, or
lvalue reference to function object type
for which, given
a value \tcode{d} of type \tcode{D} and a value
\tcode{ptr} of type \tcode{unique_ptr<T, D>::pointer}, the expression
\tcode{d(ptr)} is valid and has the effect of disposing of the
pointer as appropriate for that deleter.

\pnum
If the deleter's type \tcode{D} is not a reference type, \tcode{D} shall satisfy
the requirements of \tcode{Destructible} (Table~\ref{tab:destructible}).

\pnum
If the \grammarterm{qualified-id} \tcode{remove_reference_t<D>::pointer} is valid and denotes a
type~(\ref{temp.deduct}), then \tcode{unique_ptr<T,
D>::pointer} shall be a synonym for \tcode{remove_reference_t<D>::pointer}. Otherwise
\tcode{unique_ptr<T, D>::pointer} shall be a synonym for \tcode{element_type*}. The type \tcode{unique_ptr<T,
D>::pointer} shall
satisfy the requirements of \tcode{NullablePointer} (\ref{nullablepointer.requirements}).

\pnum
\begin{example} Given an allocator type \tcode{X}~(\ref{allocator.requirements}) and
letting \tcode{A} be a synonym for \tcode{allocator_traits<X>}, the types \tcode{A::pointer},
\tcode{A::const_pointer}, \tcode{A::void_pointer}, and \tcode{A::const_void_pointer}
may be used as \tcode{unique_ptr<T, D>::pointer}. \end{example}

\rSec4[unique.ptr.single.ctor]{\tcode{unique_ptr} constructors}

\indexlibrary{\idxcode{unique_ptr}!constructor}%
\begin{itemdecl}
constexpr unique_ptr() noexcept;
\end{itemdecl}

\begin{itemdescr}
\pnum
\requires \tcode{D} shall
satisfy the requirements of \tcode{DefaultConstructible} (Table~\ref{tab:defaultconstructible}),
and that construction shall not throw an exception.

\pnum
\effects Constructs a \tcode{unique_ptr} object that owns
nothing, value-initializing the stored pointer and the stored deleter.

\pnum
\postconditions \tcode{get() == nullptr}. \tcode{get_deleter()}
returns a reference to the stored deleter.

\pnum
\remarks If this constructor is instantiated with a pointer type or reference type
for the template argument \tcode{D}, the program is ill-formed.
\end{itemdescr}

\indexlibrary{\idxcode{unique_ptr}!constructor}%
\begin{itemdecl}
explicit unique_ptr(pointer p) noexcept;
\end{itemdecl}

\begin{itemdescr}
\pnum
\requires \tcode{D} shall
satisfy the requirements of \tcode{DefaultConstructible} (Table~\ref{tab:defaultconstructible}),
and that construction shall not throw an exception.

\pnum
\effects Constructs a \tcode{unique_ptr} which owns
\tcode{p}, initializing the stored pointer with \tcode{p} and
value-initializing the stored deleter.

\pnum
\postconditions \tcode{get() == p}. \tcode{get_deleter()}
returns a reference to the stored deleter.

\pnum
\remarks If this constructor is instantiated with a pointer type or reference type
for the template argument \tcode{D}, the program is ill-formed.
\end{itemdescr}

\indexlibrary{\idxcode{unique_ptr}!constructor}%
\begin{itemdecl}
unique_ptr(pointer p, @\seebelow@ d1) noexcept;
unique_ptr(pointer p, @\seebelow@ d2) noexcept;
\end{itemdecl}

\begin{itemdescr}
\pnum
The signature of these constructors depends upon whether \tcode{D}
is a reference type. If \tcode{D} is a non-reference type
\tcode{A}, then the signatures are:

\begin{codeblock}
unique_ptr(pointer p, const A& d);
unique_ptr(pointer p, A&& d);
\end{codeblock}

\pnum
If \tcode{D} is an lvalue reference type \tcode{A\&},
then the signatures are:

\begin{codeblock}
unique_ptr(pointer p, A& d);
unique_ptr(pointer p, A&& d);
\end{codeblock}

\pnum
If \tcode{D} is an lvalue reference type \tcode{const A\&},
then the signatures are:

\begin{codeblock}
unique_ptr(pointer p, const A& d);
unique_ptr(pointer p, const A&& d);
\end{codeblock}

\pnum
\requires
\begin{itemize}
\item If \tcode{D} is not an lvalue reference type then

\begin{itemize}
\item If \tcode{d} is an lvalue or \tcode{const} rvalue then
the first constructor of this pair will be selected. \tcode{D}
shall satisfy the requirements of
\tcode{CopyConstructible} (Table~\ref{tab:copyconstructible}), and
the copy constructor of \tcode{D} shall
not throw an exception.
This \tcode{unique_ptr} will hold
a copy of \tcode{d}.

\item Otherwise, \tcode{d} is a non-const rvalue and the second
constructor of this pair will be selected. \tcode{D}
shall satisfy the requirements of
\tcode{MoveConstructible} (Table~\ref{tab:moveconstructible}), and the
move constructor of \tcode{D} shall not throw an exception.
This \tcode{unique_ptr} will
hold a value move constructed from \tcode{d}.
\end{itemize}

\item Otherwise \tcode{D} is an lvalue reference type. \tcode{d}
shall be reference-compatible with one of the constructors. If \tcode{d} is
an rvalue, it will bind to the second constructor of this pair and the program is
ill-formed. \begin{note} The diagnostic could
be implemented using a \tcode{static_assert} which assures that
\tcode{D} is not a reference type. \end{note} Else \tcode{d}
is an lvalue and will bind to the first constructor of this pair. The type
which \tcode{D} references need not be \tcode{CopyConstructible}
nor \tcode{MoveConstructible}. This \tcode{unique_ptr} will
hold a \tcode{D} which refers to the lvalue \tcode{d}.
\begin{note} \tcode{D} may not be an rvalue reference type.
\end{note}
\end{itemize}

\pnum
\effects Constructs a \tcode{unique_ptr} object which owns \tcode{p}, initializing
the stored pointer with \tcode{p} and initializing the deleter as described above.

\pnum
\postconditions \tcode{get() == p}.
\tcode{get_deleter()} returns a reference to the stored
deleter. If \tcode{D} is a reference type then \tcode{get_deleter()}
returns a reference to the lvalue \tcode{d}.

\begin{example}

\begin{codeblock}
D d;
unique_ptr<int, D> p1(new int, D());        // \tcode{D} must be \tcode{MoveConstructible}
unique_ptr<int, D> p2(new int, d);          // \tcode{D} must be \tcode{CopyConstructible}
unique_ptr<int, D&> p3(new int, d);         // \tcode{p3} holds a reference to \tcode{d}
unique_ptr<int, const D&> p4(new int, D()); // error: rvalue deleter object combined
                                            // with reference deleter type
\end{codeblock}

\end{example}

\end{itemdescr}

\indexlibrary{\idxcode{unique_ptr}!constructor}%
\begin{itemdecl}
unique_ptr(unique_ptr&& u) noexcept;
\end{itemdecl}

\begin{itemdescr}
\pnum
\requires If \tcode{D} is not a reference type,
\tcode{D} shall satisfy the requirements of \tcode{MoveConstructible}
(Table~\ref{tab:moveconstructible}).
Construction
of the deleter from an rvalue of type \tcode{D} shall not
throw an exception.

\pnum
\effects Constructs a \tcode{unique_ptr} by transferring ownership from
\tcode{u} to \tcode{*this}. If \tcode{D} is a reference type, this
deleter is copy constructed from \tcode{u}'s deleter; otherwise, this
deleter is move constructed from \tcode{u}'s deleter. \begin{note} The
deleter constructor can be implemented with \tcode{std::forward<D>}. \end{note}

\pnum
\postconditions \tcode{get()} yields the value \tcode{u.get()}
yielded before the construction. \tcode{get_deleter()} returns a reference
to the stored deleter that was constructed from
\tcode{u.get_deleter()}. If \tcode{D} is a reference type then
\tcode{get_deleter()} and \tcode{u.get_deleter()} both reference
the same lvalue deleter.
\end{itemdescr}

\indexlibrary{\idxcode{unique_ptr}!constructor}%
\begin{itemdecl}
template <class U, class E> unique_ptr(unique_ptr<U, E>&& u) noexcept;
\end{itemdecl}

\begin{itemdescr}
\pnum
\requires If \tcode{E} is not a reference type,
construction of the deleter from an rvalue of type
\tcode{E} shall be well formed and shall not throw an exception.
Otherwise, \tcode{E} is a reference type and construction of the deleter from an
lvalue of type \tcode{E} shall be well formed and shall not throw an exception.

\pnum
\remarks This constructor shall not participate in overload resolution unless:

\begin{itemize}
\item \tcode{unique_ptr<U, E>::pointer} is implicitly convertible to \tcode{pointer},
\item \tcode{U} is not an array type, and
\item either \tcode{D} is a reference type and \tcode{E} is the same type as \tcode{D}, or
\tcode{D} is not a reference type and \tcode{E} is implicitly convertible to \tcode{D}.
\end{itemize}

\pnum
\effects Constructs a \tcode{unique_ptr} by transferring ownership from \tcode{u}
to \tcode{*this}. If \tcode{E} is a reference type, this deleter is copy constructed from
\tcode{u}'s deleter; otherwise, this deleter is move constructed from \tcode{u}'s
deleter. \begin{note} The deleter constructor can be implemented with
\tcode{std::forward<E>}. \end{note}

\pnum
\postconditions \tcode{get()} yields the value \tcode{u.get()}
yielded before the construction.
\tcode{get_deleter()} returns a reference
to the stored deleter that was constructed from
\tcode{u.get_deleter()}.
\end{itemdescr}

\rSec4[unique.ptr.single.dtor]{\tcode{unique_ptr} destructor}

\indexlibrary{\idxcode{unique_ptr}!destructor}%
\begin{itemdecl}
~unique_ptr();
\end{itemdecl}

\begin{itemdescr}
\pnum
\requires The expression \tcode{get_deleter()(get())} shall be well formed,
shall have well-defined behavior, and shall not throw exceptions. \begin{note} The
use of \tcode{default_delete} requires \tcode{T} to be a complete type.
\end{note}

\pnum
\effects If \tcode{get() == nullptr} there are no effects.
Otherwise \tcode{get_deleter()(get())}.
\end{itemdescr}

\rSec4[unique.ptr.single.asgn]{\tcode{unique_ptr} assignment}

\indexlibrarymember{operator=}{unique_ptr}%
\begin{itemdecl}
unique_ptr& operator=(unique_ptr&& u) noexcept;
\end{itemdecl}

\begin{itemdescr}
\pnum
\requires If \tcode{D} is not a reference type, \tcode{D} shall satisfy the
requirements of \tcode{MoveAssignable} (Table~\ref{tab:moveassignable}) and assignment
of the deleter from an rvalue of type \tcode{D} shall not throw an exception.
Otherwise, \tcode{D} is a reference type;
\tcode{remove_reference_t<D>} shall satisfy the \tcode{CopyAssignable}
requirements and assignment of the deleter from an
lvalue of type \tcode{D} shall not throw an exception.

\pnum
\effects
Transfers ownership from \tcode{u} to \tcode{*this} as if by calling
\tcode{reset(u.release())} followed by
\tcode{get_deleter() = std::forward<D>(u.get_deleter())}.

\pnum
\returns \tcode{*this}.
\end{itemdescr}

\indexlibrarymember{operator=}{unique_ptr}%
\begin{itemdecl}
template <class U, class E> unique_ptr& operator=(unique_ptr<U, E>&& u) noexcept;
\end{itemdecl}

\begin{itemdescr}
\pnum
\requires If \tcode{E} is not a reference type, assignment of the deleter from
an rvalue of type \tcode{E} shall be well-formed and shall not throw an exception.
Otherwise, \tcode{E} is a reference type and assignment of the deleter from an lvalue
of type \tcode{E} shall be well-formed and shall not throw an exception.

\pnum
\remarks This operator shall not participate in overload resolution unless:

\begin{itemize}
\item \tcode{unique_ptr<U, E>::pointer} is implicitly convertible to \tcode{pointer}, and
\item \tcode{U} is not an array type, and
\item \tcode{is_assignable_v<D\&, E\&\&>} is \tcode{true}.
\end{itemize}

\pnum
\effects Transfers ownership from \tcode{u} to \tcode{*this} as if by calling
\tcode{reset(u.release())} followed by
\tcode{get_deleter() = std::forward<E>(u.get_deleter())}.

\pnum
\returns \tcode{*this}.
\end{itemdescr}

\indexlibrarymember{operator=}{unique_ptr}%
\begin{itemdecl}
unique_ptr& operator=(nullptr_t) noexcept;
\end{itemdecl}

\begin{itemdescr}
\pnum
\effects As if by \tcode{reset()}.

\pnum
\postconditions \tcode{get() == nullptr}.

\pnum
\returns \tcode{*this}.
\end{itemdescr}

\rSec4[unique.ptr.single.observers]{\tcode{unique_ptr} observers}

\indexlibrarymember{operator*}{unique_ptr}%
\begin{itemdecl}
add_lvalue_reference_t<T> operator*() const;
\end{itemdecl}

\begin{itemdescr}
\pnum
\requires \tcode{get() != nullptr}.

\pnum
\returns \tcode{*get()}.

\end{itemdescr}

\indexlibrarymember{operator->}{unique_ptr}%
\begin{itemdecl}
pointer operator->() const noexcept;
\end{itemdecl}

\begin{itemdescr}
\pnum
\requires \tcode{get() != nullptr}.

\pnum
\returns \tcode{get()}.

\pnum
\realnote use typically requires that \tcode{T} be a complete type.
\end{itemdescr}

\indexlibrarymember{get}{unique_ptr}%
\begin{itemdecl}
pointer get() const noexcept;
\end{itemdecl}

\begin{itemdescr}
\pnum
\returns The stored pointer.
\end{itemdescr}

\indexlibrarymember{get_deleter}{unique_ptr}%
\begin{itemdecl}
deleter_type& get_deleter() noexcept;
const deleter_type& get_deleter() const noexcept;
\end{itemdecl}

\begin{itemdescr}
\pnum
\returns A reference to the stored deleter.
\end{itemdescr}

\indexlibrarymember{operator bool}{unique_ptr}%
\begin{itemdecl}
explicit operator bool() const noexcept;
\end{itemdecl}

\begin{itemdescr}
\pnum
\returns \tcode{get() != nullptr}.
\end{itemdescr}

\rSec4[unique.ptr.single.modifiers]{\tcode{unique_ptr} modifiers}

\indexlibrarymember{release}{unique_ptr}%
\begin{itemdecl}
pointer release() noexcept;
\end{itemdecl}

\begin{itemdescr}
\pnum
\postconditions \tcode{get() == nullptr}.

\pnum
\returns The value \tcode{get()} had at the start of
the call to \tcode{release}.
\end{itemdescr}

\indexlibrarymember{reset}{unique_ptr}%
\begin{itemdecl}
void reset(pointer p = pointer()) noexcept;
\end{itemdecl}

\begin{itemdescr}
\pnum
\requires The expression \tcode{get_deleter()(get())} shall be well formed, shall have
well-defined behavior, and shall not throw exceptions.

\pnum
\effects Assigns \tcode{p} to the stored pointer, and then if the old value of the
stored pointer, \tcode{old_p}, was not equal to \tcode{nullptr}, calls
\tcode{get_deleter()(old_p)}. \begin{note} The order of these operations is significant
because the call to \tcode{get_deleter()} may destroy \tcode{*this}. \end{note}

\pnum
\postconditions \tcode{get() == p}.
\begin{note} The postcondition does not hold if the call to \tcode{get_deleter()}
destroys \tcode{*this} since \tcode{this->get()} is no longer a valid expression.
\end{note}
\end{itemdescr}

\indexlibrarymember{swap}{unique_ptr}%
\begin{itemdecl}
void swap(unique_ptr& u) noexcept;
\end{itemdecl}

\begin{itemdescr}
\pnum
\requires \tcode{get_deleter()} shall be
swappable~(\ref{swappable.requirements}) and shall
not throw an exception
under \tcode{swap}.

\pnum
\effects Invokes \tcode{swap} on the stored pointers and on the stored
deleters of \tcode{*this} and \tcode{u}.
\end{itemdescr}

\rSec3[unique.ptr.runtime]{\tcode{unique_ptr} for array objects with a runtime length}

\indexlibrary{\idxcode{unique_ptr}}%
\begin{codeblock}
namespace std {
  template <class T, class D> class unique_ptr<T[], D> {
  public:
    using pointer      = @\seebelow@;
    using element_type = T;
    using deleter_type = D;

    // \ref{unique.ptr.runtime.ctor}, constructors
    constexpr unique_ptr() noexcept;
    template <class U> explicit unique_ptr(U p) noexcept;
    template <class U> unique_ptr(U p, @\seebelow@ d) noexcept;
    template <class U> unique_ptr(U p, @\seebelow@ d) noexcept;
    unique_ptr(unique_ptr&& u) noexcept;
    template <class U, class E>
      unique_ptr(unique_ptr<U, E>&& u) noexcept;
    constexpr unique_ptr(nullptr_t) noexcept : unique_ptr() { }

    // destructor
    ~unique_ptr();

    // assignment
    unique_ptr& operator=(unique_ptr&& u) noexcept;
    template <class U, class E>
      unique_ptr& operator=(unique_ptr<U, E>&& u) noexcept;
    unique_ptr& operator=(nullptr_t) noexcept;

    // \ref{unique.ptr.runtime.observers}, observers
    T& operator[](size_t i) const;
    pointer get() const noexcept;
    deleter_type& get_deleter() noexcept;
    const deleter_type& get_deleter() const noexcept;
    explicit operator bool() const noexcept;

    // \ref{unique.ptr.runtime.modifiers} modifiers
    pointer release() noexcept;
    template <class U> void reset(U p) noexcept;
    void reset(nullptr_t = nullptr) noexcept;
    void swap(unique_ptr& u) noexcept;

    // disable copy from lvalue
    unique_ptr(const unique_ptr&) = delete;
    unique_ptr& operator=(const unique_ptr&) = delete;
  };
}
\end{codeblock}

\pnum
A specialization for array types is provided with a slightly altered
interface.

\begin{itemize}
\item Conversions between different types of
\tcode{unique_ptr<T[], D>}
that would be disallowed for the corresponding pointer-to-array types,
and conversions to or from the non-array forms of
\tcode{unique_ptr}, produce an ill-formed program.

\item Pointers to types derived from \tcode{T} are
rejected by the constructors, and by \tcode{reset}.

\item The observers \tcode{operator*} and
\tcode{operator->} are not provided.

\item The indexing observer \tcode{operator[]} is provided.

\item The default deleter will call \tcode{delete[]}.
\end{itemize}

\pnum
Descriptions are provided below only for members that
differ from the primary template.

\pnum
The template argument \tcode{T} shall be a complete type.

\rSec4[unique.ptr.runtime.ctor]{\tcode{unique_ptr} constructors}

\indexlibrary{\idxcode{unique_ptr}!constructor}%
\begin{itemdecl}
template <class U> explicit unique_ptr(U p) noexcept;
template <class U> unique_ptr(U p, @\seebelow@ d) noexcept;
template <class U> unique_ptr(U p, @\seebelow@ d) noexcept;
\end{itemdecl}

\begin{itemdescr}
\pnum
These constructors behave the same as
the constructors that take a \tcode{pointer} parameter
in the primary template
except that they
shall not participate in overload resolution unless either

\begin{itemize}
\item \tcode{U} is the same type as \tcode{pointer},
\item \tcode{U} is \tcode{nullptr_t}, or
\item \tcode{pointer} is the same type as \tcode{element_type*},
      \tcode{U} is a pointer type \tcode{V*}, and
      \tcode{V(*)[]} is convertible to \tcode{element_type(*)[]}.
\end{itemize}
\end{itemdescr}

\indexlibrary{\idxcode{unique_ptr}!constructor}%
\begin{itemdecl}
template <class U, class E>
  unique_ptr(unique_ptr<U, E>&& u) noexcept;
\end{itemdecl}

\begin{itemdescr}
\pnum
This constructor behaves the same as in the primary template,
except that it shall not participate in overload resolution
unless all of the following conditions hold,
where \tcode{UP} is \tcode{unique_ptr<U, E>}:

\begin{itemize}
\item \tcode{U} is an array type, and
\item \tcode{pointer} is the same type as \tcode{element_type*}, and
\item \tcode{UP::pointer} is the same type as \tcode{UP::element_type*}, and
\item \tcode{UP::element_type(*)[]} is convertible to \tcode{element_type(*)[]}, and
\item either \tcode{D} is a reference type and \tcode{E} is the same type as \tcode{D},
      or \tcode{D} is not a reference type and \tcode{E} is implicitly convertible to \tcode{D}.
\end{itemize}

\begin{note}
This replaces the overload-resolution specification of the primary template
\end{note}
\end{itemdescr}

\rSec4[unique.ptr.runtime.asgn]{\tcode{unique_ptr} assignment}

\indexlibrarymember{operator=}{unique_ptr}%
\begin{itemdecl}
template <class U, class E>
  unique_ptr& operator=(unique_ptr<U, E>&& u)noexcept;
\end{itemdecl}

\begin{itemdescr}
\pnum
This operator behaves the same as in the primary template,
except that it shall not participate in overload resolution
unless all of the following conditions hold,
where \tcode{UP} is \tcode{unique_ptr<U, E>}:

\begin{itemize}
\item \tcode{U} is an array type, and
\item \tcode{pointer} is the same type as \tcode{element_type*}, and
\item \tcode{UP::pointer} is the same type as \tcode{UP::element_type*}, and
\item \tcode{UP::element_type(*)[]} is convertible to \tcode{element_type(*)[]}, and
\item \tcode{is_assignable_v<D\&, E\&\&>} is \tcode{true}.
\end{itemize}

\begin{note}
This replaces the overload-resolution specification of the primary template
\end{note}
\end{itemdescr}

\rSec4[unique.ptr.runtime.observers]{\tcode{unique_ptr} observers}

\indexlibrarymember{operator[]}{unique_ptr}%
\begin{itemdecl}
T& operator[](size_t i) const;
\end{itemdecl}

\begin{itemdescr}
\pnum
\requires \tcode{i <} the
number of elements in the array to which
the stored pointer points.

\pnum
\returns \tcode{get()[i]}.
\end{itemdescr}

\rSec4[unique.ptr.runtime.modifiers]{\tcode{unique_ptr} modifiers}

\indexlibrarymember{reset}{unique_ptr}%
\begin{itemdecl}
void reset(nullptr_t p = nullptr) noexcept;
\end{itemdecl}

\begin{itemdescr}
\pnum
\effects Equivalent to \tcode{reset(pointer())}.
\end{itemdescr}

\indexlibrarymember{reset}{unique_ptr}%
\begin{itemdecl}
template <class U> void reset(U p) noexcept;
\end{itemdecl}

\begin{itemdescr}
\pnum
This function behaves the same as
the \tcode{reset} member of the primary template,
except that it shall not participate in overload resolution
unless either

\begin{itemize}
\item \tcode{U} is the same type as \tcode{pointer}, or
\item \tcode{pointer} is the same type as \tcode{element_type*},
      \tcode{U} is a pointer type \tcode{V*}, and
      \tcode{V(*)[]} is convertible to \tcode{element_type(*)[]}.
\end{itemize}
\end{itemdescr}

\rSec3[unique.ptr.create]{\tcode{unique_ptr} creation}

\indexlibrary{\idxcode{make_unique}}%
\begin{itemdecl}
template <class T, class... Args> unique_ptr<T> make_unique(Args&&... args);
\end{itemdecl}

\begin{itemdescr}
\pnum
\remarks This function shall not participate in overload resolution unless \tcode{T} is not an array.

\pnum
\returns \tcode{unique_ptr<T>(new T(std::forward<Args>(args)...))}.

\end{itemdescr}

\indexlibrary{\idxcode{make_unique}}%
\begin{itemdecl}
template <class T> unique_ptr<T> make_unique(size_t n);
\end{itemdecl}

\begin{itemdescr}
\pnum
\remarks This function shall not participate in overload resolution unless \tcode{T} is an array of unknown bound.

\pnum
\returns \tcode{unique_ptr<T>(new remove_extent_t<T>[n]())}.

\end{itemdescr}

\indexlibrary{\idxcode{make_unique}}%
\begin{itemdecl}
template <class T, class... Args> @\unspec@ make_unique(Args&&...) = delete;
\end{itemdecl}

\begin{itemdescr}
\pnum
\remarks This function shall not participate in overload resolution unless \tcode{T} is an array of known bound.

\end{itemdescr}

\rSec3[unique.ptr.special]{\tcode{unique_ptr} specialized algorithms}

\indexlibrary{\idxcode{swap(unique_ptr\&, unique_ptr\&)}}%
\begin{itemdecl}
template <class T, class D> void swap(unique_ptr<T, D>& x, unique_ptr<T, D>& y) noexcept;
\end{itemdecl}

\begin{itemdescr}
\pnum
\remarks This function shall not participate in overload resolution
unless \tcode{is_swappable_v<D>} is \tcode{true}.

\pnum
\effects Calls \tcode{x.swap(y)}.
\end{itemdescr}

\indexlibrarymember{operator==}{unique_ptr}%
\begin{itemdecl}
template <class T1, class D1, class T2, class D2>
  bool operator==(const unique_ptr<T1, D1>& x, const unique_ptr<T2, D2>& y);
\end{itemdecl}

\begin{itemdescr}
\pnum
\returns \tcode{x.get() == y.get()}.
\end{itemdescr}

\indexlibrarymember{operator"!=}{unique_ptr}%
\begin{itemdecl}
template <class T1, class D1, class T2, class D2>
  bool operator!=(const unique_ptr<T1, D1>& x, const unique_ptr<T2, D2>& y);
\end{itemdecl}

\begin{itemdescr}
\pnum
\returns \tcode{x.get() != y.get()}.
\end{itemdescr}

\indexlibrarymember{operator<}{unique_ptr}%
\begin{itemdecl}
template <class T1, class D1, class T2, class D2>
  bool operator<(const unique_ptr<T1, D1>& x, const unique_ptr<T2, D2>& y);
\end{itemdecl}

\begin{itemdescr}
\pnum
\requires Let \tcode{\placeholder{CT}} denote
\begin{codeblock}
common_type_t<typename unique_ptr<T1, D1>::pointer,
              typename unique_ptr<T2, D2>::pointer>
\end{codeblock}
Then the specialization
\tcode{less<\placeholder{CT}>} shall be a function object type~(\ref{function.objects}) that
induces a strict weak ordering~(\ref{alg.sorting}) on the pointer values.

\pnum
\returns \tcode{less<\placeholder{CT}>()(x.get(), y.get())}.

\pnum
\remarks If \tcode{unique_ptr<T1, D1>::pointer} is not implicitly convertible
to \tcode{\placeholder{CT}} or \tcode{unique_ptr<T2, D2>::pointer} is not implicitly
convertible to \tcode{\placeholder{CT}}, the program is ill-formed.
\end{itemdescr}

\indexlibrarymember{operator<=}{unique_ptr}%
\begin{itemdecl}
template <class T1, class D1, class T2, class D2>
  bool operator<=(const unique_ptr<T1, D1>& x, const unique_ptr<T2, D2>& y);
\end{itemdecl}

\begin{itemdescr}
\pnum
\returns \tcode{!(y < x)}.
\end{itemdescr}

\indexlibrarymember{operator>}{unique_ptr}%
\begin{itemdecl}
template <class T1, class D1, class T2, class D2>
  bool operator>(const unique_ptr<T1, D1>& x, const unique_ptr<T2, D2>& y);
\end{itemdecl}

\begin{itemdescr}
\pnum
\returns \tcode{y < x}.
\end{itemdescr}

\indexlibrarymember{operator>=}{unique_ptr}%
\begin{itemdecl}
template <class T1, class D1, class T2, class D2>
  bool operator>=(const unique_ptr<T1, D1>& x, const unique_ptr<T2, D2>& y);
\end{itemdecl}

\begin{itemdescr}
\pnum
\returns \tcode{!(x < y)}.
\end{itemdescr}

\indexlibrarymember{operator==}{unique_ptr}%
\begin{itemdecl}
template <class T, class D>
  bool operator==(const unique_ptr<T, D>& x, nullptr_t) noexcept;
template <class T, class D>
  bool operator==(nullptr_t, const unique_ptr<T, D>& x) noexcept;
\end{itemdecl}

\begin{itemdescr}
\pnum
\returns \tcode{!x}.
\end{itemdescr}

\indexlibrarymember{operator"!=}{unique_ptr}%
\begin{itemdecl}
template <class T, class D>
  bool operator!=(const unique_ptr<T, D>& x, nullptr_t) noexcept;
template <class T, class D>
  bool operator!=(nullptr_t, const unique_ptr<T, D>& x) noexcept;
\end{itemdecl}

\begin{itemdescr}
\pnum
\returns \tcode{(bool)x}.
\end{itemdescr}

\indexlibrarymember{operator<}{unique_ptr}%
\begin{itemdecl}
template <class T, class D>
  bool operator<(const unique_ptr<T, D>& x, nullptr_t);
template <class T, class D>
  bool operator<(nullptr_t, const unique_ptr<T, D>& x);
\end{itemdecl}

\begin{itemdescr}
\pnum
\requires The specialization \tcode{less<unique_ptr<T, D>::pointer>} shall be
a function object type~(\ref{function.objects}) that induces a strict weak
ordering~(\ref{alg.sorting}) on the pointer values.

\pnum
\returns
The first function template returns
\tcode{less<unique_ptr<T, D>::pointer>()(x.get(),\\nullptr)}.
The second function template returns
\tcode{less<unique_ptr<T, D>::pointer>()(nullptr, x.get())}.
\end{itemdescr}

\indexlibrarymember{operator>}{unique_ptr}%
\begin{itemdecl}
template <class T, class D>
  bool operator>(const unique_ptr<T, D>& x, nullptr_t);
template <class T, class D>
  bool operator>(nullptr_t, const unique_ptr<T, D>& x);
\end{itemdecl}

\begin{itemdescr}
\pnum
\returns
The first function template returns \tcode{nullptr < x}.
The second function template returns \tcode{x < nullptr}.
\end{itemdescr}

\indexlibrarymember{operator<=}{unique_ptr}%
\begin{itemdecl}
template <class T, class D>
  bool operator<=(const unique_ptr<T, D>& x, nullptr_t);
template <class T, class D>
  bool operator<=(nullptr_t, const unique_ptr<T, D>& x);
\end{itemdecl}

\begin{itemdescr}
\pnum
\returns
The first function template returns \tcode{!(nullptr < x)}.
The second function template returns \tcode{!(x < nullptr)}.
\end{itemdescr}

\indexlibrarymember{operator>=}{unique_ptr}%
\begin{itemdecl}
template <class T, class D>
  bool operator>=(const unique_ptr<T, D>& x, nullptr_t);
template <class T, class D>
  bool operator>=(nullptr_t, const unique_ptr<T, D>& x);
\end{itemdecl}

\begin{itemdescr}
\pnum
\returns
The first function template returns \tcode{!(x < nullptr)}.
The second function template returns \tcode{!(nullptr < x)}.
\end{itemdescr}

\indextext{smart pointers|(}%
\rSec2[util.smartptr]{Shared-ownership pointers}

\rSec3[util.smartptr.weak.bad]{Class \tcode{bad_weak_ptr}}
\indexlibrary{\idxcode{bad_weak_ptr}}%
\begin{codeblock}
namespace std {
  class bad_weak_ptr : public exception {
  public:
    bad_weak_ptr() noexcept;
  };
}
\end{codeblock}

\pnum
An exception of type \tcode{bad_weak_ptr} is thrown by the \tcode{shared_ptr}
constructor taking a \tcode{weak_ptr}.

\indexlibrary{\idxcode{bad_weak_ptr}!constructor}%
\indexlibrarymember{what}{bad_weak_ptr}%
\begin{itemdecl}
bad_weak_ptr() noexcept;
\end{itemdecl}

\begin{itemdescr}
\pnum\postconditions  \tcode{what()} returns an
\impldef{return value of \tcode{bad_weak_ptr::what}} \ntbs.

\end{itemdescr}

\rSec3[util.smartptr.shared]{Class template \tcode{shared_ptr}}

\pnum
\indexlibrary{\idxcode{shared_ptr}}%
The \tcode{shared_ptr} class template stores a pointer, usually obtained
via \tcode{new}. \tcode{shared_ptr} implements semantics of shared ownership;
the last remaining owner of the pointer is responsible for destroying
the object, or otherwise releasing the resources associated with the stored pointer. A
\tcode{shared_ptr} object is \term{empty} if it does not own a pointer.

\begin{codeblock}
namespace std {
  template<class T> class shared_ptr {
  public:
    using element_type = T;
    using weak_type    = weak_ptr<T>;

    // \ref{util.smartptr.shared.const}, constructors:
    constexpr shared_ptr() noexcept;
    template<class Y> explicit shared_ptr(Y* p);
    template<class Y, class D> shared_ptr(Y* p, D d);
    template<class Y, class D, class A> shared_ptr(Y* p, D d, A a);
    template <class D> shared_ptr(nullptr_t p, D d);
    template <class D, class A> shared_ptr(nullptr_t p, D d, A a);
    template<class Y> shared_ptr(const shared_ptr<Y>& r, T* p) noexcept;
    shared_ptr(const shared_ptr& r) noexcept;
    template<class Y> shared_ptr(const shared_ptr<Y>& r) noexcept;
    shared_ptr(shared_ptr&& r) noexcept;
    template<class Y> shared_ptr(shared_ptr<Y>&& r) noexcept;
    template<class Y> explicit shared_ptr(const weak_ptr<Y>& r);
    template <class Y, class D> shared_ptr(unique_ptr<Y, D>&& r);
    constexpr shared_ptr(nullptr_t) noexcept : shared_ptr() { }

    // \ref{util.smartptr.shared.dest}, destructor:
    ~shared_ptr();

    // \ref{util.smartptr.shared.assign}, assignment:
    shared_ptr& operator=(const shared_ptr& r) noexcept;
    template<class Y> shared_ptr& operator=(const shared_ptr<Y>& r) noexcept;
    shared_ptr& operator=(shared_ptr&& r) noexcept;
    template<class Y> shared_ptr& operator=(shared_ptr<Y>&& r) noexcept;
    template <class Y, class D> shared_ptr& operator=(unique_ptr<Y, D>&& r);

    // \ref{util.smartptr.shared.mod}, modifiers:
    void swap(shared_ptr& r) noexcept;
    void reset() noexcept;
    template<class Y> void reset(Y* p);
    template<class Y, class D> void reset(Y* p, D d);
    template<class Y, class D, class A> void reset(Y* p, D d, A a);

    // \ref{util.smartptr.shared.obs}, observers:
    T* get() const noexcept;
    T& operator*() const noexcept;
    T* operator->() const noexcept;
    long use_count() const noexcept;
    bool unique() const noexcept;
    explicit operator bool() const noexcept;
    template<class U> bool owner_before(shared_ptr<U> const& b) const;
    template<class U> bool owner_before(weak_ptr<U> const& b) const;
  };

  // \ref{util.smartptr.shared.create}, shared_ptr creation
  template<class T, class... Args> shared_ptr<T> make_shared(Args&&... args);
  template<class T, class A, class... Args>
    shared_ptr<T> allocate_shared(const A& a, Args&&... args);

  // \ref{util.smartptr.shared.cmp}, shared_ptr comparisons:
  template<class T, class U>
    bool operator==(const shared_ptr<T>& a, const shared_ptr<U>& b) noexcept;
  template<class T, class U>
    bool operator!=(const shared_ptr<T>& a, const shared_ptr<U>& b) noexcept;
  template<class T, class U>
    bool operator<(const shared_ptr<T>& a, const shared_ptr<U>& b) noexcept;
  template<class T, class U>
    bool operator>(const shared_ptr<T>& a, const shared_ptr<U>& b) noexcept;
  template<class T, class U>
    bool operator<=(const shared_ptr<T>& a, const shared_ptr<U>& b) noexcept;
  template<class T, class U>
    bool operator>=(const shared_ptr<T>& a, const shared_ptr<U>& b) noexcept;

  template <class T>
    bool operator==(const shared_ptr<T>& a, nullptr_t) noexcept;
  template <class T>
    bool operator==(nullptr_t, const shared_ptr<T>& b) noexcept;
  template <class T>
    bool operator!=(const shared_ptr<T>& a, nullptr_t) noexcept;
  template <class T>
    bool operator!=(nullptr_t, const shared_ptr<T>& b) noexcept;
  template <class T>
    bool operator<(const shared_ptr<T>& a, nullptr_t) noexcept;
  template <class T>
    bool operator<(nullptr_t, const shared_ptr<T>& b) noexcept;
  template <class T>
    bool operator<=(const shared_ptr<T>& a, nullptr_t) noexcept;
  template <class T>
    bool operator<=(nullptr_t, const shared_ptr<T>& b) noexcept;
  template <class T>
    bool operator>(const shared_ptr<T>& a, nullptr_t) noexcept;
  template <class T>
    bool operator>(nullptr_t, const shared_ptr<T>& b) noexcept;
  template <class T>
    bool operator>=(const shared_ptr<T>& a, nullptr_t) noexcept;
  template <class T>
    bool operator>=(nullptr_t, const shared_ptr<T>& b) noexcept;

  // \ref{util.smartptr.shared.spec}, shared_ptr specialized algorithms:
  template<class T> void swap(shared_ptr<T>& a, shared_ptr<T>& b) noexcept;

  // \ref{util.smartptr.shared.cast}, shared_ptr casts:
  template<class T, class U>
    shared_ptr<T> static_pointer_cast(const shared_ptr<U>& r) noexcept;
  template<class T, class U>
    shared_ptr<T> dynamic_pointer_cast(const shared_ptr<U>& r) noexcept;
  template<class T, class U>
    shared_ptr<T> const_pointer_cast(const shared_ptr<U>& r) noexcept;

  // \ref{util.smartptr.getdeleter}, shared_ptr get_deleter:
  template<class D, class T> D* get_deleter(const shared_ptr<T>& p) noexcept;

  // \ref{util.smartptr.shared.io}, shared_ptr I/O:
  template<class E, class T, class Y>
    basic_ostream<E, T>& operator<< (basic_ostream<E, T>& os, const shared_ptr<Y>& p);
}
\end{codeblock}

\pnum
Specializations of \tcode{shared_ptr} shall be \tcode{CopyConstructible},
\tcode{CopyAssignable}, and \tcode{LessThanComparable}, allowing their use in standard
containers. Specializations of \tcode{shared_ptr} shall be
contextually convertible to \tcode{bool},
allowing their use in boolean expressions and declarations in conditions. The template
parameter \tcode{T} of \tcode{shared_ptr} may be an incomplete type.

\pnum
\begin{example}
\begin{codeblock}
if (shared_ptr<X> px = dynamic_pointer_cast<X>(py)) {
  // do something with px
}
\end{codeblock}
\end{example}

\pnum
For purposes of determining the presence of a data race, member functions shall
access and modify only the \tcode{shared_ptr} and \tcode{weak_ptr} objects
themselves and not objects they refer to. Changes in \tcode{use_count()} do not
reflect modifications that can introduce data races.

\rSec4[util.smartptr.shared.const]{\tcode{shared_ptr} constructors}

\pnum
In the constructor definitions below,
\term{enables \tcode{shared_from_this} with \tcode{p}},
for a pointer \tcode{p} of type \tcode{Y*},
means that if \tcode{Y} has an unambiguous and accessible base class
that is a specialization of \tcode{enable_shared_from_this}~(\ref{util.smartptr.enab}),
then \tcode{remove_cv_t<Y>*} shall be implicitly convertible to \tcode{T*} and
the constructor evaluates the statement:
\begin{codeblock}
if (p != nullptr && p->weak_this.expired())
  p->weak_this = shared_ptr<remove_cv_t<Y>>(*this, const_cast<remove_cv_t<Y>*>(p));
\end{codeblock}
The assignment to the \tcode{weak_this} member is not atomic and
conflicts with any potentially concurrent access to the same object~(\ref{intro.multithread}).

\indexlibrary{\idxcode{shared_ptr}!constructor}%
\begin{itemdecl}
constexpr shared_ptr() noexcept;
\end{itemdecl}

\begin{itemdescr}
\pnum\effects  Constructs an \textit{empty} \tcode{shared_ptr} object.

\pnum\postconditions  \tcode{use_count() == 0 \&\& get() == nullptr}.
\end{itemdescr}

\indexlibrary{\idxcode{shared_ptr}!constructor}%
\begin{itemdecl}
template<class Y> explicit shared_ptr(Y* p);
\end{itemdecl}

\begin{itemdescr}
\pnum\requires  \tcode{p} shall be convertible to \tcode{T*}.
\tcode{Y} shall be a complete type. The expression \tcode{delete p}
shall be well formed, shall have well defined behavior, and shall not
throw exceptions.

\pnum\effects Constructs a \tcode{shared_ptr} object
that \textit{owns} the pointer \tcode{p}.
Enables \tcode{shared_from_this} with \tcode{p}.
If an exception is thrown, \tcode{delete p} is called.

\pnum\postconditions  \tcode{use_count() == 1 \&\& get() == p}.

\pnum\throws \tcode{bad_alloc}, or an \impldef{exception type when \tcode{shared_ptr}
constructor fails} exception when a resource other than memory could not be obtained.
\end{itemdescr}

\indexlibrary{\idxcode{shared_ptr}!constructor}%
\begin{itemdecl}
template<class Y, class D> shared_ptr(Y* p, D d);
template<class Y, class D, class A> shared_ptr(Y* p, D d, A a);
template <class D> shared_ptr(nullptr_t p, D d);
template <class D, class A> shared_ptr(nullptr_t p, D d, A a);
\end{itemdecl}

\begin{itemdescr}
\pnum\requires  \tcode{p} shall be convertible to \tcode{T*}. \tcode{D} shall be
\tcode{CopyConstructible} and such construction shall not throw exceptions.
The destructor of \tcode{D}
shall not throw exceptions. The expression \tcode{d(p)} shall be
well formed, shall have well defined behavior, and shall not throw exceptions.
\tcode{A} shall be an allocator~(\ref{allocator.requirements}).
The copy constructor and destructor of \tcode{A} shall not throw exceptions.

\pnum\effects  Constructs a \tcode{shared_ptr} object that \textit{owns} the
object \tcode{p} and the deleter \tcode{d}.
The first and second constructors enable \tcode{shared_from_this} with \tcode{p}.
The second and fourth constructors shall use a copy of \tcode{a} to
allocate memory for internal use.
If an exception is thrown, \tcode{d(p)} is called.

\pnum\postconditions  \tcode{use_count() == 1 \&\& get() == p}.

\pnum\throws  \tcode{bad_alloc}, or an \impldef{exception type when \tcode{shared_ptr}
constructor fails} exception
when a resource other than memory could not be obtained.
\end{itemdescr}

\indexlibrary{\idxcode{shared_ptr}!constructor}%
\begin{itemdecl}
template<class Y> shared_ptr(const shared_ptr<Y>& r, T* p) noexcept;
\end{itemdecl}

\begin{itemdescr}
\pnum
\effects Constructs a \tcode{shared_ptr} instance that
stores \tcode{p} and \textit{shares ownership} with \tcode{r}.

\pnum
\postconditions \tcode{get() == p \&\& use_count() == r.use_count()}.

\pnum
\begin{note} To avoid the possibility of a dangling pointer, the
user of this constructor must ensure that \tcode{p} remains valid at
least until the ownership group of \tcode{r} is destroyed. \end{note}

\pnum
\begin{note} This constructor allows creation of an \textit{empty}
\tcode{shared_ptr} instance with a non-null stored pointer. \end{note}
\end{itemdescr}

\indexlibrary{\idxcode{shared_ptr}!constructor}%
\begin{itemdecl}
shared_ptr(const shared_ptr& r) noexcept;
template<class Y> shared_ptr(const shared_ptr<Y>& r) noexcept;
\end{itemdecl}

\begin{itemdescr}
\pnum\remarks
The second constructor shall not participate in overload resolution unless
\tcode{Y*} is implicitly convertible to \tcode{T*}.

\pnum\effects  If \tcode{r} is \textit{empty}, constructs
an \textit{empty} \tcode{shared_ptr} object; otherwise, constructs
a \tcode{shared_ptr} object that \textit{shares ownership} with \tcode{r}.

\pnum\postconditions  \tcode{get() == r.get() \&\& use_count() == r.use_count()}.
\end{itemdescr}

\indexlibrary{\idxcode{shared_ptr}!constructor}%
\begin{itemdecl}
shared_ptr(shared_ptr&& r) noexcept;
template<class Y> shared_ptr(shared_ptr<Y>&& r) noexcept;
\end{itemdecl}

\begin{itemdescr}
\pnum
\remarks The second constructor shall not participate in overload resolution unless
\tcode{Y*} is convertible to \tcode{T*}.

\pnum
\effects Move constructs a \tcode{shared_ptr} instance from \tcode{r}.

\pnum
\postconditions \tcode{*this} shall contain the old value of
\tcode{r}. \tcode{r} shall be \textit{empty}. \tcode{r.get() == nullptr.}
\end{itemdescr}

\indexlibrary{\idxcode{shared_ptr}!constructor}%
\indexlibrary{\idxcode{weak_ptr}}%
\begin{itemdecl}
template<class Y> explicit shared_ptr(const weak_ptr<Y>& r);
\end{itemdecl}

\begin{itemdescr}
\pnum\requires \tcode{Y*} shall be convertible to \tcode{T*}.

\pnum\effects  Constructs a \tcode{shared_ptr} object that \textit{shares ownership} with
\tcode{r} and stores a copy of the pointer stored in \tcode{r}.
If an exception is thrown, the constructor has no effect.

\pnum\postconditions  \tcode{use_count() == r.use_count()}.

\pnum\throws  \tcode{bad_weak_ptr} when \tcode{r.expired()}.
\end{itemdescr}

\indexlibrary{\idxcode{shared_ptr}!constructor}%
\indexlibrary{\idxcode{unique_ptr}}%
\begin{itemdecl}
template <class Y, class D> shared_ptr(unique_ptr<Y, D>&& r);
\end{itemdecl}

\begin{itemdescr}
\pnum\remarks This constructor shall not participate in overload resolution
unless \tcode{unique_ptr<Y, D>::\brk{}pointer} is convertible to \tcode{T*}.

\pnum
\effects If \tcode{r.get() == nullptr}, equivalent to \tcode{shared_ptr()}.
Otherwise, if \tcode{D} is not a reference type,
equivalent to \tcode{shared_ptr(r.release(), r.get_deleter())}.
Otherwise, equivalent to \tcode{shared_ptr(r.release(), ref(r.get_deleter()))}.
If an exception is thrown, the constructor has no effect.
\end{itemdescr}

\rSec4[util.smartptr.shared.dest]{\tcode{shared_ptr} destructor}

\indexlibrary{\idxcode{shared_ptr}!destructor}%
\begin{itemdecl}
~shared_ptr();
\end{itemdecl}

\begin{itemdescr}
\pnum\effects
\begin{itemize}
\item If \tcode{*this} is \textit{empty} or shares ownership with another
\tcode{shared_ptr} instance (\tcode{use_count() > 1}), there are no side effects.

\item
Otherwise, if \tcode{*this} \textit{owns} an object
\tcode{p} and a deleter \tcode{d}, \tcode{d(p)} is called.

\item Otherwise, \tcode{*this} \textit{owns} a pointer \tcode{p},
and \tcode{delete p} is called.
\end{itemize}
\end{itemdescr}

\pnum
\begin{note} Since the destruction of \tcode{*this}
decreases the number of instances that share ownership with \tcode{*this}
by one,
after \tcode{*this} has been destroyed
all \tcode{shared_ptr} instances that shared ownership with
\tcode{*this} will report a \tcode{use_count()} that is one less
than its previous value. \end{note}

\rSec4[util.smartptr.shared.assign]{\tcode{shared_ptr} assignment}

\indexlibrarymember{operator=}{shared_ptr}%
\begin{itemdecl}
shared_ptr& operator=(const shared_ptr& r) noexcept;
template<class Y> shared_ptr& operator=(const shared_ptr<Y>& r) noexcept;
\end{itemdecl}

\begin{itemdescr}
\pnum\effects  Equivalent to \tcode{shared_ptr(r).swap(*this)}.

\pnum\returns  \tcode{*this}.

\pnum \begin{note}
The use count updates caused by the temporary object
construction and destruction are not observable side
effects, so the implementation may meet the effects (and the
implied guarantees) via different means, without creating a
temporary. In particular, in the example:
\begin{codeblock}
shared_ptr<int> p(new int);
shared_ptr<void> q(p);
p = p;
q = p;
\end{codeblock}
both assignments may be no-ops. \end{note}
\end{itemdescr}

\indexlibrarymember{operator=}{shared_ptr}%
\begin{itemdecl}
shared_ptr& operator=(shared_ptr&& r) noexcept;
template<class Y> shared_ptr& operator=(shared_ptr<Y>&& r) noexcept;
\end{itemdecl}

\begin{itemdescr}
\pnum
\effects Equivalent to \tcode{shared_ptr(std::move(r)).swap(*this)}.

\pnum
\returns \tcode{*this}.
\end{itemdescr}

\indexlibrarymember{operator=}{shared_ptr}%
\begin{itemdecl}
template <class Y, class D> shared_ptr& operator=(unique_ptr<Y, D>&& r);
\end{itemdecl}

\begin{itemdescr}
\pnum
\effects Equivalent to \tcode{shared_ptr(std::move(r)).swap(*this)}.

\pnum
\returns \tcode{*this}.
\end{itemdescr}



\rSec4[util.smartptr.shared.mod]{\tcode{shared_ptr} modifiers}

\indexlibrarymember{swap}{shared_ptr}%
\begin{itemdecl}
void swap(shared_ptr& r) noexcept;
\end{itemdecl}

\begin{itemdescr}

\pnum\effects  Exchanges the contents of \tcode{*this} and \tcode{r}.
\end{itemdescr}

\indexlibrarymember{reset}{shared_ptr}%
\begin{itemdecl}
void reset() noexcept;
\end{itemdecl}

\begin{itemdescr}
\pnum\effects  Equivalent to \tcode{shared_ptr().swap(*this)}.
\end{itemdescr}

\indexlibrarymember{reset}{shared_ptr}%
\begin{itemdecl}
template<class Y> void reset(Y* p);
\end{itemdecl}

\begin{itemdescr}
\pnum\effects  Equivalent to \tcode{shared_ptr(p).swap(*this)}.
\end{itemdescr}

\indexlibrarymember{reset}{shared_ptr}%
\begin{itemdecl}
template<class Y, class D> void reset(Y* p, D d);
\end{itemdecl}

\begin{itemdescr}
\pnum\effects  Equivalent to \tcode{shared_ptr(p, d).swap(*this)}.
\end{itemdescr}

\indexlibrarymember{reset}{shared_ptr}%
\begin{itemdecl}
template<class Y, class D, class A> void reset(Y* p, D d, A a);
\end{itemdecl}

\begin{itemdescr}
\pnum
\effects  Equivalent to \tcode{shared_ptr(p, d, a).swap(*this)}.
\end{itemdescr}

\rSec4[util.smartptr.shared.obs]{\tcode{shared_ptr} observers}
\indexlibrarymember{get}{shared_ptr}%
\begin{itemdecl}
T* get() const noexcept;
\end{itemdecl}

\begin{itemdescr}
\pnum\returns The stored pointer.
\end{itemdescr}

\indexlibrarymember{operator*}{shared_ptr}%
\begin{itemdecl}
T& operator*() const noexcept;
\end{itemdecl}

\begin{itemdescr}
\pnum\requires  \tcode{get() != 0}.

\pnum\returns  \tcode{*get()}.

\pnum\remarks When \tcode{T} is (possibly cv-qualified) \tcode{void},
it is unspecified whether this
member function is declared. If it is declared, it is unspecified what its
return type is, except that the declaration (although not necessarily the
definition) of the function shall be well formed.
\end{itemdescr}

\indexlibrarymember{operator->}{shared_ptr}%
\begin{itemdecl}
T* operator->() const noexcept;
\end{itemdecl}

\begin{itemdescr}
\pnum\requires  \tcode{get() != 0}.

\pnum\returns  \tcode{get()}.
\end{itemdescr}

\indexlibrarymember{use_count}{shared_ptr}%
\begin{itemdecl}
long use_count() const noexcept;
\end{itemdecl}

\begin{itemdescr}
\pnum\returns  The number of \tcode{shared_ptr} objects, \tcode{*this} included,
that \textit{share ownership} with \tcode{*this}, or \tcode{0} when \tcode{*this} is
\textit{empty}.
\end{itemdescr}

\indexlibrarymember{unique}{shared_ptr}%
\begin{itemdecl}
bool unique() const noexcept;
\end{itemdecl}

\begin{itemdescr}
\pnum\returns  \tcode{use_count() == 1}.

\pnum \begin{note} If you are
using \tcode{unique()} to implement copy on write, do not rely on a
specific value when \tcode{get() == nullptr}. \end{note}
\end{itemdescr}

\indexlibrarymember{operator bool}{shared_ptr}%
\begin{itemdecl}
explicit operator bool() const noexcept;
\end{itemdecl}

\begin{itemdescr}
\pnum\returns \tcode{get() != 0}.
\end{itemdescr}

\indexlibrarymember{owner_before}{shared_ptr}%
\begin{itemdecl}
template<class U> bool owner_before(shared_ptr<U> const& b) const;
template<class U> bool owner_before(weak_ptr<U> const& b) const;
\end{itemdecl}

\begin{itemdescr}
\pnum
\returns An unspecified value such that

\begin{itemize}
\item \tcode{x.owner_before(y)} defines a strict weak ordering as defined in~\ref{alg.sorting};

\item under the equivalence relation defined by \tcode{owner_before},
\tcode{!a.owner_before(b) \&\& !b.owner_before(a)}, two \tcode{shared_ptr} or
\tcode{weak_ptr} instances are equivalent if and only if they share ownership or
are both empty.
\end{itemize}

\end{itemdescr}


\rSec4[util.smartptr.shared.create]{\tcode{shared_ptr} creation}

\indexlibrary{\idxcode{make_shared}}%
\indexlibrary{\idxcode{allocate_shared}}%
\begin{itemdecl}
template<class T, class... Args> shared_ptr<T> make_shared(Args&&... args);
template<class T, class A, class... Args>
  shared_ptr<T> allocate_shared(const A& a, Args&&... args);
\end{itemdecl}

\begin{itemdescr}
\pnum
\requires The expression \tcode{::new (pv) T(std::forward<Args>(args)...)},
where \tcode{pv} has type \tcode{void*} and points to storage suitable
to hold an object of type \tcode{T}, shall be well formed. \tcode{A} shall
be an \textit{allocator}~(\ref{allocator.requirements}). The copy constructor
and destructor of \tcode{A} shall not throw exceptions.

\pnum
\effects Allocates memory suitable for an object of type \tcode{T}
and constructs an object in that memory via the placement
\grammarterm{new-expression}
\tcode{::new (pv) T(std::forward<Args>(args)...)}.
The template \tcode{allocate_shared} uses a copy of \tcode{a} to
allocate memory. If an exception is thrown, the functions have no effect.

\pnum
\returns A \tcode{shared_ptr} instance that stores and owns
the address of the newly constructed object of type \tcode{T}.

\pnum
\postconditions \tcode{get() != 0 \&\& use_count() == 1}.

\pnum
\throws \tcode{bad_alloc}, or an exception thrown from
\tcode{A::allocate} or from the constructor of \tcode{T}.

\pnum
\remarks Implementations should
perform no more than one memory allocation. \begin{note} This provides
efficiency equivalent to an intrusive smart pointer. \end{note}

\pnum
\begin{note} These functions will typically allocate more memory
than \tcode{sizeof(T)} to allow for internal bookkeeping structures such
as the reference counts. \end{note}
\end{itemdescr}

\rSec4[util.smartptr.shared.cmp]{\tcode{shared_ptr} comparison}

\indexlibrarymember{operator==}{shared_ptr}%
\begin{itemdecl}
template<class T, class U> bool operator==(const shared_ptr<T>& a, const shared_ptr<U>& b) noexcept;
\end{itemdecl}

\begin{itemdescr}
\pnum\returns  \tcode{a.get() == b.get()}.
\end{itemdescr}

\indexlibrarymember{operator<}{shared_ptr}%
\begin{itemdecl}
template<class T, class U> bool operator<(const shared_ptr<T>& a, const shared_ptr<U>& b) noexcept;
\end{itemdecl}

\begin{itemdescr}
\pnum\returns \tcode{less<V>()(a.get(), b.get())},
where \tcode{V} is the composite pointer type (Clause~\ref{expr}) of \tcode{T*} and \tcode{U*}.

\pnum \begin{note}
Defining a comparison operator allows \tcode{shared_ptr} objects to be
used as keys in associative containers.
\end{note}

\end{itemdescr}

\indexlibrarymember{operator==}{shared_ptr}%
\begin{itemdecl}
template <class T>
  bool operator==(const shared_ptr<T>& a, nullptr_t) noexcept;
template <class T>
  bool operator==(nullptr_t, const shared_ptr<T>& a) noexcept;
\end{itemdecl}

\begin{itemdescr}
\pnum
\returns \tcode{!a}.
\end{itemdescr}

\indexlibrarymember{operator"!=}{shared_ptr}%
\begin{itemdecl}
template <class T>
  bool operator!=(const shared_ptr<T>& a, nullptr_t) noexcept;
template <class T>
  bool operator!=(nullptr_t, const shared_ptr<T>& a) noexcept;
\end{itemdecl}

\begin{itemdescr}
\pnum
\returns \tcode{(bool)a}.
\end{itemdescr}

\indexlibrarymember{operator<}{shared_ptr}%
\begin{itemdecl}
template <class T>
  bool operator<(const shared_ptr<T>& a, nullptr_t) noexcept;
template <class T>
  bool operator<(nullptr_t, const shared_ptr<T>& a) noexcept;
\end{itemdecl}

\begin{itemdescr}
\pnum
\returns
The first function template returns
\tcode{less<T*>()(a.get(), nullptr)}.
The second function template returns
\tcode{less<T*>()(nullptr, a.get())}.
\end{itemdescr}

\indexlibrarymember{operator>}{shared_ptr}%
\begin{itemdecl}
template <class T>
  bool operator>(const shared_ptr<T>& a, nullptr_t) noexcept;
template <class T>
  bool operator>(nullptr_t, const shared_ptr<T>& a) noexcept;
\end{itemdecl}

\begin{itemdescr}
\pnum
\returns
The first function template returns \tcode{nullptr < a}.
The second function template returns \tcode{a < nullptr}.
\end{itemdescr}

\indexlibrarymember{operator<=}{shared_ptr}%
\begin{itemdecl}
template <class T>
  bool operator<=(const shared_ptr<T>& a, nullptr_t) noexcept;
template <class T>
  bool operator<=(nullptr_t, const shared_ptr<T>& a) noexcept;
\end{itemdecl}

\begin{itemdescr}
\pnum
\returns
The first function template returns \tcode{!(nullptr < a)}.
The second function template returns \tcode{!(a < nullptr)}.
\end{itemdescr}

\indexlibrarymember{operator>=}{shared_ptr}%
\begin{itemdecl}
template <class T>
  bool operator>=(const shared_ptr<T>& a, nullptr_t) noexcept;
template <class T>
  bool operator>=(nullptr_t, const shared_ptr<T>& a) noexcept;
\end{itemdecl}

\begin{itemdescr}
\pnum
\returns
The first function template returns \tcode{!(a < nullptr)}.
The second function template returns \tcode{!(nullptr < a)}.
\end{itemdescr}

\rSec4[util.smartptr.shared.spec]{\tcode{shared_ptr} specialized algorithms}

\indexlibrarymember{swap}{shared_ptr}%
\begin{itemdecl}
template<class T> void swap(shared_ptr<T>& a, shared_ptr<T>& b) noexcept;
\end{itemdecl}

\begin{itemdescr}
\pnum\effects  Equivalent to \tcode{a.swap(b)}.
\end{itemdescr}

\rSec4[util.smartptr.shared.cast]{\tcode{shared_ptr} casts}

\indexlibrarymember{static_pointer_cast}{shared_ptr}%
\begin{itemdecl}
template<class T, class U> shared_ptr<T> static_pointer_cast(const shared_ptr<U>& r) noexcept;
\end{itemdecl}

\begin{itemdescr}
\pnum\requires  The expression \tcode{static_cast<T*>(r.get())} shall
be well formed.

\pnum\returns  If \tcode{r} is \textit{empty}, an \textit{empty}
\tcode{shared_ptr<T>}; otherwise, a \tcode{shared_ptr<T>} object that
\textit{shares ownership}
with \tcode{r}.

\pnum
\postconditions \tcode{w.get() == static_cast<T*>(r.get())} and
\tcode{w.use_count() == r.use_count()}, where \tcode{w} is the return value.

\pnum
\begin{note} The seemingly equivalent expression
\tcode{shared_ptr<T>(static_cast<T*>(r.get()))}
will eventually result in undefined behavior, attempting to delete the
same object twice. \end{note}
\end{itemdescr}

\indexlibrarymember{dynamic_pointer_cast}{shared_ptr}%
\begin{itemdecl}
template<class T, class U> shared_ptr<T> dynamic_pointer_cast(const shared_ptr<U>& r) noexcept;
\end{itemdecl}

\begin{itemdescr}
\pnum\requires  The expression \tcode{dynamic_cast<T*>(r.get())}
shall be well formed and shall have well defined behavior.

\pnum\returns
\begin{itemize}
\item When \tcode{dynamic_cast<T*>(r.get())} returns a nonzero value, a
  \tcode{shared_ptr<T>} object that
  \textit{shares ownership}
  with \tcode{r};

\item Otherwise, an \textit{empty} \tcode{shared_ptr<T>} object.
\end{itemize}

\pnum
\postconditions \tcode{w.get() == dynamic_cast<T*>(r.get())}, where \tcode{w} is the return value.

\pnum \begin{note}  The seemingly equivalent expression
\tcode{shared_ptr<T>(dynamic_cast<T*>(r.get()))} will eventually result in
undefined behavior, attempting to delete the same object twice. \end{note}
\end{itemdescr}

\indexlibrarymember{const_pointer_cast}{shared_ptr}%
\begin{itemdecl}
template<class T, class U> shared_ptr<T> const_pointer_cast(const shared_ptr<U>& r) noexcept;
\end{itemdecl}

\begin{itemdescr}
\pnum\requires  The expression \tcode{const_cast<T*>(r.get())} shall
be well formed.

\pnum\returns  If \tcode{r} is empty, an empty \tcode{shared_ptr<T>}; otherwise, a
\tcode{shared_ptr<T>} object that
shares ownership
with \tcode{r}.

\pnum
\postconditions \tcode{w.get() == const_cast<T*>(r.get())} and
\tcode{w.use_count() == r.use_count()},\newline
where \tcode{w} is the return value.

\pnum \begin{note} The seemingly equivalent expression
\tcode{shared_ptr<T>(const_cast<T*>(r.get()))} will eventually result in
undefined behavior, attempting to delete the same object twice. \end{note}
\end{itemdescr}

\rSec4[util.smartptr.getdeleter]{\tcode{get_deleter}}

\indexlibrarymember{get_deleter}{shared_ptr}%
\begin{itemdecl}
template<class D, class T> D* get_deleter(const shared_ptr<T>& p) noexcept;
\end{itemdecl}

\begin{itemdescr}
\pnum\returns  If \tcode{p} \textit{owns} a deleter \tcode{d} of type cv-unqualified
\tcode{D}, returns \tcode{addressof(d)}; otherwise returns \tcode{nullptr}.
The returned
pointer remains valid as long as there exists a \tcode{shared_ptr} instance
that owns \tcode{d}. \begin{note} It is unspecified whether the pointer
remains valid longer than that. This can happen if the implementation doesn't destroy
the deleter until all \tcode{weak_ptr} instances that share ownership with
\tcode{p} have been destroyed. \end{note}
\end{itemdescr}

\rSec4[util.smartptr.shared.io]{\tcode{shared_ptr} I/O}

\indexlibrarymember{operator\shl}{shared_ptr}%
\begin{itemdecl}
template<class E, class T, class Y>
  basic_ostream<E, T>& operator<< (basic_ostream<E, T>& os, shared_ptr<Y> const& p);
\end{itemdecl}

\begin{itemdescr}
\pnum\effects As if by: \tcode{os <{}< p.get();}

\pnum\returns  \tcode{os}.
\end{itemdescr}

\rSec3[util.smartptr.weak]{Class template \tcode{weak_ptr}}

\pnum
\indexlibrary{\idxcode{weak_ptr}}%
The \tcode{weak_ptr} class template stores a weak reference to an object
that is already managed by a \tcode{shared_ptr}. To access the object, a
\tcode{weak_ptr} can be converted to a \tcode{shared_ptr} using the member
function \tcode{lock}.

\begin{codeblock}
namespace std {
  template<class T> class weak_ptr {
  public:
    using element_type = T;

    // \ref{util.smartptr.weak.const}, constructors
    constexpr weak_ptr() noexcept;
    template<class Y> weak_ptr(shared_ptr<Y> const& r) noexcept;
    weak_ptr(weak_ptr const& r) noexcept;
    template<class Y> weak_ptr(weak_ptr<Y> const& r) noexcept;
    weak_ptr(weak_ptr&& r) noexcept;
    template<class Y> weak_ptr(weak_ptr<Y>&& r) noexcept;

    // \ref{util.smartptr.weak.dest}, destructor
    ~weak_ptr();

    // \ref{util.smartptr.weak.assign}, assignment
    weak_ptr& operator=(weak_ptr const& r) noexcept;
    template<class Y> weak_ptr& operator=(weak_ptr<Y> const& r) noexcept;
    template<class Y> weak_ptr& operator=(shared_ptr<Y> const& r) noexcept;
    weak_ptr& operator=(weak_ptr&& r) noexcept;
    template<class Y> weak_ptr& operator=(weak_ptr<Y>&& r) noexcept;    

    // \ref{util.smartptr.weak.mod}, modifiers
    void swap(weak_ptr& r) noexcept;
    void reset() noexcept;

    // \ref{util.smartptr.weak.obs}, observers
    long use_count() const noexcept;
    bool expired() const noexcept;
    shared_ptr<T> lock() const noexcept;
    template<class U> bool owner_before(shared_ptr<U> const& b) const;
    template<class U> bool owner_before(weak_ptr<U> const& b) const;
  };

  // \ref{util.smartptr.weak.spec}, specialized algorithms
  template<class T> void swap(weak_ptr<T>& a, weak_ptr<T>& b) noexcept;
}
\end{codeblock}

\pnum
Specializations of \tcode{weak_ptr} shall be \tcode{CopyConstructible} and
\tcode{CopyAssignable}, allowing their use in standard
containers.  The template parameter \tcode{T} of \tcode{weak_ptr} may be an
incomplete type.

\rSec4[util.smartptr.weak.const]{\tcode{weak_ptr} constructors}

\indexlibrary{\idxcode{weak_ptr}!constructor}%
\begin{itemdecl}
constexpr weak_ptr() noexcept;
\end{itemdecl}

\begin{itemdescr}
\pnum\effects  Constructs an \textit{empty} \tcode{weak_ptr} object.

\pnum\postconditions  \tcode{use_count() == 0}.
\end{itemdescr}

\indexlibrary{\idxcode{weak_ptr}!constructor}%
\begin{itemdecl}
weak_ptr(const weak_ptr& r) noexcept;
template<class Y> weak_ptr(const weak_ptr<Y>& r) noexcept;
template<class Y> weak_ptr(const shared_ptr<Y>& r) noexcept;
\end{itemdecl}

\begin{itemdescr}
\pnum\remarks The second and third constructors shall not participate in
overload resolution unless \tcode{Y*} is implicitly convertible to \tcode{T*}.

\pnum\effects  If \tcode{r} is \textit{empty}, constructs
an \textit{empty} \tcode{weak_ptr} object; otherwise, constructs
a \tcode{weak_ptr} object that \textit{shares ownership}
with \tcode{r} and stores a copy of the pointer stored in \tcode{r}.

\pnum\postconditions  \tcode{use_count() == r.use_count()}.
\end{itemdescr}

\indexlibrary{\idxcode{weak_ptr}!constructor}%
\begin{itemdecl}
weak_ptr(weak_ptr&& r) noexcept;
template<class Y> weak_ptr(weak_ptr<Y>&& r) noexcept;
\end{itemdecl}

\begin{itemdescr}
\pnum\remarks The second constructor shall not participate in overload resolution unless
\tcode{Y*} is implicitly convertible to \tcode{T*}.

\pnum\effects Move constructs a \tcode{weak_ptr} instance from \tcode{r}.

\pnum\postconditions \tcode{*this} shall contain the old value of \tcode{r}.
\tcode{r} shall be \textit{empty}. \tcode{r.use_count() == 0}.
\end{itemdescr}

\rSec4[util.smartptr.weak.dest]{\tcode{weak_ptr} destructor}

\indexlibrary{\idxcode{weak_ptr}!destructor}%
\begin{itemdecl}
~weak_ptr();
\end{itemdecl}

\begin{itemdescr}
\pnum\effects  Destroys this \tcode{weak_ptr} object but has no
effect on the object its stored pointer points to.
\end{itemdescr}

\rSec4[util.smartptr.weak.assign]{\tcode{weak_ptr} assignment}

\indexlibrarymember{operator=}{weak_ptr}%
\begin{itemdecl}
weak_ptr& operator=(const weak_ptr& r) noexcept;
template<class Y> weak_ptr& operator=(const weak_ptr<Y>& r) noexcept;
template<class Y> weak_ptr& operator=(const shared_ptr<Y>& r) noexcept;
\end{itemdecl}

\begin{itemdescr}
\pnum\effects  Equivalent to \tcode{weak_ptr(r).swap(*this)}.

\pnum\remarks  The implementation may meet the effects (and the
implied guarantees) via different means, without creating a temporary.

\pnum\returns  \tcode{*this}.
\end{itemdescr}

\indexlibrarymember{operator=}{weak_ptr}%
\begin{itemdecl}
weak_ptr& operator=(weak_ptr&& r) noexcept;
template<class Y> weak_ptr& operator=(weak_ptr<Y>&& r) noexcept;
\end{itemdecl}

\begin{itemdescr}
\pnum\effects  Equivalent to \tcode{weak_ptr(std::move(r)).swap(*this)}.

\pnum\returns  \tcode{*this}.
\end{itemdescr}

\rSec4[util.smartptr.weak.mod]{\tcode{weak_ptr} modifiers}
\indexlibrarymember{swap}{weak_ptr}%
\begin{itemdecl}
void swap(weak_ptr& r) noexcept;
\end{itemdecl}

\begin{itemdescr}
\pnum\effects  Exchanges the contents of \tcode{*this} and \tcode{r}.
\end{itemdescr}

\indexlibrarymember{reset}{weak_ptr}%
\begin{itemdecl}
void reset() noexcept;
\end{itemdecl}

\begin{itemdescr}
\pnum\effects  Equivalent to \tcode{weak_ptr().swap(*this)}.
\end{itemdescr}

\rSec4[util.smartptr.weak.obs]{\tcode{weak_ptr} observers}
\indexlibrarymember{use_count}{weak_ptr}%
\begin{itemdecl}
long use_count() const noexcept;
\end{itemdecl}

\begin{itemdescr}
\pnum\returns  \tcode{0} if \tcode{*this} is \textit{empty};
otherwise, the number of \tcode{shared_ptr} instances
that \textit{share ownership} with \tcode{*this}.
\end{itemdescr}

\indexlibrarymember{expired}{weak_ptr}%
\begin{itemdecl}
bool expired() const noexcept;
\end{itemdecl}

\begin{itemdescr}
\pnum\returns  \tcode{use_count() == 0}.
\end{itemdescr}

\indexlibrarymember{lock}{weak_ptr}%
\begin{itemdecl}
shared_ptr<T> lock() const noexcept;
\end{itemdecl}

\begin{itemdescr}
\pnum\returns  \tcode{expired() ? shared_ptr<T>() : shared_ptr<T>(*this)}, executed atomically.
\end{itemdescr}

\indexlibrarymember{owner_before}{weak_ptr}%
\begin{itemdecl}
template<class U> bool owner_before(shared_ptr<U> const& b) const;
template<class U> bool owner_before(weak_ptr<U> const& b) const;
\end{itemdecl}

\begin{itemdescr}
\pnum
\returns An unspecified value such that

\begin{itemize}
\item \tcode{x.owner_before(y)} defines a strict weak ordering as defined in~\ref{alg.sorting};

\item under the equivalence relation defined by \tcode{owner_before},
\tcode{!a.owner_before(b) \&\& !b.owner_before(a)}, two \tcode{shared_ptr} or
\tcode{weak_ptr} instances are equivalent if and only if they share ownership or are
both empty.
\end{itemize}
\end{itemdescr}


\rSec4[util.smartptr.weak.spec]{\tcode{weak_ptr} specialized algorithms}

\indexlibrarymember{swap}{weak_ptr}%
\begin{itemdecl}
template<class T> void swap(weak_ptr<T>& a, weak_ptr<T>& b) noexcept;
\end{itemdecl}

\begin{itemdescr}
\pnum\effects  Equivalent to \tcode{a.swap(b)}.
\end{itemdescr}

\rSec3[util.smartptr.ownerless]{Class template \tcode{owner_less}}

\pnum
The class template \tcode{owner_less} allows ownership-based mixed comparisons of shared
and weak pointers.

\indexlibrary{\idxcode{struct owner_less}}%
\begin{codeblock}
namespace std {
  template<class T = void> struct owner_less;

  template<class T> struct owner_less<shared_ptr<T>> {
    bool operator()(shared_ptr<T> const&, shared_ptr<T> const&) const;
    bool operator()(shared_ptr<T> const&, weak_ptr<T> const&) const;
    bool operator()(weak_ptr<T> const&, shared_ptr<T> const&) const;
  };

  template<class T> struct owner_less<weak_ptr<T>> {
    bool operator()(weak_ptr<T> const&, weak_ptr<T> const&) const;
    bool operator()(shared_ptr<T> const&, weak_ptr<T> const&) const;
    bool operator()(weak_ptr<T> const&, shared_ptr<T> const&) const;
  };

  template<> struct owner_less<void> {
    template<class T, class U>
      bool operator()(shared_ptr<T> const&, shared_ptr<U> const&) const;
    template<class T, class U>
      bool operator()(shared_ptr<T> const&, weak_ptr<U> const&) const;
    template<class T, class U>
      bool operator()(weak_ptr<T> const&, shared_ptr<U> const&) const;
    template<class T, class U>
      bool operator()(weak_ptr<T> const&, weak_ptr<U> const&) const;

    using is_transparent = @\unspec@;
  };
}
\end{codeblock}

\indexlibrarymember{operator()}{owner_less}%
\pnum \tcode{operator()(x, y)} shall return \tcode{x.owner_before(y)}. \begin{note}
Note that

\begin{itemize}
\item \tcode{operator()} defines a strict weak ordering as defined in~\ref{alg.sorting};

\item under the equivalence relation defined by \tcode{operator()},
\tcode{!operator()(a, b) \&\& !operator()(b, a)}, two \tcode{shared_ptr} or
\tcode{weak_ptr} instances are equivalent if and only if they share ownership or are
both empty.
\end{itemize} \end{note}

\rSec3[util.smartptr.enab]{Class template \tcode{enable_shared_from_this}}

\pnum
\indexlibrary{\idxcode{enable_shared_from_this}}%
A class \tcode{T} can inherit from \tcode{enable_shared_from_this<T>}
to inherit the \tcode{shared_from_this} member functions that obtain
a \textit{shared_ptr} instance pointing to \tcode{*this}.

\pnum
\begin{example}

\begin{codeblock}
struct X: public enable_shared_from_this<X> {
};

int main() {
  shared_ptr<X> p(new X);
  shared_ptr<X> q = p->shared_from_this();
  assert(p == q);
  assert(!p.owner_before(q) && !q.owner_before(p)); // p and q share ownership
}
\end{codeblock}
\end{example}

\begin{codeblock}
namespace std {
  template<class T> class enable_shared_from_this {
  protected:
    constexpr enable_shared_from_this() noexcept;
    enable_shared_from_this(enable_shared_from_this const&) noexcept;
    enable_shared_from_this& operator=(enable_shared_from_this const&) noexcept;
    ~enable_shared_from_this();
  public:
    shared_ptr<T> shared_from_this();
    shared_ptr<T const> shared_from_this() const;
    weak_ptr<T> weak_from_this() noexcept;
    weak_ptr<T const> weak_from_this() const noexcept;
  private:
    mutable weak_ptr<T> weak_this; // \expos
  };
}
\end{codeblock}

\pnum
The template parameter \tcode{T} of \tcode{enable_shared_from_this}
may be an incomplete type.

\indexlibrary{\idxcode{enable_shared_from_this}!constructor}%
\begin{itemdecl}
constexpr enable_shared_from_this() noexcept;
enable_shared_from_this(const enable_shared_from_this<T>&) noexcept;
\end{itemdecl}

\begin{itemdescr}
\pnum\effects  Value-initializes \tcode{weak_this}.
\end{itemdescr}

\indexlibrarymember{operator=}{enable_shared_from_this}%
\begin{itemdecl}
enable_shared_from_this<T>& operator=(const enable_shared_from_this<T>&) noexcept;
\end{itemdecl}

\begin{itemdescr}
\pnum\returns  \tcode{*this}.

\pnum\begin{note} \tcode{weak_this} is not changed. \end{note}
\end{itemdescr}

\indexlibrary{\idxcode{shared_ptr}}%
\indexlibrarymember{shared_from_this}{enable_shared_from_this}%
\begin{itemdecl}
shared_ptr<T>       shared_from_this();
shared_ptr<T const> shared_from_this() const;
\end{itemdecl}

\begin{itemdescr}
\pnum\returns  \tcode{shared_ptr<T>(weak_this)}.

\pnum\postconditions  \tcode{r.get() == this}.
\end{itemdescr}

\indexlibrary{\idxcode{weak_ptr}}%
\indexlibrarymember{weak_from_this}{enable_shared_from_this}%
\begin{itemdecl}
weak_ptr<T>       weak_from_this() noexcept;
weak_ptr<T const> weak_from_this() const noexcept;
\end{itemdecl}

\begin{itemdescr}
\pnum\returns  \tcode{weak_this}.
\end{itemdescr}

\rSec3[util.smartptr.shared.atomic]{\tcode{shared_ptr} atomic access}

\pnum
Concurrent access to a \tcode{shared_ptr} object from multiple threads does not
introduce a data race if the access is done exclusively via the functions in
this section and the instance is passed as their first argument.

\pnum
The meaning of the arguments of type \tcode{memory_order} is explained in~\ref{atomics.order}.

\indexlibrarymember{atomic_is_lock_free}{shared_ptr}%
\begin{itemdecl}
template<class T>
  bool atomic_is_lock_free(const shared_ptr<T>* p);
\end{itemdecl}

\begin{itemdescr}
\pnum
\requires \tcode{p} shall not be null.

\pnum
\returns \tcode{true} if atomic access to \tcode{*p} is lock-free, \tcode{false} otherwise.

\pnum
\throws Nothing.
\end{itemdescr}

\indexlibrarymember{atomic_load}{shared_ptr}%
\begin{itemdecl}
template<class T>
  shared_ptr<T> atomic_load(const shared_ptr<T>* p);
\end{itemdecl}

\begin{itemdescr}
\pnum
\requires \tcode{p} shall not be null.

\pnum
\returns \tcode{atomic_load_explicit(p, memory_order_seq_cst)}.

\pnum
\throws Nothing.
\end{itemdescr}

\indexlibrarymember{atomic_load_explicit}{shared_ptr}%
\begin{itemdecl}
template<class T>
  shared_ptr<T> atomic_load_explicit(const shared_ptr<T>* p, memory_order mo);
\end{itemdecl}

\begin{itemdescr}
\pnum
\requires \tcode{p} shall not be null.

\pnum
\requires \tcode{mo} shall not be \tcode{memory_order_release} or \tcode{memory_order_acq_rel}.

\pnum
\returns \tcode{*p}.

\pnum
\throws Nothing.
\end{itemdescr}

\indexlibrarymember{atomic_store}{shared_ptr}%
\begin{itemdecl}
template<class T>
  void atomic_store(shared_ptr<T>* p, shared_ptr<T> r);
\end{itemdecl}

\begin{itemdescr}
\pnum
\requires \tcode{p} shall not be null.

\pnum
\effects As if by \tcode{atomic_store_explicit(p, r, memory_order_seq_cst)}.

\pnum
\throws Nothing.
\end{itemdescr}

\indexlibrarymember{atomic_store_explicit}{shared_ptr}%
\begin{itemdecl}
template<class T>
  void atomic_store_explicit(shared_ptr<T>* p, shared_ptr<T> r, memory_order mo);
\end{itemdecl}

\begin{itemdescr}
\pnum
\requires \tcode{p} shall not be null.

\pnum
\requires \tcode{mo} shall not be \tcode{memory_order_acquire} or \tcode{memory_order_acq_rel}.

\pnum
\effects As if by \tcode{p->swap(r)}.

\pnum
\throws Nothing.
\end{itemdescr}

\indexlibrarymember{atomic_exchange}{shared_ptr}%
\begin{itemdecl}
template<class T>
  shared_ptr<T> atomic_exchange(shared_ptr<T>* p, shared_ptr<T> r);
\end{itemdecl}

\begin{itemdescr}
\pnum
\requires \tcode{p} shall not be null.

\pnum
\returns \tcode{atomic_exchange_explicit(p, r, memory_order_seq_cst)}.

\pnum
\throws Nothing.
\end{itemdescr}

\indexlibrarymember{atomic_exchange_explicit}{shared_ptr}%
\begin{itemdecl}
template<class T>
  shared_ptr<T> atomic_exchange_explicit(shared_ptr<T>* p, shared_ptr<T> r,
                                         memory_order mo);
\end{itemdecl}

\begin{itemdescr}
\pnum
\requires \tcode{p} shall not be null.

\pnum
\effects As if by \tcode{p->swap(r)}.

\pnum
\returns The previous value of \tcode{*p}.

\pnum
\throws Nothing.
\end{itemdescr}

\indexlibrarymember{atomic_compare_exchange_weak}{shared_ptr}%
\begin{itemdecl}
template<class T>
  bool atomic_compare_exchange_weak(
    shared_ptr<T>* p, shared_ptr<T>* v, shared_ptr<T> w);
\end{itemdecl}

\begin{itemdescr}
\pnum
\requires \tcode{p} shall not be null and \tcode{v} shall not be null.

\pnum
\returns \tcode{atomic_compare_exchange_weak_explicit(p, v, w,
memory_order_seq_cst, memory_order_seq_cst)}.

\pnum
\throws Nothing.
\end{itemdescr}

\indexlibrarymember{atomic_compare_exchange_strong}{shared_ptr}%
\begin{itemdecl}
template<class T>
  bool atomic_compare_exchange_strong(
    shared_ptr<T>* p, shared_ptr<T>* v, shared_ptr<T> w);
\end{itemdecl}

\begin{itemdescr}
\pnum
\returns \tcode{atomic_compare_exchange_strong_explicit(p, v, w,}
\tcode{memory_order_seq_cst, memory_order_seq_cst)}.
\end{itemdescr}

\indexlibrarymember{atomic_compare_exchange_weak_explicit}{shared_ptr}%
\indexlibrarymember{atomic_compare_exchange_strong_explicit}{shared_ptr}%
\begin{itemdecl}
template<class T>
  bool atomic_compare_exchange_weak_explicit(
    shared_ptr<T>* p, shared_ptr<T>* v, shared_ptr<T> w,
    memory_order success, memory_order failure);
template<class T>
  bool atomic_compare_exchange_strong_explicit(
    shared_ptr<T>* p, shared_ptr<T>* v, shared_ptr<T> w,
    memory_order success, memory_order failure);
\end{itemdecl}

\begin{itemdescr}
\pnum
\requires \tcode{p} shall not be null and \tcode{v} shall not be null.

\pnum
\requires \tcode{failure} shall not be \tcode{memory_order_release},
\tcode{memory_order_acq_rel}, or stronger than \tcode{success}.

\pnum
\effects If \tcode{*p} is equivalent to \tcode{*v}, assigns \tcode{w} to
\tcode{*p} and has synchronization semantics corresponding to the value of
\tcode{success}, otherwise assigns \tcode{*p} to \tcode{*v} and has
synchronization semantics corresponding to the value of \tcode{failure}.

\pnum
\returns \tcode{true} if \tcode{*p} was equivalent to \tcode{*v}, \tcode{false} otherwise.

\pnum
\throws Nothing.

\pnum
\remarks two \tcode{shared_ptr} objects are equivalent if they store the same
pointer value and share ownership.

\pnum
\remarks the weak forms may fail spuriously. See~\ref{atomics.types.operations}.
\end{itemdescr}

\rSec3[util.smartptr.hash]{Smart pointer hash support}

\indexlibrary{\idxcode{hash}!\idxcode{unique_ptr}}%
\begin{itemdecl}
template <class T, class D> struct hash<unique_ptr<T, D>>;
\end{itemdecl}

\begin{itemdescr}
\pnum The template specialization shall meet the requirements of class
template \tcode{hash}~(\ref{unord.hash}). For an object \tcode{p} of type \tcode{UP},
where \tcode{UP} is \tcode{unique_ptr<T, D>}, \tcode{hash<UP>()(p)} shall evaluate to
the same value as \tcode{hash<typename UP::pointer>()(p.get())}.

\pnum
\requires The specialization \tcode{hash<typename UP::pointer>} shall be
well-formed and well-defined, and shall meet the requirements of class
template \tcode{hash}~(\ref{unord.hash}).
\end{itemdescr}

\indexlibrary{\idxcode{hash}!\idxcode{shared_ptr}}%
\begin{itemdecl}
template <class T> struct hash<shared_ptr<T>>;
\end{itemdecl}

\begin{itemdescr}
\pnum
The template specialization shall meet the requirements of class
template \tcode{hash}~(\ref{unord.hash}). For an object \tcode{p} of type \tcode{shared_ptr<T>},
\tcode{hash<shared_ptr<T>>()(p)} shall evaluate to
the same value as \tcode{hash<T*>()(p.get())}.
\end{itemdescr}%
\indextext{smart pointers|)}

\rSec1[mem.res]{Memory resources}

\rSec2[mem.res.syn]{Header \tcode{<memory_resource>} synopsis}

\indexlibrary{\idxhdr{memory_resource}}%
\begin{codeblock}
namespace std::pmr {
  // \ref{mem.res.class}, class \tcode{memory_resource}
  class memory_resource;

  bool operator==(const memory_resource& a, const memory_resource& b) noexcept;
  bool operator!=(const memory_resource& a, const memory_resource& b) noexcept;

  // \ref{mem.poly.allocator.class}, class \tcode{polymorphic_allocator}
  template <class Tp> class polymorphic_allocator;

  template <class T1, class T2>
    bool operator==(const polymorphic_allocator<T1>& a,
                    const polymorphic_allocator<T2>& b) noexcept;
  template <class T1, class T2>
    bool operator!=(const polymorphic_allocator<T1>& a,
                    const polymorphic_allocator<T2>& b) noexcept;

  // \ref{mem.res.global}, global memory resources
  memory_resource* new_delete_resource() noexcept;
  memory_resource* null_memory_resource() noexcept;
  memory_resource* set_default_resource(memory_resource* r) noexcept;
  memory_resource* get_default_resource() noexcept;

  // \ref{mem.res.pool}, pool resource classes
  struct pool_options;
  class synchronized_pool_resource;
  class unsynchronized_pool_resource;
  class monotonic_buffer_resource;
}
\end{codeblock}

\rSec2[mem.res.class]{Class \tcode{memory_resource}}

\pnum
The \tcode{memory_resource} class is an abstract interface to an unbounded set of classes encapsulating memory resources.

\indexlibrary{\idxcode{memory_resource}}%
\begin{codeblock}
class memory_resource {
  static constexpr size_t max_align = alignof(max_align_t); // \expos

public:
  virtual ~memory_resource();

  void* allocate(size_t bytes, size_t alignment = max_align);
  void deallocate(void* p, size_t bytes, size_t alignment = max_align);

  bool is_equal(const memory_resource& other) const noexcept;

private:
  virtual void* do_allocate(size_t bytes, size_t alignment) = 0;
  virtual void do_deallocate(void* p, size_t bytes, size_t alignment) = 0;

  virtual bool do_is_equal(const memory_resource& other) const noexcept = 0;
};
\end{codeblock}


\rSec3[mem.res.public]{\tcode{memory_resource} public member functions}

\indexlibrary{\idxcode{memory_resource}!destructor}%
\begin{itemdecl}
~memory_resource();
\end{itemdecl}

\begin{itemdescr}
\pnum
\effects
Destroys this \tcode{memory_resource}.
\end{itemdescr}

\indexlibrarymember{allocate}{memory_resource}%
\begin{itemdecl}
void* allocate(size_t bytes, size_t alignment = max_align);
\end{itemdecl}

\begin{itemdescr}
\pnum
\effects
Equivalent to: \tcode{return do_allocate(bytes, alignment);}
\end{itemdescr}

\indexlibrarymember{deallocate}{memory_resource}%
\begin{itemdecl}
void deallocate(void* p, size_t bytes, size_t alignment = max_align);
\end{itemdecl}

\begin{itemdescr}
\pnum
\effects
Equivalent to: \tcode{do_deallocate(p, bytes, alignment);}
\end{itemdescr}

\indexlibrarymember{is_equal}{memory_resource}%
\begin{itemdecl}
bool is_equal(const memory_resource& other) const noexcept;
\end{itemdecl}

\begin{itemdescr}
\pnum
\effects
Equivalent to: \tcode{return do_is_equal(other);}
\end{itemdescr}


\rSec3[mem.res.private]{\tcode{memory_resource} private virtual member functions}

\indexlibrarymember{do_allocate}{memory_resource}%
\begin{itemdecl}
virtual void* do_allocate(size_t bytes, size_t alignment) = 0;
\end{itemdecl}

\begin{itemdescr}
\pnum
\requires
\tcode{alignment} shall be a power of two.

\pnum
\returns
A derived class shall implement this function to return a pointer to allocated storage (\ref{basic.stc.dynamic.deallocation}) with a size of at least \tcode{bytes}.
The returned storage is aligned to the specified alignment, if such alignment is supported (\ref{basic.align});
otherwise it is aligned to \tcode{max_align}.

\pnum
\throws
A derived class implementation shall throw an appropriate exception if it is unable to allocate memory with the requested size and alignment.
\end{itemdescr}

\indexlibrarymember{do_deallocate}{memory_resource}%
\begin{itemdecl}
virtual void do_deallocate(void* p, size_t bytes, size_t alignment) = 0;
\end{itemdecl}

\begin{itemdescr}
\pnum
\requires
\tcode{p} shall have been returned from a prior call to \tcode{allocate(bytes, alignment)} on a memory resource equal to \tcode{*this},
and the storage at \tcode{p} shall not yet have been deallocated.

\pnum
\effects
A derived class shall implement this function to dispose of allocated storage.

\pnum
\throws
Nothing.
\end{itemdescr}

\indexlibrarymember{do_is_equal}{memory_resource}%
\begin{itemdecl}
virtual bool do_is_equal(const memory_resource& other) const noexcept = 0;
\end{itemdecl}

\begin{itemdescr}
\pnum
\returns
A derived class shall implement this function to return \tcode{true} if memory allocated from \tcode{this} can be deallocated from \tcode{other} and vice-versa,
otherwise \tcode{false}.
\begin{note}
The most-derived type of \tcode{other} might not match the type of \tcode{this}.
For a derived class \tcode{D}, a typical implementation of this function
will immediately return \tcode{false}
if \tcode{dynamic_cast<const D*>(\&other) == nullptr}.\end{note}
\end{itemdescr}

\rSec3[mem.res.eq]{\tcode{memory_resource} equality}

\indexlibrarymember{operator==}{memory_resource}%
\begin{itemdecl}
bool operator==(const memory_resource& a, const memory_resource& b) noexcept;
\end{itemdecl}

\begin{itemdescr}
\pnum
\returns
\tcode{\&a == \&b || a.is_equal(b)}.
\end{itemdescr}

\indexlibrarymember{operator"!=}{memory_resource}%
\begin{itemdecl}
bool operator!=(const memory_resource& a, const memory_resource& b) noexcept;
\end{itemdecl}

\begin{itemdescr}
\pnum
\returns
\tcode{!(a == b)}.
\end{itemdescr}

\rSec2[mem.poly.allocator.class]{Class template \tcode{polymorphic_allocator}}

\pnum
A specialization of class template \tcode{pmr::polymorphic_allocator}
conforms to the \tcode{Allocator} requirements (\ref{allocator.requirements}).
Constructed with different memory resources,
different instances of the same specialization of \tcode{pmr::polymorphic_allocator}
can exhibit entirely different allocation behavior.
This runtime polymorphism allows objects that use \tcode{polymorphic_allocator}
to behave as if they used different allocator types at run time
even though they use the same static allocator type.

\indexlibrary{\idxcode{polymorphic_allocator}}%
\begin{codeblock}
template <class Tp>
class polymorphic_allocator {
  memory_resource* memory_rsrc; // \expos

public:
  using value_type = Tp;

  // \ref{mem.poly.allocator.ctor}, constructors:
  polymorphic_allocator() noexcept;
  polymorphic_allocator(memory_resource* r);

  polymorphic_allocator(const polymorphic_allocator& other) = default;

  template <class U>
    polymorphic_allocator(const polymorphic_allocator<U>& other) noexcept;

  polymorphic_allocator&
    operator=(const polymorphic_allocator& rhs) = delete;

  // \ref{mem.poly.allocator.mem}, member functions:
  Tp* allocate(size_t n);
  void deallocate(Tp* p, size_t n);

  template <class T, class... Args>
  void construct(T* p, Args&&... args);

  template <class T1, class T2, class... Args1, class... Args2>
    void construct(pair<T1,T2>* p, piecewise_construct_t,
                   tuple<Args1...> x, tuple<Args2...> y);
  template <class T1, class T2>
    void construct(pair<T1,T2>* p);
  template <class T1, class T2, class U, class V>
    void construct(pair<T1,T2>* p, U&& x, V&& y);
  template <class T1, class T2, class U, class V>
    void construct(pair<T1,T2>* p, const pair<U, V>& pr);
  template <class T1, class T2, class U, class V>
    void construct(pair<T1,T2>* p, pair<U, V>&& pr);

  template <class T>
    void destroy(T* p);

  polymorphic_allocator select_on_container_copy_construction() const;

  memory_resource* resource() const;
};
\end{codeblock}


\rSec3[mem.poly.allocator.ctor]{\tcode{polymorphic_allocator} constructors}

\indexlibrary{\idxcode{polymorphic_allocator}!constructor}%
\begin{itemdecl}
polymorphic_allocator() noexcept;
\end{itemdecl}

\begin{itemdescr}
\pnum
\effects
Sets \tcode{memory_rsrc} to \tcode{get_default_resource()}.
\end{itemdescr}

\indexlibrary{\idxcode{polymorphic_allocator}!constructor}%
\begin{itemdecl}
polymorphic_allocator(memory_resource* r);
\end{itemdecl}

\begin{itemdescr}
\pnum
\requires
\tcode{r} is non-null.

\pnum
\effects
Sets \tcode{memory_rsrc} to \tcode{r}.

\pnum
\throws
Nothing.

\pnum
\realnotes
This constructor provides an implicit conversion from \tcode{memory_resource*}.
\end{itemdescr}

\indexlibrary{\idxcode{polymorphic_allocator}!constructor}%
\begin{itemdecl}
template <class U>
  polymorphic_allocator(const polymorphic_allocator<U>& other) noexcept;
\end{itemdecl}

\begin{itemdescr}
\pnum
\effects
Sets \tcode{memory_rsrc} to \tcode{other.resource()}.
\end{itemdescr}


\rSec3[mem.poly.allocator.mem]{\tcode{polymorphic_allocator} member functions}

\indexlibrarymember{allocate}{polymorphic_allocator}%
\begin{itemdecl}
Tp* allocate(size_t n);
\end{itemdecl}

\begin{itemdescr}
\pnum
\returns
Equivalent to
\begin{codeblock}
return static_cast<Tp*>(memory_rsrc->allocate(n * sizeof(Tp), alignof(Tp)));
\end{codeblock}
\end{itemdescr}

\indexlibrarymember{deallocate}{polymorphic_allocator}%
\begin{itemdecl}
void deallocate(Tp* p, size_t n);
\end{itemdecl}

\begin{itemdescr}
\pnum
\requires
\tcode{p} was allocated from a memory resource \tcode{x},
equal to \tcode{*memory_rsrc},
using \tcode{x.allocate(n * sizeof(Tp), alignof(Tp))}.

\pnum
\effects
Equivalent to \tcode{memory_rsrc->deallocate(p, n * sizeof(Tp), alignof(Tp))}.

\pnum
\throws
Nothing.
\end{itemdescr}

\indexlibrarymember{construct}{polymorphic_allocator}%
\begin{itemdecl}
template <class T, class... Args>
  void construct(T* p, Args&&... args);
\end{itemdecl}

\begin{itemdescr}
\pnum
\requires
Uses-allocator construction of \tcode{T}
with allocator \tcode{resource()} (see~\ref{allocator.uses.construction})
and constructor arguments \tcode{std::forward<Args>(args)...} is well-formed.
\begin{note}
Uses-allocator construction is always well formed
for types that do not use allocators.\end{note}

\pnum
\effects
Construct a \tcode{T} object in the storage
whose address is represented by \tcode{p}
by uses-allocator construction with allocator \tcode{resource()}
and constructor arguments \tcode{std::forward<Args>(args)...}.

\pnum
\throws
Nothing unless the constructor for \tcode{T} throws.
\end{itemdescr}

\indexlibrarymember{construct}{polymorphic_allocator}%
\begin{itemdecl}
template <class T1, class T2, class... Args1, class... Args2>
  void construct(pair<T1,T2>* p, piecewise_construct_t,
                 tuple<Args1...> x, tuple<Args2...> y);
\end{itemdecl}

\begin{itemdescr}
\pnum
\begin{note}
This method and the \tcode{construct} methods that follow
are overloads for piecewise construction of pairs~(\ref{pairs.pair}).
\end{note}

\pnum
\effects
Let \tcode{xprime} be a \tcode{tuple} constructed from \tcode{x}
according to the appropriate rule from the following list.
\begin{note}
The following description can be summarized as
constructing a \tcode{pair<T1, T2>} object
in the storage whose address is represented by \tcode{p},
as if by separate uses-allocator construction
with allocator \tcode{resource()}~(\ref{allocator.uses.construction})
of \tcode{p->first} using the elements of \tcode{x}
and \tcode{p->second} using the elements of \tcode{y}.
\end{note}
\begin{itemize}
\item
If \tcode{uses_allocator_v<T1,memory_resource*>} is \tcode{false}
\\
and
\tcode{is_constructible_v<T,Args1...>} is \tcode{true},
\\
then \tcode{xprime} is \tcode{x}.
\item
Otherwise, if \tcode{uses_allocator_v<T1,memory_resource*>} is \tcode{true}
\\
and
\tcode{is_constructible_v<T1,allocator_arg_t,memory_resource*,Args1...>} is \tcode{true},
\\
then \tcode{xprime} is \tcode{tuple_cat(make_tuple(allocator_arg, resource()), std::move(x))}.
\item
Otherwise, if \tcode{uses_allocator_v<T1,memory_resource*>} is \tcode{true}
\\
and
\tcode{is_constructible_v<T1,Args1...,memory_resource*>} is \tcode{true},
\\
then \tcode{xprime} is \tcode{tuple_cat(std::move(x), make_tuple(resource()))}.
\item
Otherwise the program is ill formed.
\end{itemize}
Let \tcode{yprime} be a tuple constructed from \tcode{y}
according to the appropriate rule from the following list:
\begin{itemize}
\item
If \tcode{uses_allocator_v<T2,memory_resource*>} is \tcode{false}
\\
and
\tcode{is_constructible_v<T,Args2...>} is \tcode{true},
\\
then \tcode{yprime} is \tcode{y}.
\item
Otherwise, if \tcode{uses_allocator_v<T2,memory_resource*>} is \tcode{true}
\\
and
\tcode{is_constructible_v<T2,allocator_arg_t,memory_resource*,Args2...>} is \tcode{true},
\\
then \tcode{yprime} is \tcode{tuple_cat(make_tuple(allocator_arg, resource()), std::move(y))}.
\item
Otherwise, if \tcode{uses_allocator_v<T2,memory_resource*>} is \tcode{true}
\\
and
\tcode{is_constructible_v<T2,Args2...,memory_resource*>} is \tcode{true},
\\
then
\tcode{yprime} is \tcode{tuple_cat(std::move(y), make_tuple(resource()))}.
\item
Otherwise the program is ill formed.
\end{itemize}

Then, using \tcode{piecewise_construct}, \tcode{xprime}, and \tcode{yprime}
as the constructor arguments,
this function constructs a \tcode{pair<T1, T2>} object
in the storage whose address is represented by \tcode{p}.
\end{itemdescr}

\indexlibrarymember{construct}{polymorphic_allocator}%
\begin{itemdecl}
template <class T1, class T2>
  void construct(pair<T1,T2>* p);
\end{itemdecl}

\begin{itemdescr}
\pnum
\effects
Equivalent to:
\begin{codeblock}
construct(p, piecewise_construct, tuple<>(), tuple<>());
\end{codeblock}
\end{itemdescr}

\indexlibrarymember{construct}{polymorphic_allocator}%
\begin{itemdecl}
template <class T1, class T2, class U, class V>
  void construct(pair<T1,T2>* p, U&& x, V&& y);
\end{itemdecl}

\begin{itemdescr}
\pnum
\effects
Equivalent to:
\begin{codeblock}
construct(p, piecewise_construct,
          forward_as_tuple(std::forward<U>(x)),
          forward_as_tuple(std::forward<V>(y)));
\end{codeblock}
\end{itemdescr}

\indexlibrarymember{construct}{polymorphic_allocator}%
\begin{itemdecl}
template <class T1, class T2, class U, class V>
  void construct(pair<T1,T2>* p, const pair<U, V>& pr);
\end{itemdecl}

\begin{itemdescr}
\pnum
\effects
Equivalent to:
\begin{codeblock}
construct(p, piecewise_construct,
          forward_as_tuple(pr.first),
          forward_as_tuple(pr.second));
\end{codeblock}
\end{itemdescr}

\indexlibrarymember{construct}{polymorphic_allocator}%
\begin{itemdecl}
template <class T1, class T2, class U, class V>
  void construct(pair<T1,T2>* p, pair<U, V>&& pr);
\end{itemdecl}

\begin{itemdescr}
\pnum
\effects
Equivalent to:
\begin{codeblock}
construct(p, piecewise_construct,
          forward_as_tuple(std::forward<U>(pr.first)),
          forward_as_tuple(std::forward<V>(pr.second)));
\end{codeblock}
\end{itemdescr}

\indexlibrarymember{destroy}{polymorphic_allocator}%
\begin{itemdecl}
template <class T>
  void destroy(T* p);
\end{itemdecl}

\begin{itemdescr}
\pnum
\effects
As if by \tcode{p->\~T()}.
\end{itemdescr}

\indexlibrarymember{select_on_container_copy_construction}{polymorphic_allocator}%
\begin{itemdecl}
polymorphic_allocator select_on_container_copy_construction() const;
\end{itemdecl}

\begin{itemdescr}
\pnum
\returns
\tcode{polymorphic_allocator()}.

\pnum
\begin{note}
The memory resource is not propagated.
\end{note}
\end{itemdescr}

\indexlibrarymember{resource}{polymorphic_allocator}%
\begin{itemdecl}
memory_resource* resource() const;
\end{itemdecl}

\begin{itemdescr}
\pnum
\returns
\tcode{memory_rsrc}.
\end{itemdescr}

\rSec3[mem.poly.allocator.eq]{\tcode{polymorphic_allocator} equality}

\indexlibrarymember{operator==}{polymorphic_allocator}%
\begin{itemdecl}
template <class T1, class T2>
  bool operator==(const polymorphic_allocator<T1>& a,
                  const polymorphic_allocator<T2>& b) noexcept;
\end{itemdecl}

\begin{itemdescr}
\pnum
\returns
\tcode{*a.resource() == *b.resource()}.
\end{itemdescr}

\indexlibrarymember{operator"!=}{polymorphic_allocator}%
\begin{itemdecl}
template <class T1, class T2>
  bool operator!=(const polymorphic_allocator<T1>& a,
                  const polymorphic_allocator<T2>& b) noexcept;
\end{itemdecl}

\begin{itemdescr}
\pnum
\returns
\tcode{!(a == b)}.
\end{itemdescr}


\rSec2[mem.res.global]{Access to program-wide \tcode{memory_resource} objects}

\indexlibrary{\idxcode{new_delete_resource}}%
\begin{itemdecl}
memory_resource* new_delete_resource() noexcept;
\end{itemdecl}

\begin{itemdescr}
\pnum
\returns
A pointer to a static-duration object of a type derived from \tcode{memory_resource}
that can serve as a resource for allocating memory
using \tcode{::operator new} and \tcode{::operator delete}.
The same value is returned every time this function is called.
For a return value \tcode{p} and a memory resource \tcode{r},
\tcode{p->is_equal(r)} returns \tcode{\&r == p}.
\end{itemdescr}

\indexlibrary{\idxcode{null_memory_resource}}%
\begin{itemdecl}
memory_resource* null_memory_resource() noexcept;
\end{itemdecl}

\begin{itemdescr}
\pnum
\returns
A pointer to a static-duration object of a type derived from \tcode{memory_resource}
for which \tcode{allocate()} always throws \tcode{bad_alloc} and
for which \tcode{deallocate()} has no effect.
The same value is returned every time this function is called.
For a return value \tcode{p} and a memory resource \tcode{r},
\tcode{p->is_equal(r)} returns \tcode{\&r == p}.
\end{itemdescr}

\pnum
The \defn{default memory resource pointer} is a pointer to a memory resource
that is used by certain facilities when an explicit memory resource
is not supplied through the interface.
Its initial value is the return value of \tcode{new_delete_resource()}.

\indexlibrary{\idxcode{set_default_resource}}%
\begin{itemdecl}
memory_resource* set_default_resource(memory_resource* r) noexcept;
\end{itemdecl}

\begin{itemdescr}
\pnum
\effects
If \tcode{r} is non-null,
sets the value of the default memory resource pointer to \tcode{r},
otherwise sets the default memory resource pointer to \tcode{new_delete_resource()}.

\pnum
\postconditions
\tcode{get_default_resource() == r}.

\pnum
\returns
The previous value of the default memory resource pointer.

\pnum
\remarks
Calling the \tcode{set_default_resource} and
\tcode{get_default_resource} functions shall not incur a data race.
A call to the \tcode{set_default_resource} function
shall synchronize with subsequent calls to
the \tcode{set_default_resource} and \tcode{get_default_resource} functions.
\end{itemdescr}

\indexlibrary{\idxcode{get_default_resource}}%
\begin{itemdecl}
memory_resource* get_default_resource() noexcept;
\end{itemdecl}

\begin{itemdescr}
\pnum
\returns
The current value of the default memory resource pointer.
\end{itemdescr}

\rSec2[mem.res.pool]{Pool resource classes}

\rSec3[mem.res.pool.overview]{Classes \tcode{synchronized_pool_resource} and \tcode{unsynchronized_pool_resource}}

\pnum
The \tcode{synchronized_pool_resource} and
\tcode{unsynchronized_pool_resource} classes
(collectively called \defn{pool resource classes})
are general-purpose memory resources having the following qualities:
\begin{itemize}
\item
Each resource \term{owns} the allocated memory, and frees it on destruction --
even if \tcode{deallocate} has not been called for some of the allocated blocks.
\item
A pool resource consists of a collection of \defn{pools},
serving requests for different block sizes.
Each individual pool manages a collection of \defn{chunks}
that are in turn divided into blocks of uniform size,
returned via calls to \tcode{do_allocate}.
Each call to \tcode{do_allocate(size, alignment)} is dispatched
to the pool serving the smallest blocks accommodating at least \tcode{size} bytes.
\item
When a particular pool is exhausted,
allocating a block from that pool results in the allocation
of an additional chunk of memory from the \defn{upstream allocator}
(supplied at construction), thus replenishing the pool.
With each successive replenishment,
the chunk size obtained increases geometrically.
\begin{note}
By allocating memory in chunks,
the pooling strategy increases the chance that consecutive allocations
will be close together in memory.\end{note}
\item
Allocation requests that exceed the largest block size of any pool
are fulfilled directly from the upstream allocator.
\item
A \tcode{pool_options} struct may be passed to the pool resource constructors
to tune the largest block size and the maximum chunk size.
\end{itemize}

\pnum
A \tcode{synchronized_pool_resource} may be accessed from multiple threads
without external synchronization
and may have thread-specific pools to reduce synchronization costs.
An \tcode{unsynchronized_pool_resource} class may not be accessed
from multiple threads simultaneously
and thus avoids the cost of synchronization entirely
in single-threaded applications.

\indexlibrary{\idxcode{pool_options}}%
\indexlibrary{\idxcode{synchronized_pool_resource}}%
\indexlibrary{\idxcode{unsynchronized_pool_resource}}%
\begin{codeblock}
struct pool_options {
  size_t max_blocks_per_chunk = 0;
  size_t largest_required_pool_block = 0;
};

class synchronized_pool_resource : public memory_resource {
public:
  synchronized_pool_resource(const pool_options &opts,
                             memory_resource *upstream);

  synchronized_pool_resource()
      : synchronized_pool_resource(pool_options(), get_default_resource()) {}
  explicit synchronized_pool_resource(memory_resource *upstream)
      : synchronized_pool_resource(pool_options(), upstream) {}
  explicit synchronized_pool_resource(const pool_options &opts)
      : synchronized_pool_resource(opts, get_default_resource()) {}

  synchronized_pool_resource(const synchronized_pool_resource &) = delete;
  virtual ~synchronized_pool_resource();

  synchronized_pool_resource &
    operator=(const synchronized_pool_resource &) = delete;

  void release();
  memory_resource *upstream_resource() const;
  pool_options options() const;

protected:
  void *do_allocate(size_t bytes, size_t alignment) override;
  void do_deallocate(void *p, size_t bytes, size_t alignment) override;

  bool do_is_equal(const memory_resource &other) const noexcept override;
};

class unsynchronized_pool_resource : public memory_resource {
public:
  unsynchronized_pool_resource(const pool_options &opts,
                               memory_resource *upstream);

  unsynchronized_pool_resource()
      : unsynchronized_pool_resource(pool_options(), get_default_resource()) {}
  explicit unsynchronized_pool_resource(memory_resource *upstream)
      : unsynchronized_pool_resource(pool_options(), upstream) {}
  explicit unsynchronized_pool_resource(const pool_options &opts)
      : unsynchronized_pool_resource(opts, get_default_resource()) {}

  unsynchronized_pool_resource(const unsynchronized_pool_resource &) = delete;
  virtual ~unsynchronized_pool_resource();

  unsynchronized_pool_resource &
    operator=(const unsynchronized_pool_resource &) = delete;

  void release();
  memory_resource *upstream_resource() const;
  pool_options options() const;

protected:
  void *do_allocate(size_t bytes, size_t alignment) override;
  void do_deallocate(void *p, size_t bytes, size_t alignment) override;

  bool do_is_equal(const memory_resource &other) const noexcept override;
};
\end{codeblock}

\rSec3[mem.res.pool.options]{\tcode{pool_options} data members}

\pnum
The members of \tcode{pool_options}
comprise a set of constructor options for pool resources.
The effect of each option on the pool resource behavior is described below:

\indexlibrary{\idxcode{pool_options}!\idxcode{max_blocks_per_chunk}}%
\begin{itemdecl}
size_t max_blocks_per_chunk;
\end{itemdecl}

\begin{itemdescr}
\pnum
The maximum number of blocks that will be allocated at once
from the upstream memory resource~(\ref{mem.res.monotonic.buffer})
to replenish a pool.
If the value of \tcode{max_blocks_per_chunk} is zero or
is greater than an \impldef{largest supported value to configure the maximum number of blocks to replenish a pool}
limit, that limit is used instead.
The implementation
may choose to use a smaller value than is specified in this field and
may use different values for different pools.
\end{itemdescr}

\indexlibrary{\idxcode{pool_options}!\idxcode{largest_required_pool_block}}%
\begin{itemdecl}
size_t largest_required_pool_block;
\end{itemdecl}

\begin{itemdescr}
\pnum
The largest allocation size that is required to be fulfilled
using the pooling mechanism.
Attempts to allocate a single block larger than this threshold
will be allocated directly from the upstream memory resource.
If \tcode{largest_required_pool_block} is zero or
is greater than an \impldef{largest supported value to configure the largest allocation satisfied directly by a pool}
limit, that limit is used instead.
The implementation may choose a pass-through threshold
larger than specified in this field.
\end{itemdescr}

\rSec3[mem.res.pool.ctor]{Pool resource constructors and destructors}

\indexlibrary{\idxcode{synchronized_pool_resource}!constructor}%
\indexlibrary{\idxcode{unsynchronized_pool_resource}!constructor}%
\begin{itemdecl}
synchronized_pool_resource(const pool_options& opts, memory_resource* upstream);
unsynchronized_pool_resource(const pool_options& opts, memory_resource* upstream);
\end{itemdecl}

\begin{itemdescr}
\pnum
\requires
\tcode{upstream} is the address of a valid memory resource.

\pnum
\effects
Constructs a pool resource object that will obtain memory from \tcode{upstream}
whenever the pool resource is unable to satisfy a memory request
from its own internal data structures.
The resulting object will hold a copy of \tcode{upstream},
but will not own the resource to which \tcode{upstream} points.
\begin{note}
The intention is that calls to \tcode{upstream->allocate()}
will be substantially fewer than calls to \tcode{this->allocate()}
in most cases.\end{note}
The behavior of the pooling mechanism is tuned
according to the value of the \tcode{opts} argument.

\pnum
\throws
Nothing unless \tcode{upstream->allocate()} throws.
It is unspecified if, or under what conditions,
this constructor calls \tcode{upstream->allocate()}.
\end{itemdescr}

\indexlibrary{\idxcode{synchronized_pool_resource}!destructor}%
\indexlibrary{\idxcode{unsynchronized_pool_resource}!destructor}%
\begin{itemdecl}
virtual ~synchronized_pool_resource();
virtual ~unsynchronized_pool_resource();
\end{itemdecl}

\begin{itemdescr}
\pnum
\effects
Calls \tcode{release()}.
\end{itemdescr}

\rSec3[mem.res.pool.mem]{Pool resource members}

\indexlibrarymember{release}{synchronized_pool_resource}%
\indexlibrarymember{release}{unsynchronized_pool_resource}%
\begin{itemdecl}
void release();
\end{itemdecl}

\begin{itemdescr}
\pnum
\effects
Calls \tcode{upstream_resource()->deallocate()} as necessary
to release all allocated memory.
\begin{note}
The memory is released back to \tcode{upstream_resource()}
even if \tcode{deallocate} has not been called
for some of the allocated blocks.\end{note}
\end{itemdescr}

\indexlibrarymember{upstream_resource}{synchronized_pool_resource}%
\indexlibrarymember{upstream_resource}{unsynchronized_pool_resource}%
\begin{itemdecl}
memory_resource* upstream_resource() const;
\end{itemdecl}

\begin{itemdescr}
\pnum
\returns
The value of the \tcode{upstream} argument
provided to the constructor of this object.
\end{itemdescr}

\indexlibrarymember{options}{synchronized_pool_resource}%
\indexlibrarymember{options}{unsynchronized_pool_resource}%
\begin{itemdecl}
pool_options options() const;
\end{itemdecl}

\begin{itemdescr}
\pnum
\returns
The options that control the pooling behavior of this resource.
The values in the returned struct may differ
from those supplied to the pool resource constructor in that
values of zero will be replaced with \impldef{default configuration of a pool}
defaults, and sizes may be rounded to unspecified granularity.
\end{itemdescr}

\indexlibrarymember{do_allocate}{synchronized_pool_resource}%
\indexlibrarymember{do_allocate}{unsynchronized_pool_resource}%
\begin{itemdecl}
void* do_allocate(size_t bytes, size_t alignment) override;
\end{itemdecl}

\begin{itemdescr}
\pnum
\returns
A pointer to allocated storage (\ref{basic.stc.dynamic.deallocation})
with a size of at least \tcode{bytes}.
The size and alignment of the allocated memory shall meet the requirements
for a class derived from \tcode{memory_resource}~(\ref{mem.res}).

\pnum
\effects
If the pool selected for a block of size \tcode{bytes}
is unable to satisfy the memory request from its own internal data structures,
it will call \tcode{upstream_resource()->allocate()} to obtain more memory.
If \tcode{bytes} is larger than that which the largest pool can handle,
then memory will be allocated using \tcode{upstream_resource()->allocate()}.

\pnum
\throws
Nothing unless \tcode{upstream_resource()->allocate()} throws.
\end{itemdescr}

\indexlibrarymember{do_deallocate}{synchronized_pool_resource}%
\indexlibrarymember{do_deallocate}{unsynchronized_pool_resource}%
\begin{itemdecl}
void do_deallocate(void* p, size_t bytes, size_t alignment) override;
\end{itemdecl}

\begin{itemdescr}
\pnum
\effects
Returns the memory at \tcode{p} to the pool.
It is unspecified if, or under what circumstances,
this operation will result in a call to \tcode{upstream_resource()->deallocate()}.

\pnum
\throws
Nothing.
\end{itemdescr}

\indexlibrarymember{do_is_equal}{unsynchronized_pool_resource}%
\begin{itemdecl}
bool unsynchronized_pool_resource::do_is_equal(const memory_resource& other) const noexcept override;
\end{itemdecl}

\begin{itemdescr}
\pnum
\returns
\tcode{this == dynamic_cast<const unsynchronized_pool_resource*>(\&other)}.
\end{itemdescr}

\indexlibrarymember{do_is_equal}{synchronized_pool_resource}%
\begin{itemdecl}
bool synchronized_pool_resource::do_is_equal(const memory_resource& other) const noexcept override;
\end{itemdecl}

\begin{itemdescr}
\pnum
\returns
\tcode{this == dynamic_cast<const synchronized_pool_resource*>(\&other)}.
\end{itemdescr}

\rSec2[mem.res.monotonic.buffer]{Class \tcode{monotonic_buffer_resource}}

\pnum
A \tcode{monotonic_buffer_resource} is a special-purpose memory resource
intended for very fast memory allocations in situations
where memory is used to build up a few objects
and then is released all at once when the memory resource object is destroyed.
It has the following qualities:
\begin{itemize}
\item
A call to \tcode{deallocate} has no effect,
thus the amount of memory consumed increases monotonically
until the resource is destroyed.
\item
The program can supply an initial buffer,
which the allocator uses to satisfy memory requests.
\item
When the initial buffer (if any) is exhausted,
it obtains additional buffers from an \defn{upstream} memory resource
supplied at construction.
Each additional buffer is larger than the previous one,
following a geometric progression.
\item
It is intended for access from one thread of control at a time.
Specifically, calls to \tcode{allocate} and \tcode{deallocate}
do not synchronize with one another.
\item
It \term{owns} the allocated memory and frees it on destruction,
even if \tcode{deallocate} has not been called for some of the allocated blocks.
\end{itemize}

\indexlibrary{\idxcode{monotonic_buffer_resource}}%
\begin{codeblock}
class monotonic_buffer_resource : public memory_resource {
  memory_resource *upstream_rsrc; // \expos
  void *current_buffer;           // \expos
  size_t next_buffer_size;        // \expos

public:
  explicit monotonic_buffer_resource(memory_resource *upstream);
  monotonic_buffer_resource(size_t initial_size, memory_resource *upstream);
  monotonic_buffer_resource(void *buffer, size_t buffer_size,
                            memory_resource *upstream);

  monotonic_buffer_resource()
      : monotonic_buffer_resource(get_default_resource()) {}
  explicit monotonic_buffer_resource(size_t initial_size)
      : monotonic_buffer_resource(initial_size, get_default_resource()) {}
  monotonic_buffer_resource(void *buffer, size_t buffer_size)
      : monotonic_buffer_resource(buffer, buffer_size, get_default_resource()) {}

  monotonic_buffer_resource(const monotonic_buffer_resource &) = delete;

  virtual ~monotonic_buffer_resource();

  monotonic_buffer_resource
    operator=(const monotonic_buffer_resource &) = delete;

  void release();
  memory_resource *upstream_resource() const;

protected:
  void *do_allocate(size_t bytes, size_t alignment) override;
  void do_deallocate(void *p, size_t bytes, size_t alignment) override;

  bool do_is_equal(const memory_resource &other) const noexcept override;
};
\end{codeblock}

\rSec3[mem.res.monotonic.buffer.ctor]{\tcode{monotonic_buffer_resource} constructor and destructor}

\indexlibrary{\idxcode{monotonic_buffer_resource}!constructor}%
\begin{itemdecl}
explicit monotonic_buffer_resource(memory_resource* upstream);
monotonic_buffer_resource(size_t initial_size, memory_resource* upstream);
\end{itemdecl}

\begin{itemdescr}
\pnum
\requires
\tcode{upstream} shall be the address of a valid memory resource.
\tcode{initial_size}, if specified, shall be greater than zero.

\pnum
\effects
Sets \tcode{upstream_rsrc} to \tcode{upstream} and
\tcode{current_buffer} to \tcode{nullptr}.
If \tcode{initial_size} is specified,
sets \tcode{next_buffer_size} to at least \tcode{initial_size};
otherwise sets \tcode{next_buffer_size} to an
\impldef{default \tcode{next_buffer_size} for a \tcode{monotonic_buffer_resource}} size.
\end{itemdescr}

\indexlibrary{\idxcode{monotonic_buffer_resource}!constructor}%
\begin{itemdecl}
monotonic_buffer_resource(void* buffer, size_t buffer_size, memory_resource* upstream);
\end{itemdecl}

\begin{itemdescr}
\pnum
\requires
\tcode{upstream} shall be the address of a valid memory resource.
\tcode{buffer_size} shall be no larger than the number of bytes in \tcode{buffer}.

\pnum
\effects
Sets \tcode{upstream_rsrc} to \tcode{upstream},
\tcode{current_buffer} to \tcode{buffer}, and
\tcode{next_buffer_size} to \tcode{initial_size} (but not less than 1),
then increases \tcode{next_buffer_size}
by an \impldef{growth factor for \tcode{monotonic_buffer_resource}} growth factor (which need not be integral).
\end{itemdescr}

\indexlibrary{\idxcode{monotonic_buffer_resource}!destructor}%
\begin{itemdecl}
~monotonic_buffer_resource();
\end{itemdecl}

\begin{itemdescr}
\pnum
\effects
Calls \tcode{release()}.
\end{itemdescr}


\rSec3[mem.res.monotonic.buffer.mem]{\tcode{monotonic_buffer_resource} members}

\indexlibrarymember{release}{monotonic_buffer_resource}%
\begin{itemdecl}
void release();
\end{itemdecl}

\begin{itemdescr}
\pnum
\effects
Calls \tcode{upstream_rsrc->deallocate()} as necessary
to release all allocated memory.

\pnum
\begin{note}
The memory is released back to \tcode{upstream_rsrc}
even if some blocks that were allocated from \tcode{this}
have not been deallocated from \tcode{this}.\end{note}
\end{itemdescr}

\indexlibrarymember{upstream_resource}{monotonic_buffer_resource}%
\begin{itemdecl}
memory_resource* upstream_resource() const;
\end{itemdecl}

\begin{itemdescr}
\pnum
\returns
The value of \tcode{upstream_rsrc}.
\end{itemdescr}

\indexlibrarymember{do_allocate}{monotonic_buffer_resource}%
\begin{itemdecl}
void* do_allocate(size_t bytes, size_t alignment) override;
\end{itemdecl}

\begin{itemdescr}
\pnum
\returns
A pointer to allocated storage (\ref{basic.stc.dynamic.deallocation})
with a size of at least \tcode{bytes}.
The size and alignment of the allocated memory shall meet the requirements
for a class derived from \tcode{memory_resource}~(\ref{mem.res}).

\pnum
\effects
If the unused space in \tcode{current_buffer}
can fit a block with the specified \tcode{bytes} and \tcode{alignment},
then allocate the return block from \tcode{current_buffer};
otherwise set \tcode{current_buffer} to \tcode{upstream_rsrc->allocate(n, m)},
where \tcode{n} is not less than \tcode{max(bytes, next_buffer_size)} and
\tcode{m} is not less than \tcode{alignment},
and increase \tcode{next_buffer_size}
by an \impldef{growth factor for \tcode{monotonic_buffer_resource}} growth factor (which need not be integral),
then allocate the return block from the newly-allocated \tcode{current_buffer}.

\pnum
\throws
Nothing unless \tcode{upstream_rsrc->allocate()} throws.
\end{itemdescr}

\indexlibrarymember{do_deallocate}{monotonic_buffer_resource}%
\begin{itemdecl}
void do_deallocate(void* p, size_t bytes, size_t alignment) override;
\end{itemdecl}

\begin{itemdescr}
\pnum
\effects
None.

\pnum
\throws
Nothing.

\pnum
\remarks
Memory used by this resource increases monotonically until its destruction.
\end{itemdescr}

\indexlibrarymember{do_is_equal}{monotonic_buffer_resource}%
\begin{itemdecl}
bool do_is_equal(const memory_resource& other) const noexcept override;
\end{itemdecl}

\begin{itemdescr}
\pnum
\returns
\tcode{this == dynamic_cast<const monotonic_buffer_resource*>(\&other)}.
\end{itemdescr}


\rSec1[allocator.adaptor]{Class template \tcode{scoped_allocator_adaptor}}

\rSec2[allocator.adaptor.syn]{Header \tcode{<scoped_allocator>} synopsis}

\indexlibrary{\idxhdr{scoped_allocator}}%
\begin{codeblock}
  // scoped allocator adaptor
  template <class OuterAlloc, class... InnerAlloc>
    class scoped_allocator_adaptor;
  template <class OuterA1, class OuterA2, class... InnerAllocs>
    bool operator==(const scoped_allocator_adaptor<OuterA1, InnerAllocs...>& a,
                    const scoped_allocator_adaptor<OuterA2, InnerAllocs...>& b) noexcept;
  template <class OuterA1, class OuterA2, class... InnerAllocs>
    bool operator!=(const scoped_allocator_adaptor<OuterA1, InnerAllocs...>& a,
                    const scoped_allocator_adaptor<OuterA2, InnerAllocs...>& b) noexcept;
\end{codeblock}

\pnum
The class template \tcode{scoped_allocator_adaptor} is an allocator template that
specifies the memory resource (the outer allocator) to be used by a container (as any
other allocator does) and also specifies an inner allocator resource to be passed to the
constructor of every element within the container. This adaptor is instantiated with one
outer and zero or more inner allocator types. If instantiated with only one allocator
type, the inner allocator becomes the \tcode{scoped_allocator_adaptor} itself, thus
using the same allocator resource for the container and every element within the
container and, if the elements themselves are containers, each of their elements
recursively. If instantiated with more than one allocator, the first allocator is the
outer allocator for use by the container, the second allocator is passed to the
constructors of the container's elements, and, if the elements themselves are
containers, the third allocator is passed to the elements' elements, and so on. If
containers are nested to a depth greater than the number of allocators, the last
allocator is used repeatedly, as in the single-allocator case, for any remaining
recursions. \begin{note} The \tcode{scoped_allocator_adaptor} is derived from the outer
allocator type so it can be substituted for the outer allocator type in most
expressions. \end{note}

\indexlibrary{\idxcode{scoped_allocator_adaptor}}%
\begin{codeblock}
namespace std {
  template <class OuterAlloc, class... InnerAllocs>
    class scoped_allocator_adaptor : public OuterAlloc {
  private:
    using OuterTraits = allocator_traits<OuterAlloc>; // \expos
    scoped_allocator_adaptor<InnerAllocs...> inner;   // \expos
  public:
    using outer_allocator_type = OuterAlloc;
    using inner_allocator_type = @\seebelow@;

    using value_type           = typename OuterTraits::value_type;
    using size_type            = typename OuterTraits::size_type;
    using difference_type      = typename OuterTraits::difference_type;
    using pointer              = typename OuterTraits::pointer;
    using const_pointer        = typename OuterTraits::const_pointer;
    using void_pointer         = typename OuterTraits::void_pointer;
    using const_void_pointer   = typename OuterTraits::const_void_pointer;

    using propagate_on_container_copy_assignment = @\seebelow@;
    using propagate_on_container_move_assignment = @\seebelow@;
    using propagate_on_container_swap            = @\seebelow@;
    using is_always_equal                        = @\seebelow@;

    template <class Tp>
      struct rebind {
        using other = scoped_allocator_adaptor<
          OuterTraits::template rebind_alloc<Tp>, InnerAllocs...>;
      };

    scoped_allocator_adaptor();
    template <class OuterA2>
      scoped_allocator_adaptor(OuterA2&& outerAlloc,
                               const InnerAllocs&... innerAllocs) noexcept;

    scoped_allocator_adaptor(const scoped_allocator_adaptor& other) noexcept;
    scoped_allocator_adaptor(scoped_allocator_adaptor&& other) noexcept;

    template <class OuterA2>
      scoped_allocator_adaptor(
        const scoped_allocator_adaptor<OuterA2, InnerAllocs...>& other) noexcept;
    template <class OuterA2>
      scoped_allocator_adaptor(
        scoped_allocator_adaptor<OuterA2, InnerAllocs...>&& other) noexcept;

    scoped_allocator_adaptor& operator=(const scoped_allocator_adaptor&) = default;
    scoped_allocator_adaptor& operator=(scoped_allocator_adaptor&&) = default;

    ~scoped_allocator_adaptor();

    inner_allocator_type& inner_allocator() noexcept;
    const inner_allocator_type& inner_allocator() const noexcept;
    outer_allocator_type& outer_allocator() noexcept;
    const outer_allocator_type& outer_allocator() const noexcept;

    pointer allocate(size_type n);
    pointer allocate(size_type n, const_void_pointer hint);
    void deallocate(pointer p, size_type n);
    size_type max_size() const;

    template <class T, class... Args>
      void construct(T* p, Args&&... args);
    template <class T1, class T2, class... Args1, class... Args2>
      void construct(pair<T1, T2>* p, piecewise_construct_t,
                     tuple<Args1...> x, tuple<Args2...> y);
    template <class T1, class T2>
      void construct(pair<T1, T2>* p);
    template <class T1, class T2, class U, class V>
      void construct(pair<T1, T2>* p, U&& x, V&& y);
    template <class T1, class T2, class U, class V>
      void construct(pair<T1, T2>* p, const pair<U, V>& x);
    template <class T1, class T2, class U, class V>
      void construct(pair<T1, T2>* p, pair<U, V>&& x);

    template <class T>
      void destroy(T* p);

    scoped_allocator_adaptor select_on_container_copy_construction() const;
  };

  template <class OuterA1, class OuterA2, class... InnerAllocs>
    bool operator==(const scoped_allocator_adaptor<OuterA1, InnerAllocs...>& a,
                    const scoped_allocator_adaptor<OuterA2, InnerAllocs...>& b) noexcept;
  template <class OuterA1, class OuterA2, class... InnerAllocs>
    bool operator!=(const scoped_allocator_adaptor<OuterA1, InnerAllocs...>& a,
                    const scoped_allocator_adaptor<OuterA2, InnerAllocs...>& b) noexcept;
}
\end{codeblock}

\rSec2[allocator.adaptor.types]{Scoped allocator adaptor member types}

\indexlibrarymember{inner_allocator_type}{scoped_allocator_adaptor}%
\begin{itemdecl}
using inner_allocator_type = @\seebelow@;
\end{itemdecl}

\begin{itemdescr}
\pnum
\ctype \tcode{scoped_allocator_adaptor<OuterAlloc>} if \tcode{sizeof...(InnerAllocs)} is
zero; otherwise,\\ \tcode{scoped_allocator_adaptor<InnerAllocs...>}.
\end{itemdescr}

\indexlibrarymember{propagate_on_container_copy_assignment}{scoped_allocator_adaptor}%
\begin{itemdecl}
using propagate_on_container_copy_assignment = @\seebelow@;
\end{itemdecl}

\begin{itemdescr}
\pnum
\ctype \tcode{true_type} if
\tcode{allocator_traits<A>::propagate_on_container_copy_assignment::value} is
\tcode{true} for any \tcode{A} in the set of \tcode{OuterAlloc} and
\tcode{InnerAllocs...}; otherwise, \tcode{false_type}.
\end{itemdescr}

\indexlibrarymember{propagate_on_container_move_assignment}{scoped_allocator_adaptor}%
\begin{itemdecl}
using propagate_on_container_move_assignment = @\seebelow@;
\end{itemdecl}

\begin{itemdescr}
\pnum
\ctype \tcode{true_type} if
\tcode{allocator_traits<A>::propagate_on_container_move_assignment::value} is
\tcode{true} for any \tcode{A} in the set of \tcode{OuterAlloc} and
\tcode{InnerAllocs...}; otherwise, \tcode{false_type}.
\end{itemdescr}

\indexlibrarymember{propagate_on_container_swap}{scoped_allocator_adaptor}%
\begin{itemdecl}
using propagate_on_container_swap = @\seebelow@;
\end{itemdecl}

\begin{itemdescr}
\pnum
\ctype \tcode{true_type} if
\tcode{allocator_traits<A>::propagate_on_container_swap::value} is
\tcode{true} for any \tcode{A} in the set of \tcode{OuterAlloc} and
\tcode{InnerAllocs...}; otherwise, \tcode{false_type}.
\end{itemdescr}

\indexlibrarymember{is_always_equal}{scoped_allocator_adaptor}%
\begin{itemdecl}
using is_always_equal = @\seebelow@;
\end{itemdecl}

\begin{itemdescr}
\pnum
\ctype \tcode{true_type} if
\tcode{allocator_traits<A>::is_always_equal::value} is
\tcode{true} for every \tcode{A} in the set of \tcode{OuterAlloc} and
\tcode{InnerAllocs...}; otherwise, \tcode{false_type}.
\end{itemdescr}

\rSec2[allocator.adaptor.cnstr]{Scoped allocator adaptor constructors}

\indexlibrary{\idxcode{scoped_allocator_adaptor}!constructor}%
\begin{itemdecl}
scoped_allocator_adaptor();
\end{itemdecl}

\begin{itemdescr}
\pnum
\effects Value-initializes the \tcode{OuterAlloc} base class and the \tcode{inner} allocator
object.
\end{itemdescr}

\indexlibrary{\idxcode{scoped_allocator_adaptor}!constructor}%
\begin{itemdecl}
template <class OuterA2>
  scoped_allocator_adaptor(OuterA2&& outerAlloc,
                           const InnerAllocs&... innerAllocs) noexcept;
\end{itemdecl}

\begin{itemdescr}
\pnum
\requires \tcode{OuterAlloc} shall be constructible from \tcode{OuterA2}.

\pnum
\effects Initializes the \tcode{OuterAlloc} base class with
\tcode{std::forward<OuterA2>(outerAlloc)} and \tcode{inner} with \tcode{innerAllocs...}
(hence recursively initializing each allocator within the adaptor with the corresponding
allocator from the argument list).
\end{itemdescr}

\indexlibrary{\idxcode{scoped_allocator_adaptor}!constructor}%
\begin{itemdecl}
scoped_allocator_adaptor(const scoped_allocator_adaptor& other) noexcept;
\end{itemdecl}

\begin{itemdescr}
\pnum
\effects Initializes each allocator within the adaptor with the corresponding allocator
from \tcode{other}.
\end{itemdescr}

\indexlibrary{\idxcode{scoped_allocator_adaptor}!constructor}%
\begin{itemdecl}
scoped_allocator_adaptor(scoped_allocator_adaptor&& other) noexcept;
\end{itemdecl}

\begin{itemdescr}
\pnum
\effects Move constructs each allocator within the adaptor with the corresponding allocator
from \tcode{other}.
\end{itemdescr}

\indexlibrary{\idxcode{scoped_allocator_adaptor}!constructor}%
\begin{itemdecl}
template <class OuterA2>
  scoped_allocator_adaptor(const scoped_allocator_adaptor<OuterA2,
                                                          InnerAllocs...>& other) noexcept;
\end{itemdecl}

\begin{itemdescr}
\pnum
\requires \tcode{OuterAlloc} shall be constructible from \tcode{OuterA2}.

\pnum
\effects Initializes each allocator within the adaptor with the corresponding allocator
from \tcode{other}.
\end{itemdescr}

\indexlibrary{\idxcode{scoped_allocator_adaptor}!constructor}%
\begin{itemdecl}
template <class OuterA2>
  scoped_allocator_adaptor(scoped_allocator_adaptor<OuterA2,
                                                    InnerAllocs...>&& other) noexcept;
\end{itemdecl}

\begin{itemdescr}
\pnum
\requires \tcode{OuterAlloc} shall be constructible from \tcode{OuterA2}.

\pnum
\effects Initializes each allocator within the adaptor with the corresponding allocator rvalue
from \tcode{other}.
\end{itemdescr}

\rSec2[allocator.adaptor.members]{Scoped allocator adaptor members}

\pnum
In the \tcode{construct} member functions,
\textit{OUTERMOST(x)} is \tcode{x} if \tcode{x} does not have an
\tcode{outer_allocator()} member function and \\
\textit{OUTERMOST(x.outer_allocator())}
otherwise;
\textit{OUTERMOST_ALLOC_TRAITS(x)} is \\
\tcode{allocator_traits<decltype(\textit{OUTERMOST}(x))>}.
\begin{note} \textit{OUTERMOST}(x) and \\
\textit{OUTERMOST_ALLOC_TRAITS}(x) are recursive operations. It
is incumbent upon the definition of \tcode{outer_allocator()} to ensure that the
recursion terminates. It will terminate for all instantiations of \\
\tcode{scoped_allocator_adaptor}. \end{note}

\indexlibrarymember{inner_allocator}{scoped_allocator_adaptor}%
\begin{itemdecl}
inner_allocator_type& inner_allocator() noexcept;
const inner_allocator_type& inner_allocator() const noexcept;
\end{itemdecl}

\begin{itemdescr}
\pnum
\returns \tcode{*this} if \tcode{sizeof...(InnerAllocs)} is zero; otherwise,
\tcode{inner}.
\end{itemdescr}

\indexlibrarymember{outer_allocator}{scoped_allocator_adaptor}%
\begin{itemdecl}
outer_allocator_type& outer_allocator() noexcept;
\end{itemdecl}

\begin{itemdescr}
\pnum
\returns \tcode{static_cast<OuterAlloc\&>(*this)}.
\end{itemdescr}

\indexlibrarymember{outer_allocator}{scoped_allocator_adaptor}%
\begin{itemdecl}
const outer_allocator_type& outer_allocator() const noexcept;
\end{itemdecl}

\begin{itemdescr}
\pnum
\returns \tcode{static_cast<const OuterAlloc\&>(*this)}.
\end{itemdescr}

\indexlibrarymember{allocate}{scoped_allocator_adaptor}%
\begin{itemdecl}
pointer allocate(size_type n);
\end{itemdecl}

\begin{itemdescr}
\pnum
\returns \tcode{allocator_traits<OuterAlloc>::allocate(outer_allocator(), n)}.
\end{itemdescr}

\indexlibrarymember{allocate}{scoped_allocator_adaptor}%
\begin{itemdecl}
pointer allocate(size_type n, const_void_pointer hint);
\end{itemdecl}

\begin{itemdescr}
\pnum
\returns \tcode{allocator_traits<OuterAlloc>::allocate(outer_allocator(), n, hint)}.
\end{itemdescr}

\indexlibrarymember{deallocate}{scoped_allocator_adaptor}%
\begin{itemdecl}
void deallocate(pointer p, size_type n) noexcept;
\end{itemdecl}

\begin{itemdescr}
\pnum
\effects As if by:
\tcode{allocator_traits<OuterAlloc>::deallocate(outer_allocator(), p, n);}
\end{itemdescr}

\indexlibrarymember{max_size}{scoped_allocator_adaptor}%
\begin{itemdecl}
size_type max_size() const;
\end{itemdecl}

\begin{itemdescr}
\pnum
\returns \tcode{allocator_traits<OuterAlloc>::max_size(outer_allocator())}.
\end{itemdescr}

\indexlibrarymember{construct}{scoped_allocator_adaptor}%
\begin{itemdecl}
template <class T, class... Args>
  void construct(T* p, Args&&... args);
\end{itemdecl}

\begin{itemdescr}
\pnum
\effects

\begin{itemize}
\item If \tcode{uses_allocator_v<T, inner_allocator_type>} is \tcode{false} and\\
\tcode{is_constructible_v<T, Args...>} is \tcode{true}, calls\\
\textit{OUTERMOST_ALLOC_TRAITS}(\tcode{*this})\tcode{::construct(\\
\textit{OUTERMOST}(*this), p, std::forward<Args>(args)...)}.

\item Otherwise, if \tcode{uses_allocator_v<T, inner_allocator_type>} is \tcode{true} and
\tcode{is_construc\-tible_v<T, allocator_arg_t, inner_allocator_type\&, Args...>} is \tcode{true}, calls
\textit{OUTERMOST_ALLOC_TRAITS}(\tcode{*this})\tcode{::construct(\textit{OUTERMOST}(*this),
p, allocator_arg,\\inner_allocator(), std::forward<Args>(args)...)}.

\item Otherwise, if \tcode{uses_allocator_v<T, inner_allocator_type>} is \tcode{true} and
\tcode{is_construct\-ible_v<T, Args..., inner_allocator_type\&>} is \tcode{true}, calls
\textit{OUTERMOST_ALLOC_TRAITS}(*this)::
\tcode{construct(\textit{OUTERMOST}(*this), p, std::forward<Args>(args)...,\\inner_allocator())}.

\item Otherwise, the program is ill-formed. \begin{note} An error will result if
\tcode{uses_allocator} evaluates to \tcode{true} but the specific constructor does not take an
allocator. This definition prevents a silent failure to pass an inner allocator to a
contained element. \end{note}
\end{itemize}
\end{itemdescr}

\indexlibrarymember{construct}{scoped_allocator_adaptor}%
\begin{itemdecl}
template <class T1, class T2, class... Args1, class... Args2>
  void construct(pair<T1, T2>* p, piecewise_construct_t,
                 tuple<Args1...> x, tuple<Args2...> y);
\end{itemdecl}

\begin{itemdescr}
\pnum
\requires all of the types in \tcode{Args1} and \tcode{Args2} shall be
\tcode{CopyConstructible} (Table~\ref{tab:copyconstructible}).

\pnum
\effects Constructs a \tcode{tuple} object \tcode{xprime} from \tcode{x} by the
following rules:

\begin{itemize}
\item If \tcode{uses_allocator_v<T1, inner_allocator_type>} is \tcode{false} and\\
\tcode{is_constructible_v<T1, Args1...>} is \tcode{true},
then \tcode{xprime} is \tcode{x}.

\item Otherwise, if \tcode{uses_allocator_v<T1, inner_allocator_type>} is \tcode{true}
and
\tcode{is_construct\-ible_v<T1, allocator_arg_t, inner_allocator_type\&, Args1...>}
is
\tcode{true}, then \tcode{xprime} is
\tcode{tuple_cat(tuple<allocator_arg_t, inner_allocator_type\&>(
allocator_arg, inner_allocator()), std::move(x))}.

\item Otherwise, if \tcode{uses_allocator_v<T1, inner_allocator_type>} is
\tcode{true} and
\tcode{is_construct\-ible_v<T1, Args1..., inner_allocator_type\&>} is \tcode{true},
then \tcode{xprime} is
\tcode{tuple_cat(std::move(x), tuple<inner_allocator_type\&>(inner_allocator()))}.

\item Otherwise, the program is ill-formed.
\end{itemize}

and constructs a \tcode{tuple} object \tcode{yprime} from \tcode{y} by the following rules:

\begin{itemize}
\item If \tcode{uses_allocator_v<T2, inner_allocator_type>} is \tcode{false} and\\
\tcode{is_constructible_v<T2,
Args2...>} is \tcode{true}, then \tcode{yprime} is \tcode{y}.

\item Otherwise, if \tcode{uses_allocator_v<T2, inner_allocator_type>} is \tcode{true}
and
\tcode{is_construct\-ible_v<T2, allocator_arg_t, inner_allocator_type\&, Args2...>}
is
\tcode{true}, then \tcode{yprime} is
\tcode{tuple_cat(tuple<allocator_arg_t, inner_allocator_type\&>(
allocator_arg, inner_allocator()), std::move(y))}.

\item Otherwise, if \tcode{uses_allocator_v<T2, inner_allocator_type>} is
\tcode{true} and
\tcode{is_construct\-ible_v<T2, Args2..., inner_allocator_type\&>} is \tcode{true},
then \tcode{yprime} is
\tcode{tuple_cat(std::move(y), tuple<inner_allocator_type\&>(inner_allocator()))}.

\item Otherwise, the program is ill-formed.
\end{itemize}

then calls \tcode{\textit{OUTERMOST_ALLOC_TRAITS}(*this)::construct(\textit{OUTERMOST}(*this), p,\\
piecewise_construct, std::move(xprime), std::move(yprime))}.
\end{itemdescr}

\indexlibrarymember{construct}{scoped_allocator_adaptor}%
\begin{itemdecl}
template <class T1, class T2>
  void construct(pair<T1, T2>* p);
\end{itemdecl}

\begin{itemdescr}
\pnum
\effects Equivalent to:
\begin{codeblock}
construct(p, piecewise_construct, tuple<>(), tuple<>());
\end{codeblock}
\end{itemdescr}

\indexlibrarymember{construct}{scoped_allocator_adaptor}%
\begin{itemdecl}
template <class T1, class T2, class U, class V>
  void construct(pair<T1, T2>* p, U&& x, V&& y);
\end{itemdecl}

\begin{itemdescr}
\pnum
\effects Equivalent to:
\begin{codeblock}
construct(p, piecewise_construct,
          forward_as_tuple(std::forward<U>(x)),
          forward_as_tuple(std::forward<V>(y)));
\end{codeblock}
\end{itemdescr}

\indexlibrarymember{construct}{scoped_allocator_adaptor}%
\begin{itemdecl}
template <class T1, class T2, class U, class V>
  void construct(pair<T1, T2>* p, const pair<U, V>& x);
\end{itemdecl}

\begin{itemdescr}
\pnum
\effects Equivalent to:
\begin{codeblock}
construct(p, piecewise_construct,
          forward_as_tuple(x.first),
          forward_as_tuple(x.second));
\end{codeblock}
\end{itemdescr}

\indexlibrarymember{construct}{scoped_allocator_adaptor}%
\begin{itemdecl}
template <class T1, class T2, class U, class V>
  void construct(pair<T1, T2>* p, pair<U, V>&& x);
\end{itemdecl}

\begin{itemdescr}
\pnum
\effects Equivalent to:
\begin{codeblock}
construct(p, piecewise_construct,
          forward_as_tuple(std::forward<U>(x.first)),
          forward_as_tuple(std::forward<V>(x.second)));
\end{codeblock}
\end{itemdescr}

\indexlibrarymember{destroy}{scoped_allocator_adaptor}%
\begin{itemdecl}
template <class T>
  void destroy(T* p);
\end{itemdecl}

\begin{itemdescr}
\pnum
\effects Calls \tcode{\textit{OUTERMOST_ALLOC_TRAITS}(*this)::destroy(\textit{OUTERMOST}(*this), p)}.
\end{itemdescr}

\indexlibrarymember{select_on_container_copy_construction}{scoped_allocator_adaptor}%
\begin{itemdecl}
scoped_allocator_adaptor select_on_container_copy_construction() const;
\end{itemdecl}

\begin{itemdescr}
\pnum
\returns A new \tcode{scoped_allocator_adaptor} object where each allocator \tcode{A} in the
adaptor is initialized from the result of calling
\tcode{allocator_traits<A>::select_on_container_copy_construction()} on the
corresponding allocator in \tcode{*this}.
\end{itemdescr}

\rSec2[scoped.adaptor.operators]{Scoped allocator operators}

\indexlibrarymember{operator==}{scoped_allocator_adaptor}%
\begin{itemdecl}
template <class OuterA1, class OuterA2, class... InnerAllocs>
  bool operator==(const scoped_allocator_adaptor<OuterA1, InnerAllocs...>& a,
                  const scoped_allocator_adaptor<OuterA2, InnerAllocs...>& b) noexcept;
\end{itemdecl}

\begin{itemdescr}
\pnum
\returns If \tcode{sizeof...(InnerAllocs)} is zero,
\begin{codeblock}
a.outer_allocator() == b.outer_allocator()
\end{codeblock}
otherwise
\begin{codeblock}
a.outer_allocator() == b.outer_allocator() && a.inner_allocator() == b.inner_allocator()
\end{codeblock}
\end{itemdescr}

\indexlibrarymember{operator"!=}{scoped_allocator_adaptor}%
\begin{itemdecl}
template <class OuterA1, class OuterA2, class... InnerAllocs>
  bool operator!=(const scoped_allocator_adaptor<OuterA1, InnerAllocs...>& a,
                  const scoped_allocator_adaptor<OuterA2, InnerAllocs...>& b) noexcept;
\end{itemdecl}

\begin{itemdescr}
\pnum
\returns \tcode{!(a == b)}.
\end{itemdescr}

\rSec1[function.objects]{Function objects}

\pnum
A \indexdefn{function object!type}\term{function object type} is an object
type~(\ref{basic.types}) that can be the type of the
\grammarterm{postfix-expression} in a function call
(\ref{expr.call},~\ref{over.match.call}).\footnote{Such a type is a function
pointer or a class type which has a member \tcode{operator()} or a class type
which has a conversion to a pointer to function.} A \defn{function object} is an
object of a function object type. In the places where one would expect to pass a
pointer to a function to an algorithmic template (Clause~\ref{algorithms}), the
interface is specified to accept a function object. This not only makes
algorithmic templates work with pointers to functions, but also enables them to
work with arbitrary function objects.

\pnum
\synopsis{Header \tcode{<functional>} synopsis}

\indexlibrary{\idxhdr{functional}}%
\begin{codeblock}
namespace std {
  // \ref{func.invoke}, invoke:
  template <class F, class... Args> result_of_t<F&&(Args&&...)> invoke(F&& f, Args&&... args);

  // \ref{refwrap}, reference_wrapper:
  template <class T> class reference_wrapper;

  template <class T> reference_wrapper<T> ref(T&) noexcept;
  template <class T> reference_wrapper<const T> cref(const T&) noexcept;
  template <class T> void ref(const T&&) = delete;
  template <class T> void cref(const T&&) = delete;

  template <class T> reference_wrapper<T> ref(reference_wrapper<T>) noexcept;
  template <class T> reference_wrapper<const T> cref(reference_wrapper<T>) noexcept;

  // \ref{arithmetic.operations}, arithmetic operations:
  template <class T = void> struct plus;
  template <class T = void> struct minus;
  template <class T = void> struct multiplies;
  template <class T = void> struct divides;
  template <class T = void> struct modulus;
  template <class T = void> struct negate;
  template <> struct plus<void>;
  template <> struct minus<void>;
  template <> struct multiplies<void>;
  template <> struct divides<void>;
  template <> struct modulus<void>;
  template <> struct negate<void>;

  // \ref{comparisons}, comparisons:
  template <class T = void> struct equal_to;
  template <class T = void> struct not_equal_to;
  template <class T = void> struct greater;
  template <class T = void> struct less;
  template <class T = void> struct greater_equal;
  template <class T = void> struct less_equal;
  template <> struct equal_to<void>;
  template <> struct not_equal_to<void>;
  template <> struct greater<void>;
  template <> struct less<void>;
  template <> struct greater_equal<void>;
  template <> struct less_equal<void>;

  // \ref{logical.operations}, logical operations:
  template <class T = void> struct logical_and;
  template <class T = void> struct logical_or;
  template <class T = void> struct logical_not;
  template <> struct logical_and<void>;
  template <> struct logical_or<void>;
  template <> struct logical_not<void>;

  // \ref{bitwise.operations}, bitwise operations:
  template <class T = void> struct bit_and;
  template <class T = void> struct bit_or;
  template <class T = void> struct bit_xor;
  template <class T = void> struct bit_not;
  template <> struct bit_and<void>;
  template <> struct bit_or<void>;
  template <> struct bit_xor<void>;
  template <> struct bit_not<void>;

  // \ref{func.not_fn}, function template \tcode{not_fn}:
  template <class F> @\unspec@ not_fn(F&& f);

  // \ref{func.bind}, bind:
  template<class T> struct is_bind_expression;
  template<class T> struct is_placeholder;

  template<class F, class... BoundArgs>
    @\unspec@ bind(F&&, BoundArgs&&...);
  template<class R, class F, class... BoundArgs>
    @\unspec@ bind(F&&, BoundArgs&&...);

  namespace placeholders {
    // M is the \impldef{number of placeholders for bind expressions} number of placeholders
    @\seebelow@ _1;
    @\seebelow@ _2;
                .
                .
                .
    @\seebelow@ _M;
  }

  // \ref{func.memfn}, member function adaptors:
  template<class R, class T> @\unspec@ mem_fn(R T::*) noexcept;

  // \ref{func.wrap}, polymorphic function wrappers:
  class bad_function_call;

  template<class> class function; // not defined
  template<class R, class... ArgTypes> class function<R(ArgTypes...)>;

  template<class R, class... ArgTypes>
    void swap(function<R(ArgTypes...)>&, function<R(ArgTypes...)>&) noexcept;

  template<class R, class... ArgTypes>
    bool operator==(const function<R(ArgTypes...)>&, nullptr_t) noexcept;
  template<class R, class... ArgTypes>
    bool operator==(nullptr_t, const function<R(ArgTypes...)>&) noexcept;
  template<class R, class... ArgTypes>
    bool operator!=(const function<R(ArgTypes...)>&, nullptr_t) noexcept;
  template<class R, class... ArgTypes>
    bool operator!=(nullptr_t, const function<R(ArgTypes...)>&) noexcept;

  // \ref{func.search}, searchers:
  template<class ForwardIterator, class BinaryPredicate = equal_to<>>
    class default_searcher;
  template<class RandomAccessIterator,
           class Hash = hash<typename iterator_traits<RandomAccessIterator>::value_type>,
           class BinaryPredicate = equal_to<>>
    class boyer_moore_searcher;
  template<class RandomAccessIterator,
           class Hash = hash<typename iterator_traits<RandomAccessIterator>::value_type>,
           class BinaryPredicate = equal_to<>>
    class boyer_moore_horspool_searcher;

  template<class ForwardIterator, class BinaryPredicate = equal_to<>>
  default_searcher<ForwardIterator, BinaryPredicate>
  make_default_searcher(ForwardIterator pat_first, ForwardIterator pat_last,
                        BinaryPredicate pred = BinaryPredicate());
  template<class RandomAccessIterator,
           class Hash = hash<typename iterator_traits<RandomAccessIterator>::value_type>,
           class BinaryPredicate = equal_to<>>
  boyer_moore_searcher<RandomAccessIterator, Hash, BinaryPredicate>
  make_boyer_moore_searcher(
      RandomAccessIterator pat_first, RandomAccessIterator pat_last,
      Hash hf = Hash(), BinaryPredicate pred = BinaryPredicate());
  template<class RandomAccessIterator,
           class Hash = hash<typename iterator_traits<RandomAccessIterator>::value_type>,
           class BinaryPredicate = equal_to<>>
  boyer_moore_horspool_searcher<RandomAccessIterator, Hash, BinaryPredicate>
  make_boyer_moore_horspool_searcher(
      RandomAccessIterator pat_first, RandomAccessIterator pat_last,
      Hash hf = Hash(), BinaryPredicate pred = BinaryPredicate());

  // \ref{unord.hash}, hash function primary template:
  template <class T> struct hash;

  // Hash function specializations
  template <> struct hash<bool>;
  template <> struct hash<char>;
  template <> struct hash<signed char>;
  template <> struct hash<unsigned char>;
  template <> struct hash<char16_t>;
  template <> struct hash<char32_t>;
  template <> struct hash<wchar_t>;
  template <> struct hash<short>;
  template <> struct hash<unsigned short>;
  template <> struct hash<int>;
  template <> struct hash<unsigned int>;
  template <> struct hash<long>;
  template <> struct hash<long long>;
  template <> struct hash<unsigned long>;
  template <> struct hash<unsigned long long>;

  template <> struct hash<float>;
  template <> struct hash<double>;
  template <> struct hash<long double>;

  template<class T> struct hash<T*>;

  // \ref{func.default.traits}, default functor traits:
  template <class T = void>
  struct default_order;

  template <class T = void>
  using default_order_t = typename default_order<T>::type; 

  // \ref{func.bind}, function object binders:
  template <class T> constexpr bool is_bind_expression_v
    = is_bind_expression<T>::value;
  template <class T> constexpr int is_placeholder_v
    = is_placeholder<T>::value;
}
\end{codeblock}

\pnum
\begin{example}
If a \Cpp program wants to have a by-element addition of two vectors \tcode{a}
and \tcode{b} containing \tcode{double} and put the result into \tcode{a},
it can do:

\begin{codeblock}
transform(a.begin(), a.end(), b.begin(), a.begin(), plus<double>());
\end{codeblock}
\end{example}

\pnum
\begin{example}
To negate every element of \tcode{a}:

\begin{codeblock}
transform(a.begin(), a.end(), a.begin(), negate<double>());
\end{codeblock}

\end{example}

\rSec2[func.def]{Definitions}

\pnum
The following definitions apply to this Clause:

\pnum
\indexdefn{call signature}%
A \term{call signature} is the name of a return type followed by a
parenthesized comma-separated list of zero or more argument types.

\pnum
\indexdefn{callable type}%
A \term{callable type} is a function object type~(\ref{function.objects}) or a pointer to member.

\pnum
\indexdefn{callable object}%
A \term{callable object} is an object of a callable type.

\pnum
\indexdefn{call wrapper!type}%
A \term{call wrapper type} is a type that holds a callable object
and supports a call operation that forwards to that object.

\pnum
\indexdefn{call wrapper}%
A \term{call wrapper} is an object of a call wrapper type.

\pnum
\indexdefn{target object}%
A \term{target object} is the callable object held by a call wrapper.

\rSec2[func.require]{Requirements}

\pnum
\indexlibrary{invoke@\tcode{\textit{INVOKE}}}%
Define \tcode{\textit{INVOKE}(f, t1, t2, ..., tN)} as follows:

\begin{itemize}
\item \tcode{(t1.*f)(t2, ..., tN)} when \tcode{f} is a pointer to a
member function of a class \tcode{T}
and \tcode{is_base_of_v<T, decay_t<decltype(t1)>>} is \tcode{true};

\item \tcode{(t1.get().*f)(t2, ..., tN)} when \tcode{f} is a pointer to a
member function of a class \tcode{T}
and \tcode{decay_t<decltype(t1)>} is a specialization of \tcode{reference_wrapper};

\item \tcode{((*t1).*f)(t2, ..., tN)} when \tcode{f} is a pointer to a
member function of a class \tcode{T}
and \tcode{t1} does not satisfy the previous two items;

\item \tcode{t1.*f} when \tcode{N == 1} and \tcode{f} is a pointer to
data member of a class \tcode{T}
and \tcode{is_base_of_v<T, decay_t<decltype(t1)>>} is \tcode{true};
    
\item \tcode{t1.get().*f} when \tcode{N == 1} and \tcode{f} is a pointer to
data member of a class \tcode{T}
and \tcode{decay_t<decltype(t1)>} is a specialization of \tcode{reference_wrapper};

\item \tcode{(*t1).*f} when \tcode{N == 1} and \tcode{f} is a pointer to
data member of a class \tcode{T}
and \tcode{t1} does not satisfy the previous two items;

\item \tcode{f(t1, t2, ..., tN)} in all other cases.
\end{itemize}

\pnum
\indexlibrary{invoke@\tcode{\textit{INVOKE}}}%
Define \tcode{\textit{INVOKE}(f, t1, t2, ..., tN, R)} as
\tcode{static_cast<void>(\textit{INVOKE}(f, t1, t2, ..., tN))}
if \tcode{R} is \cv{} \tcode{void}, otherwise
\tcode{\textit{INVOKE}(f, t1, t2, ..., tN)} implicitly converted
to \tcode{R}.

\pnum
\indextext{call wrapper}%
\indextext{call wrapper!simple}%
\indextext{call wrapper!forwarding}%
\indextext{simple call wrapper}%
\indextext{forwarding call wrapper}%
Every call wrapper~(\ref{func.def}) shall be
\tcode{MoveConstructible}.
A \term{forwarding call wrapper} is a
call wrapper that can be called with an arbitrary argument list
and delivers the arguments to the wrapped callable object as references.
This forwarding step shall ensure that rvalue arguments are delivered as rvalue references
and lvalue arguments are delivered as lvalue references.
A \term{simple call wrapper} is a forwarding call wrapper that is
\tcode{CopyConstructible} and \tcode{CopyAssignable} and
whose copy constructor, move constructor, and assignment operator
do not throw exceptions.
\begin{note} In a typical implementation
forwarding call wrappers have an overloaded function call
operator of
the form

\begin{codeblock}
template<class... UnBoundArgs>
R operator()(UnBoundArgs&&... unbound_args) @\textit{cv-qual}@;
\end{codeblock}
\end{note}

\rSec2[func.invoke]{Function template \tcode{invoke}}
\indexlibrary{\idxcode{invoke}}%
\indexlibrary{invoke@\tcode{\textit{INVOKE}}}%
\begin{itemdecl}
template <class F, class... Args>
  result_of_t<F&&(Args&&...)> invoke(F&& f, Args&&... args);
\end{itemdecl}

\begin{itemdescr}
\pnum
\returns
\tcode{\textit{INVOKE}(std::forward<F>(f), std::forward<Args>(args)...)}~(\ref{func.require}).
\end{itemdescr}

\rSec2[refwrap]{Class template \tcode{reference_wrapper}}

\indexlibrary{\idxcode{reference_wrapper}}%
\indextext{function object!\idxcode{reference_wrapper}}%
\begin{codeblock}
namespace std {
  template <class T> class reference_wrapper {
  public :
    // types
    using type = T;

    // construct/copy/destroy
    reference_wrapper(T&) noexcept;
    reference_wrapper(T&&) = delete;     // do not bind to temporary objects
    reference_wrapper(const reference_wrapper& x) noexcept;

    // assignment
    reference_wrapper& operator=(const reference_wrapper& x) noexcept;

    // access
    operator T& () const noexcept;
    T& get() const noexcept;

    // invocation
    template <class... ArgTypes>
    result_of_t<T&(ArgTypes&&...)>
    operator() (ArgTypes&&...) const;
  };
}
\end{codeblock}

\pnum
\tcode{reference_wrapper<T>} is a \tcode{CopyConstructible} and \tcode{CopyAssignable} wrapper
around a reference to an object or function of type \tcode{T}.

\pnum
\tcode{reference_wrapper<T>} shall be a trivially copyable type~(\ref{basic.types}).

\rSec3[refwrap.const]{\tcode{reference_wrapper} construct/copy/destroy}

\indexlibrary{\idxcode{reference_wrapper}!constructor}%
\begin{itemdecl}
reference_wrapper(T& t) noexcept;
\end{itemdecl}

\begin{itemdescr}
\pnum
\effects Constructs a \tcode{reference_wrapper} object that stores a
reference to \tcode{t}.
\end{itemdescr}

\indexlibrary{\idxcode{reference_wrapper}!constructor}%
\begin{itemdecl}
reference_wrapper(const reference_wrapper& x) noexcept;
\end{itemdecl}

\begin{itemdescr}
\pnum\effects Constructs a \tcode{reference_wrapper} object that
stores a reference to \tcode{x.get()}.
\end{itemdescr}

\rSec3[refwrap.assign]{\tcode{reference_wrapper} assignment}

\indexlibrarymember{operator=}{reference_wrapper}%
\begin{itemdecl}
reference_wrapper& operator=(const reference_wrapper& x) noexcept;
\end{itemdecl}

\begin{itemdescr}
\pnum\postconditions \tcode{*this} stores a reference to  \tcode{x.get()}.
\end{itemdescr}

\rSec3[refwrap.access]{\tcode{reference_wrapper} access}

\indexlibrarymember{operator T\&}{reference_wrapper}%
\begin{itemdecl}
operator T& () const noexcept;
\end{itemdecl}

\begin{itemdescr}
\pnum\returns The stored reference.
\end{itemdescr}

\indexlibrarymember{get}{reference_wrapper}%
\begin{itemdecl}
T& get() const noexcept;
\end{itemdecl}

\begin{itemdescr}
\pnum\returns The stored reference.
\end{itemdescr}


\rSec3[refwrap.invoke]{reference_wrapper invocation}

\indexlibrarymember{operator()}{reference_wrapper}%
\begin{itemdecl}
template <class... ArgTypes>
  result_of_t<T&(ArgTypes&&... )>
    operator()(ArgTypes&&... args) const;
\end{itemdecl}

\begin{itemdescr}
\pnum\returns \tcode{\textit{INVOKE}(get(), std::forward<ArgTypes>(args)...)}.~(\ref{func.require})
\end{itemdescr}


\rSec3[refwrap.helpers]{reference_wrapper helper functions}

\indexlibrarymember{ref}{reference_wrapper}%
\begin{itemdecl}
template <class T> reference_wrapper<T> ref(T& t) noexcept;
\end{itemdecl}

\begin{itemdescr}
\pnum\returns \tcode{reference_wrapper<T>(t)}.
\end{itemdescr}

\indexlibrarymember{ref}{reference_wrapper}%
\begin{itemdecl}
template <class T> reference_wrapper<T> ref(reference_wrapper<T> t) noexcept;
\end{itemdecl}

\begin{itemdescr}
\pnum\returns \tcode{ref(t.get())}.
\end{itemdescr}

\indexlibrarymember{cref}{reference_wrapper}%
\begin{itemdecl}
template <class T> reference_wrapper<const T> cref(const T& t) noexcept;
\end{itemdecl}

\begin{itemdescr}
\pnum\returns \tcode{reference_wrapper <const T>(t)}.
\end{itemdescr}

\indexlibrarymember{cref}{reference_wrapper}%
\begin{itemdecl}
template <class T> reference_wrapper<const T> cref(reference_wrapper<T> t) noexcept;
\end{itemdecl}

\begin{itemdescr}
\pnum\returns \tcode{cref(t.get())}.
\end{itemdescr}

\rSec2[arithmetic.operations]{Arithmetic operations}

\pnum
The library provides basic function object classes for all of the arithmetic
operators in the language~(\ref{expr.mul}, \ref{expr.add}).

\rSec3[arithmetic.operations.plus]{class template \tcode{plus}}

\indexlibrary{\idxcode{plus}}%
\begin{itemdecl}
template <class T = void> struct plus {
  constexpr T operator()(const T& x, const T& y) const;
};
\end{itemdecl}

\indexlibrarymember{operator()}{plus}%
\begin{itemdecl}
constexpr T operator()(const T& x, const T& y) const;
\end{itemdecl}

\begin{itemdescr}
\pnum\returns \tcode{x + y}.
\end{itemdescr}

\indexlibrary{\idxcode{plus<>}}%
\begin{itemdecl}
template <> struct plus<void> {
  template <class T, class U> constexpr auto operator()(T&& t, U&& u) const
    -> decltype(std::forward<T>(t) + std::forward<U>(u));

  using is_transparent = @\unspec@;
};
\end{itemdecl}

\indexlibrarymember{operator()}{plus<>}%
\begin{itemdecl}
template <class T, class U> constexpr auto operator()(T&& t, U&& u) const
    -> decltype(std::forward<T>(t) + std::forward<U>(u));
\end{itemdecl}

\begin{itemdescr}
\pnum\returns \tcode{std::forward<T>(t) + std::forward<U>(u)}.
\end{itemdescr}

\rSec3[arithmetic.operations.minus]{class template \tcode{minus}}

\indexlibrary{\idxcode{minus}}%
\begin{itemdecl}
template <class T = void> struct minus {
  constexpr T operator()(const T& x, const T& y) const;
};
\end{itemdecl}

\indexlibrarymember{operator()}{minus}%
\begin{itemdecl}
constexpr T operator()(const T& x, const T& y) const;
\end{itemdecl}

\begin{itemdescr}
\pnum\returns \tcode{x - y}.
\end{itemdescr}

\indexlibrary{\idxcode{minus<>}}%
\begin{itemdecl}
template <> struct minus<void> {
  template <class T, class U> constexpr auto operator()(T&& t, U&& u) const
    -> decltype(std::forward<T>(t) - std::forward<U>(u));

  using is_transparent = @\unspec@;
};
\end{itemdecl}

\indexlibrarymember{operator()}{minus<>}%
\begin{itemdecl}
template <class T, class U> constexpr auto operator()(T&& t, U&& u) const
    -> decltype(std::forward<T>(t) - std::forward<U>(u));
\end{itemdecl}

\begin{itemdescr}
\pnum\returns \tcode{std::forward<T>(t) - std::forward<U>(u)}.
\end{itemdescr}

\rSec3[arithmetic.operations.multiplies]{class template \tcode{multiplies}}

\indexlibrary{\idxcode{multiplies}}%
\begin{itemdecl}
template <class T = void> struct multiplies {
  constexpr T operator()(const T& x, const T& y) const;
};
\end{itemdecl}

\indexlibrarymember{operator()}{multiplies}%
\begin{itemdecl}
constexpr T operator()(const T& x, const T& y) const;
\end{itemdecl}

\begin{itemdescr}
\pnum\returns \tcode{x * y}.
\end{itemdescr}

\indexlibrary{\idxcode{multiplies<>}}%
\begin{itemdecl}
template <> struct multiplies<void> {
  template <class T, class U> constexpr auto operator()(T&& t, U&& u) const
    -> decltype(std::forward<T>(t) * std::forward<U>(u));

  using is_transparent = @\unspec@;
};
\end{itemdecl}

\indexlibrarymember{operator()}{multiplies<>}%
\begin{itemdecl}
template <class T, class U> constexpr auto operator()(T&& t, U&& u) const
    -> decltype(std::forward<T>(t) * std::forward<U>(u));
\end{itemdecl}

\begin{itemdescr}
\pnum\returns \tcode{std::forward<T>(t) * std::forward<U>(u)}.
\end{itemdescr}

\rSec3[arithmetic.operations.divides]{class template \tcode{divides}}

\indexlibrary{\idxcode{divides}}%
\begin{itemdecl}
template <class T = void> struct divides {
  constexpr T operator()(const T& x, const T& y) const;
};
\end{itemdecl}

\indexlibrarymember{operator()}{divides}%
\begin{itemdecl}
constexpr T operator()(const T& x, const T& y) const;
\end{itemdecl}

\begin{itemdescr}
\pnum\returns \tcode{x / y}.
\end{itemdescr}

\indexlibrary{\idxcode{divides<>}}%
\begin{itemdecl}
template <> struct divides<void> {
  template <class T, class U> constexpr auto operator()(T&& t, U&& u) const
    -> decltype(std::forward<T>(t) / std::forward<U>(u));

  using is_transparent = @\unspec@;
};
\end{itemdecl}

\indexlibrarymember{operator()}{divides<>}%
\begin{itemdecl}
template <class T, class U> constexpr auto operator()(T&& t, U&& u) const
    -> decltype(std::forward<T>(t) / std::forward<U>(u));
\end{itemdecl}

\begin{itemdescr}
\pnum\returns \tcode{std::forward<T>(t) / std::forward<U>(u)}.
\end{itemdescr}

\rSec3[arithmetic.operations.modulus]{class template \tcode{modulus}}

\indexlibrary{\idxcode{modulus}}%
\begin{itemdecl}
template <class T = void> struct modulus {
  constexpr T operator()(const T& x, const T& y) const;
};
\end{itemdecl}

\indexlibrarymember{operator()}{modulus}%
\begin{itemdecl}
constexpr T operator()(const T& x, const T& y) const;
\end{itemdecl}

\begin{itemdescr}
\pnum\returns \tcode{x \% y}.
\end{itemdescr}

\indexlibrary{\idxcode{modulus<>}}%
\begin{itemdecl}
template <> struct modulus<void> {
  template <class T, class U> constexpr auto operator()(T&& t, U&& u) const
    -> decltype(std::forward<T>(t) % std::forward<U>(u));

  using is_transparent = @\unspec@;
};
\end{itemdecl}

\indexlibrarymember{operator()}{modulus<>}%
\begin{itemdecl}
template <class T, class U> constexpr auto operator()(T&& t, U&& u) const
    -> decltype(std::forward<T>(t) % std::forward<U>(u));
\end{itemdecl}

\begin{itemdescr}
\pnum\returns \tcode{std::forward<T>(t) \% std::forward<U>(u)}.
\end{itemdescr}

\rSec3[arithmetic.operations.negate]{class template \tcode{negate}}

\indexlibrary{\idxcode{negate}}%
\begin{itemdecl}
template <class T = void> struct negate {
  constexpr T operator()(const T& x) const;
};
\end{itemdecl}

\indexlibrarymember{operator()}{negate}%
\begin{itemdecl}
constexpr T operator()(const T& x) const;
\end{itemdecl}

\begin{itemdescr}
\pnum\returns \tcode{-x}.
\end{itemdescr}

\indexlibrary{\idxcode{negate<>}}%
\begin{itemdecl}
template <> struct negate<void> {
  template <class T> constexpr auto operator()(T&& t) const
    -> decltype(-std::forward<T>(t));

  using is_transparent = @\unspec@;
};
\end{itemdecl}

\indexlibrarymember{operator()}{negate<>}%
\begin{itemdecl}
template <class T> constexpr auto operator()(T&& t) const
    -> decltype(-std::forward<T>(t));
\end{itemdecl}

\begin{itemdescr}
\pnum\returns \tcode{-std::forward<T>(t)}.
\end{itemdescr}


\rSec2[comparisons]{Comparisons}

\pnum
The library provides basic function object classes for all of the comparison
operators in the language~(\ref{expr.rel}, \ref{expr.eq}).

\pnum
For templates \tcode{greater}, \tcode{less}, \tcode{greater_equal}, and
\tcode{less_equal}, the specializations for any pointer type yield a total order,
even if the built-in operators \tcode{<}, \tcode{>}, \tcode{<=}, \tcode{>=}
do not.
For template specializations \tcode{greater<void>}, \tcode{less<void>},
\tcode{greater_equal<void>}, and \tcode{less_equal<void>},
if the call operator calls a built-in operator comparing pointers,
the call operator yields a total order.

\rSec3[comparisons.equal_to]{class template \tcode{equal_to}}

\indexlibrary{\idxcode{equal_to}}%
\begin{itemdecl}
template <class T = void> struct equal_to {
  constexpr bool operator()(const T& x, const T& y) const;
};
\end{itemdecl}

\indexlibrarymember{operator()}{equal_to}%
\begin{itemdecl}
constexpr bool operator()(const T& x, const T& y) const;
\end{itemdecl}

\begin{itemdescr}
\pnum\returns \tcode{x == y}.
\end{itemdescr}

\indexlibrary{\idxcode{equal_to<>}}%
\begin{itemdecl}
template <> struct equal_to<void> {
  template <class T, class U> constexpr auto operator()(T&& t, U&& u) const
    -> decltype(std::forward<T>(t) == std::forward<U>(u));

  using is_transparent = @\unspec@;
};
\end{itemdecl}

\indexlibrarymember{operator()}{equal_to<>}%
\begin{itemdecl}
template <class T, class U> constexpr auto operator()(T&& t, U&& u) const
    -> decltype(std::forward<T>(t) == std::forward<U>(u));
\end{itemdecl}

\begin{itemdescr}
\pnum\returns \tcode{std::forward<T>(t) == std::forward<U>(u)}.
\end{itemdescr}

\rSec3[comparisons.not_equal_to]{class template \tcode{not_equal_to}}

\indexlibrary{\idxcode{not_equal_to}}%
\begin{itemdecl}
template <class T = void> struct not_equal_to {
  constexpr bool operator()(const T& x, const T& y) const;
};
\end{itemdecl}

\indexlibrarymember{operator()}{not_equal_to}%
\begin{itemdecl}
constexpr bool operator()(const T& x, const T& y) const;
\end{itemdecl}

\begin{itemdescr}
\pnum\returns \tcode{x != y}.
\end{itemdescr}

\indexlibrary{\idxcode{not_equal_to<>}}%
\begin{itemdecl}
template <> struct not_equal_to<void> {
  template <class T, class U> constexpr auto operator()(T&& t, U&& u) const
    -> decltype(std::forward<T>(t) != std::forward<U>(u));

  using is_transparent = @\unspec@;
};
\end{itemdecl}

\indexlibrarymember{operator()}{not_equal_to<>}%
\begin{itemdecl}
template <class T, class U> constexpr auto operator()(T&& t, U&& u) const
    -> decltype(std::forward<T>(t) != std::forward<U>(u));
\end{itemdecl}

\begin{itemdescr}
\pnum\returns \tcode{std::forward<T>(t) != std::forward<U>(u)}.
\end{itemdescr}

\rSec3[comparisons.greater]{class template \tcode{greater}}

\indexlibrary{\idxcode{greater}}%
\begin{itemdecl}
template <class T = void> struct greater {
  constexpr bool operator()(const T& x, const T& y) const;
};
\end{itemdecl}

\indexlibrarymember{operator()}{greater}%
\begin{itemdecl}
constexpr bool operator()(const T& x, const T& y) const;
\end{itemdecl}

\begin{itemdescr}
\pnum\returns \tcode{x > y}.
\end{itemdescr}

\indexlibrary{\idxcode{greater<>}}%
\begin{itemdecl}
template <> struct greater<void> {
  template <class T, class U> constexpr auto operator()(T&& t, U&& u) const
    -> decltype(std::forward<T>(t) > std::forward<U>(u));

  using is_transparent = @\unspec@;
};
\end{itemdecl}

\indexlibrarymember{operator()}{greater<>}%
\begin{itemdecl}
template <class T, class U> constexpr auto operator()(T&& t, U&& u) const
    -> decltype(std::forward<T>(t) > std::forward<U>(u));
\end{itemdecl}

\begin{itemdescr}
\pnum\returns \tcode{std::forward<T>(t) > std::forward<U>(u)}.
\end{itemdescr}

\rSec3[comparisons.less]{class template \tcode{less}}

\indexlibrary{\idxcode{less}}%
\begin{itemdecl}
template <class T = void> struct less {
  constexpr bool operator()(const T& x, const T& y) const;
};
\end{itemdecl}

\indexlibrarymember{operator()}{less}%
\begin{itemdecl}
constexpr bool operator()(const T& x, const T& y) const;
\end{itemdecl}

\begin{itemdescr}
\pnum\returns \tcode{x < y}.
\end{itemdescr}

\indexlibrary{\idxcode{less<>}}%
\begin{itemdecl}
template <> struct less<void> {
  template <class T, class U> constexpr auto operator()(T&& t, U&& u) const
    -> decltype(std::forward<T>(t) < std::forward<U>(u));

  using is_transparent = @\unspec@;
};
\end{itemdecl}

\indexlibrarymember{operator()}{less<>}%
\begin{itemdecl}
template <class T, class U> constexpr auto operator()(T&& t, U&& u) const
    -> decltype(std::forward<T>(t) < std::forward<U>(u));
\end{itemdecl}

\begin{itemdescr}
\pnum\returns \tcode{std::forward<T>(t) < std::forward<U>(u)}.
\end{itemdescr}

\rSec3[comparisons.greater_equal]{class template \tcode{greater_equal}}

\indexlibrary{\idxcode{greater_equal}}%
\begin{itemdecl}
template <class T = void> struct greater_equal {
  constexpr bool operator()(const T& x, const T& y) const;
};
\end{itemdecl}

\indexlibrarymember{operator()}{greater_equal}%
\begin{itemdecl}
constexpr bool operator()(const T& x, const T& y) const;
\end{itemdecl}

\begin{itemdescr}
\pnum\returns \tcode{x >= y}.
\end{itemdescr}

\indexlibrary{\idxcode{greater_equal<>}}%
\begin{itemdecl}
template <> struct greater_equal<void> {
  template <class T, class U> constexpr auto operator()(T&& t, U&& u) const
    -> decltype(std::forward<T>(t) >= std::forward<U>(u));

  using is_transparent = @\unspec@;
};
\end{itemdecl}

\indexlibrarymember{operator()}{greater_equal<>}%
\begin{itemdecl}
template <class T, class U> constexpr auto operator()(T&& t, U&& u) const
    -> decltype(std::forward<T>(t) >= std::forward<U>(u));
\end{itemdecl}

\begin{itemdescr}
\pnum\returns \tcode{std::forward<T>(t) >= std::forward<U>(u)}.
\end{itemdescr}

\rSec3[comparisons.less_equal]{class template \tcode{less_equal}}

\indexlibrary{\idxcode{less_equal}}%
\begin{itemdecl}
template <class T = void> struct less_equal {
  constexpr bool operator()(const T& x, const T& y) const;
};
\end{itemdecl}

\indexlibrarymember{operator()}{less_equal}%
\begin{itemdecl}
constexpr bool operator()(const T& x, const T& y) const;
\end{itemdecl}

\begin{itemdescr}
\pnum\returns \tcode{x <= y}.
\end{itemdescr}

\indexlibrary{\idxcode{less_equal<>}}%
\begin{itemdecl}
template <> struct less_equal<void> {
  template <class T, class U> constexpr auto operator()(T&& t, U&& u) const
    -> decltype(std::forward<T>(t) <= std::forward<U>(u));

  using is_transparent = @\unspec@;
};
\end{itemdecl}

\indexlibrarymember{operator()}{less_equal<>}%
\begin{itemdecl}
template <class T, class U> constexpr auto operator()(T&& t, U&& u) const
    -> decltype(std::forward<T>(t) <= std::forward<U>(u));
\end{itemdecl}

\begin{itemdescr}
\pnum\returns \tcode{std::forward<T>(t) <= std::forward<U>(u)}.
\end{itemdescr}


\rSec2[logical.operations]{Logical operations}

\pnum
The library provides basic function object classes for all of the logical
operators in the language~(\ref{expr.log.and}, \ref{expr.log.or}, \ref{expr.unary.op}).

\rSec3[logical.operations.and]{class template \tcode{logical_and}}

\indexlibrary{\idxcode{logical_and}}%
\begin{itemdecl}
template <class T = void> struct logical_and {
  constexpr bool operator()(const T& x, const T& y) const;
};
\end{itemdecl}

\indexlibrarymember{operator()}{logical_and}%
\begin{itemdecl}
constexpr bool operator()(const T& x, const T& y) const;
\end{itemdecl}

\begin{itemdescr}
\pnum\returns \tcode{x \&\& y}.
\end{itemdescr}

\indexlibrary{\idxcode{logical_and<>}}%
\begin{itemdecl}
template <> struct logical_and<void> {
  template <class T, class U> constexpr auto operator()(T&& t, U&& u) const
    -> decltype(std::forward<T>(t) && std::forward<U>(u));

  using is_transparent = @\unspec@;
};
\end{itemdecl}

\indexlibrarymember{operator()}{logical_and<>}%
\begin{itemdecl}
template <class T, class U> constexpr auto operator()(T&& t, U&& u) const
    -> decltype(std::forward<T>(t) && std::forward<U>(u));
\end{itemdecl}

\begin{itemdescr}
\pnum\returns \tcode{std::forward<T>(t) \&\& std::forward<U>(u)}.
\end{itemdescr}

\rSec3[logical.operations.or]{class template \tcode{logical_or}}

\indexlibrary{\idxcode{logical_or}}%
\begin{itemdecl}
template <class T = void> struct logical_or {
  constexpr bool operator()(const T& x, const T& y) const;
};
\end{itemdecl}

\indexlibrarymember{operator()}{logical_or}%
\begin{itemdecl}
constexpr bool operator()(const T& x, const T& y) const;
\end{itemdecl}

\begin{itemdescr}
\pnum\returns \tcode{x || y}.
\end{itemdescr}

\indexlibrary{\idxcode{logical_or<>}}%
\begin{itemdecl}
template <> struct logical_or<void> {
  template <class T, class U> constexpr auto operator()(T&& t, U&& u) const
    -> decltype(std::forward<T>(t) || std::forward<U>(u));

  using is_transparent = @\unspec@;
};
\end{itemdecl}

\indexlibrarymember{operator()}{logical_or<>}%
\begin{itemdecl}
template <class T, class U> constexpr auto operator()(T&& t, U&& u) const
    -> decltype(std::forward<T>(t) || std::forward<U>(u));
\end{itemdecl}

\begin{itemdescr}
\pnum\returns \tcode{std::forward<T>(t) || std::forward<U>(u)}.
\end{itemdescr}

\rSec3[logical.operations.not]{class template \tcode{logical_not}}

\indexlibrary{\idxcode{logical_not}}%
\begin{itemdecl}
template <class T = void> struct logical_not {
  constexpr bool operator()(const T& x) const;
};
\end{itemdecl}

\indexlibrarymember{operator()}{logical_not}%
\begin{itemdecl}
constexpr bool operator()(const T& x) const;
\end{itemdecl}

\begin{itemdescr}
\pnum\returns \tcode{!x}.
\end{itemdescr}

\indexlibrary{\idxcode{logical_not<>}}%
\begin{itemdecl}
template <> struct logical_not<void> {
  template <class T> constexpr auto operator()(T&& t) const
    -> decltype(!std::forward<T>(t));

  using is_transparent = @\unspec@;
};
\end{itemdecl}

\indexlibrarymember{operator()}{logical_not<>}%
\begin{itemdecl}
template <class T> constexpr auto operator()(T&& t) const
    -> decltype(!std::forward<T>(t));
\end{itemdecl}

\begin{itemdescr}
\pnum\returns \tcode{!std::forward<T>(t)}.
\end{itemdescr}


\rSec2[bitwise.operations]{Bitwise operations}

\pnum
The library provides basic function object classes for all of the bitwise
operators in the language~(\ref{expr.bit.and}, \ref{expr.or},
\ref{expr.xor}, \ref{expr.unary.op}).

\rSec3[bitwise.operations.and]{class template \tcode{bit_and}}

\indexlibrary{\idxcode{bit_and}}%
\begin{itemdecl}
template <class T = void> struct bit_and {
  constexpr T operator()(const T& x, const T& y) const;
};
\end{itemdecl}

\indexlibrarymember{operator()}{bit_and}%
\begin{itemdecl}
constexpr T operator()(const T& x, const T& y) const;
\end{itemdecl}

\begin{itemdescr}
\pnum\returns \tcode{x \& y}.
\end{itemdescr}

\indexlibrary{\idxcode{bit_and<>}}%
\begin{itemdecl}
template <> struct bit_and<void> {
  template <class T, class U> constexpr auto operator()(T&& t, U&& u) const
    -> decltype(std::forward<T>(t) & std::forward<U>(u));

  using is_transparent = @\unspec@;
};
\end{itemdecl}

\indexlibrarymember{operator()}{bit_and<>}%
\begin{itemdecl}
template <class T, class U> constexpr auto operator()(T&& t, U&& u) const
    -> decltype(std::forward<T>(t) & std::forward<U>(u));
\end{itemdecl}

\begin{itemdescr}
\pnum\returns \tcode{std::forward<T>(t) \& std::forward<U>(u)}.
\end{itemdescr}

\rSec3[bitwise.operations.or]{class template \tcode{bit_or}}

\indexlibrary{\idxcode{bit_or}}%
\begin{itemdecl}
template <class T = void> struct bit_or {
  constexpr T operator()(const T& x, const T& y) const;
};
\end{itemdecl}

\indexlibrarymember{operator()}{bit_or}%
\begin{itemdecl}
constexpr T operator()(const T& x, const T& y) const;
\end{itemdecl}

\begin{itemdescr}
\pnum\returns \tcode{x | y}.
\end{itemdescr}

\indexlibrary{\idxcode{bit_or<>}}%
\begin{itemdecl}
template <> struct bit_or<void> {
  template <class T, class U> constexpr auto operator()(T&& t, U&& u) const
    -> decltype(std::forward<T>(t) | std::forward<U>(u));

  using is_transparent = @\unspec@;
};
\end{itemdecl}

\indexlibrarymember{operator()}{bit_or<>}%
\begin{itemdecl}
template <class T, class U> constexpr auto operator()(T&& t, U&& u) const
    -> decltype(std::forward<T>(t) | std::forward<U>(u));
\end{itemdecl}

\begin{itemdescr}
\pnum\returns \tcode{std::forward<T>(t) | std::forward<U>(u)}.
\end{itemdescr}

\rSec3[bitwise.operations.xor]{class template \tcode{bit_xor}}

\indexlibrary{\idxcode{bit_xor}}%
\begin{itemdecl}
template <class T = void> struct bit_xor {
  constexpr T operator()(const T& x, const T& y) const;
};
\end{itemdecl}

\indexlibrarymember{operator()}{bit_xor}%
\begin{itemdecl}
constexpr T operator()(const T& x, const T& y) const;
\end{itemdecl}

\begin{itemdescr}
\pnum\returns \tcode{x \caret{} y}.
\end{itemdescr}

\indexlibrary{\idxcode{bit_xor<>}}%
\begin{itemdecl}
template <> struct bit_xor<void> {
  template <class T, class U> constexpr auto operator()(T&& t, U&& u) const
    -> decltype(std::forward<T>(t) ^ std::forward<U>(u));

  using is_transparent = @\unspec@;
};
\end{itemdecl}

\indexlibrarymember{operator()}{bit_xor<>}%
\begin{itemdecl}
template <class T, class U> constexpr auto operator()(T&& t, U&& u) const
    -> decltype(std::forward<T>(t) ^ std::forward<U>(u));
\end{itemdecl}

\begin{itemdescr}
\pnum\returns \tcode{std::forward<T>(t) \caret{} std::forward<U>(u)}.
\end{itemdescr}

\rSec3[bitwise.operations.not]{class template \tcode{bit_not}}

\begin{itemdecl}
template <class T = void> struct bit_not {
  constexpr T operator()(const T& x) const;
};
\end{itemdecl}

\indexlibrarymember{operator()}{bit_not}%
\begin{itemdecl}
constexpr T operator()(const T& x) const;
\end{itemdecl}

\begin{itemdescr}
\pnum\returns \tcode{\~{}x}.
\end{itemdescr}

\indexlibrary{\idxcode{bit_not<>}}%
\begin{itemdecl}
template <> struct bit_not<void> {
  template <class T> constexpr auto operator()(T&& t) const
    -> decltype(~std::forward<T>(t));

  using is_transparent = @\unspec@;
};
\end{itemdecl}

\indexlibrarymember{operator()}{bit_not<>}%
\begin{itemdecl}
template <class T> constexpr auto operator()(T&&) const
    -> decltype(~std::forward<T>(t));
\end{itemdecl}

\begin{itemdescr}
\pnum\returns \tcode{\~{}std::forward<T>(t)}.
\end{itemdescr}


\rSec2[func.not_fn]{Function template \tcode{not_fn}}

\indexlibrary{\idxcode{not_fn}}%
\begin{itemdecl}
template <class F> @\unspec@ not_fn(F&& f);
\end{itemdecl}

\begin{itemdescr}
\pnum
\effects
Equivalent to \tcode{return \textit{call_wrapper}(std::forward<F>(f));}
where \textit{call_wrapper} is an exposition only class defined as follows:
\begin{codeblock}
class @\textit{call_wrapper}@
{
   using FD = decay_t<F>;
   explicit @\textit{call_wrapper}@(F&& f);

public:
   @\textit{call_wrapper}@(@\textit{call_wrapper}@&&) = default;
   @\textit{call_wrapper}@(@\textit{call_wrapper}@ const&) = default;

   template<class... Args>
     auto operator()(Args&&...) &
       -> decltype(!declval<result_of_t<FD&(Args...)>>());

   template<class... Args>
     auto operator()(Args&&...) const&
       -> decltype(!declval<result_of_t<FD const&(Args...)>>());

   template<class... Args>
     auto operator()(Args&&...) &&
       -> decltype(!declval<result_of_t<FD(Args...)>>());

   template<class... Args>
     auto operator()(Args&&...) const&&
       -> decltype(!declval<result_of_t<FD const(Args...)>>());

private:
  FD fd;
};
\end{codeblock}
\end{itemdescr}

\begin{itemdecl}
explicit @\textit{call_wrapper}@(F&& f);
\end{itemdecl}

\begin{itemdescr}
\pnum
\requires
\tcode{FD} shall satisfy the requirements of \tcode{MoveConstructible}.
\tcode{is_constructible_v<FD, F>} shall be \tcode{true}.
\tcode{fd} shall be a callable object~(\ref{func.def}).

\pnum
\effects
Initializes \tcode{fd} from \tcode{std::forward<F>(f)}.

\pnum
\throws
Any exception thrown by construction of \tcode{fd}.
\end{itemdescr}

\begin{itemdecl}
template<class... Args>
  auto operator()(Args&&... args) &
    -> decltype(!declval<result_of_t<FD&(Args...)>>());
template<class... Args>
  auto operator()(Args&&... args) const&
    -> decltype(!declval<result_of_t<FD const&(Args...)>>());
\end{itemdecl}

\begin{itemdescr}
\pnum
\effects
Equivalent to:
\tcode{return !\textit{INVOKE}(fd, std::forward<Args>(args)...);}~(\ref{func.require}).
\end{itemdescr}

\begin{itemdecl}
template<class... Args>
  auto operator()(Args&&... args) &&
    -> decltype(!declval<result_of_t<FD(Args...)>>());
template<class... Args>
  auto operator()(Args&&... args) const&&
    -> decltype(!declval<result_of_t<FD const(Args...)>>());
\end{itemdecl}

\begin{itemdescr}
\pnum
\effects
Equivalent to:
\tcode{return !\textit{INVOKE}(std::move(fd), std::forward<Args>(args)...);}~(\ref{func.require}).
\end{itemdescr}

\rSec2[func.bind]{Function object binders}%
\indextext{function object!binders|(}

\pnum
This subclause describes a uniform mechanism for binding
arguments of callable objects.

\rSec3[func.bind.isbind]{Class template \tcode{is_bind_expression}}

\indexlibrary{\idxcode{is_bind_expression}}%
\begin{codeblock}
namespace std {
  template<class T> struct is_bind_expression; // see below
}
\end{codeblock}

\pnum
\tcode{is_bind_expression} can be used to detect function objects
generated by \tcode{bind}. \tcode{bind}
uses \tcode{is_bind_expression} to detect subexpressions.

\pnum
Instantiations of the \tcode{is_bind_expression} template shall meet
the UnaryTypeTrait requirements~(\ref{meta.rqmts}). The implementation
shall provide a definition that has a BaseCharacteristic of
\tcode{true_type} if \tcode{T} is a type returned from \tcode{bind},
otherwise it shall have a BaseCharacteristic of \tcode{false_type}.
A program may specialize this template for a user-defined type \tcode{T}
to have a BaseCharacteristic of \tcode{true_type} to indicate that
\tcode{T} should be treated as a subexpression in a \tcode{bind} call.

\rSec3[func.bind.isplace]{Class template \tcode{is_placeholder}}

\indexlibrary{\idxcode{is_placeholder}}%
\begin{codeblock}
namespace std {
  template<class T> struct is_placeholder; // see below
}
\end{codeblock}

\pnum
\tcode{is_placeholder} can be used to detect the standard placeholders
\tcode{_1}, \tcode{_2}, and so on. \tcode{bind} uses
\tcode{is_placeholder} to detect placeholders.

\pnum
Instantiations of the \tcode{is_placeholder} template shall meet
the UnaryTypeTrait requirements~(\ref{meta.rqmts}). The implementation
shall provide a definition that has the BaseCharacteristic of
\tcode{integral_constant<int, J>} if \tcode{T} is the type of
\tcode{std::placeholders::_J}, otherwise it shall have a
BaseCharacteristic of \tcode{integral_constant<int, 0>}. A program
may specialize this template for a user-defined type \tcode{T} to
have a BaseCharacteristic of \tcode{integral_constant<int, \textit{N}>}
with \tcode{\textit{N} > 0} to indicate that \tcode{T} should be
treated as a placeholder type.

\rSec3[func.bind.bind]{Function template \tcode{bind}}
\indexlibrary{\idxcode{bind}|(}

\pnum
In the text that follows:

\begin{itemize}
\item \tcode{FD} is the type \tcode{decay_t<F>},
\item \tcode{fd} is an lvalue of type \tcode{FD} constructed from \tcode{std::forward<F>(f)},
\item \tcode{Ti} is the $i^{th}$ type in the template parameter pack \tcode{BoundArgs},
\item \tcode{TiD} is the type \tcode{decay_t<Ti>},
\item \tcode{ti} is the $i^{th}$ argument in the function parameter pack \tcode{bound_args},
\item \tcode{tid} is an lvalue of type \tcode{TiD} constructed from \tcode{std::forward<Ti>(ti)},
\item \tcode{Uj} is the $j^{th}$ deduced type of the \tcode{UnBoundArgs\&\&...} parameter
  of the forwarding call wrapper, and
\item \tcode{uj} is the $j^{th}$ argument associated with \tcode{Uj}.
\end{itemize}

\indexlibrary{\idxcode{bind}}%
\begin{itemdecl}
template<class F, class... BoundArgs>
  @\unspec@ bind(F&& f, BoundArgs&&... bound_args);
\end{itemdecl}

\begin{itemdescr}
\pnum
\requires
\tcode{is_constructible_v<FD, F>} shall be \tcode{true}. For each \tcode{Ti}
in \tcode{BoundArgs}, \tcode{is_cons\-tructible_v<TiD, Ti>} shall be \tcode{true}.
\tcode{\textit{INVOKE} (fd, w1, w2, ...,
wN)}~(\ref{func.require}) shall be a valid expression for some
values \textit{w1, w2, ..., wN}, where
\tcode{N == sizeof...(bound_args)}.
The cv-qualifiers \cv{} of the call wrapper \tcode{g},
as specified below, shall be neither \tcode{volatile} nor \tcode{const volatile}.

\pnum\returns
A forwarding call wrapper \tcode{g}~(\ref{func.require}).
The effect of \tcode{g(u1, u2, ..., uM)} shall
be \tcode{\textit{INVOKE}(fd, std::forward<V1>(v1), std::forward<V2>(v2), ..., std::forward<VN>(vN))},
where the values and types of the bound
arguments \tcode{v1, v2, ..., vN} are determined as specified below.
The copy constructor and move constructor of the forwarding call wrapper shall throw an
exception if and only if the corresponding constructor of \tcode{FD} or of any of the types
\tcode{TiD} throws an exception.

\pnum
\throws Nothing unless the construction of
\tcode{fd} or of one of the values \tcode{tid} throws an exception.

\pnum
\remarks The return type shall satisfy the requirements of \tcode{MoveConstructible}. If all
of \tcode{FD} and \tcode{TiD} satisfy the requirements of \tcode{CopyConstructible}, then the
return type shall satisfy the requirements of \tcode{CopyConstructible}. \begin{note} This implies
that all of \tcode{FD} and \tcode{TiD} are \tcode{MoveConstructible}. \end{note}
\end{itemdescr}

\indexlibrary{\idxcode{bind}}%
\begin{itemdecl}
template<class R, class F, class... BoundArgs>
  @\unspec@ bind(F&& f, BoundArgs&&... bound_args);
\end{itemdecl}

\begin{itemdescr}
\pnum
\requires
\tcode{is_constructible_v<FD, F>} shall be \tcode{true}. For each \tcode{Ti}
in \tcode{BoundArgs}, \tcode{is_con\-structible_v<TiD, Ti>} shall be \tcode{true}.
\tcode{\textit{INVOKE}(fd, w1, w2, ..., wN)} shall be  a valid
expression for some
values \textit{w1, w2, ..., wN}, where
\tcode{N == sizeof...(bound_args)}.
The cv-qualifiers \cv{} of the call wrapper \tcode{g},
as specified below, shall be neither \tcode{volatile} nor \tcode{const volatile}.

\pnum
\returns
A forwarding call wrapper \tcode{g}~(\ref{func.require}).
The effect of
\tcode{g(u1, u2, ..., uM)} shall be \tcode{\textit{INVOKE}(fd,
std::forward<V1>(v1), std::forward<V2>(v2), ...,
std::forward<VN>(vN), R)}, where the values and types of the bound
arguments \tcode{v1, v2, ..., vN} are determined as specified below.
The copy constructor and move constructor of the forwarding call wrapper shall throw an
exception if and only if the corresponding constructor of \tcode{FD} or of any of the types
\tcode{TiD} throws an exception.

\pnum
\throws Nothing unless the construction of
\tcode{fd} or of one of the values \tcode{tid} throws an exception.

\pnum
\remarks The return type shall satisfy the requirements of \tcode{MoveConstructible}. If all
of \tcode{FD} and \tcode{TiD} satisfy the requirements of \tcode{CopyConstructible}, then the
return type shall satisfy the requirements of \tcode{CopyConstructible}. \begin{note} This implies
that all of \tcode{FD} and \tcode{TiD} are \tcode{MoveConstructible}. \end{note}
\end{itemdescr}

\pnum
\indextext{bound arguments}%
The values of the \techterm{bound arguments} \tcode{v1, v2, ..., vN} and their
corresponding types \tcode{V1, V2, ..., VN} depend on the
types \tcode{TiD} derived from
the call to \tcode{bind} and the
cv-qualifiers \cv{} of the call wrapper \tcode{g} as follows:

\begin{itemize}
\item if \tcode{TiD} is \tcode{reference_wrapper<T>}, the
argument is \tcode{tid.get()} and its type \tcode{Vi} is \tcode{T\&};

\item if the value of \tcode{is_bind_expression_v<TiD>}
is \tcode{true}, the argument is \tcode{tid(std::forward<Uj>(\brk{}uj)...)}  and its
type \tcode{Vi} is
\tcode{result_of_t<TiD \cv{} \& (Uj\&\&...)>\&\&};

\item if the value \tcode{j} of \tcode{is_placeholder_v<TiD>}
is not zero, the  argument is \tcode{std::forward<Uj>(uj)}
and its type \tcode{Vi}
is \tcode{Uj\&\&};

\item otherwise, the value is \tcode{tid} and its type \tcode{Vi}
is \tcode{TiD \cv{} \&}.
\end{itemize}

\rSec3[func.bind.place]{Placeholders}

\indexlibrary{\idxcode{placeholders}}%
\indexlibrary{1@\tcode{_1}}%
\begin{codeblock}
namespace std::placeholders {
  // M is the \impldef{number of placeholders for bind expressions} number of placeholders
  @\seebelow@ _1;
  @\seebelow@ _2;
              .
              .
              .
  @\seebelow@ _M;
}
\end{codeblock}

\pnum
All placeholder types shall be \tcode{DefaultConstructible} and
\tcode{CopyConstructible}, and their default constructors and copy/move
constructors shall not throw exceptions. It is \impldef{assignability of placeholder
objects} whether
placeholder types are \tcode{CopyAssignable}. \tcode{CopyAssignable} placeholders' copy
assignment operators shall not throw exceptions.%
\indexlibrary{\idxcode{bind}|)}%
\indextext{function object!binders|)}

\pnum
Placeholders should be defined as:
\begin{codeblock}
constexpr @\unspec@ _1{};
\end{codeblock}
If they are not, they shall be declared as:
\begin{codeblock}
extern @\unspec@ _1;
\end{codeblock}

\rSec2[func.memfn]{Function template \tcode{mem_fn}}%
\indextext{function object!\idxcode{mem_fn}|(}

\indexlibrary{\idxcode{mem_fn}}%
\begin{itemdecl}
template<class R, class T> @\unspec@ mem_fn(R T::* pm) noexcept;
\end{itemdecl}

\begin{itemdescr}
\pnum
\returns A simple call wrapper~(\ref{func.def}) \tcode{fn}
such that the expression \tcode{fn(t, a2, ..., aN)} is equivalent
to \tcode{\textit{INVOKE}(pm, t, a2, ..., aN)}~(\ref{func.require}).
\end{itemdescr}
\indextext{function object!\idxcode{mem_fn}|)}

\rSec2[func.wrap]{Polymorphic function wrappers}%
\indextext{function object!wrapper|(}

\pnum
This subclause describes a polymorphic wrapper class that
encapsulates arbitrary callable objects.

\rSec3[func.wrap.badcall]{Class \tcode{bad_function_call}}%
\indexlibrary{\idxcode{bad_function_call}}%

\pnum
An exception of type \tcode{bad_function_call} is thrown by
\tcode{function::operator()}~(\ref{func.wrap.func.inv})
when the function wrapper object has no target.

\begin{codeblock}
namespace std {
  class bad_function_call : public exception {
  public:
    // \ref{func.wrap.badcall.const}, constructor:
    bad_function_call() noexcept;
  };
}
\end{codeblock}

\rSec4[func.wrap.badcall.const]{\tcode{bad_function_call} constructor}

\indexlibrary{\idxcode{bad_function_call}!constructor}%
\indexlibrarymember{what}{bad_function_call}%
\begin{itemdecl}
bad_function_call() noexcept;
\end{itemdecl}

\begin{itemdescr}
\pnum\effects Constructs a \tcode{bad_function_call} object.
\end{itemdescr}

\begin{itemdescr}
\pnum\postconditions  \tcode{what()} returns an
\impldef{return value of \tcode{bad_function_call::what}} \ntbs.
\end{itemdescr}

\rSec3[func.wrap.func]{Class template \tcode{function}}
\indexlibrary{\idxcode{function}}%

\begin{codeblock}
namespace std {
  template<class> class function; // not defined

  template<class R, class... ArgTypes>
  class function<R(ArgTypes...)> {
  public:
    using result_type = R;

    // \ref{func.wrap.func.con}, construct/copy/destroy:
    function() noexcept;
    function(nullptr_t) noexcept;
    function(const function&);
    function(function&&);
    template<class F> function(F);

    function& operator=(const function&);
    function& operator=(function&&);
    function& operator=(nullptr_t) noexcept;
    template<class F> function& operator=(F&&);
    template<class F> function& operator=(reference_wrapper<F>) noexcept;

    ~function();

    // \ref{func.wrap.func.mod}, function modifiers:
    void swap(function&) noexcept;

    // \ref{func.wrap.func.cap}, function capacity:
    explicit operator bool() const noexcept;

    // \ref{func.wrap.func.inv}, function invocation:
    R operator()(ArgTypes...) const;

    // \ref{func.wrap.func.targ}, function target access:
    const type_info& target_type() const noexcept;
    template<class T>       T* target() noexcept;
    template<class T> const T* target() const noexcept;

  };

  // \ref{func.wrap.func.nullptr}, Null pointer comparisons:
  template <class R, class... ArgTypes>
    bool operator==(const function<R(ArgTypes...)>&, nullptr_t) noexcept;

  template <class R, class... ArgTypes>
    bool operator==(nullptr_t, const function<R(ArgTypes...)>&) noexcept;

  template <class R, class... ArgTypes>
    bool operator!=(const function<R(ArgTypes...)>&, nullptr_t) noexcept;

  template <class R, class... ArgTypes>
    bool operator!=(nullptr_t, const function<R(ArgTypes...)>&) noexcept;

  // \ref{func.wrap.func.alg}, specialized algorithms:
  template <class R, class... ArgTypes>
    void swap(function<R(ArgTypes...)>&, function<R(ArgTypes...)>&) noexcept;
}
\end{codeblock}

\pnum
The \tcode{function} class template provides polymorphic wrappers that
generalize the notion of a function pointer. Wrappers can store, copy,
and call arbitrary callable objects~(\ref{func.def}), given a call
signature~(\ref{func.def}), allowing functions to be first-class objects.

\pnum
\indextext{callable type}%
A callable type~(\ref{func.def}) \tcode{F}
is \defn{Lvalue-Callable} for argument
types \tcode{ArgTypes}
and return type \tcode{R}
if the expression
\tcode{\textit{INVOKE}(declval<F\&>(), declval<ArgTypes>()..., R)},
considered as an unevaluated operand (Clause~\ref{expr}), is
well formed~(\ref{func.require}).

\pnum
The \tcode{function} class template is a call
wrapper~(\ref{func.def}) whose call signature~(\ref{func.def})
is \tcode{R(ArgTypes...)}.

\rSec4[func.wrap.func.con]{\tcode{function} construct/copy/destroy}

\indexlibrary{\idxcode{function}!constructor}%
\begin{itemdecl}
function() noexcept;
\end{itemdecl}

\begin{itemdescr}
\pnum\postconditions \tcode{!*this}.
\end{itemdescr}

\indexlibrary{\idxcode{function}!constructor}%
\begin{itemdecl}
function(nullptr_t) noexcept;
\end{itemdecl}

\begin{itemdescr}
\pnum
\postconditions \tcode{!*this}.
\end{itemdescr}

\indexlibrary{\idxcode{function}!constructor}%
\begin{itemdecl}
function(const function& f);
\end{itemdecl}

\begin{itemdescr}
\pnum
\postconditions \tcode{!*this} if \tcode{!f}; otherwise,
\tcode{*this} targets a copy of \tcode{f.target()}.

\pnum
\throws shall not throw exceptions if \tcode{f}'s target is
a callable object passed via \tcode{reference_wrapper} or
a function pointer. Otherwise, may throw \tcode{bad_alloc}
or any exception thrown by the copy constructor of the stored callable object.
\begin{note} Implementations are encouraged to avoid the use of
dynamically allocated memory for small callable objects, for example, where
\tcode{f}'s target is an object holding only a pointer or reference
to an object and a member function pointer. \end{note}
\end{itemdescr}

\indexlibrary{\idxcode{function}!constructor}%
\begin{itemdecl}
function(function&& f);
\end{itemdecl}

\begin{itemdescr}
\pnum
\effects If \tcode{!f}, \tcode{*this} has
no target; otherwise, move constructs the target of \tcode{f}
into the target of \tcode{*this}, leaving \tcode{f} in
a valid state with an unspecified value.

\pnum
\throws shall not throw exceptions if \tcode{f}'s target is
a callable object passed via \tcode{reference_wrapper} or
a function pointer. Otherwise, may throw \tcode{bad_alloc} or
any exception thrown by the copy or move constructor
of the stored callable object.
\begin{note} Implementations are encouraged to avoid the use of
dynamically allocated memory for small callable objects, for example,
where \tcode{f}'s target is an object holding only a pointer or reference
to an object and a member function pointer. \end{note}
\end{itemdescr}

\indexlibrary{\idxcode{function}!constructor}%
\begin{itemdecl}
template<class F> function(F f);
\end{itemdecl}

\begin{itemdescr}
\pnum
\requires \tcode{F} shall be \tcode{CopyConstructible}.

\pnum
\remarks This constructor template shall not participate in overload resolution unless
\tcode{F} is Lvalue-Callable~(\ref{func.wrap.func}) for argument types
\tcode{ArgTypes...} and return type \tcode{R}.

\pnum
\postconditions \tcode{!*this} if any of the following hold:
\begin{itemize}
\item \tcode{f} is a null function pointer value.
\item \tcode{f} is a null member pointer value.
\item \tcode{F} is an instance of the \tcode{function} class template, and
  \tcode{!f}.
\end{itemize}

\pnum
Otherwise, \tcode{*this} targets a copy of \tcode{f}
initialized with \tcode{std::move(f)}.
\begin{note} Implementations are encouraged to avoid the use of
dynamically allocated memory for small callable objects, for example,
where \tcode{f} is an object holding only a pointer or
reference to an object and a member function pointer. \end{note}

\pnum
\throws shall not throw exceptions when \tcode{f} is a function pointer
or a \tcode{reference_wrapper<T>} for some \tcode{T}. Otherwise,
may throw \tcode{bad_alloc} or any exception thrown by \tcode{F}'s copy
or move constructor.
\end{itemdescr}

\indexlibrarymember{operator=}{function}%
\begin{itemdecl}
function& operator=(const function& f);
\end{itemdecl}

\begin{itemdescr}
\pnum
\effects As if by \tcode{function(f).swap(*this);}

\pnum
\returns \tcode{*this}.
\end{itemdescr}

\indexlibrarymember{operator=}{function}%
\begin{itemdecl}
function& operator=(function&& f);
\end{itemdecl}

\begin{itemdescr}
\pnum
\effects Replaces the target of \tcode{*this}
with the target of \tcode{f}.

\pnum
\returns \tcode{*this}.
\end{itemdescr}

\indexlibrarymember{operator=}{function}%
\begin{itemdecl}
function& operator=(nullptr_t) noexcept;
\end{itemdecl}

\begin{itemdescr}
\pnum\effects If \tcode{*this != nullptr}, destroys the target of \tcode{this}.

\pnum\postconditions \tcode{!(*this)}.

\pnum\returns \tcode{*this}.
\end{itemdescr}

\indexlibrarymember{operator=}{function}%
\begin{itemdecl}
template<class F> function& operator=(F&& f);
\end{itemdecl}

\begin{itemdescr}
\pnum\effects As if by: \tcode{function(std::forward<F>(f)).swap(*this);}

\pnum\returns \tcode{*this}.

\pnum\remarks This assignment operator shall not participate in overload
resolution unless \tcode{decay_t<F>} is
Lvalue-Callable~(\ref{func.wrap.func}) for argument types \tcode{ArgTypes...} and
return type \tcode{R}.
\end{itemdescr}

\indexlibrarymember{operator=}{function}%
\begin{itemdecl}
template<class F> function& operator=(reference_wrapper<F> f) noexcept;
\end{itemdecl}

\begin{itemdescr}
\pnum\effects As if by: \tcode{function(f).swap(*this);}

\pnum
\returns \tcode{*this}.
\end{itemdescr}

\indexlibrary{\idxcode{function}!destructor}%
\begin{itemdecl}
~function();
\end{itemdecl}

\begin{itemdescr}
\pnum\effects If \tcode{*this != nullptr}, destroys the target of \tcode{this}.
\end{itemdescr}

\rSec4[func.wrap.func.mod]{\tcode{function} modifiers}

\indexlibrarymember{swap}{function}%
\begin{itemdecl}
void swap(function& other) noexcept;
\end{itemdecl}

\begin{itemdescr}
\pnum\effects interchanges the targets of \tcode{*this} and \tcode{other}.
\end{itemdescr}

\rSec4[func.wrap.func.cap]{\tcode{function} capacity}

\indexlibrarymember{operator bool}{function}%
\begin{itemdecl}
explicit operator bool() const noexcept;
\end{itemdecl}

\begin{itemdescr}
\pnum
\returns \tcode{true} if \tcode{*this} has a target, otherwise \tcode{false}.
\end{itemdescr}

\rSec4[func.wrap.func.inv]{\tcode{function} invocation}

\indexlibrary{\idxcode{function}!invocation}%
\indexlibrarymember{operator()}{function}%
\begin{itemdecl}
R operator()(ArgTypes... args) const;
\end{itemdecl}

\begin{itemdescr}
\pnum
\returns \tcode{\textit{INVOKE}(f, std::forward<ArgTypes>(args)..., R)}~(\ref{func.require}),
where \tcode{f} is the target object~(\ref{func.def}) of \tcode{*this}.

\pnum\throws
\tcode{bad_function_call} if \tcode{!*this}; otherwise, any
exception thrown by the wrapped callable object.
\end{itemdescr}

\rSec4[func.wrap.func.targ]{function target access}

\indexlibrarymember{target_type}{function}%
\begin{itemdecl}
const type_info& target_type() const noexcept;
\end{itemdecl}

\begin{itemdescr}
\pnum\returns If \tcode{*this} has a target of type \tcode{T},
  \tcode{typeid(T)}; otherwise, \tcode{typeid(void)}.
\end{itemdescr}

\indexlibrarymember{target}{function}%
\begin{itemdecl}
template<class T>       T* target() noexcept;
template<class T> const T* target() const noexcept;
\end{itemdecl}

\begin{itemdescr}
\pnum
\requires \tcode{T} shall be a type that is
Lvalue-Callable~(\ref{func.wrap.func}) for parameter types
\tcode{ArgTypes}
and return type \tcode{R}.

\pnum\returns If \tcode{target_type() == typeid(T)}
a pointer to the stored function target; otherwise a null pointer.
\end{itemdescr}

\rSec4[func.wrap.func.nullptr]{null pointer comparison operators}

\indexlibrarymember{operator==}{function}%
\begin{itemdecl}
template <class R, class... ArgTypes>
  bool operator==(const function<R(ArgTypes...)>& f, nullptr_t) noexcept;
template <class R, class... ArgTypes>
  bool operator==(nullptr_t, const function<R(ArgTypes...)>& f) noexcept;
\end{itemdecl}

\begin{itemdescr}
\pnum\returns \tcode{!f}.
\end{itemdescr}

\indexlibrarymember{operator"!=}{function}%
\begin{itemdecl}
template <class R, class... ArgTypes>
  bool operator!=(const function<R(ArgTypes...)>& f, nullptr_t) noexcept;
template <class R, class... ArgTypes>
  bool operator!=(nullptr_t, const function<R(ArgTypes...)>& f) noexcept;
\end{itemdecl}

\begin{itemdescr}
\pnum\returns \tcode{(bool)f}.
\end{itemdescr}

\rSec4[func.wrap.func.alg]{specialized algorithms}

\indexlibrarymember{swap}{function}%
\begin{itemdecl}
template<class R, class... ArgTypes>
  void swap(function<R(ArgTypes...)>& f1, function<R(ArgTypes...)>& f2) noexcept;
\end{itemdecl}

\begin{itemdescr}
\pnum\effects As if by: \tcode{f1.swap(f2);}
\end{itemdescr}%
\indextext{function object!wrapper|)}

\rSec2[func.search]{Searchers}

\pnum
This subclause provides function object types (\ref{function.objects}) for
operations that search for a sequence \range{pat_first}{pat_last} in another
sequence \range{first}{last} that is provided to the object's function call
operator.  The first sequence (the pattern to be searched for) is provided to
the object's constructor, and the second (the sequence to be searched) is
provided to the function call operator.

\pnum
Each specialization of a class template specified in this subclause \ref{func.search} shall meet the \tcode{CopyConstructible} and \tcode{CopyAssignable} requirements.
Template parameters named
\begin{itemize}
\item \tcode{ForwardIterator},
\item \tcode{ForwardIterator1},
\item \tcode{ForwardIterator2},
\item \tcode{RandomAccessIterator},
\item \tcode{RandomAccessIterator1},
\item \tcode{RandomAccessIterator2}, and
\item \tcode{BinaryPredicate}
\end{itemize}
of templates specified in this subclause
\ref{func.search} shall meet the same requirements and semantics as
specified in \ref{algorithms.general}.
Template parameters named \tcode{Hash} shall meet the requirements as specified in \ref{hash.requirements}.

\pnum
The Boyer-Moore searcher implements the Boyer-Moore search algorithm.
The Boyer-Moore-Horspool searcher implements the Boyer-Moore-Horspool search algorithm.
In general, the Boyer-Moore searcher will use more memory and give better runtime performance than Boyer-Moore-Horspool

\rSec3[func.search.default]{Class template \tcode{default_searcher}}

\indexlibrary{\idxcode{default_searcher}}%
\begin{codeblock}
template <class ForwardIterator1, class BinaryPredicate = equal_to<>>
class default_searcher {
public:
  default_searcher(ForwardIterator1 pat_first, ForwardIterator1 pat_last,
                   BinaryPredicate pred = BinaryPredicate());

  template <class ForwardIterator2>
    pair<ForwardIterator2, ForwardIterator2>
      operator()(ForwardIterator2 first, ForwardIterator2 last) const;

private:
  ForwardIterator1 pat_first_; // \expos
  ForwardIterator1 pat_last_;  // \expos
  BinaryPredicate pred_;       // \expos
};
\end{codeblock}

\indexlibrary{\idxcode{default_searcher}!constructor}%
\begin{itemdecl}
default_searcher(ForwardIterator pat_first, ForwardIterator pat_last,
                 BinaryPredicate pred = BinaryPredicate());
\end{itemdecl}

\begin{itemdescr}
\pnum
\effects
% FIXME: The mbox prevents TeX from adding a bizarre hyphen after pat_last_.
Constructs a \tcode{default_searcher} object, initializing \tcode{pat_first_}
with \tcode{pat_first}, \mbox{\tcode{pat_last_}} with \tcode{pat_last}, and
\tcode{pred_} with \tcode{pred}.

\pnum
\throws
Any exception thrown by the copy constructor of \tcode{BinaryPredicate} or
\tcode{ForwardIterator1}.
\end{itemdescr}

\indexlibrarymember{operator()}{default_searcher}%
\begin{itemdecl}
template<class ForwardIterator2>
  pair<ForwardIterator2, ForwardIterator2>
    operator()(ForwardIterator2 first, ForwardIterator2 last) const;
\end{itemdecl}

\begin{itemdescr}
\pnum
\effects
Returns a pair of iterators \tcode{i} and \tcode{j} such that
\begin{itemize}
\item \tcode{i == search(first, last, pat_first_, pat_last_, pred_)}, and
\item if \tcode{i == last}, then \tcode{j == last},
otherwise \tcode{j == next(i, distance(pat_first_, pat_last_))}.
\end{itemize}
\end{itemdescr}

\rSec4[func.search.default.creation]{\tcode{default_searcher} creation functions}

\indexlibrary{\idxcode{make_default_searcher}}%
\begin{itemdecl}
template <class ForwardIterator, class BinaryPredicate = equal_to<>>
  default_searcher<ForwardIterator, BinaryPredicate>
    make_default_searcher(ForwardIterator pat_first, ForwardIterator pat_last,
                          BinaryPredicate pred = BinaryPredicate());
\end{itemdecl}

\begin{itemdescr}
\pnum
\effects
Equivalent to:
\begin{codeblock}
return default_searcher<ForwardIterator, BinaryPredicate>(pat_first, pat_last, pred);
\end{codeblock}
\end{itemdescr}

\rSec3[func.search.bm]{Class template \tcode{boyer_moore_searcher}}

\indexlibrary{\idxcode{boyer_moore_searcher}}%
\begin{codeblock}
template <class RandomAccessIterator1,
          class Hash = hash<typename iterator_traits<RandomAccessIterator1>::value_type>,
          class BinaryPredicate = equal_to<>>
class boyer_moore_searcher {
public:
  boyer_moore_searcher(RandomAccessIterator1 pat_first,
                       RandomAccessIterator1 pat_last, Hash hf = Hash(),
                       BinaryPredicate pred = BinaryPredicate());

  template <class RandomAccessIterator2>
    pair<RandomAccessIterator2, RandomAccessIterator2>
      operator()(RandomAccessIterator2 first,
                 RandomAccessIterator2 last) const;

private:
  RandomAccessIterator1 pat_first_; // \expos
  RandomAccessIterator1 pat_last_;  // \expos
  Hash hash_;                       // \expos
  BinaryPredicate pred_;            // \expos
};
\end{codeblock}

\indexlibrary{\idxcode{boyer_moore_searcher}!constructor}%
\begin{itemdecl}
boyer_moore_searcher(RandomAccessIterator1 pat_first,
                     RandomAccessIterator1 pat_last, Hash hf = Hash(),
                     BinaryPredicate pred = BinaryPredicate());
\end{itemdecl}

\begin{itemdescr}
\pnum
\requires
The value type of \tcode{RandomAccessIterator1} shall meet the \tcode{DefaultConstructible} requirements, the \tcode{CopyConstructible} requirements, and the \tcode{CopyAssignable} requirements.

\pnum
\requires
For any two values \tcode{A} and \tcode{B} of the type \tcode{iterator_traits<RandomAccessIterator1>::value_type}, if \tcode{pred(A,B)==true}, then \tcode{hf(A)==hf(B)} shall be \tcode{true}.

\pnum
\effects
Constructs a \tcode{boyer_moore_searcher} object, initializing \tcode{pat_first_} with \tcode{pat_first}, \tcode{pat_last_} with \tcode{pat_last}, \tcode{hash_} with \tcode{hf}, and \tcode{pred_} with \tcode{pred}.

\pnum
\throws
Any exception thrown by the copy constructor of \tcode{RandomAccessIterator1},
or by the default constructor, copy constructor, or the copy assignment operator of the value type of \tcode{RandomAccess\-Iterator1},
or the copy constructor or \tcode{operator()} of \tcode{BinaryPredicate} or \tcode{Hash}.
May throw \tcode{bad_alloc} if additional memory needed for internal data structures cannot be allocated.
\end{itemdescr}

\indexlibrarymember{operator()}{boyer_moore_searcher}%
\begin{itemdecl}
template <class RandomAccessIterator2>
  pair<RandomAccessIterator2, RandomAccessIterator2>
    operator()(RandomAccessIterator2 first, RandomAccessIterator2 last) const;
\end{itemdecl}

\begin{itemdescr}
\pnum
\requires
\tcode{RandomAccessIterator1} and \tcode{RandomAccessIterator2} shall have the same value type.

\pnum
\effects
Finds a subsequence of equal values in a sequence.

\pnum
\returns
A pair of iterators \tcode{i} and \tcode{j} such that
\begin{itemize}
\item \tcode{i} is the first iterator
in the range \range{first}{last - (pat_last_ - pat_first_)} such that
for every non-negative integer \tcode{n} less than \tcode{pat_last_ - pat_first_}
the following condition holds:
\tcode{pred(*(i + n), *(pat_first_ + n)) != false}, and
\item \tcode{j == next(i, distance(pat_first_, pat_last_))}.
\end{itemize}
Returns \tcode{make_pair(first, first)} if \range{pat_first_}{pat_last_} is empty,
otherwise returns \tcode{make_pair(last, last)} if no such iterator is found.

\pnum
\complexity
At most \tcode{(last - first) * (pat_last_ - pat_first_)} applications of the predicate.
\end{itemdescr}

\rSec4[func.search.bm.creation]{\tcode{boyer_moore_searcher} creation functions}

\indexlibrary{\idxcode{make_boyer_moore_searcher}}%
\begin{itemdecl}
template <class RandomAccessIterator,
          class Hash = hash<typename iterator_traits<RandomAccessIterator>::value_type>,
          class BinaryPredicate = equal_to<>>
  boyer_moore_searcher<RandomAccessIterator, Hash, BinaryPredicate>
    make_boyer_moore_searcher(RandomAccessIterator pat_first,
                              RandomAccessIterator pat_last, Hash hf = Hash(),
                              BinaryPredicate pred = BinaryPredicate());
\end{itemdecl}

\begin{itemdescr}
\pnum
\effects
Equivalent to:
\begin{codeblock}
return boyer_moore_searcher<RandomAccessIterator, Hash, BinaryPredicate>(
         pat_first, pat_last, hf, pred);
\end{codeblock}
\end{itemdescr}

\rSec3[func.search.bmh]{Class template \tcode{boyer_moore_horspool_searcher}}

\indexlibrary{\idxcode{boyer_moore_horspool_searcher}}%
\begin{codeblock}
template <class RandomAccessIterator1,
          class Hash = hash<typename iterator_traits<RandomAccessIterator1>::value_type>,
          class BinaryPredicate = equal_to<>>
class boyer_moore_horspool_searcher {
public:
  boyer_moore_horspool_searcher(RandomAccessIterator1 pat_first,
                                RandomAccessIterator1 pat_last,
                                Hash hf = Hash(),
                                BinaryPredicate pred = BinaryPredicate());

  template <class RandomAccessIterator2>
    pair<RandomAccessIterator2, RandomAccessIterator2>
      operator()(RandomAccessIterator2 first,
                 RandomAccessIterator2 last) const;

private:
  RandomAccessIterator1 pat_first_; // \expos
  RandomAccessIterator1 pat_last_;  // \expos
  Hash hash_;                       // \expos
  BinaryPredicate pred_;            // \expos
};
\end{codeblock}

\indexlibrary{\idxcode{boyer_moore_horspool_searcher}!constructor}%
\begin{itemdecl}
boyer_moore_horspool_searcher(RandomAccessIterator1 pat_first,
                              RandomAccessIterator1 pat_last, Hash hf = Hash(),
                              BinaryPredicate pred = BinaryPredicate());
\end{itemdecl}

\begin{itemdescr}
\pnum
\requires
The value type of \tcode{RandomAccessIterator1} shall meet the \tcode{DefaultConstructible}, \tcode{Copy\-Constructible}, and \tcode{CopyAssignable} requirements.

\pnum
\requires
For any two values \tcode{A} and \tcode{B} of the type \tcode{iterator_traits<RandomAccessIterator1>::value_type},
if \tcode{pred(A,B)==true}, then \tcode{hf(A)==hf(B)} shall be \tcode{true}.

\pnum
\effects
Constructs a \tcode{boyer_moore_horspool_searcher} object, initializing \tcode{pat_first_} with \tcode{pat_first},
\tcode{pat_last_} with \tcode{pat_last}, \tcode{hash_} with \tcode{hf}, and \tcode{pred_} with \tcode{pred}.

\pnum
\throws
Any exception thrown by the copy constructor of \tcode{RandomAccessIterator1},
or by the default constructor, copy constructor, or the copy assignment operator of the value type of \tcode{RandomAccess\-Iterator1}
or the copy constructor or \tcode{operator()} of \tcode{BinaryPredicate} or \tcode{Hash}.
May throw \tcode{bad_alloc} if additional memory needed for internal data structures cannot be allocated.
\end{itemdescr}

\indexlibrarymember{operator()}{boyer_moore_horspool_searcher}%
\begin{itemdecl}
template <class RandomAccessIterator2>
  pair<RandomAccessIterator2, RandomAccessIterator2>
    operator()(RandomAccessIterator2 first, RandomAccessIterator2 last) const;
\end{itemdecl}

\begin{itemdescr}
\pnum
\requires
\tcode{RandomAccessIterator1} and \tcode{RandomAccessIterator2} shall have the same value type.

\pnum
\effects
Finds a subsequence of equal values in a sequence.

\pnum
\returns
A pair of iterators \tcode{i} and \tcode{j} such that
\begin{itemize}
\item \tcode{i} is the first iterator \tcode{i} in the range
\range{first}{last - (pat_last_ - pat_first_)} such that
for every non-negative integer \tcode{n} less than \tcode{pat_last_ - pat_first_}
the following condition holds:
\tcode{pred(*(i + n), *(pat_first_ + n)) != false}, and
\item \tcode{j == next(i, distance(pat_first_, pat_last_))}.
\end{itemize}
Returns \tcode{make_pair(first, first)} if \range{pat_first_}{pat_last_} is empty,
otherwise returns \tcode{make_pair(last, last)} if no such iterator is found.

\pnum
\complexity
At most \tcode{(last - first) * (pat_last_ - pat_first_)} applications of the predicate.
\end{itemdescr}

\rSec4[func.search.bmh.creation]{\tcode{boyer_moore_horspool_searcher} creation functions}

\indexlibrary{\idxcode{make_boyer_moore_horspool_searcher}}%
\begin{itemdecl}
template <class RandomAccessIterator,
          class Hash = hash<typename iterator_traits<RandomAccessIterator>::value_type>,
          class BinaryPredicate = equal_to<>>
  boyer_moore_horspool_searcher<RandomAccessIterator, Hash, BinaryPredicate>
    make_boyer_moore_horspool_searcher(
      RandomAccessIterator pat_first, RandomAccessIterator pat_last,
      Hash hf = Hash(), BinaryPredicate pred = BinaryPredicate());
\end{itemdecl}

\begin{itemdescr}

\pnum
\effects
Equivalent to:
\begin{codeblock}
return boyer_moore_horspool_searcher<RandomAccessIterator, Hash, BinaryPredicate>(
         pat_first, pat_last, hf, pred);
\end{codeblock}
\end{itemdescr}

\rSec2[unord.hash]{Class template \tcode{hash}}

\pnum
\indexlibrary{\idxcode{hash}}%
\indextext{\idxcode{hash}!instantiation restrictions}%
The unordered associative containers defined in \ref{unord} use
specializations of the class template \tcode{hash} as the default hash function.
For all object types \tcode{Key} for which there exists a specialization \tcode{hash<Key>},
and for all integral and enumeration types~(\ref{dcl.enum}) \tcode{Key},
the instantiation \tcode{hash<Key>} shall:

\begin{itemize}
\item satisfy the \tcode{Hash} requirements~(\ref{hash.requirements}),
with \tcode{Key} as the function
call argument type, the \tcode{Default\-Constructible} requirements (Table~\ref{tab:defaultconstructible}),
the \tcode{CopyAssignable} requirements (Table~\ref{tab:copyassignable}),
\item be swappable~(\ref{swappable.requirements}) for lvalues,
\item satisfy the requirement that if \tcode{k1 == k2} is \tcode{true}, \tcode{h(k1) == h(k2)} is
also \tcode{true}, where \tcode{h} is an object of type \tcode{hash<Key>} and \tcode{k1} and \tcode{k2}
are objects of type \tcode{Key};
\item satisfy the requirement that the expression \tcode{h(k)}, where \tcode{h}
is an object of type \tcode{hash<Key>} and \tcode{k} is an object of type
\tcode{Key}, shall not throw an exception unless \tcode{hash<Key>} is a
user-defined specialization that depends on at least one user-defined type.
\end{itemize}

\indexlibrary{\idxcode{hash}}%
\begin{itemdecl}
template <> struct hash<bool>;
template <> struct hash<char>;
template <> struct hash<signed char>;
template <> struct hash<unsigned char>;
template <> struct hash<char16_t>;
template <> struct hash<char32_t>;
template <> struct hash<wchar_t>;
template <> struct hash<short>;
template <> struct hash<unsigned short>;
template <> struct hash<int>;
template <> struct hash<unsigned int>;
template <> struct hash<long>;
template <> struct hash<unsigned long>;
template <> struct hash<long long>;
template <> struct hash<unsigned long long>;
template <> struct hash<float>;
template <> struct hash<double>;
template <> struct hash<long double>;
template <class T> struct hash<T*>;
\end{itemdecl}

\begin{itemdescr}
\pnum
The template specializations shall meet the requirements of class template
\tcode{hash}~(\ref{unord.hash}).
\end{itemdescr}

\rSec2[func.default.traits]{Default functor traits}

\indexlibrary{\idxcode{default_order}}%
\begin{codeblock}
namespace std {
  template <class T = void>
  struct default_order {
    using type = less<T>;
  };
}
\end{codeblock}

\pnum
The class template \tcode{default_order} provides a trait that users can
specialize for user-defined types to provide a strict weak ordering for that
type, which the library can use where a default strict weak order is needed.
For example, the associative containers (\ref{associative}) and
\tcode{priority_queue} (\ref{priority.queue}) use this trait.

\pnum
\begin{example}
\begin{codeblock}
namespace sales {
  struct account {
    int id;
    std::string name;
  };

  struct order_accounts {
    bool operator()(const Account& lhs, const Account& rhs) const {
      return lhs.id < rhs.id;
    }
  };
}

namespace std {
  template<>
  struct default_order<sales::account> {
    using type = sales::order_accounts;
  };
}
\end{codeblock}
\end{example}

\rSec1[meta]{Metaprogramming and type traits}

\pnum
This subclause describes components used by \Cpp programs, particularly in
templates, to support the widest possible range of types, optimise
template code usage, detect type related user errors, and perform
type inference and transformation at compile time. It includes type
classification traits, type property inspection traits, and type
transformations. The type classification traits describe a complete taxonomy
of all possible \Cpp types, and state where in that taxonomy a given
type belongs. The type property inspection traits allow important
characteristics of types or of combinations of types to be inspected. The
type transformations allow certain properties of types to be manipulated.

\rSec2[meta.rqmts]{Requirements}

\pnum
A \defn{UnaryTypeTrait} describes a property
of a type. It shall be a class template that takes one template type
argument and, optionally, additional arguments that help define the
property being described. It shall be \tcode{DefaultConstructible},
\tcode{CopyConstructible},
and publicly and unambiguously derived, directly or indirectly, from
its \defn{BaseCharacteristic}, which is
a specialization of the template
\tcode{integral_constant}~(\ref{meta.help}), with
the arguments to the template \tcode{integral_constant} determined by the
requirements for the particular property being described.
The member names of the BaseCharacteristic shall not be hidden and shall be
unambiguously available in the UnaryTypeTrait.

\pnum
A \defn{BinaryTypeTrait} describes a
relationship between two types. It shall be a class template that
takes two template type arguments and, optionally, additional
arguments that help define the relationship being described. It shall
be \tcode{DefaultConstructible}, \tcode{CopyConstructible},
and publicly and unambiguously derived, directly or
indirectly, from
its \term{BaseCharacteristic}, which is a specialization
of the template
\tcode{integral_constant}~(\ref{meta.help}), with
the arguments to the template \tcode{integral_constant} determined by the
requirements for the particular relationship being described.
The member names of the BaseCharacteristic shall not be hidden and shall be
unambiguously available in the BinaryTypeTrait.

\pnum
A \defn{TransformationTrait}
modifies a property
of a type. It shall be a class template that takes one
template type argument and, optionally, additional arguments that help
define the modification. It shall define a publicly accessible nested type
named \tcode{type}, which shall be a synonym for the modified type.

\indexlibrary{\idxhdr{type_traits}}%
\rSec2[meta.type.synop]{Header \tcode{<type_traits>} synopsis}
\begin{codeblock}
namespace std {
  // \ref{meta.help}, helper class:
  template <class T, T v> struct integral_constant;

  template <bool B>
    using bool_constant = integral_constant<bool, B>;
  using true_type  = bool_constant<true>;
  using false_type = bool_constant<false>;

  // \ref{meta.unary.cat}, primary type categories:
  template <class T> struct is_void;
  template <class T> struct is_null_pointer;
  template <class T> struct is_integral;
  template <class T> struct is_floating_point;
  template <class T> struct is_array;
  template <class T> struct is_pointer;
  template <class T> struct is_lvalue_reference;
  template <class T> struct is_rvalue_reference;
  template <class T> struct is_member_object_pointer;
  template <class T> struct is_member_function_pointer;
  template <class T> struct is_enum;
  template <class T> struct is_union;
  template <class T> struct is_class;
  template <class T> struct is_function;

  // \ref{meta.unary.comp}, composite type categories:
  template <class T> struct is_reference;
  template <class T> struct is_arithmetic;
  template <class T> struct is_fundamental;
  template <class T> struct is_object;
  template <class T> struct is_scalar;
  template <class T> struct is_compound;
  template <class T> struct is_member_pointer;

  // \ref{meta.unary.prop}, type properties:
  template <class T> struct is_const;
  template <class T> struct is_volatile;
  template <class T> struct is_trivial;
  template <class T> struct is_trivially_copyable;
  template <class T> struct is_standard_layout;
  template <class T> struct is_pod;
  template <class T> struct is_empty;
  template <class T> struct is_polymorphic;
  template <class T> struct is_abstract;
  template <class T> struct is_final;

  template <class T> struct is_signed;
  template <class T> struct is_unsigned;

  template <class T, class... Args> struct is_constructible;
  template <class T> struct is_default_constructible;
  template <class T> struct is_copy_constructible;
  template <class T> struct is_move_constructible;

  template <class T, class U> struct is_assignable;
  template <class T> struct is_copy_assignable;
  template <class T> struct is_move_assignable;

  template <class T, class U> struct is_swappable_with;
  template <class T> struct is_swappable;

  template <class T> struct is_destructible;

  template <class T, class... Args> struct is_trivially_constructible;
  template <class T> struct is_trivially_default_constructible;
  template <class T> struct is_trivially_copy_constructible;
  template <class T> struct is_trivially_move_constructible;

  template <class T, class U> struct is_trivially_assignable;
  template <class T> struct is_trivially_copy_assignable;
  template <class T> struct is_trivially_move_assignable;
  template <class T> struct is_trivially_destructible;

  template <class T, class... Args> struct is_nothrow_constructible;
  template <class T> struct is_nothrow_default_constructible;
  template <class T> struct is_nothrow_copy_constructible;
  template <class T> struct is_nothrow_move_constructible;

  template <class T, class U> struct is_nothrow_assignable;
  template <class T> struct is_nothrow_copy_assignable;
  template <class T> struct is_nothrow_move_assignable;

  template <class T, class U> struct is_nothrow_swappable_with;
  template <class T> struct is_nothrow_swappable;

  template <class T> struct is_nothrow_destructible;

  template <class T> struct has_virtual_destructor;

  template <class T> struct has_unique_object_representations;

  // \ref{meta.unary.prop.query}, type property queries:
  template <class T> struct alignment_of;
  template <class T> struct rank;
  template <class T, unsigned I = 0> struct extent;

  // \ref{meta.rel}, type relations:
  template <class T, class U> struct is_same;
  template <class Base, class Derived> struct is_base_of;
  template <class From, class To> struct is_convertible;

  template <class, class R = void> struct is_callable; // not defined
  template <class Fn, class... ArgTypes, class R>
    struct is_callable<Fn(ArgTypes...), R>;
     
  template <class, class R = void> struct is_nothrow_callable; // not defined
  template <class Fn, class... ArgTypes, class R>
    struct is_nothrow_callable<Fn(ArgTypes...), R>;

  // \ref{meta.trans.cv}, const-volatile modifications:
  template <class T> struct remove_const;
  template <class T> struct remove_volatile;
  template <class T> struct remove_cv;
  template <class T> struct add_const;
  template <class T> struct add_volatile;
  template <class T> struct add_cv;

  template <class T>
    using remove_const_t    = typename remove_const<T>::type;
  template <class T>
    using remove_volatile_t = typename remove_volatile<T>::type;
  template <class T>
    using remove_cv_t       = typename remove_cv<T>::type;
  template <class T>
    using add_const_t       = typename add_const<T>::type;
  template <class T>
    using add_volatile_t    = typename add_volatile<T>::type;
  template <class T>
    using add_cv_t          = typename add_cv<T>::type;

  // \ref{meta.trans.ref}, reference modifications:
  template <class T> struct remove_reference;
  template <class T> struct add_lvalue_reference;
  template <class T> struct add_rvalue_reference;

  template <class T>
    using remove_reference_t     = typename remove_reference<T>::type;
  template <class T>
    using add_lvalue_reference_t = typename add_lvalue_reference<T>::type;
  template <class T>
    using add_rvalue_reference_t = typename add_rvalue_reference<T>::type;

  // \ref{meta.trans.sign}, sign modifications:
  template <class T> struct make_signed;
  template <class T> struct make_unsigned;

  template <class T>
    using make_signed_t   = typename make_signed<T>::type;
  template <class T>
    using make_unsigned_t = typename make_unsigned<T>::type;

  // \ref{meta.trans.arr}, array modifications:
  template <class T> struct remove_extent;
  template <class T> struct remove_all_extents;

  template <class T>
    using remove_extent_t      = typename remove_extent<T>::type;
  template <class T>
    using remove_all_extents_t = typename remove_all_extents<T>::type;

  // \ref{meta.trans.ptr}, pointer modifications:
  template <class T> struct remove_pointer;
  template <class T> struct add_pointer;

  template <class T>
    using remove_pointer_t = typename remove_pointer<T>::type;
  template <class T>
    using add_pointer_t    = typename add_pointer<T>::type;

  // \ref{meta.trans.other}, other transformations:
  template <size_t Len,
            size_t Align = @\textit{default-alignment}@> // see \ref{meta.trans.other}
    struct aligned_storage;      
  template <size_t Len, class... Types> struct aligned_union;
  template <class T> struct decay;
  template <bool, class T = void> struct enable_if;
  template <bool, class T, class F> struct conditional;
  template <class... T> struct common_type;
  template <class T> struct underlying_type;
  template <class> class result_of;   // not defined
  template <class F, class... ArgTypes> class result_of<F(ArgTypes...)>;

  template <size_t Len,
            size_t Align = @\textit{default-alignment}@> // see \ref{meta.trans.other}
    using aligned_storage_t = typename aligned_storage<Len, Align>::type;
  template <size_t Len, class... Types>
    using aligned_union_t   = typename aligned_union<Len, Types...>::type;
  template <class T>
    using decay_t           = typename decay<T>::type;
  template <bool b, class T = void>
    using enable_if_t       = typename enable_if<b, T>::type;
  template <bool b, class T, class F>
    using conditional_t     = typename conditional<b, T, F>::type;
  template <class... T>
    using common_type_t     = typename common_type<T...>::type;
  template <class T>
    using underlying_type_t = typename underlying_type<T>::type;
  template <class T>
    using result_of_t       = typename result_of<T>::type;  
  template <class...>
    using void_t            = void;

  // \ref{meta.logical}, logical operator traits:
  template<class... B> struct conjunction;
  template<class... B> struct disjunction;
  template<class B> struct negation;

  // \ref{meta.unary.cat}, primary type categories
  template <class T> constexpr bool is_void_v
    = is_void<T>::value;
  template <class T> constexpr bool is_null_pointer_v
    = is_null_pointer<T>::value;
  template <class T> constexpr bool is_integral_v
    = is_integral<T>::value;
  template <class T> constexpr bool is_floating_point_v
    = is_floating_point<T>::value;
  template <class T> constexpr bool is_array_v
    = is_array<T>::value;
  template <class T> constexpr bool is_pointer_v
    = is_pointer<T>::value;
  template <class T> constexpr bool is_lvalue_reference_v
    = is_lvalue_reference<T>::value;
  template <class T> constexpr bool is_rvalue_reference_v
    = is_rvalue_reference<T>::value;
  template <class T> constexpr bool is_member_object_pointer_v
    = is_member_object_pointer<T>::value;
  template <class T> constexpr bool is_member_function_pointer_v
    = is_member_function_pointer<T>::value;
  template <class T> constexpr bool is_enum_v
    = is_enum<T>::value;
  template <class T> constexpr bool is_union_v
    = is_union<T>::value;
  template <class T> constexpr bool is_class_v
    = is_class<T>::value;
  template <class T> constexpr bool is_function_v
    = is_function<T>::value;

  // \ref{meta.unary.comp}, composite type categories
  template <class T> constexpr bool is_reference_v
    = is_reference<T>::value;
  template <class T> constexpr bool is_arithmetic_v
    = is_arithmetic<T>::value;
  template <class T> constexpr bool is_fundamental_v
    = is_fundamental<T>::value;
  template <class T> constexpr bool is_object_v
    = is_object<T>::value;
  template <class T> constexpr bool is_scalar_v
    = is_scalar<T>::value;
  template <class T> constexpr bool is_compound_v
    = is_compound<T>::value;
  template <class T> constexpr bool is_member_pointer_v
    = is_member_pointer<T>::value;

  // \ref{meta.unary.prop}, type properties
  template <class T> constexpr bool is_const_v
    = is_const<T>::value;
  template <class T> constexpr bool is_volatile_v
    = is_volatile<T>::value;
  template <class T> constexpr bool is_trivial_v
    = is_trivial<T>::value;
  template <class T> constexpr bool is_trivially_copyable_v
    = is_trivially_copyable<T>::value;
  template <class T> constexpr bool is_standard_layout_v
    = is_standard_layout<T>::value;
  template <class T> constexpr bool is_pod_v
    = is_pod<T>::value;
  template <class T> constexpr bool is_empty_v
    = is_empty<T>::value;
  template <class T> constexpr bool is_polymorphic_v
    = is_polymorphic<T>::value;
  template <class T> constexpr bool is_abstract_v
    = is_abstract<T>::value;
  template <class T> constexpr bool is_final_v
    = is_final<T>::value;
  template <class T> constexpr bool is_signed_v
    = is_signed<T>::value;
  template <class T> constexpr bool is_unsigned_v
    = is_unsigned<T>::value;
  template <class T, class... Args> constexpr bool is_constructible_v
    = is_constructible<T, Args...>::value;
  template <class T> constexpr bool is_default_constructible_v
    = is_default_constructible<T>::value;
  template <class T> constexpr bool is_copy_constructible_v
    = is_copy_constructible<T>::value;
  template <class T> constexpr bool is_move_constructible_v
    = is_move_constructible<T>::value;
  template <class T, class U> constexpr bool is_assignable_v
    = is_assignable<T, U>::value;
  template <class T> constexpr bool is_copy_assignable_v
    = is_copy_assignable<T>::value;
  template <class T> constexpr bool is_move_assignable_v
    = is_move_assignable<T>::value;
  template <class T, class U> constexpr bool is_swappable_with_v
    = is_swappable_with<T, U>::value;
  template <class T> constexpr bool is_swappable_v
    = is_swappable<T>::value;
  template <class T> constexpr bool is_destructible_v
    = is_destructible<T>::value;
  template <class T, class... Args> constexpr bool is_trivially_constructible_v
    = is_trivially_constructible<T, Args...>::value;
  template <class T> constexpr bool is_trivially_default_constructible_v
    = is_trivially_default_constructible<T>::value;
  template <class T> constexpr bool is_trivially_copy_constructible_v
    = is_trivially_copy_constructible<T>::value;
  template <class T> constexpr bool is_trivially_move_constructible_v
    = is_trivially_move_constructible<T>::value;
  template <class T, class U> constexpr bool is_trivially_assignable_v
    = is_trivially_assignable<T, U>::value;
  template <class T> constexpr bool is_trivially_copy_assignable_v
    = is_trivially_copy_assignable<T>::value;
  template <class T> constexpr bool is_trivially_move_assignable_v
    = is_trivially_move_assignable<T>::value;
  template <class T> constexpr bool is_trivially_destructible_v
    = is_trivially_destructible<T>::value;
  template <class T, class... Args> constexpr bool is_nothrow_constructible_v
    = is_nothrow_constructible<T, Args...>::value;
  template <class T> constexpr bool is_nothrow_default_constructible_v
    = is_nothrow_default_constructible<T>::value;
  template <class T> constexpr bool is_nothrow_copy_constructible_v
    = is_nothrow_copy_constructible<T>::value;
  template <class T> constexpr bool is_nothrow_move_constructible_v
    = is_nothrow_move_constructible<T>::value;
  template <class T, class U> constexpr bool is_nothrow_assignable_v
    = is_nothrow_assignable<T, U>::value;
  template <class T> constexpr bool is_nothrow_copy_assignable_v
    = is_nothrow_copy_assignable<T>::value;
  template <class T> constexpr bool is_nothrow_move_assignable_v
    = is_nothrow_move_assignable<T>::value;
  template <class T, class U> constexpr bool is_nothrow_swappable_with_v
    = is_nothrow_swappable_with<T, U>::value;
  template <class T> constexpr bool is_nothrow_swappable_v
    = is_nothrow_swappable<T>::value;
  template <class T> constexpr bool is_nothrow_destructible_v
    = is_nothrow_destructible<T>::value;
  template <class T> constexpr bool has_virtual_destructor_v
    = has_virtual_destructor<T>::value;
  template <class T> constexpr bool has_unique_object_representations_v
    = has_unique_object_representations<T>::value;

  // See \ref{meta.unary.prop.query}, type property queries
  template <class T> constexpr size_t alignment_of_v
    = alignment_of<T>::value;
  template <class T> constexpr size_t rank_v
    = rank<T>::value;
  template <class T, unsigned I = 0> constexpr size_t extent_v
    = extent<T, I>::value;

  // See \ref{meta.rel}, type relations
  template <class T, class U> constexpr bool is_same_v
    = is_same<T, U>::value;
  template <class Base, class Derived> constexpr bool is_base_of_v
    = is_base_of<Base, Derived>::value;
  template <class From, class To> constexpr bool is_convertible_v
    = is_convertible<From, To>::value;
  template <class T, class R = void> constexpr bool is_callable_v
    = is_callable<T, R>::value;
  template <class T, class R = void> constexpr bool is_nothrow_callable_v
    = is_nothrow_callable<T, R>::value;

  // \ref{meta.logical}, logical operator traits:
  template<class... B> constexpr bool conjunction_v = conjunction<B...>::value;
  template<class... B> constexpr bool disjunction_v = disjunction<B...>::value;
  template<class B> constexpr bool negation_v = negation<B>::value;
}
\end{codeblock}

\pnum
The behavior of a program that adds specializations for any of
the templates defined in this subclause is undefined unless otherwise specified.

\pnum
Unless otherwise specified, an incomplete type may be used
to instantiate a template in this subclause.

\rSec2[meta.help]{Helper classes}

\begin{codeblock}
namespace std {
  template <class T, T v>
  struct integral_constant {
    static constexpr T value = v;
    using value_type = T;
    using type       = integral_constant<T, v>;
    constexpr operator value_type() const noexcept { return value; }
    constexpr value_type operator()() const noexcept { return value; }
  };
}
\end{codeblock}

\indexlibrary{\idxcode{integral_constant}}%
\indexlibrary{\idxcode{bool_constant}}%
\indexlibrary{\idxcode{true_type}}%
\indexlibrary{\idxcode{false_type}}%
\pnum
The class template \tcode{integral_constant},
alias template \tcode{bool_constant}, and
its associated \grammarterm{typedef-name}{s}
\tcode{true_type} and \tcode{false_type}
are used as base classes to define
the interface for various type traits.

\rSec2[meta.unary]{Unary type traits}

\pnum
This subclause contains templates that may be used to query the
properties of a type at compile time.

\pnum
Each of these templates shall be a
UnaryTypeTrait~(\ref{meta.rqmts})
with a BaseCharacteristic of
\tcode{true_type} if the corresponding condition is \tcode{true}, otherwise
\tcode{false_type}.

\rSec3[meta.unary.cat]{Primary type categories}

\pnum
The primary type categories correspond to the descriptions given in
section~\ref{basic.types} of the \Cpp standard.

\pnum
For any given type \tcode{T}, the result of applying one of these templates to
\tcode{T} and to \cv{} \tcode{T} shall yield the same result.

\pnum
\begin{note}
For any given type \tcode{T}, exactly one of the primary type categories
has a \tcode{value} member that evaluates to \tcode{true}.
\end{note}

\begin{libreqtab3e}{Primary type category predicates}{tab:type-traits.primary}
\\ \topline
\lhdr{Template} &   \chdr{Condition}    &   \rhdr{Comments} \\\capsep
\endfirsthead
\continuedcaption\\
\topline
\lhdr{Template} &   \chdr{Condition}    &   \rhdr{Comments} \\ \capsep
\endhead
\indexlibrary{\idxcode{is_void}}%
\tcode{template <class T>}\br
 \tcode{struct is_void;}                &
\tcode{T} is \tcode{void}       &   \\ \rowsep
\indexlibrary{\idxcode{is_null_pointer}}%
\tcode{template <class T>}\br
 \tcode{struct is_null_pointer;}                &
\tcode{T} is \tcode{nullptr_t}~(\ref{basic.fundamental})       &   \\ \rowsep
\indexlibrary{\idxcode{is_integral}}%
\tcode{template <class T>}\br
 \tcode{struct is_integral;}        &
\tcode{T} is an integral type~(\ref{basic.fundamental})                 &   \\ \rowsep
\indexlibrary{\idxcode{is_floating_point}}%
\tcode{template <class T>}\br
 \tcode{struct is_floating_point;}  &
\tcode{T} is a floating point type~(\ref{basic.fundamental})            &   \\ \rowsep
\indexlibrary{\idxcode{is_array}}%
\tcode{template <class T>}\br
 \tcode{struct is_array;}           &
\tcode{T} is an array type~(\ref{basic.compound}) of known or unknown extent    &
Class template \tcode{array}~(\ref{array})
is not an array type.                   \\ \rowsep
\indexlibrary{\idxcode{is_pointer}}%
\tcode{template <class T>}\br
 \tcode{struct is_pointer;}         &
\tcode{T} is a pointer type~(\ref{basic.compound})                      &
Includes pointers to functions
but not pointers to non-static members.                        \\ \rowsep
\indexlibrary{\idxcode{is_lvalue_reference}}%
\tcode{template <class T>}\br
 \tcode{struct is_lvalue_reference;}    &
 \tcode{T} is an lvalue reference type~(\ref{dcl.ref})   &   \\ \rowsep
\indexlibrary{\idxcode{is_rvalue_reference}}%
\tcode{template <class T>}\br
 \tcode{struct is_rvalue_reference;}    &
 \tcode{T} is an rvalue reference type~(\ref{dcl.ref})   &   \\ \rowsep
\indexlibrary{\idxcode{is_member_object_pointer}}%
\tcode{template <class T>}\br
 \tcode{struct is_member_object_pointer;}&
 \tcode{T} is a pointer to non-static data member                              &   \\ \rowsep
\indexlibrary{\idxcode{is_member_function_pointer}}%
\tcode{template <class T>}\br
 \tcode{struct is_member_function_pointer;}&
\tcode{T} is a pointer to non-static member function                           &   \\ \rowsep
\indexlibrary{\idxcode{is_enum}}%
\tcode{template <class T>}\br
 \tcode{struct is_enum;}            &
\tcode{T} is an enumeration type~(\ref{basic.compound})                 &   \\ \rowsep
\indexlibrary{\idxcode{is_union}}%
\tcode{template <class T>}\br
 \tcode{struct is_union;}           &
\tcode{T} is a union type~(\ref{basic.compound})                        &   \\ \rowsep
\indexlibrary{\idxcode{is_class}}%
\tcode{template <class T>}\br
 \tcode{struct is_class;}           &
\tcode{T} is a non-union class type~(\ref{basic.compound}) & \\ \rowsep
\indexlibrary{\idxcode{is_function}}%
\tcode{template <class T>}\br
 \tcode{struct is_function;}        &
\tcode{T} is a function type~(\ref{basic.compound})                     &   \\
\end{libreqtab3e}

\rSec3[meta.unary.comp]{Composite type traits}

\pnum
These templates provide convenient compositions of the primary type
categories, corresponding to the descriptions given in section~\ref{basic.types}.

\pnum
For any given type \tcode{T}, the result of applying one of these templates to
\tcode{T} and to \cv{} \tcode{T} shall yield the same result.

\begin{libreqtab3b}{Composite type category predicates}{tab:type-traits.composite}
\\ \topline
\lhdr{Template} &   \chdr{Condition}    &   \rhdr{Comments} \\ \capsep
\endfirsthead
\continuedcaption\\
\topline
\lhdr{Template} &   \chdr{Condition}    &   \rhdr{Comments} \\ \capsep
\endhead
\indexlibrary{\idxcode{is_reference}}%
\tcode{template <class T>}\br
 \tcode{struct is_reference;}   &
 \tcode{T} is an lvalue reference or an rvalue reference &  \\ \rowsep
\indexlibrary{\idxcode{is_arithmetic}}%
\tcode{template <class T>}\br
 \tcode{struct is_arithmetic;}          &
 \tcode{T} is an arithmetic type~(\ref{basic.fundamental})              &   \\ \rowsep
\indexlibrary{\idxcode{is_fundamental}}%
\tcode{template <class T>}\br
 \tcode{struct is_fundamental;}         &
 \tcode{T} is a fundamental type~(\ref{basic.fundamental})              &   \\ \rowsep
\indexlibrary{\idxcode{is_object}}%
\tcode{template <class T>}\br
 \tcode{struct is_object;}              &
 \tcode{T} is an object type~(\ref{basic.types})                            &   \\ \rowsep
\indexlibrary{\idxcode{is_scalar}}%
\tcode{template <class T>}\br
 \tcode{struct is_scalar;}              &
 \tcode{T} is a scalar type~(\ref{basic.types})                         &   \\ \rowsep
\indexlibrary{\idxcode{is_compound}}%
\tcode{template <class T>}\br
 \tcode{struct is_compound;}            &
 \tcode{T} is a compound type~(\ref{basic.compound})                        &   \\ \rowsep
\indexlibrary{\idxcode{is_member_pointer}}%
\tcode{template <class T>}\br
 \tcode{struct is_member_pointer;}      &
 \tcode{T} is a pointer to non-static data member
 or non-static member function             &   \\
\end{libreqtab3b}

\rSec3[meta.unary.prop]{Type properties}

\pnum
These templates provide access to some of the more important
properties of types.

\pnum
It is unspecified whether the library defines any full or partial
specializations of any of these templates.

\pnum
For all of the class templates \tcode{X} declared in this subclause,
instantiating that template with a template-argument that is a class
template specialization may result in the implicit instantiation of
the template argument if and only if the semantics of \tcode{X} require that
the argument must be a complete type.

\pnum
For the purpose of defining the templates in this subclause,
a function call expression \tcode{declval<T>()} for any type \tcode{T}
is considered to be a trivial~(\ref{basic.types}, \ref{special}) function call
that is not an odr-use~(\ref{basic.def.odr}) of \tcode{declval}
in the context of the corresponding definition
notwithstanding the restrictions of~\ref{declval}.

\begin{libreqtab3b}{Type property predicates}{tab:type-traits.properties}
\\ \topline
\lhdr{Template} &   \chdr{Condition}    &   \rhdr{Preconditions}    \\ \capsep
\endfirsthead
\continuedcaption\\
\topline
\lhdr{Template} &   \chdr{Condition}    &   \rhdr{Preconditions}    \\ \capsep
\endhead

\indexlibrary{\idxcode{is_const}}%
\tcode{template <class T>}\br
 \tcode{struct is_const;}               &
 \tcode{T} is const-qualified~(\ref{basic.type.qualifier})                  &   \\ \rowsep

\indexlibrary{\idxcode{is_volatile}}%
\tcode{template <class T>}\br
 \tcode{struct is_volatile;}            &
 \tcode{T} is volatile-qualified~(\ref{basic.type.qualifier})                   &   \\ \rowsep


\indexlibrary{\idxcode{is_trivial}}%
\tcode{template <class T>}\br
 \tcode{struct is_trivial;}                 &
 \tcode{T} is a trivial type~(\ref{basic.types})     &
 \tcode{remove_all_extents_t<T>} shall be a complete
 type or (possibly cv-qualified) \tcode{void}.                \\ \rowsep

\indexlibrary{\idxcode{is_trivially_copyable}}%
\tcode{template <class T>}\br
 \tcode{struct is_trivially_copyable;}      &
 \tcode{T} is a trivially copyable type~(\ref{basic.types}) &
 \tcode{remove_all_extents_t<T>} shall be a complete type or
 (possibly cv-qualified) \tcode{void}.                               \\ \rowsep

\indexlibrary{\idxcode{is_standard_layout}}%
\tcode{template <class T>}\br
 \tcode{struct is_standard_layout;}                 &
 \tcode{T} is a standard-layout type~(\ref{basic.types})   &
 \tcode{remove_all_extents_t<T>} shall be a complete
 type or (possibly cv-qualified) \tcode{void}.                \\ \rowsep

\indexlibrary{\idxcode{is_pod}}%
\tcode{template <class T>}\br
 \tcode{struct is_pod;}                 &
 \tcode{T} is a POD type~(\ref{basic.types})                                &
 \tcode{remove_all_extents_t<T>} shall be a complete
 type or (possibly cv-qualified) \tcode{void}.                \\ \rowsep

\indexlibrary{\idxcode{is_empty}!class}%
\tcode{template <class T>}\br
 \tcode{struct is_empty;}               &
 \tcode{T} is a class type, but not a union type, with no non-static data
 members other than bit-fields of length 0, no virtual member functions,
 no virtual base classes, and no base class \tcode{B} for
 which \tcode{is_empty_v<B>} is \tcode{false}. &
 If \tcode{T} is a non-union class type, \tcode{T} shall be a complete type.                               \\ \rowsep

\indexlibrary{\idxcode{is_polymorphic}}%
\tcode{template <class T>}\br
 \tcode{struct is_polymorphic;}         &
 \tcode{T} is a polymorphic class~(\ref{class.virtual})                             &
 If \tcode{T} is a non-union class type, \tcode{T} shall be a complete type.                \\ \rowsep

\indexlibrary{\idxcode{is_abstract}}%
\tcode{template <class T>}\br
 \tcode{struct is_abstract;}            &
 \tcode{T} is an abstract class~(\ref{class.abstract})                              &
 If \tcode{T} is a non-union class type, \tcode{T} shall be a complete type.                \\ \rowsep

\indexlibrary{\idxcode{is_final}}%
\tcode{template <class T>}\br
 \tcode{struct is_final;}               &
 \tcode{T} is a class type marked with the \grammarterm{class-virt-specifier}
 \tcode{final} (Clause~\ref{class}). \begin{note} A union is a class type that
 can be marked with \tcode{final}. \end{note}                                        &
 If \tcode{T} is a class type, \tcode{T} shall be a complete type.                          \\ \rowsep

\indexlibrary{\idxcode{is_signed}!class}%
\tcode{template <class T>}\br
  \tcode{struct is_signed;}              &
  If \tcode{is_arithmetic_v<T>} is \tcode{true}, the same result as
  \tcode{T(-1) < T(0)};
  otherwise, \tcode{false}   &   \\  \rowsep

\indexlibrary{\idxcode{is_unsigned}}%
\tcode{template <class T>}\br
  \tcode{struct is_unsigned;}            &
  If \tcode{is_arithmetic_v<T>} is \tcode{true}, the same result as
  \tcode{T(0) < T(-1)};
  otherwise, \tcode{false}   &   \\  \rowsep

\indexlibrary{\idxcode{is_constructible}}%
\tcode{template <class T, class... Args>}\br
 \tcode{struct is_constructible;}   &
 For a function type \tcode{T},
 \tcode{is_constructible_v<T, Args...>} is \tcode{false},
 otherwise \seebelow                &
 \tcode{T} and all types in the parameter pack \tcode{Args}
 shall be complete types, (possibly cv-qualified) \tcode{void},
 or arrays of unknown bound.  \\ \rowsep

\indexlibrary{\idxcode{is_default_constructible}}%
\tcode{template <class T>}\br
  \tcode{struct is_default_constructible;} &
  \tcode{is_constructible_v<T>} is \tcode{true}. &
  \tcode{T} shall be a complete type, (possibly cv-qualified) \tcode{void},
  or an array of unknown bound. \\ \rowsep

\indexlibrary{\idxcode{is_copy_constructible}}%
\tcode{template <class T>}\br
  \tcode{struct is_copy_constructible;} &
  For a referenceable type \tcode{T} (\ref{defns.referenceable}), the same result as
  \tcode{is_constructible_v<T, const T\&>}, otherwise \tcode{false}. &
  \tcode{T} shall be a complete type, (possibly cv-qualified) \tcode{void},
  or an array of unknown bound. \\ \rowsep

\indexlibrary{\idxcode{is_move_constructible}}%
\tcode{template <class T>}\br
  \tcode{struct is_move_constructible;} &
  For a referenceable type \tcode{T}, the same result as
  \tcode{is_constructible_v<T, T\&\&>}, otherwise \tcode{false}. &
  \tcode{T} shall be a complete type, (possibly cv-qualified) \tcode{void},
  or an array of unknown bound. \\ \rowsep

\indexlibrary{\idxcode{is_assignable}}%
\tcode{template <class T, class U>}\br
  \tcode{struct is_assignable;} &
  The expression \tcode{declval<T>() =} \tcode{declval<U>()} is well-formed
  when treated as an unevaluated
  operand (Clause~\ref{expr}). Access checking is performed as if in a context
  unrelated to \tcode{T} and \tcode{U}. Only the validity of the immediate context
  of the assignment expression is considered. \begin{note} The compilation of the
  expression can result in side effects such as the instantiation of class template
  specializations and function template specializations, the generation of
  implicitly-defined functions, and so on. Such side effects are not in the ``immediate
  context'' and can result in the program being ill-formed. \end{note} &
  \tcode{T} and \tcode{U} shall be complete types, (possibly cv-qualified) \tcode{void},
  or arrays of unknown bound. \\ \rowsep

\indexlibrary{\idxcode{is_copy_assignable}}%
\tcode{template <class T>}\br
  \tcode{struct is_copy_assignable;} &
  For a referenceable type \tcode{T}, the same result as
  \tcode{is_assignable_v<T\&, const T\&>}, otherwise \tcode{false}. &
  \tcode{T} shall be a complete type, (possibly cv-qualified) \tcode{void},
  or an array of unknown bound. \\ \rowsep

\indexlibrary{\idxcode{is_move_assignable}}%
\tcode{template <class T>}\br
  \tcode{struct is_move_assignable;} &
  For a referenceable type \tcode{T}, the same result as
  \tcode{is_assignable_v<T\&, T\&\&>}, otherwise \tcode{false}. &
  \tcode{T} shall be a complete type, (possibly cv-qualified) \tcode{void},
  or an array of unknown bound. \\ \rowsep

\indexlibrary{\idxcode{is_swappable_with}}%
\tcode{template <class T, class U>}\br
  \tcode{struct is_swappable_with;} &
  The expressions \tcode{swap(declval<T>(), declval<U>())} and
  \tcode{swap(declval<U>(), declval<T>())} are each well-formed
  when treated as an unevaluated operand (Clause~\ref{expr})
  in an overload-resolution context
  for swappable values~(\ref{swappable.requirements}).
  Access checking is performed as if in a context
  unrelated to \tcode{T} and \tcode{U}.
  Only the validity of the immediate context
  of the \tcode{swap} expressions is considered.
  \begin{note}
  The compilation of the expressions can result in side effects
  such as the instantiation of class template specializations and
  function template specializations,
  the generation of implicitly-defined functions, and so on.
  Such side effects are not in the ``immediate context'' and
  can result in the program being ill-formed.
  \end{note} &
  \tcode{T} and \tcode{U} shall be complete types,
  (possibly cv-qualified) \tcode{void}, or
  arrays of unknown bound.  \\ \rowsep

\indexlibrary{\idxcode{is_swappable}}%
\tcode{template <class T>}\br
  \tcode{struct is_swappable;} &
  For a referenceable type \tcode{T},
  the same result as \tcode{is_swappable_with_v<T\&, T\&>},
  otherwise \tcode{false}. &
  \tcode{T} shall be a complete type,
  (possibly cv-qualified) \tcode{void}, or
  an array of unknown bound. \\ \rowsep

\indexlibrary{\idxcode{is_destructible}}%
\tcode{template <class T>}\br
  \tcode{struct is_destructible;} &
  Either \tcode{T} is a reference type,
  or \tcode{T} is a complete object type
  for which the expression
  \tcode{declval<U\&>().\~U()}
  is well-formed
  when treated as an unevaluated operand (Clause \ref{expr}),
  where \tcode{U} is
  \tcode{remove_all_extents<T>}. &
  \tcode{T} shall be a complete type, (possibly cv-qualified) \tcode{void},
  or an array of unknown bound. \\ \rowsep

\indexlibrary{\idxcode{is_trivially_constructible}}%
\tcode{template <class T, class... Args>}\br
  \tcode{struct}\br
  \tcode{is_trivially_constructible;} &
  \tcode{is_constructible_v<T,}\br
  \tcode{Args...>} is \tcode{true} and the variable
  definition for \tcode{is_constructible}, as defined below, is known to call
  no operation that is not trivial (~\ref{basic.types},~\ref{special}). &
  \tcode{T} and all types in the parameter pack \tcode{Args} shall be complete types,
  (possibly cv-qualified) \tcode{void}, or arrays of unknown bound. \\ \rowsep

\indexlibrary{\idxcode{is_trivially_default_constructible}}%
\tcode{template <class T>}\br
 \tcode{struct is_trivially_default_constructible;} &
 \tcode{is_trivially_constructible_v<T>} is \tcode{true}. &
 \tcode{T} shall be a complete type,
 (possibly cv-qualified) \tcode{void}, or an array of unknown
 bound.                \\ \rowsep

\indexlibrary{\idxcode{is_trivially_copy_constructible}}%
\tcode{template <class T>}\br
 \tcode{struct is_trivially_copy_constructible;}      &
  For a referenceable type \tcode{T}, the same result as
 \tcode{is_trivially_constructible_v<T, const T\&>}, otherwise \tcode{false}. &
  \tcode{T} shall be a complete type,
 (possibly cv-qualified) \tcode{void}, or an array of unknown
 bound.                \\ \rowsep

\indexlibrary{\idxcode{is_trivially_move_constructible}}%
\tcode{template <class T>}\br
 \tcode{struct is_trivially_move_constructible;}      &
  For a referenceable type \tcode{T}, the same result as
 \tcode{is_trivially_constructible_v<T, T\&\&>}, otherwise \tcode{false}. &
  \tcode{T} shall be a complete type,
 (possibly cv-qualified) \tcode{void}, or an array of unknown
 bound.                \\ \rowsep

\indexlibrary{\idxcode{is_trivially_assignable}}%
\tcode{template <class T, class U>}\br
  \tcode{struct is_trivially_assignable;} &
  \tcode{is_assignable_v<T, U>} is \tcode{true} and the assignment, as defined by
  \tcode{is_assignable}, is known to call no operation that is not trivial
  (\ref{basic.types},~\ref{special}). &
  \tcode{T} and \tcode{U} shall be complete types, (possibly cv-qualified) \tcode{void},
  or arrays of unknown bound. \\ \rowsep

\indexlibrary{\idxcode{is_trivially_copy_assignable}}%
\tcode{template <class T>}\br
 \tcode{struct is_trivially_copy_assignable;} &
  For a referenceable type \tcode{T}, the same result as
 \tcode{is_trivially_assignable_v<T\&, const T\&>}, otherwise \tcode{false}. &
 \tcode{T} shall be a complete type,
 (possibly cv-qualified) \tcode{void}, or an array of unknown
 bound.                \\ \rowsep

\indexlibrary{\idxcode{is_trivially_move_assignable}}%
\tcode{template <class T>}\br
 \tcode{struct is_trivially_move_assignable;} &
  For a referenceable type \tcode{T}, the same result as
 \tcode{is_trivially_assignable_v<T\&, T\&\&>}, otherwise \tcode{false}. &
 \tcode{T} shall be a complete type,
 (possibly cv-qualified) \tcode{void}, or an array of unknown bound.                \\ \rowsep

\indexlibrary{\idxcode{is_trivially_destructible}}%
\tcode{template <class T>}\br
 \tcode{struct is_trivially_destructible;} &
 \tcode{is_destructible_v<T>} is \tcode{true} and the indicated destructor is known
 to be trivial. &
 \tcode{T} shall be a complete type,
 (possibly cv-qualified) \tcode{void}, or an array of unknown
 bound.                \\ \rowsep

\indexlibrary{\idxcode{is_nothrow_constructible}}%
\tcode{template <class T, class... Args>}\br
 \tcode{struct is_nothrow_constructible;}   &
 \tcode{is_constructible_v<T,} \tcode{ Args...>} is \tcode{true}
 and the
 variable definition for \tcode{is_constructible}, as defined below, is known not to
 throw any exceptions~(\ref{expr.unary.noexcept}).
 &
 \tcode{T} and all types in the parameter pack \tcode{Args}
 shall be complete types, (possibly cv-qualified) \tcode{void},
 or arrays of unknown bound.  \\ \rowsep

\indexlibrary{\idxcode{is_nothrow_default_constructible}}%
\tcode{template <class T>}\br
 \tcode{struct is_nothrow_default_constructible;} &
 \tcode{is_nothrow_constructible_v<T>} is \tcode{true}.  &
 \tcode{T} shall be a complete type,
 (possibly cv-qualified) \tcode{void}, or an array of unknown
 bound.                \\ \rowsep

\indexlibrary{\idxcode{is_nothrow_copy_constructible}}%
\tcode{template <class T>}\br
 \tcode{struct is_nothrow_copy_constructible;}      &
  For a referenceable type \tcode{T}, the same result as
 \tcode{is_nothrow_constructible_v<T, const T\&>}, otherwise \tcode{false}. &
 \tcode{T} shall be a complete type,
 (possibly cv-qualified) \tcode{void}, or an array of unknown
 bound.                \\ \rowsep

\indexlibrary{\idxcode{is_nothrow_move_constructible}}%
\tcode{template <class T>}\br
 \tcode{struct is_nothrow_move_constructible;}      &
  For a referenceable type \tcode{T}, the same result as
 \tcode{is_nothrow_constructible_v<T, T\&\&>}, otherwise \tcode{false}. &
 \tcode{T} shall be a complete type,
 (possibly cv-qualified) \tcode{void}, or an array of unknown bound.                \\ \rowsep

\indexlibrary{\idxcode{is_nothrow_assignable}}%
\tcode{template <class T, class U>}\br
  \tcode{struct is_nothrow_assignable;} &
  \tcode{is_assignable_v<T, U>} is \tcode{true} and the assignment is known not to
  throw any exceptions~(\ref{expr.unary.noexcept}). &
  \tcode{T} and \tcode{U} shall be complete types, (possibly cv-qualified) \tcode{void},
  or arrays of unknown bound. \\ \rowsep

\indexlibrary{\idxcode{is_nothrow_copy_assignable}}%
\tcode{template <class T>}\br
 \tcode{struct is_nothrow_copy_assignable;} &
  For a referenceable type \tcode{T}, the same result as
 \tcode{is_nothrow_assignable_v<T\&, const T\&>}, otherwise \tcode{false}. &
 \tcode{T} shall be a complete type,
 (possibly cv-qualified) \tcode{void}, or an array of unknown
 bound.                \\ \rowsep

\indexlibrary{\idxcode{is_nothrow_move_assignable}}%
\tcode{template <class T>}\br
  \tcode{struct is_nothrow_move_assignable;} &
  For a referenceable type \tcode{T}, the same result as
  \tcode{is_nothrow_assignable_v<T\&, T\&\&>}, otherwise \tcode{false}. &
 \tcode{T} shall be a complete type,
 (possibly cv-qualified) \tcode{void}, or an array of unknown
 bound.                \\ \rowsep

\indexlibrary{\idxcode{is_nothrow_swappable_with}}%
\tcode{template <class T, class U>}\br
  \tcode{struct is_nothrow_swappable_with;} &
  \tcode{is_swappable_with_v<T, U>} is \tcode{true} and
  each \tcode{swap} expression of the definition of
  \tcode{is_swappable_with<T, U>} is known not to throw
  any exceptions~(\ref{expr.unary.noexcept}). &
  \tcode{T} and \tcode{U} shall be complete types,
  (possibly cv-qualified) \tcode{void}, or
  arrays of unknown bound. \\ \rowsep

\indexlibrary{\idxcode{is_nothrow_swappable}}%
\tcode{template <class T>}\br
  \tcode{struct is_nothrow_swappable;} &
  For a referenceable type \tcode{T},
  the same result as \tcode{is_nothrow_swappable_with_v<T\&, T\&>},
  otherwise \tcode{false}. &
  \tcode{T} shall be a complete type,
  (possibly cv-qualified) \tcode{void}, or
  an array of unknown bound. \\ \rowsep

\indexlibrary{\idxcode{is_nothrow_destructible}}%
\tcode{template <class T>}\br
  \tcode{struct is_nothrow_destructible;} &
  \tcode{is_destructible_v<T>} is \tcode{true} and the indicated destructor is known
  not to throw any exceptions~(\ref{expr.unary.noexcept}). &
  \tcode{T} shall be a complete type,
  (possibly cv-qualified) \tcode{void}, or an array of unknown
  bound.                \\ \rowsep

\indexlibrary{\idxcode{has_virtual_destructor}}%
\tcode{template <class T>}\br
 \tcode{struct has_virtual_destructor;} &
 \tcode{T} has a virtual destructor~(\ref{class.dtor}) &
 If \tcode{T} is a non-union class type, \tcode{T} shall be a complete type.                \\ \rowsep

\indexlibrary{\idxcode{has_unique_object_representations}}%
\tcode{template <class T>}\br
  \tcode{struct has_unique_object_representations;} &
  For an array type \tcode{T}, the same result as
  \tcode{has_unique_object_representations_v<remove_all_extents_t<T>>},
  otherwise \seebelow. &
  \tcode{T} shall be a complete type, (possibly cv-qualified) \tcode{void}, or
  an array of unknown bound. \\ \rowsep

\end{libreqtab3b}

\pnum
\begin{example}
\begin{codeblock}
is_const_v<const volatile int>     // \tcode{true}
is_const_v<const int*>             // \tcode{false}
is_const_v<const int&>             // \tcode{false}
is_const_v<int[3]>                 // \tcode{false}
is_const_v<const int[3]>           // \tcode{true}
\end{codeblock}
\end{example}

\pnum
\begin{example}
\begin{codeblock}
remove_const_t<const volatile int>  // \tcode{volatile int}
remove_const_t<const int* const>    // \tcode{const int*}
remove_const_t<const int&>          // \tcode{const int\&}
remove_const_t<const int[3]>        // \tcode{int[3]}
\end{codeblock}
\end{example}

\pnum
\begin{example}
\begin{codeblock}
// Given:
struct P final { };
union U1 { };
union U2 final { };

// the following assertions hold:
static_assert(!is_final_v<int>, "Error!");
static_assert( is_final_v<P>,  "Error!");
static_assert(!is_final_v<U1>, "Error!");
static_assert( is_final_v<U2>, "Error!");
\end{codeblock}
\end{example}

\indexlibrary{\idxcode{is_constructible}}%
\pnum
The predicate condition for a template specialization
\tcode{is_constructible<T, Args...>} shall be satisfied if and only if the
following variable definition would be well-formed for some invented variable \tcode{t}:

\begin{codeblock}
T t(declval<Args>()...);
\end{codeblock}

\begin{note} These tokens are never interpreted as a function declaration.
\end{note} Access checking is performed as if in a context unrelated to \tcode{T}
and any of the \tcode{Args}. Only the validity of the immediate context of the
variable initialization is considered. \begin{note} The evaluation of the
initialization can result in side effects such as the instantiation of class
template specializations and function template specializations, the generation
of implicitly-defined functions, and so on. Such side effects are not in the
``immediate context'' and can result in the program being ill-formed. \end{note}

\indexlibrary{\idxcode{has_unique_object_representations}}%
\pnum
The predicate condition for a template specialization
\tcode{has_unique_object_representations<T>}
shall be satisfied if and only if:
\begin{itemize}
\item \tcode{T} is trivially copyable, and
\item any two objects of type \tcode{T} with the same value
have the same object representation, where
two objects of array or non-union class type are considered to have the same value
if their respective sequences of direct subobjects have the same values, and
two objects of union type are considered to have the same value
if they have the same active member and the corresponding members have the same value.
\end{itemize}
The set of scalar types for which this condition holds is
\impldef{which scalar types have unique object representations}.
\begin{note} If a type has padding bits, the condition does not hold;
otherwise, the condition holds true for unsigned integral types. \end{note}

\rSec2[meta.unary.prop.query]{Type property queries}

\pnum
This subclause contains templates that may be used to query
properties of types at compile time.

\begin{libreqtab2a}{Type property queries}{tab:type-traits.properties.queries}
\\ \topline
\lhdr{Template} &   \rhdr{Value}    \\ \capsep
\endfirsthead
\continuedcaption\\
\topline
\lhdr{Template} &   \rhdr{Value}    \\ \capsep
\endhead

\indexlibrary{\idxcode{alignment_of}}%
\tcode{template <class T>\br
 struct alignment_of;}      &
 \tcode{alignof(T)}.\br
 \requires{}
 \tcode{alignof(T)} shall be a valid expression~(\ref{expr.alignof})  \\  \rowsep

\indexlibrary{\idxcode{rank}}%
\tcode{template <class T>\br
 struct rank;}      &
 If \tcode{T} names an array type, an integer value representing
 the number of dimensions of \tcode{T}; otherwise, 0. \\    \rowsep

\indexlibrary{\idxcode{extent}}%
\tcode{template <class T,\br
 unsigned I = 0>\br
 struct extent;}        &
 If \tcode{T} is not an array type, or if it has rank less
 than or equal to \tcode{I}, or if \tcode{I} is 0 and \tcode{T}
 has type ``array of unknown bound of \tcode{U}'', then
 0; otherwise, the bound~(\ref{dcl.array}) of the \tcode{I}'th dimension of
\tcode{T}, where indexing of \tcode{I} is zero-based \\
\end{libreqtab2a}

\pnum
Each of these templates shall be a \tcode{UnaryTypeTrait}~(\ref{meta.rqmts}) with a
\tcode{BaseCharacteristic} of \tcode{integral_constant<size_t, Value>}.

\pnum
\begin{example}
\begin{codeblock}
// the following assertions hold:
assert(rank_v<int> == 0);
assert(rank_v<int[2]> == 1);
assert(rank_v<int[][4]> == 2);
\end{codeblock}
\end{example}

\pnum
\begin{example}
\begin{codeblock}
 // the following assertions hold:
assert(extent_v<int> == 0);
assert(extent_v<int[2]> == 2);
assert(extent_v<int[2][4]> == 2);
assert(extent_v<int[][4]> == 0);
assert((extent_v<int, 1>) == 0);
assert((extent_v<int[2], 1>) == 0);
assert((extent_v<int[2][4], 1>) == 4);
assert((extent_v<int[][4], 1>) == 4);
\end{codeblock}
\end{example}

\rSec2[meta.rel]{Relationships between types}

\pnum
This subclause contains templates that may be used to query
relationships between types at compile time.

\pnum
Each of these templates shall be a
BinaryTypeTrait~(\ref{meta.rqmts})
with a BaseCharacteristic of
\tcode{true_type} if the corresponding condition is true, otherwise
\tcode{false_type}.

\begin{libreqtab3f}{Type relationship predicates}{tab:type-traits.relationship}
\\ \topline
\lhdr{Template} &   \chdr{Condition}    &   \rhdr{Comments} \\ \capsep
\endfirsthead
\continuedcaption\\
\topline
\lhdr{Template} &   \chdr{Condition}    &   \rhdr{Comments} \\ \capsep
\endhead
\tcode{template <class T, class U>}\br
 \tcode{struct is_same;}                    &
 \tcode{T} and \tcode{U} name the same type with the same cv-qualifications                            &   \\ \rowsep

\indexlibrary{\idxcode{is_base_of}}%
\tcode{template <class Base, class Derived>}\br
 \tcode{struct is_base_of;}                 &
 \tcode{Base} is a base class of \tcode{Derived} (Clause~\ref{class.derived})
 without regard to cv-qualifiers
 or \tcode{Base} and \tcode{Derived} are not unions and
 name the same class type
 without regard to cv-qualifiers            &
 If \tcode{Base} and
 \tcode{Derived} are non-union class types and are
not possibly cv-qualified versions of the same type,
 \tcode{Derived} shall be a complete
 type.
 \begin{note} Base classes that are private, protected, or ambiguous
 are, nonetheless, base classes. \end{note} \\ \rowsep

\indexlibrary{\idxcode{is_convertible}}%
\tcode{template <class From, class To>}\br
 \tcode{struct is_convertible;}             &
 \seebelow                                  &
 \tcode{From} and \tcode{To} shall be complete
 types, arrays of unknown
 bound, or (possibly cv-qualified) \tcode{void} types.                \\ \rowsep

\indexlibrary{\idxcode{is_callable}}%
\tcode{template <class Fn, class... ArgTypes, class R>}\br
 \tcode{struct is_callable<}\br
 \tcode{Fn(ArgTypes...), R>;}                      &
 The expression \tcode{\textit{INVOKE}(declval<Fn>(), declval<ArgTypes>()..., R)}
 is well formed when treated as an unevaluated operand                &
 \tcode{Fn}, \tcode{R}, and all types in the parameter pack \tcode{ArgTypes}
 shall be complete types, (possibly cv-qualified) \tcode{void}, or
 arrays of unknown bound.                                             \\ \rowsep

\indexlibrary{\idxcode{is_nothrow_callable}}%
\tcode{template <class Fn, class... ArgTypes, class R>}\br
 \tcode{struct is_nothrow_callable<}\br
 \tcode{Fn(ArgTypes...), R>;}              &
 \tcode{is_callable_v<}\br\tcode{Fn(ArgTypes...), R>} is \tcode{true} and
 the expression \tcode{\textit{INVOKE}(declval<Fn>(), declval<ArgTypes>()..., R)}
 is known not to throw any exceptions                                 &
 \tcode{Fn}, \tcode{R}, and all types in the parameter pack \tcode{ArgTypes}
 shall be complete types, (possibly cv-qualified) \tcode{void}, or
 arrays of unknown bound.                                             \\
\end{libreqtab3f}

\pnum
For the purpose of defining the templates in this subclause,
a function call expression \tcode{declval<T>()} for any type \tcode{T}
is considered to be a trivial~(\ref{basic.types}, \ref{special}) function call
that is not an odr-use~(\ref{basic.def.odr}) of \tcode{declval}
in the context of the corresponding definition
notwithstanding the restrictions of~\ref{declval}.

\pnum
\begin{example}
\begin{codeblock}
struct B {};
struct B1 : B {};
struct B2 : B {};
struct D : private B1, private B2 {};

is_base_of_v<B, D>         // \tcode{true}
is_base_of_v<const B, D>   // \tcode{true}
is_base_of_v<B, const D>   // \tcode{true}
is_base_of_v<B, const B>   // \tcode{true}
is_base_of_v<D, B>         // \tcode{false}
is_base_of_v<B&, D&>       // \tcode{false}
is_base_of_v<B[3], D[3]>   // \tcode{false}
is_base_of_v<int, int>     // \tcode{false}
\end{codeblock}
\end{example}

\indexlibrary{\idxcode{is_convertible}}%
\pnum
The predicate condition for a template specialization \tcode{is_convertible<From, To>}
shall be satisfied if and only if the return expression in the following code would be
well-formed, including any implicit conversions to the return type of the function:

\begin{codeblock}
To test() {
  return declval<From>();
}
\end{codeblock}

\begin{note} This requirement gives well defined results for reference types, void
types, array types, and function types.\end{note} Access checking is performed as
if in a context unrelated to \tcode{To} and \tcode{From}. Only the validity of
the immediate context of the expression of the \grammarterm{return-statement}
(including conversions to the return type) is considered. \begin{note} The
evaluation of the conversion can result in side effects such as the
instantiation of class template specializations and function template
specializations, the generation of implicitly-defined functions, and so on. Such
side effects are not in the ``immediate context'' and can result in the program
being ill-formed. \end{note}

\rSec2[meta.trans]{Transformations between types}
\pnum
This subclause contains templates that may be used to transform one
type to another following some predefined rule.

\pnum
Each of the templates in this subclause shall be a
\term{TransformationTrait}~(\ref{meta.rqmts}).

\rSec3[meta.trans.cv]{Const-volatile modifications}

\begin{libreqtab2a}{Const-volatile modifications}{tab:type-traits.const-volatile}
\\ \topline
\lhdr{Template} &   \rhdr{Comments} \\ \capsep
\endfirsthead
\continuedcaption\\
\topline
\lhdr{Template} &   \rhdr{Comments} \\ \capsep
\endhead

\indexlibrary{\idxcode{remove_const}}%
\tcode{template <class T>\br
 struct remove_const;}                  &
 The member typedef \tcode{type} shall name
 the same type as \tcode{T}
 except that any top-level const-qualifier has been removed.
 \begin{example} \tcode{remove_const_t<const volatile int>} evaluates
 to \tcode{volatile int}, whereas \tcode{remove_const_t<const int*>} evaluates to
 \tcode{const int*}. \end{example}                          \\  \rowsep

\indexlibrary{\idxcode{remove_volatile}}%
\tcode{template <class T>\br
 struct remove_volatile;}               &
 The member typedef \tcode{type} shall name
 the same type as \tcode{T}
 except that any top-level volatile-qualifier has been removed.
 \begin{example} \tcode{remove_volatile_t<const volatile int>}
 evaluates to \tcode{const int},
 whereas \tcode{remove_volatile_t<volatile int*>} evaluates to \tcode{volatile int*}.
 \end{example}                                              \\  \rowsep

\indexlibrary{\idxcode{remove_cv}}%
\tcode{template <class T>\br
 struct remove_cv;}                 &
 The member typedef \tcode{type} shall be the same as \tcode{T}
 except that any top-level cv-qualifier has been removed.
 \begin{example} \tcode{remove_cv_t<const volatile int>}
 evaluates to \tcode{int}, whereas \tcode{remove_cv_t<const volatile int*>}
 evaluates to \tcode{const volatile int*}. \end{example}  \\  \rowsep

\indexlibrary{\idxcode{add_const}}%
\tcode{template <class T>\br
 struct add_const;}                 &
 If \tcode{T} is a reference, function, or top-level const-qualified
 type, then \tcode{type} shall name
 the same type as \tcode{T}, otherwise
 \tcode{T const}.                                                           \\  \rowsep

\indexlibrary{\idxcode{add_volatile}}%
\tcode{template <class T>\br
 struct add_volatile;}                  &
 If \tcode{T} is a reference, function, or top-level volatile-qualified
 type, then \tcode{type} shall name
 the same type as \tcode{T}, otherwise
 \tcode{T volatile}.                                                            \\  \rowsep

\indexlibrary{\idxcode{add_cv}}%
\tcode{template <class T>\br
 struct add_cv;}                    &
 The member typedef \tcode{type} shall name
 the same type as
 \tcode{add_const_t<add_volatile_t<T>{>}}.                               \\
\end{libreqtab2a}

\rSec3[meta.trans.ref]{Reference modifications}

\begin{libreqtab2a}{Reference modifications}{tab:type-traits.reference}
\\ \topline
\lhdr{Template} &   \rhdr{Comments} \\ \capsep
\endfirsthead
\continuedcaption\\
\topline
\lhdr{Template} &   \rhdr{Comments} \\ \capsep
\endhead

\indexlibrary{\idxcode{remove_reference}}%
\tcode{template <class T>\br
 struct remove_reference;}                  &
 If \tcode{T} has type ``reference to \tcode{T1}'' then the
 member typedef \tcode{type} shall name \tcode{T1};
 otherwise, \tcode{type} shall name \tcode{T}.\\ \rowsep

\indexlibrary{\idxcode{add_lvalue_reference}}%
\tcode{template <class T>\br
 struct add_lvalue_reference;}                     &
 If \tcode{T} names a referenceable type (\ref{defns.referenceable}) then
 the member typedef \tcode{type} shall name \tcode{T\&};
 otherwise, \tcode{type} shall name \tcode{T}.
 \begin{note}
 This rule reflects the semantics of reference collapsing~(\ref{dcl.ref}).
 \end{note}\\ \rowsep

\indexlibrary{\idxcode{add_rvalue_reference}}%
\tcode{template <class T>}\br
 \tcode{struct add_rvalue_reference;}    &
 If \tcode{T} names a referenceable type then
 the member typedef \tcode{type} shall name \tcode{T\&\&};
 otherwise, \tcode{type} shall name \tcode{T}.
 \begin{note} This rule reflects the semantics of reference collapsing~(\ref{dcl.ref}).
 For example, when a type \tcode{T} names a type \tcode{T1\&}, the type
 \tcode{add_rvalue_reference_t<T>} is not an rvalue reference.
 \end{note} \\
\end{libreqtab2a}

\rSec3[meta.trans.sign]{Sign modifications}
\begin{libreqtab2a}{Sign modifications}{tab:type-traits.sign}
\\ \topline
\lhdr{Template} &   \rhdr{Comments} \\ \capsep
\endfirsthead
\continuedcaption\\
\topline
\lhdr{Template} &   \rhdr{Comments} \\ \capsep
\endhead

\indexlibrary{\idxcode{make_signed}}%
\tcode{template <class T>}\br
 \tcode{struct make_signed;} &
 If \tcode{T} names a (possibly cv-qualified) signed integer
 type~(\ref{basic.fundamental}) then the member typedef
 \tcode{type} shall name the type \tcode{T}; otherwise,
 if \tcode{T} names a (possibly cv-qualified) unsigned integer
 type then \tcode{type} shall name the corresponding
 signed integer type, with the same cv-qualifiers as \tcode{T};
 otherwise, \tcode{type} shall name the signed integer type with smallest
 rank~(\ref{conv.rank}) for which
 \tcode{sizeof(T) == sizeof(type)}, with the same
 cv-qualifiers as \tcode{T}.\br
 \requires{} \tcode{T} shall be a (possibly cv-qualified)
 integral type or enumeration
 but not a \tcode{bool} type.\\ \rowsep

\indexlibrary{\idxcode{make_unsigned}}%
\tcode{template <class T>}\br
 \tcode{struct make_unsigned;} &
 If \tcode{T} names a (possibly cv-qualified) unsigned integer
 type~(\ref{basic.fundamental}) then the member typedef
 \tcode{type} shall name the type \tcode{T}; otherwise,
 if \tcode{T} names a (possibly cv-qualified) signed integer
 type then \tcode{type} shall name the corresponding
 unsigned integer type, with the same cv-qualifiers as \tcode{T};
 otherwise, \tcode{type} shall name the unsigned integer type with smallest
 rank~(\ref{conv.rank}) for which
 \tcode{sizeof(T) == sizeof(type)}, with the same
 cv-qualifiers as \tcode{T}.\br
 \requires{} \tcode{T} shall be a (possibly cv-qualified)
 integral type or enumeration
 but not a \tcode{bool} type.\\
\end{libreqtab2a}
\clearpage

\rSec3[meta.trans.arr]{Array modifications}
\begin{libreqtab2a}{Array modifications}{tab:type-traits.array}
\\ \topline
\lhdr{Template} &   \rhdr{Comments} \\ \capsep
\endfirsthead
\continuedcaption\\
\topline
\lhdr{Template} &   \rhdr{Comments} \\ \capsep
\endhead

\indexlibrary{\idxcode{remove_extent}}%
\tcode{template <class T>\br
 struct remove_extent;}                 &
 If \tcode{T} names a type ``array of \tcode{U}'',
 the member typedef \tcode{type} shall
 be \tcode{U}, otherwise \tcode{T}.
 \begin{note} For multidimensional arrays, only the first array dimension is
 removed. For a type ``array of \tcode{const U}'', the resulting type is
 \tcode{const U}. \end{note}                                 \\  \rowsep

\indexlibrary{\idxcode{remove_all_extents}}%
\tcode{template <class T>\br
 struct remove_all_extents;}                &
 If \tcode{T} is ``multi-dimensional array of \tcode{U}'', the resulting member
 typedef \tcode{type} is \tcode{U}, otherwise \tcode{T}.                                       \\
\end{libreqtab2a}

\pnum
\begin{example}
\begin{codeblock}
// the following assertions hold:
assert((is_same_v<remove_extent_t<int>, int>));
assert((is_same_v<remove_extent_t<int[2]>, int>));
assert((is_same_v<remove_extent_t<int[2][3]>, int[3]>));
assert((is_same_v<remove_extent_t<int[][3]>, int[3]>));
\end{codeblock}
\end{example}

\pnum
\begin{example}
\begin{codeblock}
// the following assertions hold:
assert((is_same_v<remove_all_extents_t<int>, int>));
assert((is_same_v<remove_all_extents_t<int[2]>, int>));
assert((is_same_v<remove_all_extents_t<int[2][3]>, int>));
assert((is_same_v<remove_all_extents_t<int[][3]>, int>));
\end{codeblock}
\end{example}

\rSec3[meta.trans.ptr]{Pointer modifications}
\begin{libreqtab2a}{Pointer modifications}{tab:type-traits.pointer}
\\ \topline
\lhdr{Template} &   \rhdr{Comments} \\ \capsep
\endfirsthead
\continuedcaption\\
\topline
\lhdr{Template} &   \rhdr{Comments} \\ \capsep
\endhead

\indexlibrary{\idxcode{remove_pointer}}%
\tcode{template <class T>\br
 struct remove_pointer;}                    &
 If \tcode{T} has type ``(possibly cv-qualified) pointer
 to \tcode{T1}'' then the member typedef \tcode{type}
 shall name \tcode{T1}; otherwise, it shall name \tcode{T}.\\ \rowsep

\indexlibrary{\idxcode{add_pointer}}%
\tcode{template <class T>\br
 struct add_pointer;}                       &
 If \tcode{T} names a referenceable type (\ref{defns.referenceable}) or a
 (possibly cv-qualified) \tcode{void} type then
 the member typedef \tcode{type} shall name the same type as
 \tcode{remove_reference_t<T>*};
 otherwise, \tcode{type} shall name \tcode{T}.             \\
\end{libreqtab2a}
\clearpage

\rSec3[meta.trans.other]{Other transformations}

\begin{libreqtab2a}{Other transformations}{tab:type-traits.other}
\\ \topline
\lhdr{Template}   &   \rhdr{Comments} \\ \capsep
\endfirsthead
\continuedcaption\\
\topline
\lhdr{Template}   &   \rhdr{Comments} \\ \capsep
\endhead

\indexlibrary{\idxcode{aligned_storage}}%
\tcode{template <size_t Len,\br
 size_t Align\br
 = \textit{default-alignment}>\br
 struct aligned_storage;}
 &
 The value of \textit{default-alignment} shall be the most
 stringent alignment requirement for any \Cpp object type whose size
 is no greater than \tcode{Len}~(\ref{basic.types}).
 The member typedef \tcode{type} shall be a POD type
 suitable for use as uninitialized storage for any object whose size
 is at most \textit{Len} and whose alignment is a divisor of \textit{Align}.\br
 \requires{} \tcode{Len} shall not be zero. \tcode{Align} shall be equal to
 \tcode{alignof(T)} for some type \tcode{T} or to \textit{default-alignment}.\\ \rowsep

\indexlibrary{\idxcode{aligned_union}}%
\tcode{template <size_t Len,\br
  class... Types>\br
  struct aligned_union;}
  &
  The member typedef \tcode{type} shall be a POD type suitable for use as
  uninitialized storage for any object whose type is listed in \tcode{Types};
  its size shall be at least \tcode{Len}. The static member \tcode{alignment_value}
  shall be an integral constant of type \tcode{size_t} whose value is the
  strictest alignment of all types listed in \tcode{Types}.\br
 \requires{} At least one type is provided.
  \\ \rowsep

\indexlibrary{\idxcode{decay}}%
\tcode{template <class T>\br struct decay;}
 &
 Let \tcode{U} be \tcode{remove_reference_t<T>}. If \tcode{is_array_v<U>} is
 \tcode{true}, the member typedef \tcode{type} shall equal
 \tcode{remove_extent_t<U>*}. If \tcode{is_function_v<U>} is \tcode{true},
 the member typedef \tcode{type} shall equal \tcode{add_pointer_t<U>}. Otherwise
 the member typedef \tcode{type} equals \tcode{remove_cv_t<U>}.
 \begin{note} This behavior is similar to the lvalue-to-rvalue~(\ref{conv.lval}),
 array-to-pointer~(\ref{conv.array}), and function-to-pointer~(\ref{conv.func})
 conversions applied when an lvalue expression is used as an rvalue, but also
 strips \cv-qualifiers from class types in order to more closely model by-value
 argument passing. \end{note}
 \\ \rowsep

\indexlibrary{\idxcode{enable_if}}%
\tcode{template <bool B, class T = void>} \tcode{struct enable_if;}
 &
 If \tcode{B} is \tcode{true}, the member typedef \tcode{type}
 shall equal \tcode{T}; otherwise, there shall be no member
 \tcode{type}. \\ \rowsep

\tcode{template <bool B, class T,}
 \tcode{class F>}\br
 \tcode{struct conditional;}
 &
 If \tcode{B} is \tcode{true},  the member typedef \tcode{type} shall equal \tcode{T}.
 If \tcode{B} is \tcode{false}, the member typedef \tcode{type} shall equal \tcode{F}. \\ \rowsep

 \tcode{template <class... T>} \tcode{struct common_type;}
 &
 The member typedef \tcode{type} shall be defined or omitted as specified below.
 If it is omitted, there shall be no member \tcode{type}. All types in
 the parameter pack \tcode{T} shall be complete or (possibly \cv) \tcode{void}.
 A program may specialize this trait if at least one template parameter in the
 specialization is a user-defined type. \begin{note} Such specializations are
 needed when only explicit conversions are desired among the template arguments.
 \end{note} \\ \rowsep

\indexlibrary{\idxcode{underlying_type}}%
\tcode{template <class T>}\br
 \tcode{struct underlying_type;}
 &
 The member typedef \tcode{type} shall name the underlying type
 of \tcode{T}.\br
 \requires{} \tcode{T} shall be a complete enumeration type~(\ref{dcl.enum}) \\ \rowsep

\tcode{template <class Fn,}\br
 \tcode{class... ArgTypes>}\br
 \tcode{struct result_of<}\br
 \tcode{Fn(ArgTypes...)>;}
 &
 If the expression \tcode{\textit{INVOKE}(declval<Fn>(), declval<ArgTypes>()...)}
 is well formed when treated as an unevaluated operand (Clause~\ref{expr}),
 the member typedef \tcode{type} shall name the type
 \tcode{decltype(\textit{INVOKE}(declval<Fn>(), declval<ArgTypes>()...))};
 otherwise, there shall be no member \tcode{type}. Access checking is
 performed as if in a context unrelated to \tcode{Fn} and
 \tcode{ArgTypes}. Only the validity of the immediate context of the
 expression is considered.
 \begin{note}
 The compilation of the expression can result in side effects such as
 the instantiation of class template specializations and function
 template specializations, the generation of implicitly-defined
 functions, and so on. Such side effects are not in the ``immediate
 context'' and can result in the program being ill-formed.
 \end{note} \br
 \requires{} \tcode{Fn} and all types in the parameter pack \tcode{ArgTypes} shall
 be complete types, (possibly cv-qualified) \tcode{void}, or arrays of
 unknown bound.\\
\end{libreqtab2a}

\indexlibrary{\idxcode{aligned_storage}}%
\pnum
\begin{note} A typical implementation would define \tcode{aligned_storage} as:

\begin{codeblock}
template <size_t Len, size_t Alignment>
struct aligned_storage {
  typedef struct {
    alignas(Alignment) unsigned char __data[Len];
  } type;
};
\end{codeblock}
\end{note}

\pnum
It is \impldef{support for extended alignment} whether any extended alignment is
supported~(\ref{basic.align}).

\indexlibrary{\idxcode{common_type}}%
\pnum
For the \tcode{common_type} trait applied to a parameter pack \tcode{T} of types,
the member \tcode{type} shall be either defined or not present as follows:

\begin{itemize}
\item If \tcode{sizeof...(T)} is zero, there shall be no member \tcode{type}.

\item If \tcode{sizeof...(T)} is one, let \tcode{T0} denote the sole type
in the pack \tcode{T}. The member typedef \tcode{type} shall denote the same
type as \tcode{decay_t<T0>}.

\item If \tcode{sizeof...(T)} is greater than one,
let \tcode{T1}, \tcode{T2}, and \tcode{R}, respectively,
denote the first, second, and (pack of) remaining types comprising \tcode{T}.
\begin{note} \tcode{sizeof...(R)} may be zero. \end{note}
Let \tcode{C} denote the type, if any,
of an unevaluated conditional expression~(\ref{expr.cond})
whose first operand is an arbitrary value of type \tcode{bool},
whose second operand is an xvalue of type \tcode{T1}, and
whose third operand is an xvalue of type \tcode{T2}.
If there is such a type \tcode{C}, the member typedef \tcode{type}
shall denote the same type, if any, as \tcode{common_type_t<C, R...>}.
Otherwise, there shall be no member \tcode{type}.
\end{itemize}

\pnum
\begin{example}
Given these definitions:
\begin{codeblock}
using PF1 = bool  (&)();
using PF2 = short (*)(long);

struct S {
  operator PF2() const;
  double operator()(char, int&);
  void fn(long) const;
  char data;
};

using PMF = void (S::*)(long) const;
using PMD = char  S::*;
\end{codeblock}

the following assertions will hold:

\begin{codeblock}
static_assert(is_same_v<result_of_t<S(int)>, short>, "Error!");
static_assert(is_same_v<result_of_t<S&(unsigned char, int&)>, double>, "Error!");
static_assert(is_same_v<result_of_t<PF1()>, bool>, "Error!");
static_assert(is_same_v<result_of_t<PMF(unique_ptr<S>, int)>, void>, "Error!");
static_assert(is_same_v<result_of_t<PMD(S)>, char&&>, "Error!");
static_assert(is_same_v<result_of_t<PMD(const S*)>, const char&>, "Error!");
\end{codeblock}
\end{example}

\rSec2[meta.logical]{Logical operator traits}

\pnum
This subclause describes type traits for applying logical operators
to other type traits.

\indexlibrary{\idxcode{conjunction}}%
\begin{itemdecl}
template<class... B> struct conjunction : @\seebelow@ { };
\end{itemdecl}

\begin{itemdescr}
\pnum
The class template \tcode{conjunction}
forms the logical conjunction of its template type arguments.
Every template type argument
shall be usable as a base class and
shall have a member \tcode{value} which
is convertible to \tcode{bool},
is not hidden, and
is unambiguously available in the type.

\pnum
The BaseCharacteristic of a specialization \tcode{conjunction<B1, ..., BN>}
is the first type \tcode{Bi} in the list \tcode{true_type, B1, ..., BN}
for which \tcode{Bi::value == false}, or
if every \tcode{Bi::value != false},
the BaseCharacteristic is the last type in the list.
\begin{note} This means a specialization of \tcode{conjunction}
does not necessarily have a BaseCharacteristic
of either \tcode{true_type} or \tcode{false_type}.
\end{note}

\pnum
For a specialization \tcode{conjunction<B1, ..., BN>},
if there is a template type argument \tcode{Bi} with \tcode{Bi::value == false},
then instantiating \tcode{conjunction<B1, ..., BN>::value}
does not require the instantiation of \tcode{Bj::value} for \tcode{j > i}.
\begin{note} This is analogous to the short-circuiting behavior of \tcode{\&\&}.
\end{note}
\end{itemdescr}

\indexlibrary{\idxcode{disjunction}}%
\begin{itemdecl}
template<class... B> struct disjunction : @\seebelow@ { };
\end{itemdecl}

\begin{itemdescr}
\pnum
The class template \tcode{disjunction}
forms the logical disjunction of its template type arguments.
Every template type argument shall be usable as a base class and
shall have a member \tcode{value} which
is convertible to \tcode{bool},
is not hidden, and
is unambiguously available in the type.

\pnum
The BaseCharacteristic of a specialization \tcode{disjunction<B1, ..., BN>}
is the first type \tcode{Bi} in the list \tcode{false_type, B1, ..., BN}
for which \tcode{Bi::value != false}, or
if every \tcode{Bi::value == false},
the BaseCharacteristic is the last type in the list.
\begin{note} This means a specialization of \tcode{disjunction}
does not necessarily have a BaseCharacteristic
of either \tcode{true_type} or \tcode{false_type}.
\end{note}

\pnum
For a specialization \tcode{disjunction<B1, ..., BN>},
if there is a template type argument \tcode{Bi} with \tcode{Bi::value != false},
then instantiating \tcode{disjunction<B1, ..., BN>::value}
does not require the instantiation of \tcode{Bj::value} for \tcode{j > i}.
\begin{note} This is analogous to the short-circuiting behavior of \tcode{||}.
\end{note}
\end{itemdescr}

\indexlibrary{\idxcode{negation}}%
\begin{itemdecl}
template<class B> struct negation : bool_constant<!B::value> { };
\end{itemdecl}

\begin{itemdescr}
\pnum
The class template \tcode{negation}
forms the logical negation of its template type argument.
The type \tcode{negation<B>}
is a UnaryTypeTrait with a BaseCharacteristic of \tcode{bool_constant<!B::value>}.
\end{itemdescr}

\rSec1[ratio]{Compile-time rational arithmetic}

\rSec2[ratio.general]{In general}

\pnum
\indexlibrary{\idxcode{ratio}}%
This subclause describes the ratio library. It provides a class template
\tcode{ratio} which exactly represents any finite rational number with a
numerator and denominator representable by compile-time constants of type
\tcode{intmax_t}.

\pnum
Throughout this subclause, the names of template parameters are used to express
type requirements. If a template parameter is named \tcode{R1} or \tcode{R2},
and the template argument is not a specialization of the \tcode{ratio} template,
the program is ill-formed.

\indexlibrary{\idxhdr{ratio}}%
\rSec2[ratio.syn]{Header \tcode{<ratio>} synopsis}

\begin{codeblockdigitsep}
namespace std {
  // \ref{ratio.ratio}, class template \tcode{ratio}
  template <intmax_t N, intmax_t D = 1> class ratio;

  // \ref{ratio.arithmetic}, ratio arithmetic
  template <class R1, class R2> using ratio_add = @\seebelow@;
  template <class R1, class R2> using ratio_subtract = @\seebelow@;
  template <class R1, class R2> using ratio_multiply = @\seebelow@;
  template <class R1, class R2> using ratio_divide = @\seebelow@;

  // \ref{ratio.comparison}, ratio comparison
  template <class R1, class R2> struct ratio_equal;
  template <class R1, class R2> struct ratio_not_equal;
  template <class R1, class R2> struct ratio_less;
  template <class R1, class R2> struct ratio_less_equal;
  template <class R1, class R2> struct ratio_greater;
  template <class R1, class R2> struct ratio_greater_equal;

  // \ref{ratio.si}, convenience SI typedefs
  using yocto = ratio<1, 1'000'000'000'000'000'000'000'000>;  // see below
  using zepto = ratio<1,     1'000'000'000'000'000'000'000>;  // see below
  using atto  = ratio<1,         1'000'000'000'000'000'000>;
  using femto = ratio<1,             1'000'000'000'000'000>;
  using pico  = ratio<1,                 1'000'000'000'000>;
  using nano  = ratio<1,                     1'000'000'000>;
  using micro = ratio<1,                         1'000'000>;
  using milli = ratio<1,                             1'000>;
  using centi = ratio<1,                               100>;
  using deci  = ratio<1,                                10>;
  using deca  = ratio<                               10, 1>;
  using hecto = ratio<                              100, 1>;
  using kilo  = ratio<                            1'000, 1>;
  using mega  = ratio<                        1'000'000, 1>;
  using giga  = ratio<                    1'000'000'000, 1>;
  using tera  = ratio<                1'000'000'000'000, 1>;
  using peta  = ratio<            1'000'000'000'000'000, 1>;
  using exa   = ratio<        1'000'000'000'000'000'000, 1>;
  using zetta = ratio<    1'000'000'000'000'000'000'000, 1>;  // see below
  using yotta = ratio<1'000'000'000'000'000'000'000'000, 1>;  // see below

  // \ref{ratio.arithmetic}, ratio comparison
  template <class R1, class R2> constexpr bool ratio_equal_v
    = ratio_equal<R1, R2>::value;
  template <class R1, class R2> constexpr bool ratio_not_equal_v
    = ratio_not_equal<R1, R2>::value;
  template <class R1, class R2> constexpr bool ratio_less_v
    = ratio_less<R1, R2>::value;
  template <class R1, class R2> constexpr bool ratio_less_equal_v
    = ratio_less_equal<R1, R2>::value;
  template <class R1, class R2> constexpr bool ratio_greater_v
    = ratio_greater<R1, R2>::value;
  template <class R1, class R2> constexpr bool ratio_greater_equal_v
    = ratio_greater_equal<R1, R2>::value;
}
\end{codeblockdigitsep}

\rSec2[ratio.ratio]{Class template \tcode{ratio}}

\indexlibrary{\idxcode{ratio}}%
\begin{codeblock}
namespace std {
  template <intmax_t N, intmax_t D = 1>
  class ratio {
  public:
    static constexpr intmax_t num;
    static constexpr intmax_t den;
    using type = ratio<num, den>;
  };
}
\end{codeblock}

\pnum
\indextext{signed integer representation!two's complement}%
If the template argument \tcode{D} is zero or the absolute values of either of the
template arguments \tcode{N} and \tcode{D} is not representable by type
\tcode{intmax_t}, the program is ill-formed. \begin{note} These rules ensure that infinite
ratios are avoided and that for any negative input, there exists a representable value
of its absolute value which is positive. In a two's complement representation, this
excludes the most negative value. \end{note}

\pnum
The static data members \tcode{num} and \tcode{den} shall have the following values,
where \tcode{gcd} represents the greatest common divisor of the absolute values of
\tcode{N} and \tcode{D}:

\begin{itemize}
\item \tcode{num} shall have the value \tcode{sign(N) * sign(D) * abs(N) / gcd}.
\item \tcode{den} shall have the value \tcode{abs(D) / gcd}.
\end{itemize}

\rSec2[ratio.arithmetic]{Arithmetic on \tcode{ratio}{s}}

\pnum
Each of the alias templates \tcode{ratio_add}, \tcode{ratio_subtract}, \tcode{ratio_multiply},
and \tcode{ratio_divide} denotes the result of an arithmetic computation on two
\tcode{ratio}{s} \tcode{R1} and \tcode{R2}. With \tcode{X} and \tcode{Y} computed (in the
absence of arithmetic overflow) as specified by Table~\ref{tab:ratio.arithmetic}, each alias
denotes a \tcode{ratio<U, V>} such that \tcode{U} is the same as \tcode{ratio<X, Y>::num} and
\tcode{V} is the same as \tcode{ratio<X, Y>::den}.

\pnum
If it is not possible to represent \tcode{U} or \tcode{V} with \tcode{intmax_t}, the program is
ill-formed. Otherwise, an implementation should yield correct values of \tcode{U} and
\tcode{V}. If it is not possible to represent \tcode{X} or \tcode{Y} with \tcode{intmax_t}, the
program is ill-formed unless the implementation yields correct values of \tcode{U} and
\tcode{V}.

\begin{floattable}{Expressions used to perform ratio arithmetic}{tab:ratio.arithmetic}
{lll}
\topline
\lhdr{Type}                     &
  \chdr{Value of \tcode{X}}     &
  \rhdr{Value of \tcode{Y}}     \\ \rowsep

\tcode{ratio_add<R1, R2>}       &
  \tcode{R1::num * R2::den +}   &
  \tcode{R1::den * R2::den}     \\
                                &
  \tcode{R2::num * R1::den}     &
                                \\ \rowsep

\tcode{ratio_subtract<R1, R2>}  &
  \tcode{R1::num * R2::den -}   &
  \tcode{R1::den * R2::den}     \\
                                &
  \tcode{R2::num * R1::den}     &
                                \\ \rowsep

\tcode{ratio_multiply<R1, R2>}  &
  \tcode{R1::num * R2::num}     &
  \tcode{R1::den * R2::den}     \\ \rowsep

\tcode{ratio_divide<R1, R2>}    &
  \tcode{R1::num * R2::den}     &
  \tcode{R1::den * R2::num}     \\
\end{floattable}

\pnum
\begin{example}

\begin{codeblock}
static_assert(ratio_add<ratio<1, 3>, ratio<1, 6>>::num == 1, "1/3+1/6 == 1/2");
static_assert(ratio_add<ratio<1, 3>, ratio<1, 6>>::den == 2, "1/3+1/6 == 1/2");
static_assert(ratio_multiply<ratio<1, 3>, ratio<3, 2>>::num == 1, "1/3*3/2 == 1/2");
static_assert(ratio_multiply<ratio<1, 3>, ratio<3, 2>>::den == 2, "1/3*3/2 == 1/2");

// The following cases may cause the program to be ill-formed under some implementations
static_assert(ratio_add<ratio<1, INT_MAX>, ratio<1, INT_MAX>>::num == 2,
  "1/MAX+1/MAX == 2/MAX");
static_assert(ratio_add<ratio<1, INT_MAX>, ratio<1, INT_MAX>>::den == INT_MAX,
  "1/MAX+1/MAX == 2/MAX");
static_assert(ratio_multiply<ratio<1, INT_MAX>, ratio<INT_MAX, 2>>::num == 1,
  "1/MAX * MAX/2 == 1/2");
static_assert(ratio_multiply<ratio<1, INT_MAX>, ratio<INT_MAX, 2>>::den == 2,
  "1/MAX * MAX/2 == 1/2");
\end{codeblock}

\end{example}

\rSec2[ratio.comparison]{Comparison of \tcode{ratio}{s}}

\indexlibrary{\idxcode{ratio_equal}}%
\begin{itemdecl}
template <class R1, class R2> struct ratio_equal
  : bool_constant<R1::num == R2::num && R1::den == R2::den> { };
\end{itemdecl}

\indexlibrary{\idxcode{ratio_not_equal}}%
\begin{itemdecl}
template <class R1, class R2> struct ratio_not_equal
  : bool_constant<!ratio_equal_v<R1, R2>> { };
\end{itemdecl}

\indexlibrary{\idxcode{ratio_less}}%
\begin{itemdecl}
template <class R1, class R2> struct ratio_less
  : bool_constant<@\seebelow@> { };
\end{itemdecl}

\begin{itemdescr}
\pnum
If \tcode{R1::num} $\times$ \tcode{R2::den} is less than \tcode{R2::num} $\times$ \tcode{R1::den},
\tcode{ratio_less<R1, R2>} shall be
derived from \tcode{bool_constant<true>}; otherwise it shall be derived from
\tcode{bool_constant<false>}. Implementations may use other algorithms to
compute this relationship to avoid overflow. If overflow occurs, the program is ill-formed.
\end{itemdescr}

\indexlibrary{\idxcode{ratio_less_equal}}%
\begin{itemdecl}
template <class R1, class R2> struct ratio_less_equal
  : bool_constant<!ratio_less_v<R2, R1>> { };
\end{itemdecl}

\indexlibrary{\idxcode{ratio_greater}}%
\begin{itemdecl}
template <class R1, class R2> struct ratio_greater
  : bool_constant<ratio_less_v<R2, R1>> { };
\end{itemdecl}

\indexlibrary{\idxcode{ratio_greater_equal}}%
\begin{itemdecl}
template <class R1, class R2> struct ratio_greater_equal
  : bool_constant<!ratio_less_v<R1, R2>> { };
\end{itemdecl}

\rSec2[ratio.si]{SI types for \tcode{ratio}}

\pnum
For each of the \grammarterm{typedef-name}{s} \tcode{yocto}, \tcode{zepto},
\tcode{zetta}, and \tcode{yotta}, if both of the constants used in its
specification are representable by \tcode{intmax_t}, the typedef shall be
defined; if either of the constants is not representable by \tcode{intmax_t},
the typedef shall not be defined.

\rSec1[time]{Time utilities}

\rSec2[time.general]{In general}

\pnum
\indexlibrary{\idxcode{chrono}}%
This subclause describes the chrono library~(\ref{time.syn}) and various C
functions~(\ref{ctime.syn}) that provide generally useful time
utilities.

\indexlibrary{\idxhdr{chrono}}%
\rSec2[time.syn]{Header \tcode{<chrono>} synopsis}

\begin{codeblock}
namespace std {
  namespace chrono {
    // \ref{time.duration}, class template \tcode{duration}
    template <class Rep, class Period = ratio<1>> class duration;

    // \ref{time.point}, class template \tcode{time_point}
    template <class Clock, class Duration = typename Clock::duration> class time_point;
  }

  // \ref{time.traits.specializations} \tcode{common_type} specializations
  template <class Rep1, class Period1, class Rep2, class Period2>
    struct common_type<chrono::duration<Rep1, Period1>,
                       chrono::duration<Rep2, Period2>>;

  template <class Clock, class Duration1, class Duration2>
    struct common_type<chrono::time_point<Clock, Duration1>,
                       chrono::time_point<Clock, Duration2>>;

  namespace chrono {
    // \ref{time.traits}, customization traits
    template <class Rep> struct treat_as_floating_point;
    template <class Rep> struct duration_values;
    template <class Rep> constexpr bool treat_as_floating_point_v
      = treat_as_floating_point<Rep>::value;

    // \ref{time.duration.nonmember}, duration arithmetic
    template <class Rep1, class Period1, class Rep2, class Period2>
      common_type_t<duration<Rep1, Period1>, duration<Rep2, Period2>>
      constexpr operator+(const duration<Rep1, Period1>& lhs,
                          const duration<Rep2, Period2>& rhs);
    template <class Rep1, class Period1, class Rep2, class Period2>
      common_type_t<duration<Rep1, Period1>, duration<Rep2, Period2>>
      constexpr operator-(const duration<Rep1, Period1>& lhs,
                          const duration<Rep2, Period2>& rhs);
    template <class Rep1, class Period, class Rep2>
      duration<common_type_t<Rep1, Rep2>, Period>
      constexpr operator*(const duration<Rep1, Period>& d, const Rep2& s);
    template <class Rep1, class Rep2, class Period>
      duration<common_type_t<Rep1, Rep2>, Period>
      constexpr operator*(const Rep1& s, const duration<Rep2, Period>& d);
    template <class Rep1, class Period, class Rep2>
      duration<common_type_t<Rep1, Rep2>, Period>
      constexpr operator/(const duration<Rep1, Period>& d, const Rep2& s);
    template <class Rep1, class Period1, class Rep2, class Period2>
      common_type_t<Rep1, Rep2>
      constexpr operator/(const duration<Rep1, Period1>& lhs,
                          const duration<Rep2, Period2>& rhs);
    template <class Rep1, class Period, class Rep2>
      duration<common_type_t<Rep1, Rep2>, Period>
      constexpr operator%(const duration<Rep1, Period>& d, const Rep2& s);
    template <class Rep1, class Period1, class Rep2, class Period2>
      common_type_t<duration<Rep1, Period1>, duration<Rep2, Period2>>
      constexpr operator%(const duration<Rep1, Period1>& lhs,
                          const duration<Rep2, Period2>& rhs);

    // \ref{time.duration.comparisons}, duration comparisons
    template <class Rep1, class Period1, class Rep2, class Period2>
      constexpr bool operator==(const duration<Rep1, Period1>& lhs,
                                const duration<Rep2, Period2>& rhs);
    template <class Rep1, class Period1, class Rep2, class Period2>
      constexpr bool operator!=(const duration<Rep1, Period1>& lhs,
                                const duration<Rep2, Period2>& rhs);
    template <class Rep1, class Period1, class Rep2, class Period2>
      constexpr bool operator< (const duration<Rep1, Period1>& lhs,
                                const duration<Rep2, Period2>& rhs);
    template <class Rep1, class Period1, class Rep2, class Period2>
      constexpr bool operator<=(const duration<Rep1, Period1>& lhs,
                                const duration<Rep2, Period2>& rhs);
    template <class Rep1, class Period1, class Rep2, class Period2>
      constexpr bool operator> (const duration<Rep1, Period1>& lhs,
                                const duration<Rep2, Period2>& rhs);
    template <class Rep1, class Period1, class Rep2, class Period2>
      constexpr bool operator>=(const duration<Rep1, Period1>& lhs,
                                const duration<Rep2, Period2>& rhs);

    // \ref{time.duration.cast}, duration_cast
    template <class ToDuration, class Rep, class Period>
      constexpr ToDuration duration_cast(const duration<Rep, Period>& d);
    template <class ToDuration, class Rep, class Period>
      constexpr ToDuration floor(const duration<Rep, Period>& d);
    template <class ToDuration, class Rep, class Period>
      constexpr ToDuration ceil(const duration<Rep, Period>& d);
    template <class ToDuration, class Rep, class Period>
      constexpr ToDuration round(const duration<Rep, Period>& d);

    // convenience typedefs
    using nanoseconds  = duration<@\term{signed integer type of at least 64 bits}@, nano>;
    using microseconds = duration<@\term{signed integer type of at least 55 bits}@, micro>;
    using milliseconds = duration<@\term{signed integer type of at least 45 bits}@, milli>;
    using seconds      = duration<@\term{signed integer type of at least 35 bits}@>;
    using minutes      = duration<@\term{signed integer type of at least 29 bits}@, ratio<  60>>;
    using hours        = duration<@\term{signed integer type of at least 23 bits}@, ratio<3600>>;

    // \ref{time.point.nonmember}, time_point arithmetic
    template <class Clock, class Duration1, class Rep2, class Period2>
      constexpr time_point<Clock, common_type_t<Duration1, duration<Rep2, Period2>>>
      operator+(const time_point<Clock, Duration1>& lhs,
                const duration<Rep2, Period2>& rhs);
    template <class Rep1, class Period1, class Clock, class Duration2>
      constexpr time_point<Clock, common_type_t<duration<Rep1, Period1>, Duration2>>
      operator+(const duration<Rep1, Period1>& lhs,
                const time_point<Clock, Duration2>& rhs);
    template <class Clock, class Duration1, class Rep2, class Period2>
      constexpr time_point<Clock, common_type_t<Duration1, duration<Rep2, Period2>>>
      operator-(const time_point<Clock, Duration1>& lhs,
                const duration<Rep2, Period2>& rhs);
    template <class Clock, class Duration1, class Duration2>
      constexpr common_type_t<Duration1, Duration2>
      operator-(const time_point<Clock, Duration1>& lhs,
                const time_point<Clock, Duration2>& rhs);

    // \ref{time.point.comparisons} time_point comparisons
    template <class Clock, class Duration1, class Duration2>
       constexpr bool operator==(const time_point<Clock, Duration1>& lhs,
                                 const time_point<Clock, Duration2>& rhs);
    template <class Clock, class Duration1, class Duration2>
       constexpr bool operator!=(const time_point<Clock, Duration1>& lhs,
                                 const time_point<Clock, Duration2>& rhs);
    template <class Clock, class Duration1, class Duration2>
       constexpr bool operator< (const time_point<Clock, Duration1>& lhs,
                                 const time_point<Clock, Duration2>& rhs);
    template <class Clock, class Duration1, class Duration2>
       constexpr bool operator<=(const time_point<Clock, Duration1>& lhs,
                                 const time_point<Clock, Duration2>& rhs);
    template <class Clock, class Duration1, class Duration2>
       constexpr bool operator> (const time_point<Clock, Duration1>& lhs,
                                 const time_point<Clock, Duration2>& rhs);
    template <class Clock, class Duration1, class Duration2>
       constexpr bool operator>=(const time_point<Clock, Duration1>& lhs,
                                 const time_point<Clock, Duration2>& rhs);

    // \ref{time.point.cast}, time_point_cast
    template <class ToDuration, class Clock, class Duration>
      constexpr time_point<Clock, ToDuration>
      time_point_cast(const time_point<Clock, Duration>& t);
    template <class ToDuration, class Clock, class Duration>
      constexpr time_point<Clock, ToDuration>
      floor(const time_point<Clock, Duration>& tp);
    template <class ToDuration, class Clock, class Duration>
      constexpr time_point<Clock, ToDuration>
      ceil(const time_point<Clock, Duration>& tp);
    template <class ToDuration, class Clock, class Duration>
      constexpr time_point<Clock, ToDuration>
      round(const time_point<Clock, Duration>& tp);

    // \ref{time.duration.alg}, specialized algorithms:
    template <class Rep, class Period>
      constexpr duration<Rep, Period> abs(duration<Rep, Period> d);

    // \ref{time.clock}, clocks
    class system_clock;
    class steady_clock;
    class high_resolution_clock;
  }

  inline namespace literals {
    inline namespace chrono_literals {
      // ~\ref{time.duration.literals}, suffixes for duration literals
      constexpr chrono::hours                                operator "" h(unsigned long long);
      constexpr chrono::duration<@\unspec,@ ratio<3600,1>> operator "" h(long double);
      constexpr chrono::minutes                              operator "" min(unsigned long long);
      constexpr chrono::duration<@\unspec,@ ratio<60,1>>   operator "" min(long double);
      constexpr chrono::seconds                              operator "" s(unsigned long long);
      constexpr chrono::duration<@\unspec@> @\itcorr[-1]@               operator "" s(long double);
      constexpr chrono::milliseconds                         operator "" ms(unsigned long long);
      constexpr chrono::duration<@\unspec,@ milli>          operator "" ms(long double);
      constexpr chrono::microseconds                         operator "" us(unsigned long long);
      constexpr chrono::duration<@\unspec,@ micro>         operator "" us(long double);
      constexpr chrono::nanoseconds                          operator "" ns(unsigned long long);
      constexpr chrono::duration<@\unspec,@ nano>          operator "" ns(long double);
    }
  }

  namespace chrono {
    using namespace literals::chrono_literals;
  }
}
\end{codeblock}

\rSec2[time.clock.req]{Clock requirements}

\pnum
A clock is a bundle consisting of a \tcode{duration}, a
\tcode{time_point}, and a function \tcode{now()} to get the current \tcode{time_point}.
The origin of the clock's \tcode{time_point} is referred to as the clock's \defn{epoch}.
 A clock shall meet the requirements in Table~\ref{tab:time.clock}.

\pnum
In Table~\ref{tab:time.clock} \tcode{C1} and \tcode{C2} denote clock types. \tcode{t1} and
\tcode{t2} are values returned by \tcode{C1::now()} where the call returning \tcode{t1} happens
before~(\ref{intro.multithread}) the call returning \tcode{t2} and both of these calls
occur
before \tcode{C1::time_point::max()}.
\begin{note} this means \tcode{C1} did not wrap around between \tcode{t1} and
\tcode{t2}. \end{note}

\begin{libreqtab3a}
{Clock requirements}
{tab:time.clock}
\\ \topline
\lhdr{Expression}       &   \chdr{Return type}  &   \rhdr{Operational semantics} \\ \capsep
\endfirsthead
\continuedcaption\\
\hline
\lhdr{Expression}       &   \chdr{Return type}  &   \rhdr{Operational semantics}       \\ \capsep
\endhead

\tcode{C1::rep} &
  An arithmetic type or a class emulating an arithmetic type &
  The representation type of \tcode{C1::duration}.  \\ \rowsep

\tcode{C1::period}  &
  a specialization of \tcode{ratio}     &
  The tick period of the clock in seconds.  \\ \rowsep

\tcode{C1::duration}  &
  \tcode{chrono::duration<C1::rep, C1::period>} &
  The \tcode{duration} type of the clock. \\ \rowsep

\tcode{C1::time_point}  &
  \tcode{chrono::time_point<C1>} or \tcode{chrono::time_point<C2, C1::duration>}  &
  The \tcode{time_point} type of the clock. \tcode{C1} and \tcode{C2} shall
  refer to the same epoch. \\ \rowsep

\tcode{C1::is_steady}  &
  \tcode{const bool}      &
  \tcode{true} if \tcode{t1 <= t2} is always \tcode{true} and the time between clock
  ticks is constant, otherwise \tcode{false}.  \\ \rowsep

\tcode{C1::now()} &
  \tcode{C1::time_point}  &
  Returns a \tcode{time_point} object representing the current point in time. \\

\end{libreqtab3a}

\pnum
\begin{note} The relative difference in durations between those reported by a given clock and the
SI definition is a measure of the quality of implementation. \end{note}

\pnum
A type \tcode{TC} meets the \tcode{TrivialClock} requirements if:

\begin{itemize}
\item \tcode{TC} satisfies the \tcode{Clock} requirements~(\ref{time.clock.req}),

\item the types \tcode{TC::rep}, \tcode{TC::duration}, and \tcode{TC::time_point}
satisfy the requirements of \tcode{EqualityCom\-parable} (Table~\ref{tab:equalitycomparable}),
\tcode{LessThanComparable} (Table~\ref{tab:lessthancomparable}),
\tcode{DefaultConstructible} (Table~\ref{tab:defaultconstructible}),
\tcode{CopyCon\-structible} (Table~\ref{tab:copyconstructible}),
\tcode{CopyAssignable} (Table~\ref{tab:copyassignable}),
\tcode{Destructible} (Table~\ref{tab:destructible}), and the requirements of
numeric types~(\ref{numeric.requirements}). \begin{note} this means, in particular,
that operations on these types will not throw exceptions. \end{note}

\item lvalues of the types \tcode{TC::rep}, \tcode{TC::duration}, and
\tcode{TC::time_point} are swappable~(\ref{swappable.requirements}),

\item the function \tcode{TC::now()} does not throw exceptions, and

\item the type \tcode{TC::time_point::clock} meets the \tcode{TrivialClock}
requirements, recursively.
\end{itemize}

\rSec2[time.traits]{Time-related traits}

\rSec3[time.traits.is_fp]{\tcode{treat_as_floating_point}}

\indexlibrary{\idxcode{treat_as_floating_point}}%
\begin{itemdecl}
template <class Rep> struct treat_as_floating_point
  : is_floating_point<Rep> { };
\end{itemdecl}

\pnum
The \tcode{duration} template uses the \tcode{treat_as_floating_point} trait to
help determine if a \tcode{duration} object can be converted to another
\tcode{duration} with a different tick \tcode{period}. If
\tcode{treat_as_floating_point_v<Rep>} is \tcode{true}, then implicit conversions
are allowed among \tcode{duration}s. Otherwise, the implicit convertibility
depends on the tick \tcode{period}s of the \tcode{duration}s.
\begin{note}
The intention of this trait is to indicate whether a given class behaves like a floating-point
type, and thus allows division of one value by another with acceptable loss of precision. If
\tcode{treat_as_floating_point_v<Rep>} is \tcode{false}, \tcode{Rep} will be treated as
if it behaved like an integral type for the purpose of these conversions.
\end{note}

\rSec3[time.traits.duration_values]{\tcode{duration_values}}

\indexlibrary{\idxcode{duration_values}}%
\begin{itemdecl}
template <class Rep>
struct duration_values {
public:
  static constexpr Rep zero();
  static constexpr Rep min();
  static constexpr Rep max();
};
\end{itemdecl}

\pnum
The \tcode{duration} template uses the \tcode{duration_values} trait to
construct special values of the durations representation (\tcode{Rep}). This is
done because the representation might be a class type with behavior which
requires some other implementation to return these special values. In that case,
the author of that class type should specialize \tcode{duration_values} to
return the indicated values.

\indexlibrarymember{zero}{duration_values}%
\begin{itemdecl}
static constexpr Rep zero();
\end{itemdecl}

\begin{itemdescr}
\pnum
\returns \tcode{Rep(0)}. \begin{note} \tcode{Rep(0)} is specified instead of
\tcode{Rep()} because \tcode{Rep()} may have some other meaning, such as an
uninitialized value. \end{note}

\pnum
\remarks The value returned shall be the additive identity.
\end{itemdescr}

\indexlibrarymember{min}{duration_values}%
\begin{itemdecl}
static constexpr Rep min();
\end{itemdecl}

\begin{itemdescr}
\pnum
\returns \tcode{numeric_limits<Rep>::lowest()}.

\pnum
\remarks The value returned shall compare less than or equal to \tcode{zero()}.
\end{itemdescr}

\indexlibrarymember{max}{duration_values}%
\begin{itemdecl}
static constexpr Rep max();
\end{itemdecl}

\begin{itemdescr}
\pnum
\returns \tcode{numeric_limits<Rep>::max()}.

\pnum
\remarks The value returned shall compare greater than \tcode{zero()}.
\end{itemdescr}

\rSec3[time.traits.specializations]{Specializations of \tcode{common_type}}

\indexlibrary{\idxcode{common_type}}%
\begin{itemdecl}
template <class Rep1, class Period1, class Rep2, class Period2>
struct common_type<chrono::duration<Rep1, Period1>, chrono::duration<Rep2, Period2>> {
  using type = chrono::duration<common_type_t<Rep1, Rep2>, @\seebelow@>;
};
\end{itemdecl}

\pnum
The \tcode{period} of the \tcode{duration} indicated by this specialization of
\tcode{common_type} shall be the greatest common divisor of \tcode{Period1} and
\tcode{Period2}. \begin{note} This can be computed by forming a ratio of the
greatest common divisor of \tcode{Period1::num} and \tcode{Period2::num} and the
least common multiple of \tcode{Period1::den} and \tcode{Period2::den}.
\end{note}

\pnum
\begin{note} The \tcode{typedef} name \tcode{type} is a synonym for the
\tcode{duration} with the largest tick \tcode{period} possible where both
\tcode{duration} arguments will convert to it without requiring a division
operation. The representation of this type is intended to be able to hold any
value resulting from this conversion with no truncation error, although
floating-point durations may have round-off errors. \end{note}

\indexlibrary{\idxcode{common_type}}%
\begin{itemdecl}
template <class Clock, class Duration1, class Duration2>
struct common_type<chrono::time_point<Clock, Duration1>, chrono::time_point<Clock, Duration2>> {
  using type = chrono::time_point<Clock, common_type_t<Duration1, Duration2>>;
};
\end{itemdecl}

\pnum
The common type of two \tcode{time_point} types is a \tcode{time_point} with the same
clock as the two types and the common type of their two \tcode{duration}s.

\rSec2[time.duration]{Class template \tcode{duration}}

\pnum
A \tcode{duration} type measures time between two points in time (\tcode{time_point}s).
A \tcode{duration} has a representation which holds a count of ticks and a tick period.
The tick period is the amount of time which occurs from one tick to the next, in units
of seconds. It is expressed as a rational constant using the template \tcode{ratio}.

\indexlibrary{\idxcode{duration}}%
\begin{codeblock}
template <class Rep, class Period = ratio<1>>
class duration {
public:
  using rep    = Rep;
  using period = Period;
private:
  rep rep_;  // \expos
public:
  // \ref{time.duration.cons}, construct/copy/destroy:
  constexpr duration() = default;
  template <class Rep2>
      constexpr explicit duration(const Rep2& r);
  template <class Rep2, class Period2>
     constexpr duration(const duration<Rep2, Period2>& d);
  ~duration() = default;
  duration(const duration&) = default;
  duration& operator=(const duration&) = default;

  // \ref{time.duration.observer}, observer:
  constexpr rep count() const;

  // \ref{time.duration.arithmetic}, arithmetic:
  constexpr duration  operator+() const;
  constexpr duration  operator-() const;
  duration& operator++();
  duration  operator++(int);
  duration& operator--();
  duration  operator--(int);

  duration& operator+=(const duration& d);
  duration& operator-=(const duration& d);

  duration& operator*=(const rep& rhs);
  duration& operator/=(const rep& rhs);
  duration& operator%=(const rep& rhs);
  duration& operator%=(const duration& rhs);

  // \ref{time.duration.special}, special values:
  static constexpr duration zero();
  static constexpr duration min();
  static constexpr duration max();
};
\end{codeblock}

\begin{itemdescr}
\pnum
\requires \tcode{Rep} shall be an arithmetic type or a class emulating an arithmetic type.

\pnum
\remarks If \tcode{duration} is instantiated with a \tcode{duration} type for the template
argument \tcode{Rep}, the program is ill-formed.

\pnum
\remarks If \tcode{Period} is not a specialization of \tcode{ratio}, the program is ill-formed.

\pnum
\remarks If \tcode{Period::num} is not positive, the program is ill-formed.

\pnum
\requires Members of \tcode{duration} shall not throw exceptions other than
those thrown by the indicated operations on their representations.

\pnum
\remarks The defaulted copy constructor of duration shall be a
\tcode{constexpr} function if and only if the required initialization
of the member \tcode{rep_} for copy and move, respectively, would
satisfy the requirements for a \tcode{constexpr} function.
\end{itemdescr}

\begin{example}
\begin{codeblock}
duration<long, ratio<60>> d0;       // holds a count of minutes using a \tcode{long}
duration<long long, milli> d1;      // holds a count of milliseconds using a \tcode{long long}
duration<double, ratio<1, 30>>  d2; // holds a count with a tick period of $\frac{1}{30}$ of a second
                                    // (30 Hz) using a \tcode{double}
\end{codeblock}
\end{example}

\rSec3[time.duration.cons]{\tcode{duration} constructors}

\indexlibrary{\idxcode{duration}!constructor}%
\begin{itemdecl}
template <class Rep2>
  constexpr explicit duration(const Rep2& r);
\end{itemdecl}

\begin{itemdescr}
\pnum
\remarks This constructor shall not participate in overload
resolution unless
\tcode{Rep2} is implicitly convertible to \tcode{rep} and
\begin{itemize}
\item \tcode{treat_as_floating_point_v<rep>} is \tcode{true} or
\item \tcode{treat_as_floating_point_v<Rep2>} is \tcode{false}.
\end{itemize}
\begin{example}
\begin{codeblock}
duration<int, milli> d(3);          // OK
duration<int, milli> d(3.5);        // error
\end{codeblock}
\end{example}

\pnum
\effects Constructs an object of type \tcode{duration}.

\pnum
\postconditions \tcode{count() == static_cast<rep>(r)}.
\end{itemdescr}

\indexlibrary{\idxcode{duration}!constructor}%
\begin{itemdecl}
template <class Rep2, class Period2>
  constexpr duration(const duration<Rep2, Period2>& d);
\end{itemdecl}

\begin{itemdescr}
\pnum
\remarks This constructor shall not participate in overload resolution unless
no overflow is induced in the conversion and
\tcode{treat_as_floating_point_v<rep>} is \tcode{true} or both
\tcode{ratio_divide<Period2, period>::den} is \tcode{1} and
\tcode{treat_as_floating_point_v<Rep2>} is \tcode{false}. \begin{note} This
requirement prevents implicit truncation error when converting between
integral-based \tcode{duration} types. Such a construction could easily lead to
confusion about the value of the \tcode{duration}. \end{note}
\begin{example}
\begin{codeblock}
duration<int, milli> ms(3);
duration<int, micro> us = ms;       // OK
duration<int, milli> ms2 = us;      // error
\end{codeblock}
\end{example}

\pnum
\effects Constructs an object of type \tcode{duration}, constructing \tcode{rep_} from\\
\tcode{duration_cast<duration>(d).count()}.
\end{itemdescr}

\rSec3[time.duration.observer]{\tcode{duration} observer}

\indexlibrarymember{count}{duration}%
\begin{itemdecl}
constexpr rep count() const;
\end{itemdecl}

\begin{itemdescr}
\pnum
\returns \tcode{rep_}.
\end{itemdescr}

\rSec3[time.duration.arithmetic]{\tcode{duration} arithmetic}

\indexlibrarymember{operator+}{duration}%
\begin{itemdecl}
constexpr duration operator+() const;
\end{itemdecl}

\begin{itemdescr}
\pnum
\returns \tcode{*this}.
\end{itemdescr}

\indexlibrarymember{operator-}{duration}%
\begin{itemdecl}
constexpr duration operator-() const;
\end{itemdecl}

\begin{itemdescr}
\pnum
\returns \tcode{duration(-rep_)}.
\end{itemdescr}

\indexlibrarymember{operator++}{duration}%
\begin{itemdecl}
duration& operator++();
\end{itemdecl}

\begin{itemdescr}
\pnum
\effects As if by \tcode{++rep_}.

\pnum
\returns \tcode{*this}.
\end{itemdescr}

\indexlibrarymember{operator++}{duration}%
\begin{itemdecl}
duration operator++(int);
\end{itemdecl}

\begin{itemdescr}
\pnum
\returns \tcode{duration(rep_++)}.
\end{itemdescr}

\indexlibrarymember{operator\dcr}{duration}%
\begin{itemdecl}
duration& operator--();
\end{itemdecl}

\begin{itemdescr}
\pnum
\effects As if by \tcode{--rep_}.

\pnum
\returns \tcode{*this}.
\end{itemdescr}

\indexlibrarymember{operator\dcr}{duration}%
\begin{itemdecl}
duration operator--(int);
\end{itemdecl}

\begin{itemdescr}
\pnum
\returns \tcode{duration(rep_-{}-)}.
\end{itemdescr}

\indexlibrarymember{operator+=}{duration}%
\begin{itemdecl}
duration& operator+=(const duration& d);
\end{itemdecl}

\begin{itemdescr}
\pnum
\effects As if by: \tcode{rep_ += d.count();}

\pnum
\returns \tcode{*this}.
\end{itemdescr}

\indexlibrarymember{operator-=}{duration}%
\begin{itemdecl}
duration& operator-=(const duration& d);
\end{itemdecl}

\begin{itemdescr}
\pnum
\effects As if by: \tcode{rep_ -= d.count();}

\pnum
\returns \tcode{*this}.
\end{itemdescr}

\indexlibrarymember{operator*=}{duration}%
\begin{itemdecl}
duration& operator*=(const rep& rhs);
\end{itemdecl}

\begin{itemdescr}
\pnum
\effects As if by: \tcode{rep_ *= rhs;}

\pnum
\returns \tcode{*this}.
\end{itemdescr}

\indexlibrarymember{operator/=}{duration}%
\begin{itemdecl}
duration& operator/=(const rep& rhs);
\end{itemdecl}

\begin{itemdescr}
\pnum
\effects As if by: \tcode{rep_ /= rhs;}

\pnum
\returns \tcode{*this}.
\end{itemdescr}

\indexlibrarymember{operator\%=}{duration}%
\begin{itemdecl}
duration& operator%=(const rep& rhs);
\end{itemdecl}

\begin{itemdescr}
\pnum
\effects As if by: \tcode{rep_ \%= rhs;}

\pnum
\returns \tcode{*this}.
\end{itemdescr}

\indexlibrarymember{operator\%=}{duration}%
\begin{itemdecl}
duration& operator%=(const duration& rhs);
\end{itemdecl}

\begin{itemdescr}
\pnum
\effects As if by: \tcode{rep_ \%= rhs.count();}

\pnum
\returns \tcode{*this}.
\end{itemdescr}


\rSec3[time.duration.special]{\tcode{duration} special values}

\indexlibrarymember{zero}{duration}%
\begin{itemdecl}
static constexpr duration zero();
\end{itemdecl}

\begin{itemdescr}
\pnum
\returns \tcode{duration(duration_values<rep>::zero())}.
\end{itemdescr}

\indexlibrarymember{min}{duration}%
\begin{itemdecl}
static constexpr duration min();
\end{itemdecl}

\begin{itemdescr}
\pnum
\returns \tcode{duration(duration_values<rep>::min())}.
\end{itemdescr}

\indexlibrarymember{max}{duration}%
\begin{itemdecl}
static constexpr duration max();
\end{itemdecl}

\begin{itemdescr}
\pnum
\returns \tcode{duration(duration_values<rep>::max())}.
\end{itemdescr}

\rSec3[time.duration.nonmember]{\tcode{duration} non-member arithmetic}

\pnum
In the function descriptions that follow, \tcode{CD} represents the return type
of the function. \tcode{CR(A, B)} represents \tcode{common_type_t<A, B>}.

\indexlibrary{\idxcode{common_type}}%
\begin{itemdecl}
template <class Rep1, class Period1, class Rep2, class Period2>
  constexpr common_type_t<duration<Rep1, Period1>, duration<Rep2, Period2>>
  operator+(const duration<Rep1, Period1>& lhs, const duration<Rep2, Period2>& rhs);
\end{itemdecl}

\begin{itemdescr}
\pnum
\returns \tcode{CD(CD(lhs).count() + CD(rhs).count())}.
\end{itemdescr}

\indexlibrary{\idxcode{common_type}}%
\begin{itemdecl}
template <class Rep1, class Period1, class Rep2, class Period2>
  constexpr common_type_t<duration<Rep1, Period1>, duration<Rep2, Period2>>
  operator-(const duration<Rep1, Period1>& lhs, const duration<Rep2, Period2>& rhs);
\end{itemdecl}

\begin{itemdescr}
\pnum
\returns \tcode{CD(CD(lhs).count() - CD(rhs).count())}.
\end{itemdescr}

\indexlibrarymember{operator*}{duration}%
\begin{itemdecl}
template <class Rep1, class Period, class Rep2>
  constexpr duration<common_type_t<Rep1, Rep2>, Period>
  operator*(const duration<Rep1, Period>& d, const Rep2& s);
\end{itemdecl}

\begin{itemdescr}
\pnum
\remarks This operator shall not participate in overload
resolution unless \tcode{Rep2} is implicitly convertible to \tcode{CR(Rep1, Rep2)}.

\pnum
\returns \tcode{CD(CD(d).count() * s)}.
\end{itemdescr}

\indexlibrarymember{operator*}{duration}%
\begin{itemdecl}
template <class Rep1, class Rep2, class Period>
  constexpr duration<common_type_t<Rep1, Rep2>, Period>
  operator*(const Rep1& s, const duration<Rep2, Period>& d);
\end{itemdecl}

\begin{itemdescr}
\pnum
\remarks This operator shall not participate in overload
resolution unless \tcode{Rep1} is implicitly convertible to \tcode{CR(Rep1, Rep2)}.

\pnum
\returns \tcode{d * s}.
\end{itemdescr}

\indexlibrarymember{operator/}{duration}%
\begin{itemdecl}
template <class Rep1, class Period, class Rep2>
  constexpr duration<common_type_t<Rep1, Rep2>, Period>
  operator/(const duration<Rep1, Period>& d, const Rep2& s);
\end{itemdecl}

\begin{itemdescr}
\pnum
\remarks This operator shall not participate in overload
resolution unless \tcode{Rep2} is implicitly convertible to \tcode{CR(Rep1, Rep2)}
and \tcode{Rep2} is not a specialization of \tcode{duration}.

\pnum
\returns \tcode{CD(CD(d).count() / s)}.
\end{itemdescr}

\indexlibrarymember{operator/}{duration}%
\begin{itemdecl}
template <class Rep1, class Period1, class Rep2, class Period2>
  constexpr common_type_t<Rep1, Rep2>
  operator/(const duration<Rep1, Period1>& lhs, const duration<Rep2, Period2>& rhs);
\end{itemdecl}

\begin{itemdescr}
\pnum
\returns \tcode{CD(lhs).count() / CD(rhs).count()}.
\end{itemdescr}

\indexlibrarymember{operator\%}{duration}%
\begin{itemdecl}
template <class Rep1, class Period, class Rep2>
  constexpr duration<common_type_t<Rep1, Rep2>, Period>
  operator%(const duration<Rep1, Period>& d, const Rep2& s);
\end{itemdecl}

\begin{itemdescr}
\pnum
\remarks This operator shall not participate in overload
resolution unless \tcode{Rep2} is implicitly convertible to \tcode{CR(Rep1, Rep2)} and
\tcode{Rep2} is not a specialization of \tcode{duration}.

\pnum
\returns \tcode{CD(CD(d).count() \% s)}.
\end{itemdescr}

\indexlibrarymember{operator\%}{duration}%
\begin{itemdecl}
template <class Rep1, class Period1, class Rep2, class Period2>
  constexpr common_type_t<duration<Rep1, Period1>, duration<Rep2, Period2>>
  operator%(const duration<Rep1, Period1>& lhs, const duration<Rep2, Period2>& rhs);
\end{itemdecl}

\begin{itemdescr}
\pnum
\returns \tcode{CD(CD(lhs).count() \% CD(rhs).count())}.
\end{itemdescr}


\rSec3[time.duration.comparisons]{\tcode{duration} comparisons}

\pnum
In the function descriptions that follow, \tcode{\placeholder{CT}} represents
\tcode{common_type_t<A, B>}, where \tcode{A} and \tcode{B} are the types of
the two arguments to the function.

\indexlibrarymember{operator==}{duration}%
\begin{itemdecl}
template <class Rep1, class Period1, class Rep2, class Period2>
  constexpr bool operator==(const duration<Rep1, Period1>& lhs,
                            const duration<Rep2, Period2>& rhs);
\end{itemdecl}

\begin{itemdescr}
\pnum
\returns \tcode{\placeholder{CT}(lhs).count() == \placeholder{CT}(rhs).count()}.
\end{itemdescr}

\indexlibrarymember{operator"!=}{duration}%
\begin{itemdecl}
template <class Rep1, class Period1, class Rep2, class Period2>
  constexpr bool operator!=(const duration<Rep1, Period1>& lhs,
                            const duration<Rep2, Period2>& rhs);
\end{itemdecl}

\begin{itemdescr}
\pnum
\returns \tcode{!(lhs == rhs)}.
\end{itemdescr}

\indexlibrarymember{operator<}{duration}%
\begin{itemdecl}
template <class Rep1, class Period1, class Rep2, class Period2>
  constexpr bool operator<(const duration<Rep1, Period1>& lhs,
                           const duration<Rep2, Period2>& rhs);
\end{itemdecl}

\begin{itemdescr}
\pnum
\returns \tcode{\placeholder{CT}(lhs).count() < \placeholder{CT}(rhs).count()}.
\end{itemdescr}

\indexlibrarymember{operator<=}{duration}%
\begin{itemdecl}
template <class Rep1, class Period1, class Rep2, class Period2>
  constexpr bool operator<=(const duration<Rep1, Period1>& lhs,
                            const duration<Rep2, Period2>& rhs);
\end{itemdecl}

\begin{itemdescr}
\pnum
\returns \tcode{!(rhs < lhs)}.
\end{itemdescr}

\indexlibrarymember{operator>}{duration}%
\begin{itemdecl}
template <class Rep1, class Period1, class Rep2, class Period2>
  constexpr bool operator>(const duration<Rep1, Period1>& lhs,
                           const duration<Rep2, Period2>& rhs);
\end{itemdecl}

\begin{itemdescr}
\pnum
\returns \tcode{rhs < lhs}.
\end{itemdescr}

\indexlibrarymember{operator>=}{duration}%
\begin{itemdecl}
template <class Rep1, class Period1, class Rep2, class Period2>
  constexpr bool operator>=(const duration<Rep1, Period1>& lhs,
                            const duration<Rep2, Period2>& rhs);
\end{itemdecl}

\begin{itemdescr}
\pnum
\returns \tcode{!(lhs < rhs)}.
\end{itemdescr}

\rSec3[time.duration.cast]{\tcode{duration_cast}}

\indexlibrary{\idxcode{duration}!\idxcode{duration_cast}}%
\indexlibrary{\idxcode{duration_cast}}%
\begin{itemdecl}
template <class ToDuration, class Rep, class Period>
  constexpr ToDuration duration_cast(const duration<Rep, Period>& d);
\end{itemdecl}

\begin{itemdescr}
\pnum
\remarks This function shall not participate in overload resolution
unless \tcode{ToDuration} is a specialization of \tcode{duration}.

\pnum
\returns Let \tcode{CF} be \tcode{ratio_divide<Period, typename
ToDuration::period>}, and \tcode{CR} be \tcode{common_type<} \tcode{typename
ToDuration::rep, Rep, intmax_t>::type}.
\begin{itemize}
\item If \tcode{CF::num == 1} and \tcode{CF::den == 1}, returns
\begin{codeblock}
ToDuration(static_cast<typename ToDuration::rep>(d.count()))
\end{codeblock}

\item otherwise, if \tcode{CF::num != 1} and \tcode{CF::den == 1}, returns
\begin{codeblock}
ToDuration(static_cast<typename ToDuration::rep>(
  static_cast<CR>(d.count()) * static_cast<CR>(CF::num)))
\end{codeblock}

\item otherwise, if \tcode{CF::num == 1} and \tcode{CF::den != 1}, returns
\begin{codeblock}
ToDuration(static_cast<typename ToDuration::rep>(
  static_cast<CR>(d.count()) / static_cast<CR>(CF::den)))
\end{codeblock}

\item otherwise, returns
\begin{codeblock}
ToDuration(static_cast<typename ToDuration::rep>(
  static_cast<CR>(d.count()) * static_cast<CR>(CF::num) / static_cast<CR>(CF::den)))
\end{codeblock}
\end{itemize}

\realnotes This function does not use any implicit conversions; all conversions
are done with \tcode{static_cast}. It avoids multiplications and divisions when
it is known at compile time that one or more arguments is 1. Intermediate
computations are carried out in the widest representation and only converted to
the destination representation at the final step.
\end{itemdescr}

\indexlibrarymember{floor}{duration}%
\begin{itemdecl}
template <class ToDuration, class Rep, class Period>
  constexpr ToDuration floor(const duration<Rep, Period>& d);
\end{itemdecl}

\begin{itemdescr}
\pnum
\remarks This function shall not participate in overload resolution
unless \tcode{ToDuration} is a specialization of \tcode{duration}.

\pnum
\returns The greatest result \tcode{t} representable in \tcode{ToDuration}
for which \tcode{t <= d}.
\end{itemdescr}

\indexlibrarymember{ceil}{duration}%
\begin{itemdecl}
template <class ToDuration, class Rep, class Period>
  constexpr ToDuration ceil(const duration<Rep, Period>& d);
\end{itemdecl}

\begin{itemdescr}
\pnum
\remarks This function shall not participate in overload resolution
unless \tcode{ToDuration} is a specialization of \tcode{duration}.

\pnum
\returns The least result \tcode{t} representable in \tcode{ToDuration}
for which \tcode{t >= d}.
\end{itemdescr}

\indexlibrarymember{round}{duration}%
\begin{itemdecl}
template <class ToDuration, class Rep, class Period>
  constexpr ToDuration round(const duration<Rep, Period>& d);
\end{itemdecl}

\begin{itemdescr}
\pnum
\remarks This function shall not participate in overload resolution
unless \tcode{ToDuration} is a specialization of \tcode{duration},
and \tcode{treat_as_floating_point_v<typename ToDuration::rep>}
is \tcode{false}.

\pnum
\returns The value of \tcode{ToDuration} that is closest to \tcode{d}.
If there are two closest values, then return the value \tcode{t}
for which \tcode{t \% 2 == 0}.
\end{itemdescr}

\rSec3[time.duration.literals]{Suffixes for duration literals}

\pnum
This section describes literal suffixes for constructing duration literals. The
suffixes \tcode{h}, \tcode{min}, \tcode{s}, \tcode{ms}, \tcode{us}, \tcode{ns}
denote duration values of the corresponding types \tcode{hours}, \tcode{minutes},
\tcode{seconds}, \tcode{milliseconds}, \tcode{microseconds}, and \tcode{nanoseconds}
respectively if they are applied to integral literals.

\pnum
If any of these suffixes are applied to a floating point literal the result is a
\tcode{chrono::duration} literal with an unspecified floating point representation.

\pnum
If any of these suffixes are applied to an integer literal and the resulting
\tcode{chrono::duration} value cannot be represented in the result type because
of overflow, the program is ill-formed.

\pnum
\begin{example}
The following code shows some duration literals.
\begin{codeblock}
using namespace std::chrono_literals;
auto constexpr aday=24h;
auto constexpr lesson=45min;
auto constexpr halfanhour=0.5h;
\end{codeblock}
\end{example}

\indexlibrarymember{operator """" h}{duration}%
\begin{itemdecl}
constexpr chrono::hours                                 operator "" h(unsigned long long hours);
constexpr chrono::duration<@\unspec,@ ratio<3600, 1>> operator "" h(long double hours);
\end{itemdecl}

\begin{itemdescr}
\pnum
\returns
A \tcode{duration} literal representing \tcode{hours} hours.
\end{itemdescr}

\indexlibrarymember{operator """" min}{duration}%
\begin{itemdecl}
constexpr chrono::minutes                             operator "" min(unsigned long long minutes);
constexpr chrono::duration<@\unspec,@ ratio<60, 1>> operator "" min(long double minutes);
\end{itemdecl}

\begin{itemdescr}
\pnum
\returns
A \tcode{duration} literal representing \tcode{minutes} minutes.
\end{itemdescr}

\indexlibrarymember{operator """" s}{duration}%
\begin{itemdecl}
constexpr chrono::seconds  @\itcorr@             operator "" s(unsigned long long sec);
constexpr chrono::duration<@\unspec@> operator "" s(long double sec);
\end{itemdecl}

\begin{itemdescr}
\pnum
\returns
A \tcode{duration} literal representing \tcode{sec} seconds.

\pnum
\begin{note}
The same suffix \tcode{s} is used for \tcode{basic_string} but there is no
conflict, since duration suffixes apply to numbers and string literal suffixes
apply to character array literals.
\end{note}
\end{itemdescr}

\indexlibrarymember{operator """" ms}{duration}%
\begin{itemdecl}
constexpr chrono::milliseconds                 operator "" ms(unsigned long long msec);
constexpr chrono::duration<@\unspec,@ milli> operator "" ms(long double msec);
\end{itemdecl}

\begin{itemdescr}
\pnum
\returns
A \tcode{duration} literal representing \tcode{msec} milliseconds.
\end{itemdescr}

\indexlibrarymember{operator """" us}{duration}%
\begin{itemdecl}
constexpr chrono::microseconds                 operator "" us(unsigned long long usec);
constexpr chrono::duration<@\unspec,@ micro> operator "" us(long double usec);
\end{itemdecl}

\begin{itemdescr}
\pnum
\returns
A \tcode{duration} literal representing \tcode{usec} microseconds.
\end{itemdescr}

\indexlibrarymember{operator """" ns}{duration}%
\begin{itemdecl}
constexpr chrono::nanoseconds                 operator "" ns(unsigned long long nsec);
constexpr chrono::duration<@\unspec,@ nano> operator "" ns(long double nsec);
\end{itemdecl}

\begin{itemdescr}
\pnum
\returns
A \tcode{duration} literal representing \tcode{nsec} nanoseconds.
\end{itemdescr}

\rSec3[time.duration.alg]{\tcode{duration} algorithms}

\indexlibrarymember{abs}{duration}%
\begin{itemdecl}
template <class Rep, class Period>
  constexpr duration<Rep, Period> abs(duration<Rep, Period> d);
\end{itemdecl}

\begin{itemdescr}
\pnum
\remarks This function shall not participate in overload resolution
unless \tcode{numeric_limits<Rep>::is_signed} is \tcode{true}.

\pnum
\returns If \tcode{d >= d.zero()}, return \tcode{d},
otherwise return \tcode{-d}.
\end{itemdescr}

\rSec2[time.point]{Class template \tcode{time_point}}

\indexlibrary{\idxcode{time_point}}%
\begin{codeblock}
template <class Clock, class Duration = typename Clock::duration>
class time_point {
public:
  using clock    = Clock;
  using duration = Duration;
  using rep      = typename duration::rep;
  using period   = typename duration::period;
private:
  duration d_;  // \expos

public:
  // \ref{time.point.cons}, construct:
  constexpr time_point();  // has value epoch
  constexpr explicit time_point(const duration& d);  // same as time_point() + d
  template <class Duration2>
    constexpr time_point(const time_point<clock, Duration2>& t);

  // \ref{time.point.observer}, observer:
  constexpr duration time_since_epoch() const;

  // \ref{time.point.arithmetic}, arithmetic:
  time_point& operator+=(const duration& d);
  time_point& operator-=(const duration& d);

  // \ref{time.point.special}, special values:
  static constexpr time_point min();
  static constexpr time_point max();
};
\end{codeblock}

\pnum
\tcode{Clock} shall meet the Clock requirements~(\ref{time.clock}).

\pnum
If \tcode{Duration} is not an instance of \tcode{duration},
the program is ill-formed.

\rSec3[time.point.cons]{\tcode{time_point} constructors}

\indexlibrary{\idxcode{time_point}!constructor}%
\begin{itemdecl}
constexpr time_point();
\end{itemdecl}

\begin{itemdescr}
\pnum
\effects Constructs an object of type \tcode{time_point}, initializing
\tcode{d_} with \tcode{duration::zero()}. Such a \tcode{time_point} object
represents the epoch.
\end{itemdescr}

\indexlibrary{\idxcode{time_point}!constructor}%
\begin{itemdecl}
constexpr explicit time_point(const duration& d);
\end{itemdecl}

\begin{itemdescr}
\pnum
\effects Constructs an object of type \tcode{time_point}, initializing
\tcode{d_} with \tcode{d}. Such a \tcode{time_point} object represents the epoch
\tcode{+ d}.
\end{itemdescr}

\indexlibrary{\idxcode{time_point}!constructor}%
\begin{itemdecl}
template <class Duration2>
  constexpr time_point(const time_point<clock, Duration2>& t);
\end{itemdecl}

\begin{itemdescr}
\pnum
\remarks This constructor shall not participate in overload resolution unless \tcode{Duration2}
is implicitly convertible to \tcode{duration}.

\pnum
\effects Constructs an object of type \tcode{time_point}, initializing
\tcode{d_} with \tcode{t.time_since_epoch()}.
\end{itemdescr}

\rSec3[time.point.observer]{\tcode{time_point} observer}

\indexlibrarymember{time_since_epoch}{time_point}%
\begin{itemdecl}
constexpr duration time_since_epoch() const;
\end{itemdecl}

\begin{itemdescr}
\pnum
\returns \tcode{d_}.
\end{itemdescr}

\rSec3[time.point.arithmetic]{\tcode{time_point} arithmetic}

\indexlibrarymember{operator+=}{time_point}%
\begin{itemdecl}
time_point& operator+=(const duration& d);
\end{itemdecl}

\begin{itemdescr}
\pnum
\effects As if by: \tcode{d_ += d;}

\pnum
\returns \tcode{*this}.
\end{itemdescr}

\indexlibrarymember{operator-=}{time_point}%
\begin{itemdecl}
time_point& operator-=(const duration& d);
\end{itemdecl}

\begin{itemdescr}
\pnum
\effects As if by: \tcode{d_ -= d;}

\pnum
\returns \tcode{*this}.
\end{itemdescr}

\rSec3[time.point.special]{\tcode{time_point} special values}

\indexlibrarymember{min}{time_point}%
\begin{itemdecl}
static constexpr time_point min();
\end{itemdecl}

\begin{itemdescr}
\pnum
\returns \tcode{time_point(duration::min())}.
\end{itemdescr}

\indexlibrarymember{max}{time_point}%
\begin{itemdecl}
static constexpr time_point max();
\end{itemdecl}

\begin{itemdescr}
\pnum
\returns \tcode{time_point(duration::max())}.
\end{itemdescr}

\rSec3[time.point.nonmember]{\tcode{time_point} non-member arithmetic}

\indexlibrarymember{operator+}{time_point}%
\indexlibrarymember{operator+}{duration}%
\begin{itemdecl}
template <class Clock, class Duration1, class Rep2, class Period2>
  constexpr time_point<Clock, common_type_t<Duration1, duration<Rep2, Period2>>>
  operator+(const time_point<Clock, Duration1>& lhs, const duration<Rep2, Period2>& rhs);
\end{itemdecl}

\begin{itemdescr}
\pnum
\returns \tcode{\placeholder{CT}(lhs.time_since_epoch() + rhs)}, where \tcode{\placeholder{CT}} is the type of the return value.
\end{itemdescr}

\indexlibrarymember{operator+}{time_point}%
\indexlibrarymember{operator+}{duration}%
\begin{itemdecl}
template <class Rep1, class Period1, class Clock, class Duration2>
  constexpr time_point<Clock, common_type_t<duration<Rep1, Period1>, Duration2>>
  operator+(const duration<Rep1, Period1>& lhs, const time_point<Clock, Duration2>& rhs);
\end{itemdecl}

\begin{itemdescr}
\pnum
\returns \tcode{rhs + lhs}.
\end{itemdescr}

\indexlibrarymember{operator-}{time_point}%
\indexlibrarymember{operator-}{duration}%
\begin{itemdecl}
template <class Clock, class Duration1, class Rep2, class Period2>
  constexpr time_point<Clock, common_type_t<Duration1, duration<Rep2, Period2>>>
  operator-(const time_point<Clock, Duration1>& lhs, const duration<Rep2, Period2>& rhs);
\end{itemdecl}

\begin{itemdescr}
\pnum
\returns \tcode{lhs + (-rhs)}.
\end{itemdescr}

\indexlibrarymember{operator-}{time_point}%
\begin{itemdecl}
template <class Clock, class Duration1, class Duration2>
  constexpr common_type_t<Duration1, Duration2>
  operator-(const time_point<Clock, Duration1>& lhs, const time_point<Clock, Duration2>& rhs);
\end{itemdecl}

\begin{itemdescr}
\pnum
\returns \tcode{lhs.time_since_epoch() - rhs.time_since_epoch()}.
\end{itemdescr}

\rSec3[time.point.comparisons]{\tcode{time_point} comparisons}

\indexlibrarymember{operator==}{time_point}%
\begin{itemdecl}
template <class Clock, class Duration1, class Duration2>
  constexpr bool operator==(const time_point<Clock, Duration1>& lhs,
                            const time_point<Clock, Duration2>& rhs);
\end{itemdecl}

\begin{itemdescr}
\pnum
\returns \tcode{lhs.time_since_epoch() == rhs.time_since_epoch()}.
\end{itemdescr}

\indexlibrarymember{operator"!=}{time_point}%
\begin{itemdecl}
template <class Clock, class Duration1, class Duration2>
  constexpr bool operator!=(const time_point<Clock, Duration1>& lhs,
                            const time_point<Clock, Duration2>& rhs);
\end{itemdecl}

\begin{itemdescr}
\pnum
\returns \tcode{!(lhs == rhs)}.
\end{itemdescr}

\indexlibrarymember{operator<}{time_point}%
\begin{itemdecl}
template <class Clock, class Duration1, class Duration2>
  constexpr bool operator<(const time_point<Clock, Duration1>& lhs,
                           const time_point<Clock, Duration2>& rhs);
\end{itemdecl}

\begin{itemdescr}
\pnum
\returns \tcode{lhs.time_since_epoch() < rhs.time_since_epoch()}.
\end{itemdescr}

\indexlibrarymember{operator<=}{time_point}%
\begin{itemdecl}
template <class Clock, class Duration1, class Duration2>
  constexpr bool operator<=(const time_point<Clock, Duration1>& lhs,
                            const time_point<Clock, Duration2>& rhs);
\end{itemdecl}

\begin{itemdescr}
\pnum
\returns \tcode{!(rhs < lhs)}.
\end{itemdescr}

\indexlibrarymember{operator>}{time_point}%
\begin{itemdecl}
template <class Clock, class Duration1, class Duration2>
  constexpr bool operator>(const time_point<Clock, Duration1>& lhs,
                           const time_point<Clock, Duration2>& rhs);
\end{itemdecl}

\begin{itemdescr}
\pnum
\returns \tcode{rhs < lhs}.
\end{itemdescr}

\indexlibrarymember{operator>=}{time_point}%
\begin{itemdecl}
template <class Clock, class Duration1, class Duration2>
  constexpr bool operator>=(const time_point<Clock, Duration1>& lhs,
                            const time_point<Clock, Duration2>& rhs);
\end{itemdecl}

\begin{itemdescr}
\pnum
\returns \tcode{!(lhs < rhs)}.
\end{itemdescr}

\rSec3[time.point.cast]{\tcode{time_point_cast}}

\indexlibrary{\idxcode{time_point}!\idxcode{time_point_cast}}%
\indexlibrary{\idxcode{time_point_cast}}%
\begin{itemdecl}
template <class ToDuration, class Clock, class Duration>
  constexpr time_point<Clock, ToDuration>
  time_point_cast(const time_point<Clock, Duration>& t);
\end{itemdecl}

\begin{itemdescr}
\pnum
\remarks This function shall not participate in overload resolution
unless \tcode{ToDuration} is a specialization of \tcode{duration}.

\pnum
\returns
\begin{codeblock}
time_point<Clock, ToDuration>(duration_cast<ToDuration>(t.time_since_epoch()))
\end{codeblock}
\end{itemdescr}

\indexlibrarymember{floor}{time_point}%
\begin{itemdecl}
template <class ToDuration, class Clock, class Duration>
  constexpr time_point<Clock, ToDuration>
  floor(const time_point<Clock, Duration>& tp);
\end{itemdecl}

\begin{itemdescr}
\pnum
\remarks This function shall not participate in overload resolution
unless \tcode{ToDuration} is a specialization of \tcode{duration}.

\pnum
\returns \tcode{time_point<Clock, ToDuration>(floor<ToDuration>(tp.time_since_epoch()))}.
\end{itemdescr}

\indexlibrarymember{ceil}{time_point}%
\begin{itemdecl}
template <class ToDuration, class Clock, class Duration>
  constexpr time_point<Clock, ToDuration>
  ceil(const time_point<Clock, Duration>& tp);
\end{itemdecl}

\begin{itemdescr}
\pnum
\remarks This function shall not participate in overload resolution
unless \tcode{ToDuration} is a specialization of \tcode{duration}.

\pnum
\returns \tcode{time_point<Clock, ToDuration>(ceil<ToDuration>(tp.time_since_epoch()))}.
\end{itemdescr}

\indexlibrarymember{round}{time_point}%
\begin{itemdecl}
template <class ToDuration, class Clock, class Duration>
  constexpr time_point<Clock, ToDuration>
  round(const time_point<Clock, Duration>& tp);
\end{itemdecl}

\begin{itemdescr}
\pnum
\remarks This function shall not participate in overload resolution
unless \tcode{ToDuration} is a specialization of \tcode{duration}, and
\tcode{treat_as_floating_point_v<typename ToDuration::rep>} is \tcode{false}.

\pnum
\returns \tcode{time_point<Clock, ToDuration>(round<ToDuration>(tp.time_since_epoch()))}.
\end{itemdescr}

\rSec2[time.clock]{Clocks}

\pnum
The types defined in this subclause shall satisfy the
\tcode{TrivialClock}
requirements~(\ref{time.clock.req}).

\rSec3[time.clock.system]{Class \tcode{system_clock}}
\indexlibrary{\idxcode{system_clock}}%

\pnum
Objects of class \tcode{system_clock} represent wall clock time from the system-wide
realtime clock.

\begin{codeblock}
class system_clock {
public:
  using rep        = @\seebelow@;
  using period     = ratio<@\unspecnc@, @\unspec{}@>;
  using duration   = chrono::duration<rep, period>;
  using time_point = chrono::time_point<system_clock>;
  static constexpr bool is_steady = @\unspec;@

  static time_point now() noexcept;

  // Map to C API
  static time_t      to_time_t  (const time_point& t) noexcept;
  static time_point  from_time_t(time_t t) noexcept;
};
\end{codeblock}

\indexlibrarymember{rep}{system_clock}%
\begin{itemdecl}
using system_clock::rep = @\unspec@;
\end{itemdecl}

\begin{itemdescr}
\pnum
\requires \tcode{system_clock::duration::min() < system_clock::duration::zero()} shall
be \tcode{true}.\\
\begin{note} This implies that \tcode{rep} is a signed type. \end{note}
\end{itemdescr}

\indexlibrarymember{to_time_t}{system_clock}%
\begin{itemdecl}
static time_t to_time_t(const time_point& t) noexcept;
\end{itemdecl}

\begin{itemdescr}
\pnum
\returns A \tcode{time_t} object that represents the same point in time as \tcode{t}
when both values are restricted to the coarser of the precisions of \tcode{time_t} and
\tcode{time_point}.
It is \impldef{whether values are rounded or truncated to the
required precision when converting between \tcode{time_t} values and \tcode{time_point} objects}
whether values are rounded or truncated to the required precision.
\end{itemdescr}

\indexlibrarymember{from_time_t}{system_clock}%
\begin{itemdecl}
static time_point from_time_t(time_t t) noexcept;
\end{itemdecl}

\begin{itemdescr}
\pnum
\returns A \tcode{time_point} object that represents the same point in time as \tcode{t}
when both values are restricted to the coarser of the precisions of \tcode{time_t} and
\tcode{time_point}.
It is \impldef{whether values are rounded or truncated to the
required precision when converting between \tcode{time_t} values and \tcode{time_point} objects}
whether values are rounded or truncated to the required precision.
\end{itemdescr}

\rSec3[time.clock.steady]{Class \tcode{steady_clock}}
\indexlibrary{\idxcode{steady_clock}}%

\pnum
Objects of class \tcode{steady_clock} represent clocks for which values of \tcode{time_point}
never decrease as physical time advances and for which values of \tcode{time_point} advance at
a steady rate relative to real time. That is, the clock may not be adjusted.

\begin{codeblock}
class steady_clock {
public:
  using rep        = @\unspec@;
  using period     = ratio<@\unspecnc@, @\unspec{}@>;
  using duration   = chrono::duration<rep, period>;
  using time_point = chrono::time_point<@\unspecnc@, duration>;
  static constexpr bool is_steady = true;

  static time_point now() noexcept;
};
\end{codeblock}

\rSec3[time.clock.hires]{Class \tcode{high_resolution_clock}}
\indexlibrary{\idxcode{high_resolution_clock}}%

\pnum
Objects of class \tcode{high_resolution_clock} represent clocks with the
shortest tick period. \tcode{high_resolution_clock} may be a synonym for
\tcode{system_clock} or \tcode{steady_clock}.

\begin{codeblock}
class high_resolution_clock {
public:
  using rep        = @\unspec@;
  using period     = ratio<@\unspecnc@, @\unspec{}@>;
  using duration   = chrono::duration<rep, period>;
  using time_point = chrono::time_point<@\unspecnc@, duration>;
  static constexpr bool is_steady = @\unspec@;

  static time_point now() noexcept;
};
\end{codeblock}

\rSec2[ctime.syn]{Header \tcode{<ctime>} synopsis}

\indextext{\idxhdr{ctime}}%
\indexlibrary{\idxcode{CLOCKS_PER_SEC}}%
\indexlibrary{\idxcode{NULL}}%
\indexlibrary{\idxcode{TIME_UTC}}%
\indexlibrary{\idxcode{asctime}}%
\indexlibrary{\idxcode{clock_t}}%
\indexlibrary{\idxcode{clock}}%
\indexlibrary{\idxcode{ctime}}%
\indexlibrary{\idxcode{difftime}}%
\indexlibrary{\idxcode{gmtime}}%
\indexlibrary{\idxcode{localtime}}%
\indexlibrary{\idxcode{mktime}}%
\indexlibrary{\idxcode{size_t}}%
\indexlibrary{\idxcode{strftime}}%
\indexlibrary{\idxcode{time_t}}%
\indexlibrary{\idxcode{timespec_get}}%
\indexlibrary{\idxcode{timespec}}%
\indexlibrary{\idxcode{time}}%
\indexlibrary{\idxcode{tm}}%
\begin{codeblock}
#define NULL @\textit{see \ref{support.types.nullptr}}@
#define CLOCKS_PER_SEC @\seebelow@
#define TIME_UTC @\seebelow@

namespace std {
  using size_t = @\textit{see \ref{support.types.layout}}@;
  using clock_t = @\seebelow@;
  using time_t = @\seebelow@;

  struct timespec;
  struct tm;

  clock_t clock();
  double difftime(time_t time1, time_t time0);
  time_t mktime(struct tm* timeptr);
  time_t time(time_t* timer);
  int timespec_get(timespec* ts, int base);
  char* asctime(const struct tm* timeptr);
  char* ctime(const time_t* timer);
  struct tm* gmtime(const time_t* timer);
  struct tm* localtime(const time_t* timer);
  size_t strftime(char* s, size_t maxsize, const char* format, const struct tm* timeptr);
}
\end{codeblock}

\pnum
\indexlibrary{\idxhdr{time.h}}%
\indexlibrary{\idxhdr{ctime}}%
The contents of the header \tcode{<ctime>} are the same as the C standard library header \tcode{<time.h>}.%
\footnote{\tcode{strftime} supports the C conversion specifiers
\tcode{C}, \tcode{D}, \tcode{e}, \tcode{F}, \tcode{g}, \tcode{G}, \tcode{h},
\tcode{r}, \tcode{R}, \tcode{t}, \tcode{T}, \tcode{u}, \tcode{V}, and
\tcode{z}, and the modifiers \tcode{E} and \tcode{O}.}

\pnum
The functions \tcode{asctime}, \tcode{ctime}, \tcode{gmtime}, and
\tcode{localtime} are not required to avoid data
races~(\ref{res.on.data.races}).

\xref ISO C~7.27.

\rSec1[type.index]{Class \tcode{type_index}}

\indexlibrary{\idxhdr{typeindex}}%
\rSec2[type.index.synopsis]{Header \tcode{<typeindex>} synopsis}

\begin{codeblock}
namespace std {
  class type_index;
  template <class T> struct hash;
  template<> struct hash<type_index>;
}
\end{codeblock}

\rSec2[type.index.overview]{\tcode{type_index} overview}

\indexlibrary{\idxcode{type_index}}%
\begin{codeblock}
namespace std {
  class type_index {
  public:
    type_index(const type_info& rhs) noexcept;
    bool operator==(const type_index& rhs) const noexcept;
    bool operator!=(const type_index& rhs) const noexcept;
    bool operator< (const type_index& rhs) const noexcept;
    bool operator<= (const type_index& rhs) const noexcept;
    bool operator> (const type_index& rhs) const noexcept;
    bool operator>= (const type_index& rhs) const noexcept;
    size_t hash_code() const noexcept;
    const char* name() const noexcept;
  private:
    const type_info* target;    // \expos
    // Note that the use of a pointer here, rather than a reference,
    // means that the default copy/move constructor and assignment
    // operators will be provided and work as expected.
  };
}
\end{codeblock}

\pnum
The class \tcode{type_index} provides a simple wrapper for
\tcode{type_info} which can be used as an index type in associative
containers~(\ref{associative}) and in unordered associative
containers~(\ref{unord}).

\rSec2[type.index.members]{\tcode{type_index} members}

\indexlibrary{\idxcode{type_index}!constructor}%
\begin{itemdecl}
type_index(const type_info& rhs) noexcept;
\end{itemdecl}

\begin{itemdescr}
\pnum
\effects Constructs a \tcode{type_index} object, the equivalent of \tcode{target = \&rhs}.
\end{itemdescr}

\indexlibrarymember{operator==}{type_index}%
\begin{itemdecl}
bool operator==(const type_index& rhs) const noexcept;
\end{itemdecl}

\begin{itemdescr}
\pnum
\returns \tcode{*target == *rhs.target}.
\end{itemdescr}

\indexlibrarymember{operator"!=}{type_index}%
\begin{itemdecl}
bool operator!=(const type_index& rhs) const noexcept;
\end{itemdecl}

\begin{itemdescr}
\pnum
\returns \tcode{*target != *rhs.target}.
\end{itemdescr}

\indexlibrarymember{operator<}{type_index}%
\begin{itemdecl}
bool operator<(const type_index& rhs) const noexcept;
\end{itemdecl}

\begin{itemdescr}
\pnum
\returns \tcode{target->before(*rhs.target)}.
\end{itemdescr}

\indexlibrarymember{operator<=}{type_index}%
\begin{itemdecl}
bool operator<=(const type_index& rhs) const noexcept;
\end{itemdecl}

\begin{itemdescr}
\pnum
\returns \tcode{!rhs.target->before(*target)}.
\end{itemdescr}

\indexlibrarymember{operator>}{type_index}%
\begin{itemdecl}
bool operator>(const type_index& rhs) const noexcept;
\end{itemdecl}

\begin{itemdescr}
\pnum
\returns \tcode{rhs.target->before(*target)}.
\end{itemdescr}

\indexlibrarymember{operator>=}{type_index}%
\begin{itemdecl}
bool operator>=(const type_index& rhs) const noexcept;
\end{itemdecl}

\begin{itemdescr}
\pnum
\returns \tcode{!target->before(*rhs.target)}.
\end{itemdescr}

\indexlibrarymember{hash_code}{type_index}%
\begin{itemdecl}
size_t hash_code() const noexcept;
\end{itemdecl}

\begin{itemdescr}
\pnum
\returns \tcode{target->hash_code()}.
\end{itemdescr}

\indexlibrarymember{name}{type_index}%
\begin{itemdecl}
const char* name() const noexcept;
\end{itemdecl}

\begin{itemdescr}
\pnum
\returns \tcode{target->name()}.
\end{itemdescr}

\rSec2[type.index.hash]{Hash support}

\indexlibrary{\idxcode{hash}!\idxcode{type_index}}%
\begin{itemdecl}
template <> struct hash<type_index>;
\end{itemdecl}

\begin{itemdescr}
\pnum
The template specialization shall meet the requirements of class template
\tcode{hash}~(\ref{unord.hash}). For an object \tcode{index} of type \tcode{type_index},
\tcode{hash<type_index>()(index)} shall evaluate to the same result as \tcode{index.hash_code()}.
\end{itemdescr}

\rSec1[execpol]{Execution policies}
\rSec2[execpol.general]{In general}

\pnum
This subclause describes classes that are \defn{execution policy} types. An
object of an execution policy type indicates the kinds of parallelism allowed
in the execution of an algorithm and expresses the consequent requirements on
the element access functions.
\begin{example}
\begin{codeblock}
using namespace std;
vector<int> v = ...

// standard sequential sort
sort(v.begin(), v.end());

// explicitly sequential sort
sort(execution::seq, v.begin(), v.end());

// permitting parallel execution
sort(execution::par, v.begin(), v.end());

// permitting vectorization as well
sort(execution::par_unseq, v.begin(), v.end());
\end{codeblock}
\end{example}
\begin{note}
Because different parallel architectures may require idiosyncratic
parameters for efficient execution, implementations
may provide additional execution policies to those described in this
standard as extensions.
\end{note}

\indexlibrary{\idxhdr{execution}}%
\rSec2[execution.syn]{Header \tcode{<execution>} synopsis}
\begin{codeblock}
namespace std {
  // \ref{execpol.type}, execution policy type trait:
  template<class T> struct is_execution_policy;
  template<class T> constexpr bool is_execution_policy_v = is_execution_policy<T>::value;
}

namespace std::execution {
  // \ref{execpol.seq}, sequenced execution policy:
  class sequenced_policy;

  // \ref{execpol.par}, parallel execution policy:
  class parallel_policy;

  // \ref{execpol.parunseq}, parallel and unsequenced execution policy:
  class parallel_unsequenced_policy;

  // \ref{execpol.objects}, execution policy objects:
  constexpr sequenced_policy            seq{ @\unspec@ };
  constexpr parallel_policy             par{ @\unspec@ };
  constexpr parallel_unsequenced_policy par_unseq{ @\unspec@ };
}
\end{codeblock}

\rSec2[execpol.type]{Execution policy type trait}

\indexlibrary{\idxcode{is_execution_policy}}%
\begin{itemdecl}
template<class T> struct is_execution_policy { @\seebelow@ };
\end{itemdecl}

\begin{itemdescr}
\pnum
\tcode{is_execution_policy} can be used to detect execution policies for the
purpose of excluding function signatures from otherwise ambiguous overload
resolution participation.

\pnum
\tcode{is_execution_policy<T>} shall be a UnaryTypeTrait with a
BaseCharacteristic of \tcode{true_type} if \tcode{T} is the type of a standard
or \impldef{additional execution policies supported by parallel algorithms}
execution policy, otherwise \tcode{false_type}.

\begin{note}
This provision reserves the privilege of creating non-standard execution
policies to the library implementation.
\end{note}

\pnum
The behavior of a program that adds specializations for
\tcode{is_execution_policy} is undefined.
\end{itemdescr}

\rSec2[execpol.seq]{Sequenced execution policy}

\indexlibrary{\idxcode{execution::sequenced_policy}}%
\begin{itemdecl}
class execution::sequenced_policy { @\unspec@ };
\end{itemdecl}

\begin{itemdescr}
\pnum
The class \tcode{execution::sequenced_policy} is an execution policy type used
as a unique type to disambiguate parallel algorithm overloading and require
that a parallel algorithm's execution may not be parallelized.
\end{itemdescr}

\rSec2[execpol.par]{Parallel execution policy}

\indexlibrary{\idxcode{execution::parallel_policy}}%
\begin{itemdecl}
class execution::parallel_policy { @\unspec@ };
\end{itemdecl}

\begin{itemdescr}
\pnum
The class \tcode{execution::parallel_policy} is an execution policy type used as
a unique type to disambiguate parallel algorithm overloading and indicate that
a parallel algorithm's execution may be parallelized.
\end{itemdescr}

\rSec2[execpol.parunseq]{Parallel and unsequenced execution policy}

\indexlibrary{\idxcode{execution::parallel_unsequenced_policy}}%
\begin{itemdecl}
class execution::parallel_unsequenced_policy { @\unspec@ };
\end{itemdecl}

\begin{itemdescr}
\pnum
The class \tcode{execution::parallel_unsequenced_policy} is an execution policy type
used as a unique type to disambiguate parallel algorithm overloading and
indicate that a parallel algorithm's execution may be parallelized and
vectorized.
\end{itemdescr}

\rSec2[execpol.objects]{Execution policy objects}

\indexlibrary{\idxcode{seq}}%
\indexlibrary{\idxcode{par}}%
\indexlibrary{\idxcode{par_unseq}}%
\indexlibrary{\idxcode{execution}!\idxcode{seq}}%
\indexlibrary{\idxcode{execution}!\idxcode{par}}%
\indexlibrary{\idxcode{execution}!\idxcode{par_unseq}}%
\begin{itemdecl}
constexpr execution::sequenced_policy            execution::seq{ @\unspec@ };
constexpr execution::parallel_policy             execution::par{ @\unspec@ };
constexpr execution::parallel_unsequenced_policy execution::par_unseq{ @\unspec@ };
\end{itemdecl}

\begin{itemdescr}
\pnum
The header \tcode{<execution>} declares global objects associated with each type of execution policy.
\end{itemdescr}
